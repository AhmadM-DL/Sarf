%\documentstyle{llncs}
%\documentclass[12pt,fullpage]{llncs}
\documentclass[12pt]{article}
%\usepackage[left=3cm,top=1.2cm,right=3cm,nohead,nofoot]{geometry}
%\usepackage[letter,left=.25in,right=.25in,top=.17in,textwidth=7.5in,textheight=9in]{geometry}
%\usepackage[left=0cm]{geometry}

%\documentclass[12pt]{report}
%\usepackage {utthesis2}              %% Preamble.

%\usepackage{amsmath}
%\usepackage{amssymb}
\usepackage{arabtex}
\usepackage{amsmath}
\usepackage{amssymb}
\usepackage{times}
\usepackage{caption}
\usepackage{epsfig}
\usepackage{subfigure}
\usepackage{color}
\usepackage{rotate}
\usepackage{rotating}
\usepackage{multirow}
%\usepackage[colorlinks=false]{color,hyperref}
\usepackage{color,hyperref}
%\usepackage{amsthm}
\usepackage{booktabs}
\usepackage{multirow}
\usepackage{natbib}
\bibpunct{[}{]}{,}{n}{}{;}

\usepackage{url}

\usepackage{relsize}
\usepackage{fancyvrb}
\usepackage{fancyhdr,lastpage}

\usepackage{utf8}
\setarab
\fullvocalize
\transtrue
\arabtrue

\newcommand{\CharCodeIn}[1]{`\CodeIn{#1}'}
\newcommand{\CodeIn}[1]{{\small\texttt{#1}}}
\newcommand{\frl}[1]{\fbox{\RL{#1}}} 
\newcommand{\noArRL}[1]{\arabfalse\RL{#1}\arabtrue} 
\newcommand{\noTrRL}[1]{\transfalse\RL{#1}\transtrue} 
\newcommand{\noTrNoVocRL}[1]{\novocalize\transfalse\RL{#1}\transtrue\vocalize} 
%\newcommand{\drawline}{\begin{picture}(6,.1) \put(0,0) {\line(1,0){6.25}}\end{picture}}

\usepackage{setspace}
%\doublespacing
\renewcommand{\baselinestretch}{1.15}
\setlength{\parindent}{0in}
%\parskip 6pt
%\parindent 0pt
%\setlength{\parskip}{.05in}
%\oddsidemargin 0in
%\evensidemargin 0in
\oddsidemargin .0in
\evensidemargin .0in
\hoffset -.75in
\voffset -.83in
\textwidth 7.5in
\textheight 9.5in

\topsep 0in
\topmargin 0.17in

%\usepackage{arabtex}
%\usepackage{utf8}

\begin{document}


\pagestyle{fancy}
%\lhead{}
\chead{}
\rhead{NPRP No.~:~~~~4-484-1-075}

\lfoot{QNRF Form}
\cfoot{}
\rfoot{Page \thepage~of~\pageref{LastPage} }
\renewcommand{\footrulewidth}{0.2pt}
\renewcommand{\headrulewidth}{0.2pt}

%\begin{titlepage}

\begin{center}
{\Large \bf Relational Arabic Text Mining Framework using 
    Morphological
    and Case-Based Analysis }

%\setlength{\unitlength}{1in}
%\setlength{\unitlength}{1in}
%\vspace{1.5in}
\vspace{.1in}

\renewcommand{\arraystretch}{.6}
\begin{tabular}{cc}
Fadi Zaraket & Rehab Duwairi \\
%Electrical and Computer Engineering &  Computer Science and Engineering Department \\
American University of Beirut & University of Qatar \\
{\tt fadi.zaraket@aub.edu.lb} & {\tt Rehab.Duwairi@qu.edu.qa}
\end{tabular}

\vspace{.1in}

\renewcommand{\arraystretch}{.6}
\begin{tabular}{c}
{\small Response to PR comments } \\
\end{tabular}
\vspace{.1in}

\date{\today}
%\pagebreak
\end{center}

%\section{Background (max of 2 pages)}
%\label{s:background}
We detail in this document our responses to the comments of
the respected reviewers of our previous Arabic text mining
framework proposal.
We rewrote the proposal to reflect changes that address the 
comments of the reviewers and changes that reflect the substantial
progress we made after submitting the initial proposal in December 
2009.
We managed to build ATSarf~\cite{ATMine09}, 
a novel case-based morphological analyzer 
and used it to automate the analysis of 
hadith literature books. 
We extracted directed acyclic graphs that 
represent a partial order relation between narrators and narrations,
and we are looking at investigaive and interrogative police reports.
This work led us to modify our approach and build the framework 
thorugh solving several case studies.
For each case study we will visit the computational components we 
needed to make it work and generalize them as library components. 
We will refine these components to make them general enough
while working on the next case study.
We adopted a case-based approach to build a flexible
low level analysis infrastructure.
The research plan document explains the approach in more details. 
This work led to manuscripts submitted for publication in 
renowned conferences. 
We hope that this addresses the {\em concern of 
the reviewers about the experience of the LPI in the area
of Arabic text mining}. 

%In the following we address the comments of the reviewers.
%We start by addressing the common concerns and
%The positive comments of most of the reviewers concerning the novelty
%of our approach encouraged us to continue 
{\bf Concern 1.Specific Arabic features.(PR1,PR2,PR3,PR4,PR5)~~}
Reviewers liked the idea that we are going to use 
features specific to the Arabic language in our analysis and had
a concern that we did not explain how we are going to do that in 
detail. 
We mentioned previously that we are going to build relaxed
Arabic grammar rules and use them in the analysis. 
We describe how to do that by building a grammar 
that formalizes the Arabic language similar to 
ElixirFM~\cite{Otakar:07}.
Our grammar will be novel in that it will follow a relational 
morphological approach rather than a functional morphological 
approach. 
That is the solutions will be expressed in a directed acyclic graph
representing a hierarchy of relations between the morphemes 
rather than moprhological trees mapping the morphemes to the 
solutions.
We think this is better since it provides a more succint
representation for the solutions and because interesting relations 
between morphemes can be computed statically before
the input is presented to the analyzer.
Morphological trees are traversals of these DAGs and are expensive 
to compute statically. 

We will compile this formalization into finite state machines that
compute solutions of input text strings.
We will simplify these FSMs before starting the analysis 
based on the user input query.
We consider the parse graph of the user input query 
which has atomic elements corresonding to a state variables
in the FSM.
We use this correspondence and use simple structural analysis 
techniques such as constant propagation and cone of influence
reductions to simplify the FSM. 
We think that this simplification is Arabic specific since
the query language will be designed to match the
structure of question sentences in common standard Arabic,
and the query itself reflects the way an Arabic user
himself thinks and searches. 

{\bf Concern 2. Details about case studies(PR1,PR2,.~~}
Peer reviewer 1 (PR1) shared a common concern with other reviewers
about missing details of the case studies. 
We addressed this concern by providing extensive details to the
hadith and security case studies and by explaining our
approach to solve the health and network content case studies.
We named the literature case study the hadith case study as 
PR1 suggested. 
We explained how the authenticity will be represented as annotations
to the graph of Figure 2 of the research document. 
We leave the quantification of the annotations to the expert
scholars %to avoid the controversy of hadith interpretation
and provide a partial order of the annotations of each narrator.
The subject scholar can decide where to draw his thresholds. 
We hope that addresses PR1's question on evaluation and quantification
of authenticity.
PR1 commented that a contradiction may not mean non authentication.
This is true and this is why the role of the tool is to flag 
inconsistencies and leave it to the expert to decide. 
As for additional sample queries, they are definitely welcome.
For example, the graph in Figure 2 allows 
for measuring the effect of one narrator 
against a certain subject covered in the hadith. 
PR1 asks for the details of how the solver will work, we address that 
in Section 4.10 as well as in the explanation of Figures 3-6 including
the diagrams of the case studies. 
PR1 also suggests involving experts in the project, the LPI is 
working with volunteers including experts in the hadith~\cite{Zar06},
health and security subjects. 
PR2 and PR5 asked how the data will be gathered, we 
addressed that in Sections
4.2--4.5 for each of the case studies. 
PR4 was concerned about the relevance and semantical intricacy 
of the case studies and suggests other case studies such as
web mining, economy, and intelligence. 
We kept our case studies as other PRs were favorable of them,
we added a web based case study and we explained how the same 
components can be used to analyze the market.
Intelligence (in the sense of machine learning) is on our future 
work list.
We understand the semantical intricacy concern of PR4 and we hope
that our preliminary results show that we can overcome 
the semantical complexities that concerned PR4. 


{\bf Concern 3. Strengthen literature review (PR1,PR2,PR3,PR4,PR5).~~}
%@inproceedings{AlSham08,
%We hope this addresses the concern of PR4 on the shortcomings
%of previous approaches and the potentials of ours. 
We went over the literature suggested by the reviewers. 
We compared our approach to current and previous approaches in the
reseach plan document. 
Here we refer to the references cited by PR5 and illustrate how
we made great use of them.
We hope this addresses the concern of PR4 on the shortcomings
of previous approaches and the potentials of ours. 

We validated our case-based approach by the observation of 
Daoud Daoud~\cite{Dao09} that the pipeline model is not 
adequate for Arabic. 
He suggests a synchronized syntactic and morphological
approach while we take the argument further to 
allow a tighter integration between the pipeline
levels up to the query in question by providing a case-based
morphological analyzer. 
%However, Mr. Daoud makes a mistake when he assumes that a noun
%cannot be followed by a preposition in Arabic to resolve
%an ambiguity in his example~\noTrNoVocRL{.s-a.hb-aa s-amy _dhb-aa 
%il_A alswq} (Sami's two companions went to the market) and decide 
%that \noTrNoVocRL{_dhb-aa} is a verb (they went)
%and not a noun since it is followed by \noTrNoVocRL{il_A}.
%Surely, one can say~\noTrNoVocRL{a_h_d  .s-a.hb-aa s-amy _dhb-aa 
%il_A alswq} (Sami's two companions took gold to the market) and 
%now \noTrNoVocRL{_dhb-aa} is a noun (gold stressed with double 
%diactric).
The work~\cite{AlSham08} reports a similar result where
the automated addition of syntax knowledge improved steeming
in terms of accuracy and efficiency. 

The work of Rogati in~\cite{Rog03} builds a stemmer based on
a parallel corpus with a document containing Arabic and English
text and another document containing only English with the
translation of the Arabic embedded in place. 
The translation is done manually and an English stemmer is used. 
This approach works for parallel corpora which is not the case
for our data of interest. 
It is also highly dependent on the accuracy of the manual translation.
Several interpretations of a word may be omitted if the context
of the corpus happened to bias one meaning.

Toutanova in~\cite{Tou09} suggested that using 
a joint model to do lemmatization and POS tagging performs better
than the direct and the pipelined approach. 
We take the first argument further to suggest a full joint NLP task 
model via using the NLP query to simplify and render the analysis
more accurate at the low level morphological analysis.

Later, Toutanova, in joint work with Smith and others~\cite{Smi10},
suggested better extraction quality from comparable or related
corpora. 
They consider two sets of documents of the same nature from
two different languages and extract parallel structures. 
We are actually considering the same genre of analysis in 
our proposal and case studies only in the same language. 
For example, the hadith narrations are comparable to the hadith 
biographies as far as the researcher is concerned with authenticity.
The same applies to the clinical reports and the pharmaceutical 
records, and the investigative reports from the crime scenes
and the interrogative reports with suspects.

We benefited a lot from the work of Maamouri~\cite{Maamouri:10}
and the SAMA~\cite{Kulick:10} analyzer. 
In~\cite{Maamouri:10} Maamouri considering adding a new corpus
to the Arabic Tree Bank and the addition challenged existing
annotation guidelines. 
The solution was to refine SAMA and have it interact with the
high level NLP task.
We find strong evidence in that to support our approach of
a case-based morphological analyzer integrated
with the linguistic computational model.

%@inproceedings{AlSham08,
%We hope this addresses the concern of PR4 on the shortcomings
%of previous approaches and the potentials of ours. 

{\bf Concern 4. Charging model.~~}

{\bf Concern 5. Budget issues (PR1,PR2,PR4)~~}

{\bf Concern 6. List of outcomes (PR1, PR2, ).~~}

{\bf Concern 7. Web based case studies (PR2, ).~~}
We addressed the concern of PR2 by adding a case study that 
analyzes web traffic. 
The opinion mining track suggested by PR2 is also of interest and 
can be another application we consider.
Opinion classification and detection of spam opinions
can make use of the traffic classification modules of the network
content casey study, and the span opinion detection can 
make use of the inconsistency checkers of the hadith case study.


{\bf Concern 8. Improve evaluation (PR3).~~}

{\bf Concern 9. Team involved (PR3).~~}

{\bf Concern 10. Novelty (PR4).~~}

{\bf Concern 11. Plan for dissemination.(PR4).~~}



we explain the term relational (PR3) in .. 
why extreme programming (PR4)
what existing techniques (PR5)

types of documents (PR5): comparable documents

logic solver (PR5)
size of documents (PR5)











\pagebreak
%\bibliography{adnan_refs}
%\bibliographystyle{ieeetr}
\bibliographystyle{abbrvnat}
%\bibliographystyle{ieeetr}
{\small
\bibliography{fzAr}  
}


\end{document}
