%\documentstyle{llncs}
%\documentclass[12pt,fullpage]{llncs}
\documentclass[12pt]{article}
%\usepackage[left=3cm,top=1.2cm,right=3cm,nohead,nofoot]{geometry}
%\usepackage[letter,left=.25in,right=.25in,top=.17in,textwidth=7.5in,textheight=9in]{geometry}
%\usepackage[left=0cm]{geometry}

%\documentclass[12pt]{report}
%\usepackage {utthesis2}              %% Preamble.

%\usepackage{amsmath}
%\usepackage{amssymb}
\usepackage{arabtex}
\usepackage{amsmath}
\usepackage{amssymb}
\usepackage{times}
\usepackage{caption}
\usepackage{epsfig}
\usepackage{subfigure}
\usepackage{color}
\usepackage{rotate}
\usepackage{rotating}
\usepackage{multirow}
%\usepackage[colorlinks=false]{color,hyperref}
\usepackage{color,hyperref}
%\usepackage{amsthm}
\usepackage{booktabs}
\usepackage{multirow}
\usepackage{natbib}
\bibpunct{[}{]}{,}{n}{}{;}

\usepackage{url}

\usepackage{relsize}
\usepackage{fancyvrb}
\usepackage{fancyhdr,lastpage}

\usepackage{utf8}
\setarab
\fullvocalize
\transtrue
\arabtrue

\newcommand{\CharCodeIn}[1]{`\CodeIn{#1}'}
\newcommand{\CodeIn}[1]{{\small\texttt{#1}}}
\newcommand{\frl}[1]{\fbox{\RL{#1}}} 
\newcommand{\noArRL}[1]{\arabfalse\RL{#1}\arabtrue} 
\newcommand{\noTrRL}[1]{\transfalse\RL{#1}\transtrue} 
\newcommand{\noTrNoVocRL}[1]{\novocalize\transfalse\RL{#1}\transtrue\vocalize} 
%\newcommand{\drawline}{\begin{picture}(6,.1) \put(0,0) {\line(1,0){6.25}}\end{picture}}

\usepackage{setspace}
%\doublespacing
\renewcommand{\baselinestretch}{1.15}
\setlength{\parindent}{0in}
%\parskip 6pt
%\parindent 0pt
%\setlength{\parskip}{.05in}
%\oddsidemargin 0in
%\evensidemargin 0in
\oddsidemargin .0in
\evensidemargin .0in
\hoffset -.75in
\voffset -.83in
\textwidth 7.5in
\textheight 9.5in

\topsep 0in
\topmargin 0.17in

%\usepackage{arabtex}
%\usepackage{utf8}

\begin{document}


\pagestyle{fancy}
%\lhead{}
\chead{}
\rhead{NPRP No.~:~~~~4-484-1-075}

\lfoot{QNRF Form}
\cfoot{}
\rfoot{Page \thepage~of~\pageref{LastPage} }
\renewcommand{\footrulewidth}{0.2pt}
\renewcommand{\headrulewidth}{0.2pt}

%\begin{titlepage}

\begin{center}
{\Large \bf Relational Arabic Text Mining Framework using 
    Morphological
    and Case-Based Analysis }

%\setlength{\unitlength}{1in}
%\setlength{\unitlength}{1in}
%\vspace{1.5in}

\renewcommand{\arraystretch}{.6}
\begin{tabular}{cc}
Fadi Zaraket & Rehab Duwairi \\
%Electrical and Computer Engineering &  Computer Science and Engineering Department \\
American University of Beirut & University of Qatar \\
{\tt fadi.zaraket@aub.edu.lb} & {\tt Rehab.Duwairi@qu.edu.qa}
\end{tabular}

%\vspace{1.5in}

\renewcommand{\arraystretch}{.6}
\begin{tabular}{c}
{\small Rebuttle to PR comments } \\
\end{tabular}
\vspace{.5in}

\date{\today}
%\pagebreak
\end{center}

%\section{Background (max of 2 pages)}
%\label{s:background}
In this document we discuss the changes made to the 
Arabic text mining framework proposal. 
We changed the proposal based on the comments of the respected reviewers
and on the progress we made during last year on this project. 

\pagebreak
%\bibliography{adnan_refs}
%\bibliographystyle{ieeetr}
\bibliographystyle{abbrvnat}
%\bibliographystyle{ieeetr}
{\small
\bibliography{fzAr}  
}


\end{document}
