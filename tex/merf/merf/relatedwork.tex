The work in \cite{dukes2011supervised} presents a collaborative effort towards morphological 
and syntactic annotation of the Quran. 
The task is held through online supervised collaboration using a multi-stage approach that 
includes automatic POS tagging, manual verification, and online supervised collaborative 
proofreading. 
Moreover, the work in \cite{dorr2010interlingual} presents a framework for interlingual 
annotation of parallel text corpora with multi-level representations. 
This corpora is important for NLP tasks including text summarization, information retrieval, 
and machine translation.
An overview of annotation tools and their Arabic-English word alignment issues 
concludes with a set of rules and guidelines needed in an Arabic annotation alignment tool
~\cite{kholidy2010towards}.
The work in \cite{kulick2010consistent} presents the integration of the Standard Arabic Morphological 
Analyzer (SAMA) into the annotation workflow of the Arabic Treebank.
Such tasks motivated us to build \framework, a morphology based open source annotation tool 
for the Arabic language.

MMAX2 is a manual multi-level linguistic annotation tool with an XML 
based data model ~\cite{mmax2}. 
It enables the user to create, browse, visualize, and query annotations
and may be able to resolve coreference tags. 
BRAT is a multi-lingual user friendly manual web-based annotator that allows the construction 
of entity and relation annotation corpora~\cite{brat}. 
BRAT provides an API for automated annotators to provide annotations.
WordFreak is similar to BRAT. It supports Arabic text and can be extended through a plug-in 
architecture to integrate with NLP and CL tasks. 
The plug-in API may enable the use of automatic annotators along with customized visualization 
and annotation specifications~\cite{wordfreak}. 
AGTK is a toolkit for the development of text and speech annotation tools~\cite{agtk}. 
It provides import APIs from other data and graphical user interface (GUI) components. 
The work in \cite{smrz2004morphotrees} presents the extension of TrEd, a customizable 
general purpose tree editor, with the Arabic MorphoTrees annotation. The MorphoTrees present 
the morphological analyses in a hierarchical organization based on common features.

\framework differs from MMAX2, BRAT, WordFreak, AGTK, and TrEd in that it allows the user to specify 
sophisticated tag types using Boolean formulae of Arabic morphological features. 
These are key in CL and NLP for Arabic text. 
Fassieh is a commercial Arabic text annotation tool that enables the production of large 
Arabic text corpora~\cite{attia2009fassieh}. 
The tool supports Arabic text factorizations including morphological analysis, POS tagging, 
full phonetic transcription, and lexical semantics analysis in an automatic mode. 
Fassieh is not directly accessible to the research community and requires commercial licensing. 
\framework is open source and differs in that it allows the user to build tag types 
using Boolean formulae of several atomic terms and define Sequential formulae based on the Boolean formulae. 

Task specific annotation tools such as~\cite{alrahabi2006semantic}
uses enunciation semantic maps to automatically annotate 
directly reported Arabic and French speech. 
AraTation is another task specific tool for semantic annotation of 
Arabic news using web ontology based semantic maps~\cite{saleh2009aratation}.
We differ in that \framework is general, and not task specific, and it uses 
morphology based features as atomic terms. 