\setcode{utf8}
\setarab

\def\pp{\ensuremath{{\cal P}}} % prefix
\def\ss{\ensuremath{{\cal S}}} % stem
\def\xx{\ensuremath{{\cal X}}} % suffix
\def\PP{\ensuremath{\mathit{POS}}} % pos
\def\GG{\ensuremath{\mathit{GLOSS}}} % gloss
\def\AC{\ensuremath{\mathit{CAT}}} % category

In this Section we define and discuss terms important to our method such as 
morphological features, finite state transducers, morphological analyzer, 
classes, labels, and tag types.

\subsection{Morphological features}

Arabic is a morphologically rich language. 
Due to the rich Arabic morphology, a single Arabic word 
might replace a complete sentence in English. 
For example, the word \RL{يَأْكُله}
stands for the English 
statement ``he/it is eating him/it''.

Form-based morphological analysis decomposes an Arabic word into
several morphemes~\cite{habash2010introduction}.
A morpheme is the smallest linguistic unit 
that has a meaning and fulfills a grammatical function. 
A morpheme can be a stem or an affix. 
Each morpheme is associated with other morphological features 
including {\em POS}, {\em gloss}, {\em lemma}, and {\em category} tags. 

A stem can be a root word, a {\em templatic}, or {\em non-templatic} stem. 
Templatic stems are formed from roots using template morphological rules. 
For example, the stem
\RL{kAtb} (writer) is a subject noun formed from the root verb \RL{ktb} (write). 
Non-templatic stems tend to be foreign names such as \RL{واشنطن} (Washington). 
A root is an atomic word with three, four, and rarely five letters.
We denote the set of all stems by the symbol \ss.

An affix can be a {\em prefix}, a {\em suffix}, or an {\em infix}. 
Prefixes attach before the stem and a word can have multiple prefixes. 
We denote the set of all prefixes by the symbol \pp. 
Suffixes attach after the stem and a word can have multiple suffixes. 
We denote the set of all suffixes by the symbol \xx.
Infixes are inserted inside the stem to form a new stem. 
In this work we consider a set of stems that includes infix morphological changes. 

The part-of-speech tag, referred to as POS, assigns a morpho-syntactic tag for a morpheme. 
Sample POS tags are {\tt NOUN} and {\tt VERB\_PERFECT}. 
We denote the set of all POS tags by the symbol \PP.

The gloss is a brief notation of the semantic meaning of a morpheme in English. 
A word might have multiple glosses attached to it as it could stand for multiple meanings. 
We denote the set of all glosses by the symbol \GG.

The lemma is a convention choice that assigns a core meaning of word forms that share similar stems. 
For example, the words \RL{byt}(house), 
\RL{llbyt}(for the house), 
and \RL{bywt}(houses) are mapped to the singular noun form \RL{byt} as a lemma.

The category is a user defined tag that the user can assign to several 
morphemes.
For example, the user can define a temporal category to include the 
prefix \RL{س} (will) and the time unit \RL{ساعة} (hour). 
We denote the set of all categories by the symbol \AC.

\subsection{Morphological analyzer}

An Arabic morphological analyzer takes an Arabic word 
and returns a set of morphological solutions. 
Each solution splits the word into its morphemes and 
associates each morpheme with corresponding tags. 
Multiple solutions can be the result of multiple valid 
segmentations of the word into morphemes, multiple 
possible tags associated with a morpheme.

\framework is integrated with {\em Sarf}, 
an in-house open source Arabic morphological analyzer based on 
finite state transducers~\cite{ZaMaColing2012DemosSarf}. 
We use Sarf to extract the morphological features of words present in a text. 
Sarf takes an Arabic word as input and returns a set of morphological solutions. 
Each solution splits the word into its morphemes and associates each morpheme with the corresponding tags. 
We refer to the morphemes and associated tags as features. 
These features are used to tag each input word with one or more user-defined tags. 
Sarf processes a word $w_{i}$ in a text and returns the relevant solution vectors. 
%Multiple solutions are the result of multiple valid segmentations of the word into morphemes, 
%or the multiple possible tags associated with a morpheme. 
%Also, a word might have multiple solution vectors as it can have multiple interpretations and meanings.
%The contribution of Sarf over previous work is that it considers the Arabic affixes as agglutinative, i.e. composed of one or more morphemes. 
%Hence, it introduces fusional compatibility rules for affix-affix concatenations. 
%This contribution results in a lexicon which is smaller in size, has less redundancy, and resolves a number of inconsistencies. 
%Figure~\ref{fig:sarfFSM} shows the finite state machine of Sarf. 
%Boxes denote legal affixes and stems, and circles denote regular nodes. 
%The edges are transitions and the labels correspond to the input letters. 
%$\epsilon$ represents an empty string and is the source of non-determinism.
%The transitions between the prefix \pp, stem \ss, and suffix \xx~sub-machines are non-deterministic to compute all valid morphological analyses.

%\transfalse
%\begin{figure}[h!]
%\begin{center}
%\resizebox{\columnwidth}{!}{
%	\input{figures/FSM.pdftex_t}
%}
%\caption{FSM of Sarf}
%\label{fig:sarfFSM}
%\end{center}
%\end{figure}
%\transtrue

%An example is shown in Table~\ref{tab:table1} below:

% Table generated by Excel2LaTeX from sheet 'Sheet1'
%\begin{table}[h]
%\setcode{utf8}
%\setarab
%  \centering
%    \begin{tabular}{|r|r|}
%    \toprule
%    \RL{وَلَد} & child/son \\
%    \midrule
%    \RL{وَلَّدَ}& generate \\
%    \midrule
%    \RL{وُلِدَ}& be born \\
%    \bottomrule
%    \end{tabular}%
%    \caption{Word with different interpretations}
%    \label{tab:table1}%
%\setcode{standard}
%\end{table}%

A sample solution vector of Sarf is shown in Table~\ref{tab:table2}. 
$w_{i}$-index shows the index of the word $w_{i}$ in the text. 
The terms pref and suf in the features refer to prefix and suffix respectively. 
The features pref-data, stem-data, and suf-data presents the prefix, stem, and suffix morphemes with diacritics. 
Sarf provides an additional feature called stem-ac. 
This feature shows a list of the abstract categories relevant to this stem. 
For example, ``Country'' is an abstract category for ``Lebanon''.

%By the end of the morphological analysis stage, Sarf returns a sequence of solution vectors. 
%These vectors contain all the possible morphological analyses of the words in the input text.
\setcode{standard}