{\em Computational Linguistics} (CL) is concerned with building 
accurate linguistic computational models. 
{\em Natural Language Processing} (NLP) is concerned with automating the 
understanding of natural language. 
CL and NLP tasks range from simple ones such as spell checking and typing 
error correction to more 
complex tasks including %text summarization, 
{\em named entity recognition} (NER), {\em cross-document analysis}, 
{\em machine translation}, 
and {\em relational entity extraction}~\cite{linckels2011natural,ferilli2011natural}.
Entities are elements of text that are of interest to an NLP task. 
Relational entities are elements that connect entities. 
{\em Annotations} relate chunks of text to {\em labels} denoting
semantic values such as entities or relational entities.
We refer to annotations and labels as {\em tags} and {\em tag types}, respectively, 
in the sequel.

Supervised and unsupervised empirical learning techniques tackle NLP and CL tasks. 
They employ machine learning without the need to manually encode the requisite 
knowledge~\cite{soudi2007arabic}. 
Supervised learning techniques require training corpora annotated 
with {\em correct} tags to learn a computational model. 
Supervised and unsupervised techniques require annotated reference corpora to 
evaluate the accuracy of the technique using metrics such as precision and 
recall~\cite{englishtreebank,arabictreebank,chinesetreebank}. 

Researchers build %the required %seed 
training and reference corpora 
either manually, incrementally using learning techniques, or using knowledge-based 
annotation techniques that recognize and extract entities and relational 
entities from text. 
Knowledge-based techniques use linguistic and 
rhetorical domain specific knowledge encoded into sets of rules 
to extract entities and relational entities~\cite{soudi2007arabic}.
While existing annotation, entity, and relational entity 
extraction tools exist
\cite{chiticariu2010systemt% SystemT
,atzmueller2008rule%Textmarker
,urbain2012user%
,settles2011closing% dualist
,mmax2%MMAX
,brat% BRAT
},
most of them lack Arabic language support, and almost all of them
lack Arabic morphological analysis support
% that is key to Arabic NLP due to the rich morphological nature of Arabic
~\cite{habash2006arabic}. 
Fassieh~\cite{attia2009fassieh} is a {\em commercial} Arabic annotation tool
with morphological analysis support and text factorization. 
However, this tool lacks support for entity and relational entity extraction.

\transtrue
\setcode{utf8}
\setarab
\transfalse
\begin{figure}[tb]
\begin{center}
\relsize{-0.5}
%\renewcommand{\arraystretch}{1.5}% Wider
\resizebox{\columnwidth}{!}{
    \begin{tabular}{|C{10cm}C{8cm}|} 
      \hline
  \begin{tabular}{R{10cm}}
\RL{تقود} \RL{وأنت} $\stackrel{u1}{\framebox[1.2\width]{\RL{الأول}}}$ $\stackrel{p_2}{\framebox[1.2\width]{\RL{التقاطع}}}$ \RL{من} $\stackrel{r_1}{\framebox[1.2\width]{\RL{بالقرب}}}$ $\stackrel{n_1}{\framebox[1.2\width]{\RL{خليفة}}}$ $\stackrel{p_1}{\framebox[1.2\width]{\RL{برج}}}$ \RL{تلاحظ} \RL{ألا} \RL{المستحيل} \RL{من} 
  \\ ~ \\
$\stackrel{r_3}{\framebox[1.2\width]{\RL{فيها}}}$ \RL{تسلك} \RL{التي} $\stackrel{u_2}{\framebox[1.2\width]{\RL{الأولى}}}$ \RL{المرّة} \RL{هذه} \RL{كانت} \RL{وإن} \RL{حتّى} $\stackrel{n_3}{\framebox[1.2\width]{\RL{زايد،}}}$ $\stackrel{n_2}{\framebox[1.2\width]{\RL{الشيخ}}}$ $\stackrel{p_3}{\framebox[1.2\width]{\RL{شارع}}}$ $\stackrel{r_2}{\framebox[1.2\width]{\RL{في}}}$ \RL{سيارتك}
  \\ ~ \\
\RL{العالم.} \RL{في} \RL{الأطول} \RL{يُعد} \RL{الذي}  $\stackrel{p_7}{\framebox[1.2\width]{\RL{المبنى}}}$ \RL{هذا} \RL{من} $\stackrel{r_4}{\framebox[1.2\width]{\RL{مقربة}}}$ \RL{على} $\stackrel{p_6}{\framebox[1.2\width]{\RL{مول}}}$ $\stackrel{p_5}{\framebox[1.2\width]{\RL{دبيّ}}}$ \RL{يقع} $\stackrel{p_4}{\framebox[1.2\width]{\RL{الطريق؛}}}$ \RL{هذا}
  \\ ~ \\
%  \RL{العالم.} \RL{في} \RL{الأطول} \RL{يُعد} \RL{الذي}  
%  \\
  \multicolumn{1}{L{10cm}}{
   \arabfalse \transtrue
   \RL{من المستحيل ألا تلاحظ برج خليفة بالقرب من التقاطع الأول وأنت تقود سيارتك في شارع الشيخ زايد، حتّى وإن كانت هذه المرّة الأولى التي تسلك فيها هذا الطريق؛ يقع دبيّ مول على مقربة من هذا المبنى الذي يُعد الأطول في العالم.}
   \arabtrue \transfalse} 
  \\ ~ \\
  \multicolumn{1}{L{10cm}}{It is impossible not to notice $\stackrel{n1}{\framebox[1.2\width]{Khalifa}}$ $\stackrel{p_1}{\framebox[1.2\width]{Tower}}$ $\stackrel{r_1}{\framebox[1.2\width]{next to}}$ the $\stackrel{u_1}{\framebox[1.2\width]{first}}$ $\stackrel{p_2}{\framebox[1.2\width]{intersection}}$ while you are driving your car $\stackrel{r_2}{\framebox[1.2\width]{on}}$ $\stackrel{n_2}{\framebox[1.2\width]{Sheikh}}$ $\stackrel{n_3}{\framebox[1.2\width]{Zayed}}$ $\stackrel{p_3}{\framebox[1.2\width]{Road}}$, even if this was $\stackrel{u_2}{\framebox[1.2\width]{the first}}$ time that you take this $\stackrel{p_4}{\framebox[1.2\width]{road}}$; $\stackrel{p_5}{\framebox[1.2\width]{Dubai}}$ $\stackrel{p_6}{\framebox[1.2\width]{Mall}}$ is located $\stackrel{r_4}{\framebox[1.2\width]{near}}$ this $\stackrel{p_7}{\framebox[1.2\width]{building}}$, which is the longest in the world.} 
  \end{tabular}
&
\resizebox{0.6\columnwidth}{!}{
  % Graphic for TeX using PGF
% Title: /home/ameen/Desktop/ergraphEx.dia
% Creator: Dia v0.97.1
% CreationDate: Sat Jan 11 03:49:31 2014
% For: ameen
% \usepackage{tikz}
% The following commands are not supported in PSTricks at present
% We define them conditionally, so when they are implemented,
% this pgf file will use them.
\ifx\du\undefined
  \newlength{\du}
\fi
\setlength{\du}{30\unitlength}
\begin{tikzpicture}
\pgftransformxscale{1.000000}
\pgftransformyscale{-1.000000}
\definecolor{dialinecolor}{rgb}{0.000000, 0.000000, 0.000000}
\pgfsetstrokecolor{dialinecolor}
\definecolor{dialinecolor}{rgb}{1.000000, 1.000000, 1.000000}
\pgfsetfillcolor{dialinecolor}
\definecolor{dialinecolor}{rgb}{1.000000, 1.000000, 1.000000}
\pgfsetfillcolor{dialinecolor}
\pgfpathellipse{\pgfpoint{4.324130\du}{10.573360\du}}{\pgfpoint{1.223030\du}{0\du}}{\pgfpoint{0\du}{0.734470\du}}
\pgfusepath{fill}
\pgfsetlinewidth{0.050000\du}
\pgfsetdash{}{0pt}
\pgfsetdash{}{0pt}
\pgfsetmiterjoin
\definecolor{dialinecolor}{rgb}{0.000000, 0.000000, 0.000000}
\pgfsetstrokecolor{dialinecolor}
\pgfpathellipse{\pgfpoint{4.324130\du}{10.573360\du}}{\pgfpoint{1.223030\du}{0\du}}{\pgfpoint{0\du}{0.734470\du}}
\pgfusepath{stroke}
% setfont left to latex
\definecolor{dialinecolor}{rgb}{0.000000, 0.000000, 0.000000}
\pgfsetstrokecolor{dialinecolor}
\node at (4.324130\du,10.465026\du){\RL{دبي مول}};
% setfont left to latex
\definecolor{dialinecolor}{rgb}{0.000000, 0.000000, 0.000000}
\pgfsetstrokecolor{dialinecolor}
\node at (4.324130\du,10.888360\du){Dubai Mall};
\definecolor{dialinecolor}{rgb}{1.000000, 1.000000, 1.000000}
\pgfsetfillcolor{dialinecolor}
\pgfpathellipse{\pgfpoint{6.415976\du}{8.893981\du}}{\pgfpoint{1.395976\du}{0\du}}{\pgfpoint{0\du}{0.720041\du}}
\pgfusepath{fill}
\pgfsetlinewidth{0.050000\du}
\pgfsetdash{}{0pt}
\pgfsetdash{}{0pt}
\pgfsetmiterjoin
\definecolor{dialinecolor}{rgb}{0.000000, 0.000000, 0.000000}
\pgfsetstrokecolor{dialinecolor}
\pgfpathellipse{\pgfpoint{6.415976\du}{8.893981\du}}{\pgfpoint{1.395976\du}{0\du}}{\pgfpoint{0\du}{0.720041\du}}
\pgfusepath{stroke}
% setfont left to latex
\definecolor{dialinecolor}{rgb}{0.000000, 0.000000, 0.000000}
\pgfsetstrokecolor{dialinecolor}
\node at (6.415976\du,8.785647\du){\RL{المبنى}};
% setfont left to latex
\definecolor{dialinecolor}{rgb}{0.000000, 0.000000, 0.000000}
\pgfsetstrokecolor{dialinecolor}
\node at (6.415976\du,9.208981\du){the building};
\definecolor{dialinecolor}{rgb}{1.000000, 1.000000, 1.000000}
\pgfsetfillcolor{dialinecolor}
\pgfpathellipse{\pgfpoint{8.776400\du}{10.616657\du}}{\pgfpoint{0.927400\du}{0\du}}{\pgfpoint{0\du}{0.546667\du}}
\pgfusepath{fill}
\pgfsetlinewidth{0.050000\du}
\pgfsetdash{}{0pt}
\pgfsetdash{}{0pt}
\pgfsetmiterjoin
\definecolor{dialinecolor}{rgb}{0.000000, 0.000000, 0.000000}
\pgfsetstrokecolor{dialinecolor}
\pgfpathellipse{\pgfpoint{8.776400\du}{10.616657\du}}{\pgfpoint{0.927400\du}{0\du}}{\pgfpoint{0\du}{0.546667\du}}
\pgfusepath{stroke}
% setfont left to latex
\definecolor{dialinecolor}{rgb}{0.000000, 0.000000, 0.000000}
\pgfsetstrokecolor{dialinecolor}
\node at (8.776400\du,10.508324\du){\RL{مقربة}};
% setfont left to latex
\definecolor{dialinecolor}{rgb}{0.000000, 0.000000, 0.000000}
\pgfsetstrokecolor{dialinecolor}
\node at (8.776400\du,10.931657\du){near};
\definecolor{dialinecolor}{rgb}{1.000000, 1.000000, 1.000000}
\pgfsetfillcolor{dialinecolor}
\pgfpathellipse{\pgfpoint{4.451892\du}{6.494554\du}}{\pgfpoint{1.451892\du}{0\du}}{\pgfpoint{0\du}{0.787194\du}}
\pgfusepath{fill}
\pgfsetlinewidth{0.050000\du}
\pgfsetdash{}{0pt}
\pgfsetdash{}{0pt}
\pgfsetmiterjoin
\definecolor{dialinecolor}{rgb}{0.000000, 0.000000, 0.000000}
\pgfsetstrokecolor{dialinecolor}
\pgfpathellipse{\pgfpoint{4.451892\du}{6.494554\du}}{\pgfpoint{1.451892\du}{0\du}}{\pgfpoint{0\du}{0.787194\du}}
\pgfusepath{stroke}
% setfont left to latex
\definecolor{dialinecolor}{rgb}{0.000000, 0.000000, 0.000000}
\pgfsetstrokecolor{dialinecolor}
\node at (4.451892\du,6.386221\du){\RL{برج خليفة}};
% setfont left to latex
\definecolor{dialinecolor}{rgb}{0.000000, 0.000000, 0.000000}
\pgfsetstrokecolor{dialinecolor}
\node at (4.451892\du,6.809554\du){Khalifa tower};
\definecolor{dialinecolor}{rgb}{1.000000, 1.000000, 1.000000}
\pgfsetfillcolor{dialinecolor}
\pgfpathellipse{\pgfpoint{6.414368\du}{4.903560\du}}{\pgfpoint{1.590768\du}{0\du}}{\pgfpoint{0\du}{0.722790\du}}
\pgfusepath{fill}
\pgfsetlinewidth{0.050000\du}
\pgfsetdash{}{0pt}
\pgfsetdash{}{0pt}
\pgfsetmiterjoin
\definecolor{dialinecolor}{rgb}{0.000000, 0.000000, 0.000000}
\pgfsetstrokecolor{dialinecolor}
\pgfpathellipse{\pgfpoint{6.414368\du}{4.903560\du}}{\pgfpoint{1.590768\du}{0\du}}{\pgfpoint{0\du}{0.722790\du}}
\pgfusepath{stroke}
% setfont left to latex
\definecolor{dialinecolor}{rgb}{0.000000, 0.000000, 0.000000}
\pgfsetstrokecolor{dialinecolor}
\node at (6.414368\du,4.795227\du){\RL{التقاطع الأول}};
% setfont left to latex
\definecolor{dialinecolor}{rgb}{0.000000, 0.000000, 0.000000}
\pgfsetstrokecolor{dialinecolor}
\node at (6.414368\du,5.218560\du){intersection 1};
\definecolor{dialinecolor}{rgb}{1.000000, 1.000000, 1.000000}
\pgfsetfillcolor{dialinecolor}
\pgfpathellipse{\pgfpoint{8.800968\du}{6.567226\du}}{\pgfpoint{1.083468\du}{0\du}}{\pgfpoint{0\du}{0.584486\du}}
\pgfusepath{fill}
\pgfsetlinewidth{0.050000\du}
\pgfsetdash{}{0pt}
\pgfsetdash{}{0pt}
\pgfsetmiterjoin
\definecolor{dialinecolor}{rgb}{0.000000, 0.000000, 0.000000}
\pgfsetstrokecolor{dialinecolor}
\pgfpathellipse{\pgfpoint{8.800968\du}{6.567226\du}}{\pgfpoint{1.083468\du}{0\du}}{\pgfpoint{0\du}{0.584486\du}}
\pgfusepath{stroke}
% setfont left to latex
\definecolor{dialinecolor}{rgb}{0.000000, 0.000000, 0.000000}
\pgfsetstrokecolor{dialinecolor}
\node at (8.800968\du,6.458893\du){\RL{بالقرب}};
% setfont left to latex
\definecolor{dialinecolor}{rgb}{0.000000, 0.000000, 0.000000}
\pgfsetstrokecolor{dialinecolor}
\node at (8.800968\du,6.882226\du){next to};
% setfont left to latex
\definecolor{dialinecolor}{rgb}{0.000000, 0.000000, 0.000000}
\pgfsetstrokecolor{dialinecolor}
\node at (6.673600\du,10.362490\du){\RL{على}};
% setfont left to latex
\definecolor{dialinecolor}{rgb}{0.000000, 0.000000, 0.000000}
\pgfsetstrokecolor{dialinecolor}
\node at (6.673600\du,10.785823\du){prep};
% setfont left to latex
\definecolor{dialinecolor}{rgb}{0.000000, 0.000000, 0.000000}
\pgfsetstrokecolor{dialinecolor}
\node at (9.098600\du,9.312490\du){\RL{من هذا}};
% setfont left to latex
\definecolor{dialinecolor}{rgb}{0.000000, 0.000000, 0.000000}
\pgfsetstrokecolor{dialinecolor}
\node at (9.098600\du,9.735823\du){from this};
\definecolor{dialinecolor}{rgb}{1.000000, 1.000000, 1.000000}
\pgfsetfillcolor{dialinecolor}
\fill (3.913600\du,8.797490\du)--(3.913600\du,9.615823\du)--(4.733600\du,9.615823\du)--(4.733600\du,8.797490\du)--cycle;
% setfont left to latex
\definecolor{dialinecolor}{rgb}{0.000000, 0.000000, 0.000000}
\pgfsetstrokecolor{dialinecolor}
\node at (4.323600\du,9.112490\du){\RL{مقربة}};
% setfont left to latex
\definecolor{dialinecolor}{rgb}{0.000000, 0.000000, 0.000000}
\pgfsetstrokecolor{dialinecolor}
\node at (4.323600\du,9.535823\du){near};
% setfont left to latex
\definecolor{dialinecolor}{rgb}{0.000000, 0.000000, 0.000000}
\pgfsetstrokecolor{dialinecolor}
\node[anchor=west] at (3.823600\du,8.012500\du){\cci{isSyn}};
% setfont left to latex
\definecolor{dialinecolor}{rgb}{0.000000, 0.000000, 0.000000}
\pgfsetstrokecolor{dialinecolor}
\node at (3.848700\du,4.987490\du){\RL{بالقرب}};
% setfont left to latex
\definecolor{dialinecolor}{rgb}{0.000000, 0.000000, 0.000000}
\pgfsetstrokecolor{dialinecolor}
\node at (3.848700\du,5.410823\du){next to};
% setfont left to latex
\definecolor{dialinecolor}{rgb}{0.000000, 0.000000, 0.000000}
\pgfsetstrokecolor{dialinecolor}
\node at (8.892600\du,5.337490\du){\RL{من}};
% setfont left to latex
\definecolor{dialinecolor}{rgb}{0.000000, 0.000000, 0.000000}
\pgfsetstrokecolor{dialinecolor}
\node at (8.892600\du,5.760823\du){from};
% setfont left to latex
\definecolor{dialinecolor}{rgb}{0.000000, 0.000000, 0.000000}
\pgfsetstrokecolor{dialinecolor}
\node[anchor=west] at (7.103800\du,7.837500\du){};
\pgfsetlinewidth{0.050000\du}
\pgfsetdash{}{0pt}
\pgfsetdash{}{0pt}
\pgfsetbuttcap
{
\definecolor{dialinecolor}{rgb}{0.000000, 0.000000, 0.000000}
\pgfsetfillcolor{dialinecolor}
% was here!!!
\definecolor{dialinecolor}{rgb}{0.000000, 0.000000, 0.000000}
\pgfsetstrokecolor{dialinecolor}
\pgfpathmoveto{\pgfpoint{8.386343\du}{6.027332\du}}
\pgfpatharc{1}{-61}{0.960686\du and 0.960686\du}
\pgfusepath{stroke}
}
\pgfsetlinewidth{0.050000\du}
\pgfsetdash{}{0pt}
\pgfsetdash{}{0pt}
\pgfsetbuttcap
{
\definecolor{dialinecolor}{rgb}{0.000000, 0.000000, 0.000000}
\pgfsetfillcolor{dialinecolor}
% was here!!!
\definecolor{dialinecolor}{rgb}{0.000000, 0.000000, 0.000000}
\pgfsetstrokecolor{dialinecolor}
\pgfpathmoveto{\pgfpoint{4.944705\du}{5.180159\du}}
\pgfpatharc{266}{181}{0.530542\du and 0.530542\du}
\pgfusepath{stroke}
}
\pgfsetlinewidth{0.050000\du}
\pgfsetdash{}{0pt}
\pgfsetdash{}{0pt}
\pgfsetbuttcap
{
\definecolor{dialinecolor}{rgb}{0.000000, 0.000000, 0.000000}
\pgfsetfillcolor{dialinecolor}
% was here!!!
\definecolor{dialinecolor}{rgb}{0.000000, 0.000000, 0.000000}
\pgfsetstrokecolor{dialinecolor}
\pgfpathmoveto{\pgfpoint{5.126282\du}{9.169518\du}}
\pgfpatharc{243}{167}{0.649761\du and 0.649761\du}
\pgfusepath{stroke}
}
\pgfsetlinewidth{0.050000\du}
\pgfsetdash{}{0pt}
\pgfsetdash{}{0pt}
\pgfsetbuttcap
{
\definecolor{dialinecolor}{rgb}{0.000000, 0.000000, 0.000000}
\pgfsetfillcolor{dialinecolor}
% was here!!!
\definecolor{dialinecolor}{rgb}{0.000000, 0.000000, 0.000000}
\pgfsetstrokecolor{dialinecolor}
\pgfpathmoveto{\pgfpoint{8.421500\du}{10.111607\du}}
\pgfpatharc{350}{297}{1.328688\du and 1.328688\du}
\pgfusepath{stroke}
}
\pgfsetlinewidth{0.050000\du}
\pgfsetdash{}{0pt}
\pgfsetdash{}{0pt}
\pgfsetbuttcap
{
\definecolor{dialinecolor}{rgb}{0.000000, 0.000000, 0.000000}
\pgfsetfillcolor{dialinecolor}
% was here!!!
\definecolor{dialinecolor}{rgb}{0.000000, 0.000000, 0.000000}
\pgfsetstrokecolor{dialinecolor}
\pgfpathmoveto{\pgfpoint{5.547041\du}{10.573278\du}}
\pgfpatharc{125}{58}{2.099405\du and 2.099405\du}
\pgfusepath{stroke}
}
\pgfsetlinewidth{0.050000\du}
\pgfsetdash{}{0pt}
\pgfsetdash{}{0pt}
\pgfsetbuttcap
{
\definecolor{dialinecolor}{rgb}{0.000000, 0.000000, 0.000000}
\pgfsetfillcolor{dialinecolor}
% was here!!!
\definecolor{dialinecolor}{rgb}{0.000000, 0.000000, 0.000000}
\pgfsetstrokecolor{dialinecolor}
\pgfpathmoveto{\pgfpoint{4.451867\du}{7.281656\du}}
\pgfpatharc{165}{113}{1.674544\du and 1.674544\du}
\pgfusepath{stroke}
}
% setfont left to latex
\definecolor{dialinecolor}{rgb}{0.000000, 0.000000, 0.000000}
\pgfsetstrokecolor{dialinecolor}
\node[anchor=west] at (6.318700\du,7.975000\du){$e_2$};
% setfont left to latex
\definecolor{dialinecolor}{rgb}{0.000000, 0.000000, 0.000000}
\pgfsetstrokecolor{dialinecolor}
\node[anchor=west] at (4.268700\du,8.825000\du){$r_4$};
% setfont left to latex
\definecolor{dialinecolor}{rgb}{0.000000, 0.000000, 0.000000}
\pgfsetstrokecolor{dialinecolor}
\node[anchor=west] at (4.198700\du,11.665000\du){$e_1$};
% setfont left to latex
\definecolor{dialinecolor}{rgb}{0.000000, 0.000000, 0.000000}
\pgfsetstrokecolor{dialinecolor}
\node[anchor=west] at (8.868700\du,11.525000\du){$r_4$};
% setfont left to latex
\definecolor{dialinecolor}{rgb}{0.000000, 0.000000, 0.000000}
\pgfsetstrokecolor{dialinecolor}
\node[anchor=west] at (8.881200\du,8.915000\du){$o_2$};
% setfont left to latex
\definecolor{dialinecolor}{rgb}{0.000000, 0.000000, 0.000000}
\pgfsetstrokecolor{dialinecolor}
\node[anchor=west] at (6.533700\du,11.315000\du){$o_1$};
% setfont left to latex
\definecolor{dialinecolor}{rgb}{0.000000, 0.000000, 0.000000}
\pgfsetstrokecolor{dialinecolor}
\node[anchor=west] at (3.046200\du,7.390000\du){$e_1$};
% setfont left to latex
\definecolor{dialinecolor}{rgb}{0.000000, 0.000000, 0.000000}
\pgfsetstrokecolor{dialinecolor}
\node[anchor=west] at (6.298700\du,3.940000\du){$e_2$};
% setfont left to latex
\definecolor{dialinecolor}{rgb}{0.000000, 0.000000, 0.000000}
\pgfsetstrokecolor{dialinecolor}
\node[anchor=west] at (8.753700\du,5.015000\du){$o_2$};
% setfont left to latex
\definecolor{dialinecolor}{rgb}{0.000000, 0.000000, 0.000000}
\pgfsetstrokecolor{dialinecolor}
\node[anchor=west] at (3.781200\du,4.590000\du){$r_1$};
% setfont left to latex
\definecolor{dialinecolor}{rgb}{0.000000, 0.000000, 0.000000}
\pgfsetstrokecolor{dialinecolor}
\node[anchor=west] at (8.708700\du,7.490000\du){$r_1$};
\end{tikzpicture}
}
\\  \hline
%\multicolumn{1}{c}{\multirow{1}{*}{\textbf{(a) Text with directions}}} &
%\multicolumn{1}{c}{\textbf{(b) Formula, matches, and entity-relation graph}} \\
    \end{tabular}
}
\end{center}
\vspace{-1em} 
  \caption{Direction example with Arabic text, annotated with entities, transliteration, translation, and extracted relational entities in a graph.} 
  \label{fig:intromotiv}
\vspace{-1em} 
\end{figure}
\transtrue
\setcode{standard}


%These tasks take as input digital text documents from sources including literature, books, news, business reports, chat messages, and emails. 
%Some documents are originally typed in digital format such as emails. 
%Other documents are automatically digitized using techniques such as {\em optical character recognition} (OCR) and 
%{\em automatic speech recognition}~\cite{margner2012guide,rashwan2012robust}. 

In this paper, 
we present \framework, a {\em morphology-based entity and relational entity
extraction framework for Arabic text }. 
\framework provides a user-friendly interface where the user defines tag types 
and associates them with \framework formulae that are regular expressions over 
\framework Boolean formulae. 
Boolean formulae are terms, negations of terms, and disjunctions of terms. 
Terms are matches to Arabic morphological features including 
prefix, stem, suffix, part of speech (POS) tags, gloss tags, extended synonym tags, 
and semantic categories. 
We discuss the importance of the morphological features supported in 
\framework terms in Section~\ref{sec:morph}; in brief, morphological preprocessing
is key to Arabic NLP.% due to the richness of Arabic morphology.

We illustrate the target of \framework using a simple example. 
Consider the text in Figure~\ref{fig:intromotiv} which contains
directions to Dubai Mall~\footnote{text taken from the Dubai Mall website
~\url{http://www.thedubaimall.com/ar/}}.
The framed words in the text are entities referring to names of people 
($n_1,n_2,n_3$), 
names of places ($p_1,\dots,p_7$), 
relative positions ($r_1,\dots,r_4$), 
and numerical terms ($u_1,u_2$). 
We would like to extract those entities, and then extract
the relational entities forming the graph in Figure~\ref{fig:intromotiv} 
where vertices express entities, 
and edges represent the relational entities.
\framework allows a regular user to do that interactively
in four phases that include morphological analysis, 
entity extraction based on morphological features, 
relational entity extraction, and entity cross-reference.

\framework regular expressions support operators such as concatenation, 
zero or one, zero or more, one or more, up to $M\in\mathbb{N}^+$, and logical 
conjunction and disjunction. 
An editor allows the user to associate an action with each sub-expression.
The user specifies the action with C++ code and uses an API to access information related to the matches such as text, position, length, morphological features, and numerical value.

\framework takes an Arabic text, a set of user-defined \framework Boolean 
formulae, and a set of user-defined \framework regular expressions. 
\framework computes the morphological solutions of the words in the 
input text then computes matches to the Boolean formulae. 
\framework then generates a {\em non-deterministic finite state 
automata} (NDFSA) for each expression and simulates it with the 
sequence of Boolean formulae matches to compute the regular 
expression matches. 
\framework generates executable code for the actions associated with
the regular expressions, 
compiles, links, and executes the generated code 
as shared object libraries. 
Finally, \framework constructs the semantic relations and 
cross-reference between entities. 
\framework also provides visualization tools to present the matches, 
and estimate their accuracy with respect to reference tags.

\subsection{Existing annotation and entity extraction tools}

Researchers proposed and evaluated empirical and 
knowledge-based techniques to extract entities and relational entities
from text. 
We briefly review them here and we discuss them and compare to them
in Section~\ref{sec:related}. 
The work in~\cite{ekbal2010named} presents a language independent 
approach for NER extraction using {\em support vector machines}. 
The work in~\cite{abdelrahman2010integrated} integrates 
a semi-supervised bootstrapping pattern recognition technique, 
and a supervised classifier based on {\em conditional random fields}
to solve NER problems. 

Knowledge-based techniques such 
as~\cite{zaghouani2010adapting,traboulsi2009arabic} propose local grammars 
with morphological stemming to perform NER. 
%The work in~
\cite{ZaMaHaCicling2012Entity} presents a method for extracting entities, 
events, and relations amongst them from Arabic text using a hierarchy of manually built 
finite state machines driven by morphological features and graph 
transformation algorithms. 
Such techniques require advanced linguistic and programming expertise. 
QARAB is a question answering system that takes 
Arabic natural language queries and provides short answers~\cite{hammo2002qarab}. 

Researchers also proposed systems for automatic IE based 
on user specifications. 
CPSL is a common pattern specification language for finite-state 
grammar~\cite{appelt1998common}. 
%\framework is an extension to CPSL with action execution and relation construction. 
%It defines three sections for declaration, rule definition, and macros. 
The work in \cite{chiticariu2010systemt} presents SystemT, a system based on 
an algebraic Approach to Declarative information extraction (IE). 
%It uses a declarative rule language and an optimizer. 
TEXTMARKER is a rule-based IE system designed to extract structured data from 
text \cite{atzmueller2008rule}. 
\cite{urbain2012user} presents a user-driven relational model 
requiring a user natural language query to extract entities and relational entities.

This work significantly extends the position 
paper~\cite{JaZaMatar} that allows for manual, and 
basic morphology annotation.
\framework enables a regular user to incrementally create 
complex annotations for Arabic text based on automatic 
extraction of morphological tags through a user friendly interactive interface. 
\framework has the following advantages.
\begin{itemize}
  \item \framework provides a novel and intuitive visual interface to build Boolean formulae over morphological features, 
    build regular expressions over the resulting Boolean formulae, and thereafter compute automatic tags.
  \item To our knowledge, this morphology-based framework  is the first for Arabic entity and relational entity extraction.
  \item \framework provides the user with the ability to rapidly create annotated Arabic text corpora with sophisticated morphology-based tags.
\end{itemize}

In \framework, we make the following contributions.
\begin{itemize}
  \item \framework enables the user to define relations in a simple manner
    and automatically detects relational entities matching the user defined relations. 
  \item \framework enables the user to associate subexpressions
    with code actions, and executes the code action 
    when a corresponding match is found. 
    It also provides an API to enable access to match 
    features such as text, position, length, numerical value, and morphological features.
  \item \framework enables the user to tag words based on a synonymic relation using the $Syn^k$ feature.
\end{itemize}

\setcode{utf8}
\setarab
\transfalse
\begin{figure}[tb!]
%\renewcommand{\arraystretch}{1.5}% Wider
\resizebox{0.9\columnwidth}{!}{
\begin{tabular}{c}
\resizebox{\columnwidth}{!}{
	\begin{tabular}{c|c|c|c}
		MBF & description & formula & matches\\ \hline
        N & name of person & $category~=~Name\_of\_Person$ & $n_1,n_2,n_3$ \\ \hline
        P & name of place & $category~=~Name\_of\_Place$ & $p_1,p_2,...,p_7$ \\ \hline
        R & relative position & $stem\in$ \{\RL{قرب},\RL{في},\dots\} & $r_1,r_2,r_3,r_4$ \\ \hline
        U & numerical term & $stem\in$ \{\RL{أول},\RL{ثاني},\dots\} & $u_1,u_2$ \\
    \end{tabular}
} 
\\ ~ \\
\resizebox{\columnwidth}{!} {
\begin{tabular}{cc}
\relsize{-2} % Graphic for TeX using PGF
% Title: /home/ameen/Desktop/match1.dia
% Creator: Dia v0.97.1
% CreationDate: Sat Jan 11 00:35:23 2014
% For: ameen
% \usepackage{tikz}
% The following commands are not supported in PSTricks at present
% We define them conditionally, so when they are implemented,
% this pgf file will use them.
\ifx\du\undefined
  \newlength{\du}
\fi
\setlength{\du}{15\unitlength}
\begin{tikzpicture}
\pgftransformxscale{1.000000}
\pgftransformyscale{-1.000000}
\definecolor{dialinecolor}{rgb}{0.000000, 0.000000, 0.000000}
\pgfsetstrokecolor{dialinecolor}
\definecolor{dialinecolor}{rgb}{1.000000, 1.000000, 1.000000}
\pgfsetfillcolor{dialinecolor}
\definecolor{dialinecolor}{rgb}{1.000000, 1.000000, 1.000000}
\pgfsetfillcolor{dialinecolor}
\fill (23.657500\du,7.687500\du)--(23.657500\du,8.537500\du)--(24.632500\du,8.537500\du)--(24.632500\du,7.687500\du)--cycle;
\pgfsetlinewidth{0.020000\du}
\pgfsetdash{}{0pt}
\pgfsetdash{}{0pt}
\pgfsetmiterjoin
\definecolor{dialinecolor}{rgb}{0.000000, 0.000000, 0.000000}
\pgfsetstrokecolor{dialinecolor}
\draw (23.657500\du,7.687500\du)--(23.657500\du,8.537500\du)--(24.632500\du,8.537500\du)--(24.632500\du,7.687500\du)--cycle;
% setfont left to latex
\definecolor{dialinecolor}{rgb}{0.000000, 0.000000, 0.000000}
\pgfsetstrokecolor{dialinecolor}
\node at (24.145000\du,8.233056\du){()};
\definecolor{dialinecolor}{rgb}{1.000000, 1.000000, 1.000000}
\pgfsetfillcolor{dialinecolor}
\pgfpathellipse{\pgfpoint{25.225101\du}{9.941465\du}}{\pgfpoint{1.185301\du}{0\du}}{\pgfpoint{0\du}{0.726265\du}}
\pgfusepath{fill}
\pgfsetlinewidth{0.020000\du}
\pgfsetdash{}{0pt}
\pgfsetdash{}{0pt}
\pgfsetmiterjoin
\definecolor{dialinecolor}{rgb}{0.000000, 0.000000, 0.000000}
\pgfsetstrokecolor{dialinecolor}
\pgfpathellipse{\pgfpoint{25.225101\du}{9.941465\du}}{\pgfpoint{1.185301\du}{0\du}}{\pgfpoint{0\du}{0.726265\du}}
\pgfusepath{stroke}
% setfont left to latex
\definecolor{dialinecolor}{rgb}{0.000000, 0.000000, 0.000000}
\pgfsetstrokecolor{dialinecolor}
\node at (25.225101\du,10.062021\du){r1=\RL{بالقرب}};
\definecolor{dialinecolor}{rgb}{1.000000, 1.000000, 1.000000}
\pgfsetfillcolor{dialinecolor}
\pgfpathellipse{\pgfpoint{22.605219\du}{13.362071\du}}{\pgfpoint{1.319819\du}{0\du}}{\pgfpoint{0\du}{0.707771\du}}
\pgfusepath{fill}
\pgfsetlinewidth{0.020000\du}
\pgfsetdash{}{0pt}
\pgfsetdash{}{0pt}
\pgfsetmiterjoin
\definecolor{dialinecolor}{rgb}{0.000000, 0.000000, 0.000000}
\pgfsetstrokecolor{dialinecolor}
\pgfpathellipse{\pgfpoint{22.605219\du}{13.362071\du}}{\pgfpoint{1.319819\du}{0\du}}{\pgfpoint{0\du}{0.707771\du}}
\pgfusepath{stroke}
% setfont left to latex
\definecolor{dialinecolor}{rgb}{0.000000, 0.000000, 0.000000}
\pgfsetstrokecolor{dialinecolor}
\node at (22.605219\du,13.482627\du){p2=\RL{التقاطع}};
\definecolor{dialinecolor}{rgb}{1.000000, 1.000000, 1.000000}
\pgfsetfillcolor{dialinecolor}
\fill (22.757500\du,9.782730\du)--(22.757500\du,10.667730\du)--(23.757500\du,10.667730\du)--(23.757500\du,9.782730\du)--cycle;
\pgfsetlinewidth{0.020000\du}
\pgfsetdash{}{0pt}
\pgfsetdash{}{0pt}
\pgfsetmiterjoin
\definecolor{dialinecolor}{rgb}{0.000000, 0.000000, 0.000000}
\pgfsetstrokecolor{dialinecolor}
\draw (22.757500\du,9.782730\du)--(22.757500\du,10.667730\du)--(23.757500\du,10.667730\du)--(23.757500\du,9.782730\du)--cycle;
% setfont left to latex
\definecolor{dialinecolor}{rgb}{0.000000, 0.000000, 0.000000}
\pgfsetstrokecolor{dialinecolor}
\node at (23.257500\du,10.345786\du){\^{}2};
\definecolor{dialinecolor}{rgb}{1.000000, 1.000000, 1.000000}
\pgfsetfillcolor{dialinecolor}
\pgfpathellipse{\pgfpoint{23.312350\du}{11.641042\du}}{\pgfpoint{0.695150\du}{0\du}}{\pgfpoint{0\du}{0.488202\du}}
\pgfusepath{fill}
\pgfsetlinewidth{0.020000\du}
\pgfsetdash{}{0pt}
\pgfsetdash{}{0pt}
\pgfsetmiterjoin
\definecolor{dialinecolor}{rgb}{0.000000, 0.000000, 0.000000}
\pgfsetstrokecolor{dialinecolor}
\pgfpathellipse{\pgfpoint{23.312350\du}{11.641042\du}}{\pgfpoint{0.695150\du}{0\du}}{\pgfpoint{0\du}{0.488202\du}}
\pgfusepath{stroke}
% setfont left to latex
\definecolor{dialinecolor}{rgb}{0.000000, 0.000000, 0.000000}
\pgfsetstrokecolor{dialinecolor}
\node at (23.312350\du,11.761598\du){\RL{من}};
\definecolor{dialinecolor}{rgb}{1.000000, 1.000000, 1.000000}
\pgfsetfillcolor{dialinecolor}
\pgfpathellipse{\pgfpoint{20.061250\du}{13.438719\du}}{\pgfpoint{1.046250\du}{0\du}}{\pgfpoint{0\du}{0.631119\du}}
\pgfusepath{fill}
\pgfsetlinewidth{0.020000\du}
\pgfsetdash{}{0pt}
\pgfsetdash{}{0pt}
\pgfsetmiterjoin
\definecolor{dialinecolor}{rgb}{0.000000, 0.000000, 0.000000}
\pgfsetstrokecolor{dialinecolor}
\pgfpathellipse{\pgfpoint{20.061250\du}{13.438719\du}}{\pgfpoint{1.046250\du}{0\du}}{\pgfpoint{0\du}{0.631119\du}}
\pgfusepath{stroke}
% setfont left to latex
\definecolor{dialinecolor}{rgb}{0.000000, 0.000000, 0.000000}
\pgfsetstrokecolor{dialinecolor}
\node at (20.061250\du,13.559275\du){u1=\RL{الأول}};
\definecolor{dialinecolor}{rgb}{1.000000, 1.000000, 1.000000}
\pgfsetfillcolor{dialinecolor}
\pgfpathellipse{\pgfpoint{25.370283\du}{13.399620\du}}{\pgfpoint{1.072583\du}{0\du}}{\pgfpoint{0\du}{0.670220\du}}
\pgfusepath{fill}
\pgfsetlinewidth{0.020000\du}
\pgfsetdash{}{0pt}
\pgfsetdash{}{0pt}
\pgfsetmiterjoin
\definecolor{dialinecolor}{rgb}{0.000000, 0.000000, 0.000000}
\pgfsetstrokecolor{dialinecolor}
\pgfpathellipse{\pgfpoint{25.370283\du}{13.399620\du}}{\pgfpoint{1.072583\du}{0\du}}{\pgfpoint{0\du}{0.670220\du}}
\pgfusepath{stroke}
% setfont left to latex
\definecolor{dialinecolor}{rgb}{0.000000, 0.000000, 0.000000}
\pgfsetstrokecolor{dialinecolor}
\node at (25.370283\du,13.520176\du){n1=\RL{خليفة}};
\definecolor{dialinecolor}{rgb}{1.000000, 1.000000, 1.000000}
\pgfsetfillcolor{dialinecolor}
\pgfpathellipse{\pgfpoint{27.556244\du}{13.449916\du}}{\pgfpoint{0.861244\du}{0\du}}{\pgfpoint{0\du}{0.619916\du}}
\pgfusepath{fill}
\pgfsetlinewidth{0.020000\du}
\pgfsetdash{}{0pt}
\pgfsetdash{}{0pt}
\pgfsetmiterjoin
\definecolor{dialinecolor}{rgb}{0.000000, 0.000000, 0.000000}
\pgfsetstrokecolor{dialinecolor}
\pgfpathellipse{\pgfpoint{27.556244\du}{13.449916\du}}{\pgfpoint{0.861244\du}{0\du}}{\pgfpoint{0\du}{0.619916\du}}
\pgfusepath{stroke}
% setfont left to latex
\definecolor{dialinecolor}{rgb}{0.000000, 0.000000, 0.000000}
\pgfsetstrokecolor{dialinecolor}
\node at (27.556244\du,13.570472\du){p1=\RL{برج}};
\definecolor{dialinecolor}{rgb}{1.000000, 1.000000, 1.000000}
\pgfsetfillcolor{dialinecolor}
\fill (26.957500\du,9.717730\du)--(26.957500\du,10.667730\du)--(27.882500\du,10.667730\du)--(27.882500\du,9.717730\du)--cycle;
\pgfsetlinewidth{0.020000\du}
\pgfsetdash{}{0pt}
\pgfsetdash{}{0pt}
\pgfsetmiterjoin
\definecolor{dialinecolor}{rgb}{0.000000, 0.000000, 0.000000}
\pgfsetstrokecolor{dialinecolor}
\draw (26.957500\du,9.717730\du)--(26.957500\du,10.667730\du)--(27.882500\du,10.667730\du)--(27.882500\du,9.717730\du)--cycle;
% setfont left to latex
\definecolor{dialinecolor}{rgb}{0.000000, 0.000000, 0.000000}
\pgfsetstrokecolor{dialinecolor}
\node at (27.420000\du,10.313286\du){$+$};
\definecolor{dialinecolor}{rgb}{1.000000, 1.000000, 1.000000}
\pgfsetfillcolor{dialinecolor}
\fill (27.082500\du,11.203540\du)--(27.082500\du,12.078540\du)--(27.982500\du,12.078540\du)--(27.982500\du,11.203540\du)--cycle;
\pgfsetlinewidth{0.020000\du}
\pgfsetdash{}{0pt}
\pgfsetdash{}{0pt}
\pgfsetmiterjoin
\definecolor{dialinecolor}{rgb}{0.000000, 0.000000, 0.000000}
\pgfsetstrokecolor{dialinecolor}
\draw (27.082500\du,11.203540\du)--(27.082500\du,12.078540\du)--(27.982500\du,12.078540\du)--(27.982500\du,11.203540\du)--cycle;
% setfont left to latex
\definecolor{dialinecolor}{rgb}{0.000000, 0.000000, 0.000000}
\pgfsetstrokecolor{dialinecolor}
\node at (27.532500\du,11.761596\du){$|$};
\definecolor{dialinecolor}{rgb}{1.000000, 1.000000, 1.000000}
\pgfsetfillcolor{dialinecolor}
\fill (21.507500\du,11.178540\du)--(21.507500\du,12.103540\du)--(22.457500\du,12.103540\du)--(22.457500\du,11.178540\du)--cycle;
\pgfsetlinewidth{0.020000\du}
\pgfsetdash{}{0pt}
\pgfsetdash{}{0pt}
\pgfsetmiterjoin
\definecolor{dialinecolor}{rgb}{0.000000, 0.000000, 0.000000}
\pgfsetstrokecolor{dialinecolor}
\draw (21.507500\du,11.178540\du)--(21.507500\du,12.103540\du)--(22.457500\du,12.103540\du)--(22.457500\du,11.178540\du)--cycle;
% setfont left to latex
\definecolor{dialinecolor}{rgb}{0.000000, 0.000000, 0.000000}
\pgfsetstrokecolor{dialinecolor}
\node at (21.982500\du,11.761596\du){$|$};
\definecolor{dialinecolor}{rgb}{1.000000, 1.000000, 1.000000}
\pgfsetfillcolor{dialinecolor}
\fill (20.382500\du,9.717730\du)--(20.382500\du,10.667730\du)--(21.307500\du,10.667730\du)--(21.307500\du,9.717730\du)--cycle;
\pgfsetlinewidth{0.020000\du}
\pgfsetdash{}{0pt}
\pgfsetdash{}{0pt}
\pgfsetmiterjoin
\definecolor{dialinecolor}{rgb}{0.000000, 0.000000, 0.000000}
\pgfsetstrokecolor{dialinecolor}
\draw (20.382500\du,9.717730\du)--(20.382500\du,10.667730\du)--(21.307500\du,10.667730\du)--(21.307500\du,9.717730\du)--cycle;
% setfont left to latex
\definecolor{dialinecolor}{rgb}{0.000000, 0.000000, 0.000000}
\pgfsetstrokecolor{dialinecolor}
\node at (20.845000\du,10.313286\du){$+$};
\pgfsetlinewidth{0.020000\du}
\pgfsetdash{}{0pt}
\pgfsetdash{}{0pt}
\pgfsetbuttcap
{
\definecolor{dialinecolor}{rgb}{0.000000, 0.000000, 0.000000}
\pgfsetfillcolor{dialinecolor}
% was here!!!
\pgfsetarrowsend{latex}
\definecolor{dialinecolor}{rgb}{0.000000, 0.000000, 0.000000}
\pgfsetstrokecolor{dialinecolor}
\draw (24.642726\du,8.428648\du)--(26.953057\du,9.896135\du);
}
\pgfsetlinewidth{0.020000\du}
\pgfsetdash{}{0pt}
\pgfsetdash{}{0pt}
\pgfsetbuttcap
{
\definecolor{dialinecolor}{rgb}{0.000000, 0.000000, 0.000000}
\pgfsetfillcolor{dialinecolor}
% was here!!!
\pgfsetarrowsend{latex}
\definecolor{dialinecolor}{rgb}{0.000000, 0.000000, 0.000000}
\pgfsetstrokecolor{dialinecolor}
\draw (27.457628\du,10.677150\du)--(27.497728\du,11.193393\du);
}
\pgfsetlinewidth{0.020000\du}
\pgfsetdash{}{0pt}
\pgfsetdash{}{0pt}
\pgfsetbuttcap
{
\definecolor{dialinecolor}{rgb}{0.000000, 0.000000, 0.000000}
\pgfsetfillcolor{dialinecolor}
% was here!!!
\pgfsetarrowsend{latex}
\definecolor{dialinecolor}{rgb}{0.000000, 0.000000, 0.000000}
\pgfsetstrokecolor{dialinecolor}
\draw (27.538372\du,12.088401\du)--(27.547978\du,12.820166\du);
}
\definecolor{dialinecolor}{rgb}{1.000000, 1.000000, 1.000000}
\pgfsetfillcolor{dialinecolor}
\fill (24.832500\du,11.191040\du)--(24.832500\du,12.091040\du)--(25.807500\du,12.091040\du)--(25.807500\du,11.191040\du)--cycle;
\pgfsetlinewidth{0.020000\du}
\pgfsetdash{}{0pt}
\pgfsetdash{}{0pt}
\pgfsetmiterjoin
\definecolor{dialinecolor}{rgb}{0.000000, 0.000000, 0.000000}
\pgfsetstrokecolor{dialinecolor}
\draw (24.832500\du,11.191040\du)--(24.832500\du,12.091040\du)--(25.807500\du,12.091040\du)--(25.807500\du,11.191040\du)--cycle;
% setfont left to latex
\definecolor{dialinecolor}{rgb}{0.000000, 0.000000, 0.000000}
\pgfsetstrokecolor{dialinecolor}
\node at (25.320000\du,11.761596\du){$|$};
\pgfsetlinewidth{0.020000\du}
\pgfsetdash{}{0pt}
\pgfsetdash{}{0pt}
\pgfsetbuttcap
{
\definecolor{dialinecolor}{rgb}{0.000000, 0.000000, 0.000000}
\pgfsetfillcolor{dialinecolor}
% was here!!!
\pgfsetarrowsend{latex}
\definecolor{dialinecolor}{rgb}{0.000000, 0.000000, 0.000000}
\pgfsetstrokecolor{dialinecolor}
\draw (26.947808\du,10.518388\du)--(25.817314\du,11.298056\du);
}
\pgfsetlinewidth{0.020000\du}
\pgfsetdash{}{0pt}
\pgfsetdash{}{0pt}
\pgfsetbuttcap
{
\definecolor{dialinecolor}{rgb}{0.000000, 0.000000, 0.000000}
\pgfsetfillcolor{dialinecolor}
% was here!!!
\pgfsetarrowsend{latex}
\definecolor{dialinecolor}{rgb}{0.000000, 0.000000, 0.000000}
\pgfsetstrokecolor{dialinecolor}
\draw (25.333062\du,12.097859\du)--(25.350826\du,12.719115\du);
}
\pgfsetlinewidth{0.020000\du}
\pgfsetdash{}{0pt}
\pgfsetdash{}{0pt}
\pgfsetbuttcap
{
\definecolor{dialinecolor}{rgb}{0.000000, 0.000000, 0.000000}
\pgfsetfillcolor{dialinecolor}
% was here!!!
\pgfsetarrowsend{latex}
\definecolor{dialinecolor}{rgb}{0.000000, 0.000000, 0.000000}
\pgfsetstrokecolor{dialinecolor}
\draw (24.400258\du,8.544736\du)--(24.816635\du,9.249798\du);
}
\pgfsetlinewidth{0.020000\du}
\pgfsetdash{}{0pt}
\pgfsetdash{}{0pt}
\pgfsetbuttcap
{
\definecolor{dialinecolor}{rgb}{0.000000, 0.000000, 0.000000}
\pgfsetfillcolor{dialinecolor}
% was here!!!
\pgfsetarrowsend{latex}
\definecolor{dialinecolor}{rgb}{0.000000, 0.000000, 0.000000}
\pgfsetstrokecolor{dialinecolor}
\draw (23.962560\du,8.546806\du)--(23.447524\du,9.772871\du);
}
\pgfsetlinewidth{0.020000\du}
\pgfsetdash{}{0pt}
\pgfsetdash{}{0pt}
\pgfsetbuttcap
{
\definecolor{dialinecolor}{rgb}{0.000000, 0.000000, 0.000000}
\pgfsetfillcolor{dialinecolor}
% was here!!!
\pgfsetarrowsend{latex}
\definecolor{dialinecolor}{rgb}{0.000000, 0.000000, 0.000000}
\pgfsetstrokecolor{dialinecolor}
\draw (23.275029\du,10.677695\du)--(23.293053\du,11.142950\du);
}
\definecolor{dialinecolor}{rgb}{1.000000, 1.000000, 1.000000}
\pgfsetfillcolor{dialinecolor}
\fill (19.635000\du,11.178540\du)--(19.635000\du,12.103540\du)--(20.585000\du,12.103540\du)--(20.585000\du,11.178540\du)--cycle;
\pgfsetlinewidth{0.020000\du}
\pgfsetdash{}{0pt}
\pgfsetdash{}{0pt}
\pgfsetmiterjoin
\definecolor{dialinecolor}{rgb}{0.000000, 0.000000, 0.000000}
\pgfsetstrokecolor{dialinecolor}
\draw (19.635000\du,11.178540\du)--(19.635000\du,12.103540\du)--(20.585000\du,12.103540\du)--(20.585000\du,11.178540\du)--cycle;
% setfont left to latex
\definecolor{dialinecolor}{rgb}{0.000000, 0.000000, 0.000000}
\pgfsetstrokecolor{dialinecolor}
\node at (20.110000\du,11.761596\du){$|$};
\pgfsetlinewidth{0.020000\du}
\pgfsetdash{}{0pt}
\pgfsetdash{}{0pt}
\pgfsetbuttcap
{
\definecolor{dialinecolor}{rgb}{0.000000, 0.000000, 0.000000}
\pgfsetfillcolor{dialinecolor}
% was here!!!
\pgfsetarrowsend{latex}
\definecolor{dialinecolor}{rgb}{0.000000, 0.000000, 0.000000}
\pgfsetstrokecolor{dialinecolor}
\draw (21.225463\du,10.677150\du)--(21.611202\du,11.168288\du);
}
\pgfsetlinewidth{0.020000\du}
\pgfsetdash{}{0pt}
\pgfsetdash{}{0pt}
\pgfsetbuttcap
{
\definecolor{dialinecolor}{rgb}{0.000000, 0.000000, 0.000000}
\pgfsetfillcolor{dialinecolor}
% was here!!!
\pgfsetarrowsend{latex}
\definecolor{dialinecolor}{rgb}{0.000000, 0.000000, 0.000000}
\pgfsetstrokecolor{dialinecolor}
\draw (20.599163\du,10.677150\du)--(20.349916\du,11.168288\du);
}
\pgfsetlinewidth{0.020000\du}
\pgfsetdash{}{0pt}
\pgfsetdash{}{0pt}
\pgfsetbuttcap
{
\definecolor{dialinecolor}{rgb}{0.000000, 0.000000, 0.000000}
\pgfsetfillcolor{dialinecolor}
% was here!!!
\pgfsetarrowsend{latex}
\definecolor{dialinecolor}{rgb}{0.000000, 0.000000, 0.000000}
\pgfsetstrokecolor{dialinecolor}
\draw (20.097182\du,12.113721\du)--(20.078639\du,12.797506\du);
}
\pgfsetlinewidth{0.020000\du}
\pgfsetdash{}{0pt}
\pgfsetdash{}{0pt}
\pgfsetbuttcap
{
\definecolor{dialinecolor}{rgb}{0.000000, 0.000000, 0.000000}
\pgfsetfillcolor{dialinecolor}
% was here!!!
\pgfsetarrowsend{latex}
\definecolor{dialinecolor}{rgb}{0.000000, 0.000000, 0.000000}
\pgfsetstrokecolor{dialinecolor}
\draw (22.153535\du,12.113735\du)--(22.350263\du,12.657440\du);
}
\pgfsetlinewidth{0.020000\du}
\pgfsetdash{}{0pt}
\pgfsetdash{}{0pt}
\pgfsetbuttcap
{
\definecolor{dialinecolor}{rgb}{0.000000, 0.000000, 0.000000}
\pgfsetfillcolor{dialinecolor}
% was here!!!
\pgfsetarrowsend{latex}
\definecolor{dialinecolor}{rgb}{0.000000, 0.000000, 0.000000}
\pgfsetstrokecolor{dialinecolor}
\draw (23.647905\du,8.425855\du)--(21.317522\du,9.894865\du);
}
% setfont left to latex
\definecolor{dialinecolor}{rgb}{0.000000, 0.000000, 0.000000}
\pgfsetstrokecolor{dialinecolor}
\node[anchor=west] at (21.340000\du,8.166810\du){match 1};
\end{tikzpicture}
 &
\relsize{-2} % Graphic for TeX using PGF
% Title: /home/ameen/Desktop/match2.dia
% Creator: Dia v0.97.1
% CreationDate: Sat Jan 11 00:36:28 2014
% For: ameen
% \usepackage{tikz}
% The following commands are not supported in PSTricks at present
% We define them conditionally, so when they are implemented,
% this pgf file will use them.
\ifx\du\undefined
  \newlength{\du}
\fi
\setlength{\du}{15\unitlength}
\begin{tikzpicture}
\pgftransformxscale{1.000000}
\pgftransformyscale{-1.000000}
\definecolor{dialinecolor}{rgb}{0.000000, 0.000000, 0.000000}
\pgfsetstrokecolor{dialinecolor}
\definecolor{dialinecolor}{rgb}{1.000000, 1.000000, 1.000000}
\pgfsetfillcolor{dialinecolor}
\definecolor{dialinecolor}{rgb}{1.000000, 1.000000, 1.000000}
\pgfsetfillcolor{dialinecolor}
\fill (24.100350\du,11.350000\du)--(24.100350\du,12.250000\du)--(25.075350\du,12.250000\du)--(25.075350\du,11.350000\du)--cycle;
\pgfsetlinewidth{0.020000\du}
\pgfsetdash{}{0pt}
\pgfsetdash{}{0pt}
\pgfsetmiterjoin
\definecolor{dialinecolor}{rgb}{0.000000, 0.000000, 0.000000}
\pgfsetstrokecolor{dialinecolor}
\draw (24.100350\du,11.350000\du)--(24.100350\du,12.250000\du)--(25.075350\du,12.250000\du)--(25.075350\du,11.350000\du)--cycle;
% setfont left to latex
\definecolor{dialinecolor}{rgb}{0.000000, 0.000000, 0.000000}
\pgfsetstrokecolor{dialinecolor}
\node at (24.587850\du,11.920556\du){()};
\definecolor{dialinecolor}{rgb}{1.000000, 1.000000, 1.000000}
\pgfsetfillcolor{dialinecolor}
\pgfpathellipse{\pgfpoint{20.350327\du}{17.174574\du}}{\pgfpoint{1.216577\du}{0\du}}{\pgfpoint{0\du}{0.632074\du}}
\pgfusepath{fill}
\pgfsetlinewidth{0.020000\du}
\pgfsetdash{}{0pt}
\pgfsetdash{}{0pt}
\pgfsetmiterjoin
\definecolor{dialinecolor}{rgb}{0.000000, 0.000000, 0.000000}
\pgfsetstrokecolor{dialinecolor}
\pgfpathellipse{\pgfpoint{20.350327\du}{17.174574\du}}{\pgfpoint{1.216577\du}{0\du}}{\pgfpoint{0\du}{0.632074\du}}
\pgfusepath{stroke}
% setfont left to latex
\definecolor{dialinecolor}{rgb}{0.000000, 0.000000, 0.000000}
\pgfsetstrokecolor{dialinecolor}
\node at (20.350327\du,17.295129\du){p7=\RL{المبنى}};
\definecolor{dialinecolor}{rgb}{1.000000, 1.000000, 1.000000}
\pgfsetfillcolor{dialinecolor}
\pgfpathellipse{\pgfpoint{23.081828\du}{17.180054\du}}{\pgfpoint{1.082178\du}{0\du}}{\pgfpoint{0\du}{0.637554\du}}
\pgfusepath{fill}
\pgfsetlinewidth{0.020000\du}
\pgfsetdash{}{0pt}
\pgfsetdash{}{0pt}
\pgfsetmiterjoin
\definecolor{dialinecolor}{rgb}{0.000000, 0.000000, 0.000000}
\pgfsetstrokecolor{dialinecolor}
\pgfpathellipse{\pgfpoint{23.081828\du}{17.180054\du}}{\pgfpoint{1.082178\du}{0\du}}{\pgfpoint{0\du}{0.637554\du}}
\pgfusepath{stroke}
% setfont left to latex
\definecolor{dialinecolor}{rgb}{0.000000, 0.000000, 0.000000}
\pgfsetstrokecolor{dialinecolor}
\node at (23.081828\du,17.300610\du){r4=\RL{مقربة}};
\definecolor{dialinecolor}{rgb}{1.000000, 1.000000, 1.000000}
\pgfsetfillcolor{dialinecolor}
\fill (21.875350\du,13.465230\du)--(21.875350\du,14.380230\du)--(22.809250\du,14.380230\du)--(22.809250\du,13.465230\du)--cycle;
\pgfsetlinewidth{0.020000\du}
\pgfsetdash{}{0pt}
\pgfsetdash{}{0pt}
\pgfsetmiterjoin
\definecolor{dialinecolor}{rgb}{0.000000, 0.000000, 0.000000}
\pgfsetstrokecolor{dialinecolor}
\draw (21.875350\du,13.465230\du)--(21.875350\du,14.380230\du)--(22.809250\du,14.380230\du)--(22.809250\du,13.465230\du)--cycle;
% setfont left to latex
\definecolor{dialinecolor}{rgb}{0.000000, 0.000000, 0.000000}
\pgfsetstrokecolor{dialinecolor}
\node at (22.342300\du,14.043286\du){\^{}2};
\definecolor{dialinecolor}{rgb}{1.000000, 1.000000, 1.000000}
\pgfsetfillcolor{dialinecolor}
\pgfpathellipse{\pgfpoint{21.712850\du}{15.353540\du}}{\pgfpoint{0.537500\du}{0\du}}{\pgfpoint{0\du}{0.437500\du}}
\pgfusepath{fill}
\pgfsetlinewidth{0.020000\du}
\pgfsetdash{}{0pt}
\pgfsetdash{}{0pt}
\pgfsetmiterjoin
\definecolor{dialinecolor}{rgb}{0.000000, 0.000000, 0.000000}
\pgfsetstrokecolor{dialinecolor}
\pgfpathellipse{\pgfpoint{21.712850\du}{15.353540\du}}{\pgfpoint{0.537500\du}{0\du}}{\pgfpoint{0\du}{0.437500\du}}
\pgfusepath{stroke}
% setfont left to latex
\definecolor{dialinecolor}{rgb}{0.000000, 0.000000, 0.000000}
\pgfsetstrokecolor{dialinecolor}
\node at (21.712850\du,15.474096\du){\RL{هذا}};
\definecolor{dialinecolor}{rgb}{1.000000, 1.000000, 1.000000}
\pgfsetfillcolor{dialinecolor}
\pgfpathellipse{\pgfpoint{22.926600\du}{15.353543\du}}{\pgfpoint{0.573750\du}{0\du}}{\pgfpoint{0\du}{0.469243\du}}
\pgfusepath{fill}
\pgfsetlinewidth{0.020000\du}
\pgfsetdash{}{0pt}
\pgfsetdash{}{0pt}
\pgfsetmiterjoin
\definecolor{dialinecolor}{rgb}{0.000000, 0.000000, 0.000000}
\pgfsetstrokecolor{dialinecolor}
\pgfpathellipse{\pgfpoint{22.926600\du}{15.353543\du}}{\pgfpoint{0.573750\du}{0\du}}{\pgfpoint{0\du}{0.469243\du}}
\pgfusepath{stroke}
% setfont left to latex
\definecolor{dialinecolor}{rgb}{0.000000, 0.000000, 0.000000}
\pgfsetstrokecolor{dialinecolor}
\node at (22.926600\du,15.474098\du){\RL{من}};
\definecolor{dialinecolor}{rgb}{1.000000, 1.000000, 1.000000}
\pgfsetfillcolor{dialinecolor}
\pgfpathellipse{\pgfpoint{27.589711\du}{17.121042\du}}{\pgfpoint{1.061861\du}{0\du}}{\pgfpoint{0\du}{0.578542\du}}
\pgfusepath{fill}
\pgfsetlinewidth{0.020000\du}
\pgfsetdash{}{0pt}
\pgfsetdash{}{0pt}
\pgfsetmiterjoin
\definecolor{dialinecolor}{rgb}{0.000000, 0.000000, 0.000000}
\pgfsetstrokecolor{dialinecolor}
\pgfpathellipse{\pgfpoint{27.589711\du}{17.121042\du}}{\pgfpoint{1.061861\du}{0\du}}{\pgfpoint{0\du}{0.578542\du}}
\pgfusepath{stroke}
% setfont left to latex
\definecolor{dialinecolor}{rgb}{0.000000, 0.000000, 0.000000}
\pgfsetstrokecolor{dialinecolor}
\node at (27.589711\du,17.241597\du){p5=\RL{دبي}};
\definecolor{dialinecolor}{rgb}{1.000000, 1.000000, 1.000000}
\pgfsetfillcolor{dialinecolor}
\pgfpathellipse{\pgfpoint{25.317211\du}{17.121042\du}}{\pgfpoint{1.061861\du}{0\du}}{\pgfpoint{0\du}{0.578542\du}}
\pgfusepath{fill}
\pgfsetlinewidth{0.020000\du}
\pgfsetdash{}{0pt}
\pgfsetdash{}{0pt}
\pgfsetmiterjoin
\definecolor{dialinecolor}{rgb}{0.000000, 0.000000, 0.000000}
\pgfsetstrokecolor{dialinecolor}
\pgfpathellipse{\pgfpoint{25.317211\du}{17.121042\du}}{\pgfpoint{1.061861\du}{0\du}}{\pgfpoint{0\du}{0.578542\du}}
\pgfusepath{stroke}
% setfont left to latex
\definecolor{dialinecolor}{rgb}{0.000000, 0.000000, 0.000000}
\pgfsetstrokecolor{dialinecolor}
\node at (25.317211\du,17.241597\du){p6=\RL{مول}};
\definecolor{dialinecolor}{rgb}{1.000000, 1.000000, 1.000000}
\pgfsetfillcolor{dialinecolor}
\pgfpathellipse{\pgfpoint{24.916600\du}{15.353542\du}}{\pgfpoint{0.658750\du}{0\du}}{\pgfpoint{0\du}{0.578542\du}}
\pgfusepath{fill}
\pgfsetlinewidth{0.020000\du}
\pgfsetdash{}{0pt}
\pgfsetdash{}{0pt}
\pgfsetmiterjoin
\definecolor{dialinecolor}{rgb}{0.000000, 0.000000, 0.000000}
\pgfsetstrokecolor{dialinecolor}
\pgfpathellipse{\pgfpoint{24.916600\du}{15.353542\du}}{\pgfpoint{0.658750\du}{0\du}}{\pgfpoint{0\du}{0.578542\du}}
\pgfusepath{stroke}
% setfont left to latex
\definecolor{dialinecolor}{rgb}{0.000000, 0.000000, 0.000000}
\pgfsetstrokecolor{dialinecolor}
\node at (24.916600\du,15.474097\du){\RL{على}};
\definecolor{dialinecolor}{rgb}{1.000000, 1.000000, 1.000000}
\pgfsetfillcolor{dialinecolor}
\fill (19.827850\du,13.430230\du)--(19.827850\du,14.380230\du)--(20.752850\du,14.380230\du)--(20.752850\du,13.430230\du)--cycle;
\pgfsetlinewidth{0.020000\du}
\pgfsetdash{}{0pt}
\pgfsetdash{}{0pt}
\pgfsetmiterjoin
\definecolor{dialinecolor}{rgb}{0.000000, 0.000000, 0.000000}
\pgfsetstrokecolor{dialinecolor}
\draw (19.827850\du,13.430230\du)--(19.827850\du,14.380230\du)--(20.752850\du,14.380230\du)--(20.752850\du,13.430230\du)--cycle;
% setfont left to latex
\definecolor{dialinecolor}{rgb}{0.000000, 0.000000, 0.000000}
\pgfsetstrokecolor{dialinecolor}
\node at (20.290350\du,14.025786\du){$+$};
\definecolor{dialinecolor}{rgb}{1.000000, 1.000000, 1.000000}
\pgfsetfillcolor{dialinecolor}
\fill (19.852850\du,14.891040\du)--(19.852850\du,15.816040\du)--(20.802850\du,15.816040\du)--(20.802850\du,14.891040\du)--cycle;
\pgfsetlinewidth{0.020000\du}
\pgfsetdash{}{0pt}
\pgfsetdash{}{0pt}
\pgfsetmiterjoin
\definecolor{dialinecolor}{rgb}{0.000000, 0.000000, 0.000000}
\pgfsetstrokecolor{dialinecolor}
\draw (19.852850\du,14.891040\du)--(19.852850\du,15.816040\du)--(20.802850\du,15.816040\du)--(20.802850\du,14.891040\du)--cycle;
% setfont left to latex
\definecolor{dialinecolor}{rgb}{0.000000, 0.000000, 0.000000}
\pgfsetstrokecolor{dialinecolor}
\node at (20.327850\du,15.474096\du){$|$};
\definecolor{dialinecolor}{rgb}{1.000000, 1.000000, 1.000000}
\pgfsetfillcolor{dialinecolor}
\fill (24.475350\du,13.465230\du)--(24.475350\du,14.380230\du)--(25.486750\du,14.380230\du)--(25.486750\du,13.465230\du)--cycle;
\pgfsetlinewidth{0.020000\du}
\pgfsetdash{}{0pt}
\pgfsetdash{}{0pt}
\pgfsetmiterjoin
\definecolor{dialinecolor}{rgb}{0.000000, 0.000000, 0.000000}
\pgfsetstrokecolor{dialinecolor}
\draw (24.475350\du,13.465230\du)--(24.475350\du,14.380230\du)--(25.486750\du,14.380230\du)--(25.486750\du,13.465230\du)--cycle;
% setfont left to latex
\definecolor{dialinecolor}{rgb}{0.000000, 0.000000, 0.000000}
\pgfsetstrokecolor{dialinecolor}
\node at (24.981050\du,14.043286\du){?};
\pgfsetlinewidth{0.020000\du}
\pgfsetdash{}{0pt}
\pgfsetdash{}{0pt}
\pgfsetbuttcap
{
\definecolor{dialinecolor}{rgb}{0.000000, 0.000000, 0.000000}
\pgfsetfillcolor{dialinecolor}
% was here!!!
\pgfsetarrowsend{latex}
\definecolor{dialinecolor}{rgb}{0.000000, 0.000000, 0.000000}
\pgfsetstrokecolor{dialinecolor}
\draw (24.459070\du,12.260050\du)--(23.260613\du,16.541370\du);
}
\pgfsetlinewidth{0.020000\du}
\pgfsetdash{}{0pt}
\pgfsetdash{}{0pt}
\pgfsetbuttcap
{
\definecolor{dialinecolor}{rgb}{0.000000, 0.000000, 0.000000}
\pgfsetfillcolor{dialinecolor}
% was here!!!
\pgfsetarrowsend{latex}
\definecolor{dialinecolor}{rgb}{0.000000, 0.000000, 0.000000}
\pgfsetstrokecolor{dialinecolor}
\draw (24.090007\du,12.043880\du)--(20.763012\du,13.673686\du);
}
\pgfsetlinewidth{0.020000\du}
\pgfsetdash{}{0pt}
\pgfsetdash{}{0pt}
\pgfsetbuttcap
{
\definecolor{dialinecolor}{rgb}{0.000000, 0.000000, 0.000000}
\pgfsetfillcolor{dialinecolor}
% was here!!!
\pgfsetarrowsend{latex}
\definecolor{dialinecolor}{rgb}{0.000000, 0.000000, 0.000000}
\pgfsetstrokecolor{dialinecolor}
\draw (20.302893\du,14.389650\du)--(20.315609\du,14.880788\du);
}
\pgfsetlinewidth{0.020000\du}
\pgfsetdash{}{0pt}
\pgfsetdash{}{0pt}
\pgfsetbuttcap
{
\definecolor{dialinecolor}{rgb}{0.000000, 0.000000, 0.000000}
\pgfsetfillcolor{dialinecolor}
% was here!!!
\pgfsetarrowsend{latex}
\definecolor{dialinecolor}{rgb}{0.000000, 0.000000, 0.000000}
\pgfsetstrokecolor{dialinecolor}
\draw (20.333667\du,15.824804\du)--(20.342397\du,16.532143\du);
}
\pgfsetlinewidth{0.020000\du}
\pgfsetdash{}{0pt}
\pgfsetdash{}{0pt}
\pgfsetbuttcap
{
\definecolor{dialinecolor}{rgb}{0.000000, 0.000000, 0.000000}
\pgfsetfillcolor{dialinecolor}
% was here!!!
\pgfsetarrowsend{latex}
\definecolor{dialinecolor}{rgb}{0.000000, 0.000000, 0.000000}
\pgfsetstrokecolor{dialinecolor}
\draw (24.672711\du,12.258128\du)--(24.894654\du,13.456310\du);
}
\pgfsetlinewidth{0.020000\du}
\pgfsetdash{}{0pt}
\pgfsetdash{}{0pt}
\pgfsetbuttcap
{
\definecolor{dialinecolor}{rgb}{0.000000, 0.000000, 0.000000}
\pgfsetfillcolor{dialinecolor}
% was here!!!
\pgfsetarrowsend{latex}
\definecolor{dialinecolor}{rgb}{0.000000, 0.000000, 0.000000}
\pgfsetstrokecolor{dialinecolor}
\draw (24.959997\du,14.390119\du)--(24.943066\du,14.765987\du);
}
\pgfsetlinewidth{0.020000\du}
\pgfsetdash{}{0pt}
\pgfsetdash{}{0pt}
\pgfsetbuttcap
{
\definecolor{dialinecolor}{rgb}{0.000000, 0.000000, 0.000000}
\pgfsetfillcolor{dialinecolor}
% was here!!!
\pgfsetarrowsend{latex}
\definecolor{dialinecolor}{rgb}{0.000000, 0.000000, 0.000000}
\pgfsetstrokecolor{dialinecolor}
\draw (24.103215\du,12.258128\du)--(22.818712\du,13.472375\du);
}
\pgfsetlinewidth{0.020000\du}
\pgfsetdash{}{0pt}
\pgfsetdash{}{0pt}
\pgfsetbuttcap
{
\definecolor{dialinecolor}{rgb}{0.000000, 0.000000, 0.000000}
\pgfsetfillcolor{dialinecolor}
% was here!!!
\pgfsetarrowsend{latex}
\definecolor{dialinecolor}{rgb}{0.000000, 0.000000, 0.000000}
\pgfsetstrokecolor{dialinecolor}
\draw (22.136684\du,14.390119\du)--(21.898028\du,14.932611\du);
}
\pgfsetlinewidth{0.020000\du}
\pgfsetdash{}{0pt}
\pgfsetdash{}{0pt}
\pgfsetbuttcap
{
\definecolor{dialinecolor}{rgb}{0.000000, 0.000000, 0.000000}
\pgfsetfillcolor{dialinecolor}
% was here!!!
\pgfsetarrowsend{latex}
\definecolor{dialinecolor}{rgb}{0.000000, 0.000000, 0.000000}
\pgfsetstrokecolor{dialinecolor}
\draw (22.533168\du,14.390120\du)--(22.741011\du,14.899078\du);
}
\definecolor{dialinecolor}{rgb}{1.000000, 1.000000, 1.000000}
\pgfsetfillcolor{dialinecolor}
\fill (27.375350\du,14.891040\du)--(27.375350\du,15.816040\du)--(28.325350\du,15.816040\du)--(28.325350\du,14.891040\du)--cycle;
\pgfsetlinewidth{0.020000\du}
\pgfsetdash{}{0pt}
\pgfsetdash{}{0pt}
\pgfsetmiterjoin
\definecolor{dialinecolor}{rgb}{0.000000, 0.000000, 0.000000}
\pgfsetstrokecolor{dialinecolor}
\draw (27.375350\du,14.891040\du)--(27.375350\du,15.816040\du)--(28.325350\du,15.816040\du)--(28.325350\du,14.891040\du)--cycle;
% setfont left to latex
\definecolor{dialinecolor}{rgb}{0.000000, 0.000000, 0.000000}
\pgfsetstrokecolor{dialinecolor}
\node at (27.850350\du,15.474096\du){$|$};
\definecolor{dialinecolor}{rgb}{1.000000, 1.000000, 1.000000}
\pgfsetfillcolor{dialinecolor}
\fill (26.825350\du,13.430230\du)--(26.825350\du,14.380230\du)--(27.750350\du,14.380230\du)--(27.750350\du,13.430230\du)--cycle;
\pgfsetlinewidth{0.020000\du}
\pgfsetdash{}{0pt}
\pgfsetdash{}{0pt}
\pgfsetmiterjoin
\definecolor{dialinecolor}{rgb}{0.000000, 0.000000, 0.000000}
\pgfsetstrokecolor{dialinecolor}
\draw (26.825350\du,13.430230\du)--(26.825350\du,14.380230\du)--(27.750350\du,14.380230\du)--(27.750350\du,13.430230\du)--cycle;
% setfont left to latex
\definecolor{dialinecolor}{rgb}{0.000000, 0.000000, 0.000000}
\pgfsetstrokecolor{dialinecolor}
\node at (27.287850\du,14.025786\du){$+$};
\definecolor{dialinecolor}{rgb}{1.000000, 1.000000, 1.000000}
\pgfsetfillcolor{dialinecolor}
\fill (25.727850\du,14.891040\du)--(25.727850\du,15.816040\du)--(26.677850\du,15.816040\du)--(26.677850\du,14.891040\du)--cycle;
\pgfsetlinewidth{0.020000\du}
\pgfsetdash{}{0pt}
\pgfsetdash{}{0pt}
\pgfsetmiterjoin
\definecolor{dialinecolor}{rgb}{0.000000, 0.000000, 0.000000}
\pgfsetstrokecolor{dialinecolor}
\draw (25.727850\du,14.891040\du)--(25.727850\du,15.816040\du)--(26.677850\du,15.816040\du)--(26.677850\du,14.891040\du)--cycle;
% setfont left to latex
\definecolor{dialinecolor}{rgb}{0.000000, 0.000000, 0.000000}
\pgfsetstrokecolor{dialinecolor}
\node at (26.202850\du,15.474096\du){$|$};
\pgfsetlinewidth{0.020000\du}
\pgfsetdash{}{0pt}
\pgfsetdash{}{0pt}
\pgfsetbuttcap
{
\definecolor{dialinecolor}{rgb}{0.000000, 0.000000, 0.000000}
\pgfsetfillcolor{dialinecolor}
% was here!!!
\pgfsetarrowsend{latex}
\definecolor{dialinecolor}{rgb}{0.000000, 0.000000, 0.000000}
\pgfsetstrokecolor{dialinecolor}
\draw (27.475991\du,14.389650\du)--(27.666741\du,14.880788\du);
}
\pgfsetlinewidth{0.020000\du}
\pgfsetdash{}{0pt}
\pgfsetdash{}{0pt}
\pgfsetbuttcap
{
\definecolor{dialinecolor}{rgb}{0.000000, 0.000000, 0.000000}
\pgfsetfillcolor{dialinecolor}
% was here!!!
\pgfsetarrowsend{latex}
\definecolor{dialinecolor}{rgb}{0.000000, 0.000000, 0.000000}
\pgfsetstrokecolor{dialinecolor}
\draw (26.924947\du,14.389650\du)--(26.557011\du,14.880788\du);
}
\pgfsetlinewidth{0.020000\du}
\pgfsetdash{}{0pt}
\pgfsetdash{}{0pt}
\pgfsetbuttcap
{
\definecolor{dialinecolor}{rgb}{0.000000, 0.000000, 0.000000}
\pgfsetfillcolor{dialinecolor}
% was here!!!
\pgfsetarrowsend{latex}
\definecolor{dialinecolor}{rgb}{0.000000, 0.000000, 0.000000}
\pgfsetstrokecolor{dialinecolor}
\draw (25.084871\du,12.187535\du)--(26.815218\du,13.536712\du);
}
\pgfsetlinewidth{0.020000\du}
\pgfsetdash{}{0pt}
\pgfsetdash{}{0pt}
\pgfsetbuttcap
{
\definecolor{dialinecolor}{rgb}{0.000000, 0.000000, 0.000000}
\pgfsetfillcolor{dialinecolor}
% was here!!!
\pgfsetarrowsend{latex}
\definecolor{dialinecolor}{rgb}{0.000000, 0.000000, 0.000000}
\pgfsetstrokecolor{dialinecolor}
\draw (27.780863\du,15.824759\du)--(27.675742\du,16.537628\du);
}
\pgfsetlinewidth{0.020000\du}
\pgfsetdash{}{0pt}
\pgfsetdash{}{0pt}
\pgfsetbuttcap
{
\definecolor{dialinecolor}{rgb}{0.000000, 0.000000, 0.000000}
\pgfsetfillcolor{dialinecolor}
% was here!!!
\pgfsetarrowsend{latex}
\definecolor{dialinecolor}{rgb}{0.000000, 0.000000, 0.000000}
\pgfsetstrokecolor{dialinecolor}
\draw (25.966737\du,15.824759\du)--(25.601325\du,16.554026\du);
}
% setfont left to latex
\definecolor{dialinecolor}{rgb}{0.000000, 0.000000, 0.000000}
\pgfsetstrokecolor{dialinecolor}
\node[anchor=west] at (21.807850\du,11.829310\du){match 2};
% setfont left to latex
\definecolor{dialinecolor}{rgb}{0.000000, 0.000000, 0.000000}
\pgfsetstrokecolor{dialinecolor}
\node[anchor=west] at (19.550350\du,17.274600\du){};
\end{tikzpicture}
 \\
    \end{tabular}
} \\
\end{tabular}
}
  \caption{Boolean formulae, and matches of regular expression $(P|N)+~O?~R~O$\^{}$2~(P|N|U)+$}
  \label{fig:motiv}
\end{figure}
\transtrue
\setcode{standard}

In the rest of this paper,
Section~\ref{sec:motivation} presents a motivational example. 
Section~\ref{sec:morph} introduces Arabic morphological analysis 
and its important role in Arabic NLP. 
%We present preliminaries necessary to understand the work 
%in Section~\ref{sec:preliminaries}. 
Section~\ref{sec:framework} explains the \framework methodology. 
Section~\ref{sec:gui} presents the visual and user friendly interface of \framework. 
Section~\ref{sec:related} presents and discusses related work.
Section~\ref{sec:results} presents the evaluation results. 
Finally, we conclude and discuss future work in Section~\ref{sec:conclusion}.