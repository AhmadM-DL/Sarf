\begin{figure}[tb!]
\centering
\resizebox{\columnwidth}{!}{
	%\relsize{+1} \begin{figure}[tb!]
\centering
\resizebox{\columnwidth}{!}{
	\relsize{+1} \begin{figure}[tb!]
\centering
\resizebox{\columnwidth}{!}{
	\relsize{+1} \input{figures/overview.tex}
}
\caption{\framework flow diagram.}
\label{f:overview}
\end{figure}

Figure~\ref{f:overview} shows the flow diagram of \framework. 
%A box node represents a \framework process and 
%an ellipse node represents input and output data objects.
The Arabic text and the reference tag chunks are the primary inputs to \framework.
Solutions, morphology-based Boolean formulae, tags, 
morphology-based regular expressions, 
tag chunks, relation and action definitions, and data structures expressing entities 
and relations are input and output data to \framework processes. 
The morphological analyzer (Sarf), $Syn^k$ detector, 
GUI for Boolean formulae definition, 
visualization annotator, GUI for regular expression and action definition, 
Boolean formula simulator, regular expression simulator, 
relation extraction and action execution, and difference and statistical analyzer 
are \framework processes.

\def\pp{\ensuremath{{\cal P}}} % prefix
\def\ss{\ensuremath{{\cal S}}} % stem
\def\xx{\ensuremath{{\cal X}}} % suffix
\def\PP{\ensuremath{\mathit{POS}}} % pos
\def\GG{\ensuremath{\mathit{GLOSS}}} % gloss
\def\AC{\ensuremath{\mathit{CAT}}} % category

%~\footnote{In this document, we use the default ArabTeX transliteration style ZDMG.}
{\bf Morphological analyzer (Sarf).~}
\label{s:s:morph}
Morphological analysis is key to Arabic NLP~\cite{arabicmorph}.
Short vowels, also known as diacritics, are typically omitted in Arabic text
and inferred by readers~\cite{habash2006arabic}. 
For example, the word \RL{'sd}%
can be interpreted as \RL{'sad} {\tt ``lion''} with a {\em fatha} diacritic on the 
letter \utfrl{سـ} or 
\vocalize \RL{'sodd} (I block) with a 
{\em damma} diacritic on the letter \utfrl{سـ} and {\em shadda} on the letter \RL{d}.

In addition, the position of an Arabic letter in a word 
(beginning, middle, end, and standalone) changes
its visual form.
Some letters have non-connecting end forms which allows visual
word separation without the need of a white space separator. 
This requires morphological analysis for even the simple tokenization 
task of Arabic text.
%For example, the word \utfrl{ياسمين} can be interpreted as
%the ``Jasmine'' flower, 
%as well as \notrutfrl{يا} (the calling word) followed by
%the word \notrutfrl{سمين} (obese). 
For example, consider the sentence 
%\utfrl{ذهب الولدإلى المدرسة} 
\notrrl{AlmdrsT}\notrrl{_dhb alwald-il_A}
\arabfalse \RL{_dhb alwald-il_A almdrsT} \arabtrue
(the kid went to school). 
The letters \notrutfrl{د} and \notrutfrl{ى} have 
non-connecting end of word forms and the words 
\notrutfrl{الولد}, \notrutfrl{الى}, and  \notrutfrl{المدرسة} 
are visually separable, 
yet there is no space character in between.
Newspaper articles with text justification requirements, 
SMS messages, and automatically digitized documents
are examples where such problems often occur. 

\framework is integrated with {\em Sarf}, 
an in-house open source Arabic morphological analyzer based on 
finite state transducers~\cite{ZaMaColing2012DemosSarf}. 
Given an Arabic word, Sarf returns 
a set of morphological solutions. 
A word might have more than one solution 
due to multiple possible segmentations and multiple tags associated 
with each word. 
A morphological solution is the internal structure of the word 
composed of several morphemes including 
{\em affixes} ({\em prefixes} and {\em suffixes}), and a
{\em stem}, where each morpheme is associated with tags such as 
POS, gloss, and category tags~\cite{arabicmorph,habash2010introduction}. 
%POS, gloss,  {\em lemma}, and category tags~\cite{arabicmorph,habash2010introduction}. 
%A morpheme is the smallest linguistic unit 
%that has a meaning and fulfills a grammatical function. 

Prefixes attach before the stem and a word can have multiple prefixes. 
Suffixes attach after the stem and a word can have multiple suffixes. 
Infixes are inserted inside the stem to form a new stem. 
In this work we consider a set of stems that includes infix morphological changes. 
The part-of-speech tag, referred to as POS, 
assigns a morpho-syntactic tag for a morpheme. 
The gloss is a brief semantic notation of morpheme in English. 
A morpheme might have multiple glosses as it could stand for multiple meanings. 
The category is a user defined tag that the user can assign to several 
morphemes.
For example, the user can define a temporal category to include the 
prefix \utfrl{س} (will) and the time unit \utfrl{ساعة} (hour). 
We denote by 
\ss,
\pp,
\xx,
\PP,
\GG, and 
\AC, the set of 
all stems,
prefixes,
suffixes,
POS,
gloss, 
and user defined category tags, respectively. 

%The lemma is a convention choice that assigns a core meaning of word forms that share similar stems. 
%For example, the words \RL{byt}(house), 
%\RL{llbyt}(for the house), 
%and \RL{bywt}(houses) are mapped to the singular noun form \RL{byt} as a lemma.

\vocalize

%\setarab
\begin{table}[tb!]
  \centering
  \caption{Sample solution vector for \utfrl{فَسَيَأْكُلها}.}
  \resizebox{\columnwidth}{!}{
    \begin{tabular}{|r|c|c|c|c|c|}
          & \multicolumn{3}{c|}{\textbf{Prefixes}} & \textbf{Stem} & \textbf{Suffix} \\
    \textbf{Data} & \utfrl{فَ} & \utfrl{سَ} & \utfrl{يَ} & \utfrl{أْكُل} & \utfrl{ها} \\
    \textbf{POS} & CONJ+ & FUT+ & IV3MS+ & VERB\_IMPERFECT & IVSUFF\_DO:3FS \\
    \textbf{Gloss} & and/so & will & he/it & eat/consume & it/them/her \\
    \textbf{index} & \multicolumn{3}{c|}{10} & 13 & 16 \\
    \textbf{length} & \multicolumn{3}{c|}{3} & 3 & 2 \\
    \end{tabular}%
    }
    \vspace{-3em}
  \label{tab:samplesolution}%
\end{table}%

Table~\ref{tab:samplesolution} shows the morphological analysis
of the \utfrl{فَسَيَأْكُلها}. 
The word is composed of the prefix morphemes 
\utfrl{فَ}, \utfrl{سَ}, and \utfrl{يَ}, followed by the 
stem \utfrl{أْكُل}, and then followed by the suffix morpheme
\utfrl{ها}. 
Each morpheme is associated with a number of morphological features.
The {\tt CONJ},
{\tt FUT}, 
{\tt IV3MS} 
{\tt VERB\_IMPERFECT}, and 
{\tt IVSUFF\_DO:3FS} POS tags indicate
conjunction, 
future, 
third person masculine singular subject pronoun,
an imperfect verb, and 
and a third person feminine singular object pronoun, respectively.
The POS and gloss notations follow the Buckwalter notation~\cite{Buckwalter:02}.

%A morpheme can be a stem or an affix. 
%Each morpheme is associated with other morphological features 
%including {\em POS}, {\em gloss}, {\em lemma}, and {\em category} tags. 

%A stem can be a root word, a {\em templatic}, or {\em non-templatic} stem. 
%Templatic stems are formed from roots using template morphological rules. 
%For example, the stem
%\RL{kAtb} (writer) is a subject noun formed from the root verb \RL{ktb} (write). 
%Non-templatic stems tend to be foreign names such as \RL{واشنطن} (Washington). 
%A root is an atomic word with three, four, and rarely five letters.

%An affix can be a {\em prefix}, a {\em suffix}, or an {\em infix}. 
%The lemma is a convention choice that assigns a core meaning of word forms that share similar stems. 
%For example, the words \RL{byt}(house), 
%\RL{llbyt}(for the house), 
%and \RL{bywt}(houses) are mapped to the singular noun form \RL{byt} as a lemma.

%We introduce this feature in detail in Section~\ref{sec:framework}.

The user interacts with \framework~through a user-friendly interfaces to 
specify morphology-based Boolean formulae and regular expressions 
and associate them with tag types that are in turn associated with
visualization legends such as 
foreground and background colors and fonts. 

{\bf Boolean formula simulator.~}
\framework~uses Sarf to compute morphological solutions for the Arabic text, 
and passes the solutions along with the user-defined tag types to the 
Boolean formula simulator.
The simulator interacts with the $Syn^k$ detector, 
computes tag type matches, and produces a tag set for each word. 
%A word might have multiple tags as its morphological solutions could 
%match multiple Boolean formulae. 
%Each tag contains information about the matching word and the relevant tag type name. 
%Word information include word index relative to the text, character position, and text.
%We formally define the MBF and explain its simulation in Section~\ref{sec:framework}.

{\bf Regular expression simulator.~}
\framework~passes the sequence of tag sets and the user-defined 
tag types with morphology based regular expressions to the regular expression 
simulator. 
The simulator computes tag chunks; i.e. sequences of tags sets, that match the 
regular expressions. 
%Each tag chunk contains information about the matching sequence of words and the tag type with the matching regular expression (MRE). 
%We formally define the MRE and explain its simulation in Section~\ref{sec:framework}.

{\bf Relation extraction and action execution.~}
\framework~enables the user to associate code actions to parts of the regular
expressions. 
The user can use an API to access information such as 
text, position, length, and morphological features of the tag chunks that 
match the sub-expression. 
\framework~also allows the user to declare relations between
matches using the relation editor. 
Each relation is specified by a source, destination, and a label tuple. 
The user selects a feature of a match of sub-expressions to specify the 
tuple entities. 

\framework~executes the user-defined actions corresponding to each 
sub-expression match in a tag chunk. 
It also evaluates the semantic relation declarations against the 
tag chunks to compute the relational matches. 
Finally, \framework~uses the \cci{isA} cross-reference relation 
to create relations across tag type matches. 
%The output of this process expresses entities and relations among them. 
%We formally define the semantic relation and explain its construction in Section~\ref{sec:framework}.

{\bf Visualization annotator.~}
\framework presents the resulting tags to the user incrementally in the form of 
text annotated with style and color legends, match trees, and graphs expressing
the relational entities. 
The annotation can be edited by the user in a user-friendly interface. 
Match trees present the match text associated with the relevant tags 
and regular expression structure. 
The graphs are the result of the user defined relations.
%We present \framework's interface in Section~\ref{sec:gui}.

{\bf Agreement and statistical analysis.~}
\framework provides statistical analysis tools that help compare sets 
of tags to reference tags and compute standard agreement and accuracy measures. 
\framework~ provides criteria for comparison including exact match and intersection. 
This comparison tool can be used to edit the automatically generated annotations. 
%\framework~provides the user with an interactive interface to 
%build the reference tags. 
%We explain the analysis process and its interface in Section~\ref{sec:gui}.

}
\caption{\framework flow diagram.}
\label{f:overview}
\end{figure}

Figure~\ref{f:overview} shows the flow diagram of \framework. 
%A box node represents a \framework process and 
%an ellipse node represents input and output data objects.
The Arabic text and the reference tag chunks are the primary inputs to \framework.
Solutions, morphology-based Boolean formulae, tags, 
morphology-based regular expressions, 
tag chunks, relation and action definitions, and data structures expressing entities 
and relations are input and output data to \framework processes. 
The morphological analyzer (Sarf), $Syn^k$ detector, 
GUI for Boolean formulae definition, 
visualization annotator, GUI for regular expression and action definition, 
Boolean formula simulator, regular expression simulator, 
relation extraction and action execution, and difference and statistical analyzer 
are \framework processes.

\def\pp{\ensuremath{{\cal P}}} % prefix
\def\ss{\ensuremath{{\cal S}}} % stem
\def\xx{\ensuremath{{\cal X}}} % suffix
\def\PP{\ensuremath{\mathit{POS}}} % pos
\def\GG{\ensuremath{\mathit{GLOSS}}} % gloss
\def\AC{\ensuremath{\mathit{CAT}}} % category

%~\footnote{In this document, we use the default ArabTeX transliteration style ZDMG.}
{\bf Morphological analyzer (Sarf).~}
\label{s:s:morph}
Morphological analysis is key to Arabic NLP~\cite{arabicmorph}.
Short vowels, also known as diacritics, are typically omitted in Arabic text
and inferred by readers~\cite{habash2006arabic}. 
For example, the word \RL{'sd}%
can be interpreted as \RL{'sad} {\tt ``lion''} with a {\em fatha} diacritic on the 
letter \utfrl{سـ} or 
\vocalize \RL{'sodd} (I block) with a 
{\em damma} diacritic on the letter \utfrl{سـ} and {\em shadda} on the letter \RL{d}.

In addition, the position of an Arabic letter in a word 
(beginning, middle, end, and standalone) changes
its visual form.
Some letters have non-connecting end forms which allows visual
word separation without the need of a white space separator. 
This requires morphological analysis for even the simple tokenization 
task of Arabic text.
%For example, the word \utfrl{ياسمين} can be interpreted as
%the ``Jasmine'' flower, 
%as well as \notrutfrl{يا} (the calling word) followed by
%the word \notrutfrl{سمين} (obese). 
For example, consider the sentence 
%\utfrl{ذهب الولدإلى المدرسة} 
\notrrl{AlmdrsT}\notrrl{_dhb alwald-il_A}
\arabfalse \RL{_dhb alwald-il_A almdrsT} \arabtrue
(the kid went to school). 
The letters \notrutfrl{د} and \notrutfrl{ى} have 
non-connecting end of word forms and the words 
\notrutfrl{الولد}, \notrutfrl{الى}, and  \notrutfrl{المدرسة} 
are visually separable, 
yet there is no space character in between.
Newspaper articles with text justification requirements, 
SMS messages, and automatically digitized documents
are examples where such problems often occur. 

\framework is integrated with {\em Sarf}, 
an in-house open source Arabic morphological analyzer based on 
finite state transducers~\cite{ZaMaColing2012DemosSarf}. 
Given an Arabic word, Sarf returns 
a set of morphological solutions. 
A word might have more than one solution 
due to multiple possible segmentations and multiple tags associated 
with each word. 
A morphological solution is the internal structure of the word 
composed of several morphemes including 
{\em affixes} ({\em prefixes} and {\em suffixes}), and a
{\em stem}, where each morpheme is associated with tags such as 
POS, gloss, and category tags~\cite{arabicmorph,habash2010introduction}. 
%POS, gloss,  {\em lemma}, and category tags~\cite{arabicmorph,habash2010introduction}. 
%A morpheme is the smallest linguistic unit 
%that has a meaning and fulfills a grammatical function. 

Prefixes attach before the stem and a word can have multiple prefixes. 
Suffixes attach after the stem and a word can have multiple suffixes. 
Infixes are inserted inside the stem to form a new stem. 
In this work we consider a set of stems that includes infix morphological changes. 
The part-of-speech tag, referred to as POS, 
assigns a morpho-syntactic tag for a morpheme. 
The gloss is a brief semantic notation of morpheme in English. 
A morpheme might have multiple glosses as it could stand for multiple meanings. 
The category is a user defined tag that the user can assign to several 
morphemes.
For example, the user can define a temporal category to include the 
prefix \utfrl{س} (will) and the time unit \utfrl{ساعة} (hour). 
We denote by 
\ss,
\pp,
\xx,
\PP,
\GG, and 
\AC, the set of 
all stems,
prefixes,
suffixes,
POS,
gloss, 
and user defined category tags, respectively. 

%The lemma is a convention choice that assigns a core meaning of word forms that share similar stems. 
%For example, the words \RL{byt}(house), 
%\RL{llbyt}(for the house), 
%and \RL{bywt}(houses) are mapped to the singular noun form \RL{byt} as a lemma.

\vocalize

%\setarab
\begin{table}[tb!]
  \centering
  \caption{Sample solution vector for \utfrl{فَسَيَأْكُلها}.}
  \resizebox{\columnwidth}{!}{
    \begin{tabular}{|r|c|c|c|c|c|}
          & \multicolumn{3}{c|}{\textbf{Prefixes}} & \textbf{Stem} & \textbf{Suffix} \\
    \textbf{Data} & \utfrl{فَ} & \utfrl{سَ} & \utfrl{يَ} & \utfrl{أْكُل} & \utfrl{ها} \\
    \textbf{POS} & CONJ+ & FUT+ & IV3MS+ & VERB\_IMPERFECT & IVSUFF\_DO:3FS \\
    \textbf{Gloss} & and/so & will & he/it & eat/consume & it/them/her \\
    \textbf{index} & \multicolumn{3}{c|}{10} & 13 & 16 \\
    \textbf{length} & \multicolumn{3}{c|}{3} & 3 & 2 \\
    \end{tabular}%
    }
    \vspace{-3em}
  \label{tab:samplesolution}%
\end{table}%

Table~\ref{tab:samplesolution} shows the morphological analysis
of the \utfrl{فَسَيَأْكُلها}. 
The word is composed of the prefix morphemes 
\utfrl{فَ}, \utfrl{سَ}, and \utfrl{يَ}, followed by the 
stem \utfrl{أْكُل}, and then followed by the suffix morpheme
\utfrl{ها}. 
Each morpheme is associated with a number of morphological features.
The {\tt CONJ},
{\tt FUT}, 
{\tt IV3MS} 
{\tt VERB\_IMPERFECT}, and 
{\tt IVSUFF\_DO:3FS} POS tags indicate
conjunction, 
future, 
third person masculine singular subject pronoun,
an imperfect verb, and 
and a third person feminine singular object pronoun, respectively.
The POS and gloss notations follow the Buckwalter notation~\cite{Buckwalter:02}.

%A morpheme can be a stem or an affix. 
%Each morpheme is associated with other morphological features 
%including {\em POS}, {\em gloss}, {\em lemma}, and {\em category} tags. 

%A stem can be a root word, a {\em templatic}, or {\em non-templatic} stem. 
%Templatic stems are formed from roots using template morphological rules. 
%For example, the stem
%\RL{kAtb} (writer) is a subject noun formed from the root verb \RL{ktb} (write). 
%Non-templatic stems tend to be foreign names such as \RL{واشنطن} (Washington). 
%A root is an atomic word with three, four, and rarely five letters.

%An affix can be a {\em prefix}, a {\em suffix}, or an {\em infix}. 
%The lemma is a convention choice that assigns a core meaning of word forms that share similar stems. 
%For example, the words \RL{byt}(house), 
%\RL{llbyt}(for the house), 
%and \RL{bywt}(houses) are mapped to the singular noun form \RL{byt} as a lemma.

%We introduce this feature in detail in Section~\ref{sec:framework}.

The user interacts with \framework~through a user-friendly interfaces to 
specify morphology-based Boolean formulae and regular expressions 
and associate them with tag types that are in turn associated with
visualization legends such as 
foreground and background colors and fonts. 

{\bf Boolean formula simulator.~}
\framework~uses Sarf to compute morphological solutions for the Arabic text, 
and passes the solutions along with the user-defined tag types to the 
Boolean formula simulator.
The simulator interacts with the $Syn^k$ detector, 
computes tag type matches, and produces a tag set for each word. 
%A word might have multiple tags as its morphological solutions could 
%match multiple Boolean formulae. 
%Each tag contains information about the matching word and the relevant tag type name. 
%Word information include word index relative to the text, character position, and text.
%We formally define the MBF and explain its simulation in Section~\ref{sec:framework}.

{\bf Regular expression simulator.~}
\framework~passes the sequence of tag sets and the user-defined 
tag types with morphology based regular expressions to the regular expression 
simulator. 
The simulator computes tag chunks; i.e. sequences of tags sets, that match the 
regular expressions. 
%Each tag chunk contains information about the matching sequence of words and the tag type with the matching regular expression (MRE). 
%We formally define the MRE and explain its simulation in Section~\ref{sec:framework}.

{\bf Relation extraction and action execution.~}
\framework~enables the user to associate code actions to parts of the regular
expressions. 
The user can use an API to access information such as 
text, position, length, and morphological features of the tag chunks that 
match the sub-expression. 
\framework~also allows the user to declare relations between
matches using the relation editor. 
Each relation is specified by a source, destination, and a label tuple. 
The user selects a feature of a match of sub-expressions to specify the 
tuple entities. 

\framework~executes the user-defined actions corresponding to each 
sub-expression match in a tag chunk. 
It also evaluates the semantic relation declarations against the 
tag chunks to compute the relational matches. 
Finally, \framework~uses the \cci{isA} cross-reference relation 
to create relations across tag type matches. 
%The output of this process expresses entities and relations among them. 
%We formally define the semantic relation and explain its construction in Section~\ref{sec:framework}.

{\bf Visualization annotator.~}
\framework presents the resulting tags to the user incrementally in the form of 
text annotated with style and color legends, match trees, and graphs expressing
the relational entities. 
The annotation can be edited by the user in a user-friendly interface. 
Match trees present the match text associated with the relevant tags 
and regular expression structure. 
The graphs are the result of the user defined relations.
%We present \framework's interface in Section~\ref{sec:gui}.

{\bf Agreement and statistical analysis.~}
\framework provides statistical analysis tools that help compare sets 
of tags to reference tags and compute standard agreement and accuracy measures. 
\framework~ provides criteria for comparison including exact match and intersection. 
This comparison tool can be used to edit the automatically generated annotations. 
%\framework~provides the user with an interactive interface to 
%build the reference tags. 
%We explain the analysis process and its interface in Section~\ref{sec:gui}.

	\input{figures/flowdiag.pdf_t}
}
\vspace{-2em}
\caption{\framework four process methodology with rounded corner blocks for GUI.}
%\vspace{-.5em}
\label{f:overview}
\end{figure}


The \framework framework is illustrated in the flow diagram of
Figure~\ref{f:overview}.
The Arabic text and the reference tag chunks are the primary inputs to \framework. 
Solutions, morphology-based Boolean formulae, tags, 
morphology-based regular expressions, 
tag chunks, relation and action definitions, and data structures expressing entities 
and relations are input and output data to processes. 
The morphological analyzer (Sarf), $Syn^k$ detector, 
GUI for Boolean formulae definition, 
visualization annotator, GUI for regular expression and action definition, 
Boolean formula simulator, regular expression simulator, 
relation extraction and action execution, and difference and statistical analyzer 
are processes.

%The user interacts with \framework~through a user-friendly interfaces to 
%specify morphology-based Boolean formulae and regular expressions 
%and associate them with tag types that are in turn associated with
%visualization legends such as 
%foreground and background colors and fonts. 
%The user also associates code actions to parts of the regular expressions.
%The user can also declare relations between matches using the relation
%editor.

\subsection{The extended synonymy feature $Syn^k$}

The motivation for $Syn^k$ is to construct a light Arabic ontology based on the lexicon of Sarf.
The sets $E, A,$ and $L$ denote all English words, Arabic words, 
and Arabic lexicon words, respectively.
Recall that $\GG$ and $\ss$ denote the set of glosses and stems in the morphological analyzer, respectively.
We have $\GG \subset E$ and $\ss \subset L \subset A$. 
Function $\alpha: \ss \rightarrow 2^{\GG}$ maps Arabic stems to 
subsets of related English glosses. %; e.g. $g_{s} = \alpha(s)\subset 2^{\GG}$.
Function $\gamma: L \rightarrow 2^{\ss}$ maps Arabic lexicon words to subsets 
of relevant Arabic stems. %; e.g. $s_{l} = \gamma(l)\subset 2^{\ss}$.

Given a word $w\in L$, 
$Sy(w)=\{u\mid u\in \ss \land\exists s\in \gamma(w)\land~\alpha(u)\cap\alpha(s)\neq\emptyset\}$
is the set of Arabic stems 
directly related to $w$ through the gloss map.

Let $Sy^{i}(w)$ denote stems related to $w$ using the gloss map of order $i$ recursively such that
$Sy^{1}(w) = Sy(w)$ and
$Sy^{i+1}(w)=\{u\mid u\in S\land\exists s\in Sy^{i}(w)\land~\alpha(u)\cap\alpha(s)\neq\phi\}$.
Formally, $Syn^k(w) = \bigcup\limits_{i=1}^{k} Sy^{i}(w)$ for $i\in[1 \ldots k]$.

The example in Figure~\ref{fig:introsynEx} illustrates the computation.
The input Arabic word \RL{mA'}, denoted $w$, shares the gloss \cci{water} with the stem \RL{n.d.h}, denoted $s_1$, i.e. $s_1\in Sy^{1}(w)$.
%The input word \RL{mA'} shares the gloss \RL{n.d.h} through the gloss intersection \cci{water}. 
Next, the stem \RL{r^s^s}, denoted $s_2$, shares the gloss \cci{spray} with $s_1$, i.e. $s_2\in Sy^{1}(s1)\subset Sy^{2}(w)$.
Therefore, $Syn^2(w)$ relates the words \RL{mA'} and \RL{r^s^s}.

\begin{figure}[tb!]
\setcode{utf8}
\begin{center}
  \resizebox{0.7\columnwidth}{!}{ 
  	{\relsize{-2} % Graphic for TeX using PGF
% Title: /home/ameen/Desktop/Syn.dia
% Creator: Dia v0.97.1
% CreationDate: Fri Mar  7 00:34:04 2014
% For: ameen
% \usepackage{tikz}
% The following commands are not supported in PSTricks at present
% We define them conditionally, so when they are implemented,
% this pgf file will use them.
\ifx\du\undefined
  \newlength{\du}
\fi
\setlength{\du}{15\unitlength}
\begin{tikzpicture}
\pgftransformxscale{1.000000}
\pgftransformyscale{-1.000000}
\definecolor{dialinecolor}{rgb}{0.000000, 0.000000, 0.000000}
\pgfsetstrokecolor{dialinecolor}
\definecolor{dialinecolor}{rgb}{1.000000, 1.000000, 1.000000}
\pgfsetfillcolor{dialinecolor}
\definecolor{dialinecolor}{rgb}{1.000000, 1.000000, 1.000000}
\pgfsetfillcolor{dialinecolor}
\pgfpathellipse{\pgfpoint{3.630147\du}{12.268195\du}}{\pgfpoint{0.803947\du}{0\du}}{\pgfpoint{0\du}{0.833295\du}}
\pgfusepath{fill}
\pgfsetlinewidth{0.020000\du}
\pgfsetdash{{\pgflinewidth}{0.200000\du}}{0cm}
\pgfsetdash{{\pgflinewidth}{0.200000\du}}{0cm}
\pgfsetmiterjoin
\definecolor{dialinecolor}{rgb}{0.000000, 0.000000, 0.000000}
\pgfsetstrokecolor{dialinecolor}
\pgfpathellipse{\pgfpoint{3.630147\du}{12.268195\du}}{\pgfpoint{0.803947\du}{0\du}}{\pgfpoint{0\du}{0.833295\du}}
\pgfusepath{stroke}
% setfont left to latex
\definecolor{dialinecolor}{rgb}{0.000000, 0.000000, 0.000000}
\pgfsetstrokecolor{dialinecolor}
\node at (3.630147\du,12.371528\du){\RL{ماء}};
\definecolor{dialinecolor}{rgb}{1.000000, 1.000000, 1.000000}
\pgfsetfillcolor{dialinecolor}
\pgfpathellipse{\pgfpoint{7.412496\du}{12.443762\du}}{\pgfpoint{0.899916\du}{0\du}}{\pgfpoint{0\du}{0.848262\du}}
\pgfusepath{fill}
\pgfsetlinewidth{0.020000\du}
\pgfsetdash{{\pgflinewidth}{0.200000\du}}{0cm}
\pgfsetdash{{\pgflinewidth}{0.200000\du}}{0cm}
\pgfsetmiterjoin
\definecolor{dialinecolor}{rgb}{0.000000, 0.000000, 0.000000}
\pgfsetstrokecolor{dialinecolor}
\pgfpathellipse{\pgfpoint{7.412496\du}{12.443762\du}}{\pgfpoint{0.899916\du}{0\du}}{\pgfpoint{0\du}{0.848262\du}}
\pgfusepath{stroke}
% setfont left to latex
\definecolor{dialinecolor}{rgb}{0.000000, 0.000000, 0.000000}
\pgfsetstrokecolor{dialinecolor}
\node at (7.412496\du,12.547096\du){\RL{نضح}};
\definecolor{dialinecolor}{rgb}{1.000000, 1.000000, 1.000000}
\pgfsetfillcolor{dialinecolor}
\pgfpathellipse{\pgfpoint{11.327704\du}{12.243392\du}}{\pgfpoint{0.887504\du}{0\du}}{\pgfpoint{0\du}{0.840661\du}}
\pgfusepath{fill}
\pgfsetlinewidth{0.020000\du}
\pgfsetdash{{\pgflinewidth}{0.200000\du}}{0cm}
\pgfsetdash{{\pgflinewidth}{0.200000\du}}{0cm}
\pgfsetmiterjoin
\definecolor{dialinecolor}{rgb}{0.000000, 0.000000, 0.000000}
\pgfsetstrokecolor{dialinecolor}
\pgfpathellipse{\pgfpoint{11.327704\du}{12.243392\du}}{\pgfpoint{0.887504\du}{0\du}}{\pgfpoint{0\du}{0.840661\du}}
\pgfusepath{stroke}
% setfont left to latex
\definecolor{dialinecolor}{rgb}{0.000000, 0.000000, 0.000000}
\pgfsetstrokecolor{dialinecolor}
\node at (11.327704\du,12.346725\du){\RL{رشّ}};
\pgfsetlinewidth{0.020000\du}
\pgfsetdash{}{0pt}
\pgfsetdash{}{0pt}
\pgfsetmiterjoin
\definecolor{dialinecolor}{rgb}{0.000000, 0.000000, 0.000000}
\pgfsetstrokecolor{dialinecolor}
\pgfpathellipse{\pgfpoint{4.446635\du}{12.287500\du}}{\pgfpoint{1.753365\du}{0\du}}{\pgfpoint{0\du}{1.150000\du}}
\pgfusepath{stroke}
% setfont left to latex
\definecolor{dialinecolor}{rgb}{0.000000, 0.000000, 0.000000}
\pgfsetstrokecolor{dialinecolor}
\node at (4.446635\du,12.390833\du){};
\pgfsetlinewidth{0.020000\du}
\pgfsetdash{}{0pt}
\pgfsetdash{}{0pt}
\pgfsetmiterjoin
\definecolor{dialinecolor}{rgb}{0.000000, 0.000000, 0.000000}
\pgfsetstrokecolor{dialinecolor}
\pgfpathellipse{\pgfpoint{7.512500\du}{12.275000\du}}{\pgfpoint{2.487500\du}{0\du}}{\pgfpoint{0\du}{1.237500\du}}
\pgfusepath{stroke}
% setfont left to latex
\definecolor{dialinecolor}{rgb}{0.000000, 0.000000, 0.000000}
\pgfsetstrokecolor{dialinecolor}
\node at (7.512500\du,12.378333\du){};
% setfont left to latex
\definecolor{dialinecolor}{rgb}{0.000000, 0.000000, 0.000000}
\pgfsetstrokecolor{dialinecolor}
\node at (5.612500\du,12.325000\du){water};
% setfont left to latex
\definecolor{dialinecolor}{rgb}{0.000000, 0.000000, 0.000000}
\pgfsetstrokecolor{dialinecolor}
\node at (7.425000\du,11.400000\du){leak};
% setfont left to latex
\definecolor{dialinecolor}{rgb}{0.000000, 0.000000, 0.000000}
\pgfsetstrokecolor{dialinecolor}
\node at (9.462500\du,12.375000\du){spray};
% setfont left to latex
\definecolor{dialinecolor}{rgb}{0.000000, 0.000000, 0.000000}
\pgfsetstrokecolor{dialinecolor}
\node at (10.250000\du,11.512500\du){splatter};
\pgfsetlinewidth{0.020000\du}
\pgfsetdash{}{0pt}
\pgfsetdash{}{0pt}
\pgfsetmiterjoin
\definecolor{dialinecolor}{rgb}{0.000000, 0.000000, 0.000000}
\pgfsetstrokecolor{dialinecolor}
\pgfpathellipse{\pgfpoint{10.606250\du}{12.237500\du}}{\pgfpoint{1.793750\du}{0\du}}{\pgfpoint{0\du}{1.225000\du}}
\pgfusepath{stroke}
% setfont left to latex
\definecolor{dialinecolor}{rgb}{0.000000, 0.000000, 0.000000}
\pgfsetstrokecolor{dialinecolor}
\node at (10.606250\du,12.340833\du){};
\pgfsetlinewidth{0.100000\du}
\pgfsetdash{}{0pt}
\pgfsetdash{}{0pt}
\pgfsetmiterjoin
\definecolor{dialinecolor}{rgb}{1.000000, 1.000000, 1.000000}
\pgfsetfillcolor{dialinecolor}
\fill (5.300000\du,10.150000\du)--(5.300000\du,10.225000\du)--(9.750000\du,10.225000\du)--(9.750000\du,10.150000\du)--cycle;
\definecolor{dialinecolor}{rgb}{1.000000, 1.000000, 1.000000}
\pgfsetstrokecolor{dialinecolor}
\draw (5.300000\du,10.150000\du)--(5.300000\du,10.225000\du)--(9.750000\du,10.225000\du)--(9.750000\du,10.150000\du)--cycle;
\pgfsetlinewidth{0.100000\du}
\pgfsetdash{}{0pt}
\pgfsetdash{}{0pt}
\pgfsetmiterjoin
\definecolor{dialinecolor}{rgb}{1.000000, 1.000000, 1.000000}
\pgfsetfillcolor{dialinecolor}
\fill (5.202500\du,14.050000\du)--(5.202500\du,14.125000\du)--(9.652500\du,14.125000\du)--(9.652500\du,14.050000\du)--cycle;
\definecolor{dialinecolor}{rgb}{1.000000, 1.000000, 1.000000}
\pgfsetstrokecolor{dialinecolor}
\draw (5.202500\du,14.050000\du)--(5.202500\du,14.125000\du)--(9.652500\du,14.125000\du)--(9.652500\du,14.050000\du)--cycle;
\pgfsetlinewidth{0.100000\du}
\pgfsetdash{}{0pt}
\pgfsetdash{}{0pt}
\pgfsetmiterjoin
\definecolor{dialinecolor}{rgb}{1.000000, 1.000000, 1.000000}
\pgfsetfillcolor{dialinecolor}
\fill (2.575000\du,10.625000\du)--(2.575000\du,13.500000\du)--(2.625000\du,13.500000\du)--(2.625000\du,10.625000\du)--cycle;
\definecolor{dialinecolor}{rgb}{1.000000, 1.000000, 1.000000}
\pgfsetstrokecolor{dialinecolor}
\draw (2.575000\du,10.625000\du)--(2.575000\du,13.500000\du)--(2.625000\du,13.500000\du)--(2.625000\du,10.625000\du)--cycle;
\pgfsetlinewidth{0.100000\du}
\pgfsetdash{}{0pt}
\pgfsetdash{}{0pt}
\pgfsetmiterjoin
\definecolor{dialinecolor}{rgb}{1.000000, 1.000000, 1.000000}
\pgfsetfillcolor{dialinecolor}
\fill (12.527500\du,10.600000\du)--(12.527500\du,13.475000\du)--(12.577500\du,13.475000\du)--(12.577500\du,10.600000\du)--cycle;
\definecolor{dialinecolor}{rgb}{1.000000, 1.000000, 1.000000}
\pgfsetstrokecolor{dialinecolor}
\draw (12.527500\du,10.600000\du)--(12.527500\du,13.475000\du)--(12.577500\du,13.475000\du)--(12.577500\du,10.600000\du)--cycle;
\pgfsetlinewidth{0.020000\du}
\pgfsetdash{}{0pt}
\pgfsetdash{}{0pt}
\pgfsetbuttcap
{
\definecolor{dialinecolor}{rgb}{1.000000, 0.000000, 0.000000}
\pgfsetfillcolor{dialinecolor}
% was here!!!
\pgfsetarrowsstart{latex}
\definecolor{dialinecolor}{rgb}{1.000000, 0.000000, 0.000000}
\pgfsetstrokecolor{dialinecolor}
\pgfpathmoveto{\pgfpoint{6.560588\du}{11.131713\du}}
\pgfpatharc{316}{224}{1.471709\du and 1.471709\du}
\pgfusepath{stroke}
}
\pgfsetlinewidth{0.020000\du}
\pgfsetdash{}{0pt}
\pgfsetdash{}{0pt}
\pgfsetbuttcap
{
\definecolor{dialinecolor}{rgb}{1.000000, 0.000000, 0.000000}
\pgfsetfillcolor{dialinecolor}
% was here!!!
\pgfsetarrowsstart{latex}
\definecolor{dialinecolor}{rgb}{1.000000, 0.000000, 0.000000}
\pgfsetstrokecolor{dialinecolor}
\pgfpathmoveto{\pgfpoint{10.606267\du}{11.012515\du}}
\pgfpatharc{311}{224}{1.549530\du and 1.549530\du}
\pgfusepath{stroke}
}
\end{tikzpicture}
} }
\setcode{standard}
\caption{$Syn^2($\RL{mA'}$)$}
\label{fig:introsynEx}
\end{center}
\end{figure}

\subsection{MRE: Morphology-based regular expressions}

Let ${\cal O} = \{ \mathit{isA}, \mathit{contains}\}$ be the set of atomic term 
predicates, and let ${\cal F} = \{ \pp, \ss, \xx, \PP, \GG, \AC\}$ be the 
set of morphological features where each entry is in turn a set of values for that feature.
Given a word $w$, a morphological feature $A\in{\cal F}$, 
a user defined constant feature value $CF\in A$, and an integer 
$k, 1\le k\le 7$, 
the following are morphology-based atomic terms (MAT), {\em terms} for short.
\begin{itemize}
  \item $a(w):= \exists m \in M(w). m=\langle p^k,s,x^l,P,G,C\rangle. r \circ CF$
where $\circ \in {\cal O}$, $r \in \{p^k,s,x^l,P,G,C\}$, and $r\in A$.
Informally, a solution vector of $w$ exists with
a feature containing or exactly matching $CF$.
The predicates $\mathit{isA}$ and $\mathit{contains}$ denote the exact match and contain criteria, respectively.
\item $a(w) := w \in Syn^k(CF), CF \in \ss$.
  Informally, this checks if $w$ is an extended synonym of a stem $CF$. 
\end{itemize}

A morphology-based Boolean formula (MBF) is of the following form.
\begin{itemize}
  \item $a$ and $\neg a$ are MBF formulae where $a$ is a MAT and $\neg$ is the negation operator. 
  \item $(f \vee g)$ is an MBF where $f$ and $g$ are MBF formulae, 
    and $\vee$ is the disjunction (union) operator. 
\end{itemize}

Moreover, \framework provides $O$ to be a default Boolean formula that tags all {\em other} words in the text that do not match a user defined formula.
We also refer to those words as {\em null} words.

\setcode{utf8}
\setarab
\transfalse
\begin{figure}[t!]
%\renewcommand{\arraystretch}{1.5}% Wider
\resizebox{\columnwidth}{!} {
	\begin{tabular}{c|c|c|c}
		MBF & description & formula & matches\\ \hline
        N & name of person & $category~=~Name\_of\_Person$ & $n_1,n_2,n_3$ \\ \hline
        P & name of place & $category~=~Name\_of\_Place$ & $p_1,p_2,...,p_7$ \\ \hline
        R & relative position & $stem\in$ \{\RL{قرب},\RL{في},\dots\} & $r_1,r_2,r_3,r_4$ \\ \hline
        U & numerical term & $stem\in$ \{\RL{أول},\RL{ثاني},\dots\} & $u_1,u_2$ \\
    \end{tabular}
}
\caption{Boolean formulae corresponding to task in Fig.~\ref{fig:intromotiv}}
  \label{fig:taskMBF}
\end{figure}
\transtrue
\setcode{standard}

Consider the task we discussed in the introduction (Figure~\ref{fig:intromotiv}) 
and recall that we are interested in identifying names of people, names of places, relative positions, and numerical terms.
The table in Figure~\ref{fig:taskMBF} presents the defined formulae.
The user denotes the ``name of person'' entities with formula $N$ 
which requires the {\em category} feature in the morphological solution of a word to be {\tt Name\_of \_Person}.
The entities $n_1$, $n_2$, and $n_3$ are matches of the formula $N$ in the text.
Similarly, the user specifies formula $P$ to denote ``name of place'' entities. 
The user specifies formula $R$ to denote ``relative position'' entities, 
and defines it as a disjunction of MAT terms that check for 
solutions matching stems such as \RL{qrb} (near) and \RL{fy} (in).
%requires the stem feature of a word to belong to a selected list of stems containing \RL{fy} and \RL{qrb}. a
Similarly, $U$ denotes numerical terms and is a disjunction of constraints 
requiring the stem feature to belong to a set of stems such as
\RL{'wl}(first), \RL{_tAny}(second), \dots \RL{`A^sr}(tenth).

%Formula $N$ in Figure~\ref{fig:motiv} checks whether a solution has a category feature matching category {\tt Name\_of\_Person}.
%Formula $R$ is the disjunction of MAT terms that check for solutions matching stems such as \RL{qrb} (near) and \RL{fy} (in).

Next, we define a morphology-based regular expression (MRE) as follows.
\begin{itemize}
\item $m$ is an MRE where $m$ is an MBF.
\item $fg$ is an MRE denoting a concatenation operation
  where $f$ and $g$ are both MRE expressions. This is 
  satisfied by a match of $f$ followed by a match of $g$. 
\item $f*,f+,f$\textasciicircum$x,$ and $f?$ are MRE expressions where $f$ is an MRE,
  and are satisfied by zero or more matches, one of more matches, up to $x$ matches, 
  and zero or only one match of $f$, respectively.
\item $f\& g,$ (conjunction) and $f|g$ (disjunction) are  MRE expressions where 
  $f$ and $g$ are MRE expressions, and are satisfied by a 
  a match of both $f$ and $g$, 
  and a match of either $f$ or $g$, respectively. 
\end{itemize}
We denote by $\ldbrack f\rdbrack$ the set of matches of an MRE $f$.

Back to the example in Figure~\ref{fig:intromotiv}.
We use the formulae defined in Figure~\ref{fig:taskMBF} to construct an MRE such as
$(P|N)\!+O?~R~O^\wedge\!2~(P|N|U)+$ where 
$|,+,?,$ and $^\wedge k$ denote disjunction, one or more, zero or one, and
up to $k$ matches, respectively.
The expression specifies a sequence of places or names of persons, 
optionally followed by a null word, 
followed by one relative position, followed by up to two possible null words, 
followed by one or more match of name of place, name of person, or numerical term.
$O?$ and $O^\wedge 2$ are used in the expression to allow for flexible matches.
%Note that the user reaches the satisfying matches by experimenting with the visualizer and the expression editor which do not require knowledge and expertise in regular expressions. 

\setcode{utf8}
\setarab
\transfalse
\begin{figure}[t!]
%\renewcommand{\arraystretch}{1.5}% Wider
\resizebox{\columnwidth}{!} {
\begin{tabular}{cc}
\relsize{-2} % Graphic for TeX using PGF
% Title: /home/ameen/Desktop/match1.dia
% Creator: Dia v0.97.1
% CreationDate: Sat Jan 11 00:35:23 2014
% For: ameen
% \usepackage{tikz}
% The following commands are not supported in PSTricks at present
% We define them conditionally, so when they are implemented,
% this pgf file will use them.
\ifx\du\undefined
  \newlength{\du}
\fi
\setlength{\du}{15\unitlength}
\begin{tikzpicture}
\pgftransformxscale{1.000000}
\pgftransformyscale{-1.000000}
\definecolor{dialinecolor}{rgb}{0.000000, 0.000000, 0.000000}
\pgfsetstrokecolor{dialinecolor}
\definecolor{dialinecolor}{rgb}{1.000000, 1.000000, 1.000000}
\pgfsetfillcolor{dialinecolor}
\definecolor{dialinecolor}{rgb}{1.000000, 1.000000, 1.000000}
\pgfsetfillcolor{dialinecolor}
\fill (23.657500\du,7.687500\du)--(23.657500\du,8.537500\du)--(24.632500\du,8.537500\du)--(24.632500\du,7.687500\du)--cycle;
\pgfsetlinewidth{0.020000\du}
\pgfsetdash{}{0pt}
\pgfsetdash{}{0pt}
\pgfsetmiterjoin
\definecolor{dialinecolor}{rgb}{0.000000, 0.000000, 0.000000}
\pgfsetstrokecolor{dialinecolor}
\draw (23.657500\du,7.687500\du)--(23.657500\du,8.537500\du)--(24.632500\du,8.537500\du)--(24.632500\du,7.687500\du)--cycle;
% setfont left to latex
\definecolor{dialinecolor}{rgb}{0.000000, 0.000000, 0.000000}
\pgfsetstrokecolor{dialinecolor}
\node at (24.145000\du,8.233056\du){()};
\definecolor{dialinecolor}{rgb}{1.000000, 1.000000, 1.000000}
\pgfsetfillcolor{dialinecolor}
\pgfpathellipse{\pgfpoint{25.225101\du}{9.941465\du}}{\pgfpoint{1.185301\du}{0\du}}{\pgfpoint{0\du}{0.726265\du}}
\pgfusepath{fill}
\pgfsetlinewidth{0.020000\du}
\pgfsetdash{}{0pt}
\pgfsetdash{}{0pt}
\pgfsetmiterjoin
\definecolor{dialinecolor}{rgb}{0.000000, 0.000000, 0.000000}
\pgfsetstrokecolor{dialinecolor}
\pgfpathellipse{\pgfpoint{25.225101\du}{9.941465\du}}{\pgfpoint{1.185301\du}{0\du}}{\pgfpoint{0\du}{0.726265\du}}
\pgfusepath{stroke}
% setfont left to latex
\definecolor{dialinecolor}{rgb}{0.000000, 0.000000, 0.000000}
\pgfsetstrokecolor{dialinecolor}
\node at (25.225101\du,10.062021\du){r1=\RL{بالقرب}};
\definecolor{dialinecolor}{rgb}{1.000000, 1.000000, 1.000000}
\pgfsetfillcolor{dialinecolor}
\pgfpathellipse{\pgfpoint{22.605219\du}{13.362071\du}}{\pgfpoint{1.319819\du}{0\du}}{\pgfpoint{0\du}{0.707771\du}}
\pgfusepath{fill}
\pgfsetlinewidth{0.020000\du}
\pgfsetdash{}{0pt}
\pgfsetdash{}{0pt}
\pgfsetmiterjoin
\definecolor{dialinecolor}{rgb}{0.000000, 0.000000, 0.000000}
\pgfsetstrokecolor{dialinecolor}
\pgfpathellipse{\pgfpoint{22.605219\du}{13.362071\du}}{\pgfpoint{1.319819\du}{0\du}}{\pgfpoint{0\du}{0.707771\du}}
\pgfusepath{stroke}
% setfont left to latex
\definecolor{dialinecolor}{rgb}{0.000000, 0.000000, 0.000000}
\pgfsetstrokecolor{dialinecolor}
\node at (22.605219\du,13.482627\du){p2=\RL{التقاطع}};
\definecolor{dialinecolor}{rgb}{1.000000, 1.000000, 1.000000}
\pgfsetfillcolor{dialinecolor}
\fill (22.757500\du,9.782730\du)--(22.757500\du,10.667730\du)--(23.757500\du,10.667730\du)--(23.757500\du,9.782730\du)--cycle;
\pgfsetlinewidth{0.020000\du}
\pgfsetdash{}{0pt}
\pgfsetdash{}{0pt}
\pgfsetmiterjoin
\definecolor{dialinecolor}{rgb}{0.000000, 0.000000, 0.000000}
\pgfsetstrokecolor{dialinecolor}
\draw (22.757500\du,9.782730\du)--(22.757500\du,10.667730\du)--(23.757500\du,10.667730\du)--(23.757500\du,9.782730\du)--cycle;
% setfont left to latex
\definecolor{dialinecolor}{rgb}{0.000000, 0.000000, 0.000000}
\pgfsetstrokecolor{dialinecolor}
\node at (23.257500\du,10.345786\du){\^{}2};
\definecolor{dialinecolor}{rgb}{1.000000, 1.000000, 1.000000}
\pgfsetfillcolor{dialinecolor}
\pgfpathellipse{\pgfpoint{23.312350\du}{11.641042\du}}{\pgfpoint{0.695150\du}{0\du}}{\pgfpoint{0\du}{0.488202\du}}
\pgfusepath{fill}
\pgfsetlinewidth{0.020000\du}
\pgfsetdash{}{0pt}
\pgfsetdash{}{0pt}
\pgfsetmiterjoin
\definecolor{dialinecolor}{rgb}{0.000000, 0.000000, 0.000000}
\pgfsetstrokecolor{dialinecolor}
\pgfpathellipse{\pgfpoint{23.312350\du}{11.641042\du}}{\pgfpoint{0.695150\du}{0\du}}{\pgfpoint{0\du}{0.488202\du}}
\pgfusepath{stroke}
% setfont left to latex
\definecolor{dialinecolor}{rgb}{0.000000, 0.000000, 0.000000}
\pgfsetstrokecolor{dialinecolor}
\node at (23.312350\du,11.761598\du){\RL{من}};
\definecolor{dialinecolor}{rgb}{1.000000, 1.000000, 1.000000}
\pgfsetfillcolor{dialinecolor}
\pgfpathellipse{\pgfpoint{20.061250\du}{13.438719\du}}{\pgfpoint{1.046250\du}{0\du}}{\pgfpoint{0\du}{0.631119\du}}
\pgfusepath{fill}
\pgfsetlinewidth{0.020000\du}
\pgfsetdash{}{0pt}
\pgfsetdash{}{0pt}
\pgfsetmiterjoin
\definecolor{dialinecolor}{rgb}{0.000000, 0.000000, 0.000000}
\pgfsetstrokecolor{dialinecolor}
\pgfpathellipse{\pgfpoint{20.061250\du}{13.438719\du}}{\pgfpoint{1.046250\du}{0\du}}{\pgfpoint{0\du}{0.631119\du}}
\pgfusepath{stroke}
% setfont left to latex
\definecolor{dialinecolor}{rgb}{0.000000, 0.000000, 0.000000}
\pgfsetstrokecolor{dialinecolor}
\node at (20.061250\du,13.559275\du){u1=\RL{الأول}};
\definecolor{dialinecolor}{rgb}{1.000000, 1.000000, 1.000000}
\pgfsetfillcolor{dialinecolor}
\pgfpathellipse{\pgfpoint{25.370283\du}{13.399620\du}}{\pgfpoint{1.072583\du}{0\du}}{\pgfpoint{0\du}{0.670220\du}}
\pgfusepath{fill}
\pgfsetlinewidth{0.020000\du}
\pgfsetdash{}{0pt}
\pgfsetdash{}{0pt}
\pgfsetmiterjoin
\definecolor{dialinecolor}{rgb}{0.000000, 0.000000, 0.000000}
\pgfsetstrokecolor{dialinecolor}
\pgfpathellipse{\pgfpoint{25.370283\du}{13.399620\du}}{\pgfpoint{1.072583\du}{0\du}}{\pgfpoint{0\du}{0.670220\du}}
\pgfusepath{stroke}
% setfont left to latex
\definecolor{dialinecolor}{rgb}{0.000000, 0.000000, 0.000000}
\pgfsetstrokecolor{dialinecolor}
\node at (25.370283\du,13.520176\du){n1=\RL{خليفة}};
\definecolor{dialinecolor}{rgb}{1.000000, 1.000000, 1.000000}
\pgfsetfillcolor{dialinecolor}
\pgfpathellipse{\pgfpoint{27.556244\du}{13.449916\du}}{\pgfpoint{0.861244\du}{0\du}}{\pgfpoint{0\du}{0.619916\du}}
\pgfusepath{fill}
\pgfsetlinewidth{0.020000\du}
\pgfsetdash{}{0pt}
\pgfsetdash{}{0pt}
\pgfsetmiterjoin
\definecolor{dialinecolor}{rgb}{0.000000, 0.000000, 0.000000}
\pgfsetstrokecolor{dialinecolor}
\pgfpathellipse{\pgfpoint{27.556244\du}{13.449916\du}}{\pgfpoint{0.861244\du}{0\du}}{\pgfpoint{0\du}{0.619916\du}}
\pgfusepath{stroke}
% setfont left to latex
\definecolor{dialinecolor}{rgb}{0.000000, 0.000000, 0.000000}
\pgfsetstrokecolor{dialinecolor}
\node at (27.556244\du,13.570472\du){p1=\RL{برج}};
\definecolor{dialinecolor}{rgb}{1.000000, 1.000000, 1.000000}
\pgfsetfillcolor{dialinecolor}
\fill (26.957500\du,9.717730\du)--(26.957500\du,10.667730\du)--(27.882500\du,10.667730\du)--(27.882500\du,9.717730\du)--cycle;
\pgfsetlinewidth{0.020000\du}
\pgfsetdash{}{0pt}
\pgfsetdash{}{0pt}
\pgfsetmiterjoin
\definecolor{dialinecolor}{rgb}{0.000000, 0.000000, 0.000000}
\pgfsetstrokecolor{dialinecolor}
\draw (26.957500\du,9.717730\du)--(26.957500\du,10.667730\du)--(27.882500\du,10.667730\du)--(27.882500\du,9.717730\du)--cycle;
% setfont left to latex
\definecolor{dialinecolor}{rgb}{0.000000, 0.000000, 0.000000}
\pgfsetstrokecolor{dialinecolor}
\node at (27.420000\du,10.313286\du){$+$};
\definecolor{dialinecolor}{rgb}{1.000000, 1.000000, 1.000000}
\pgfsetfillcolor{dialinecolor}
\fill (27.082500\du,11.203540\du)--(27.082500\du,12.078540\du)--(27.982500\du,12.078540\du)--(27.982500\du,11.203540\du)--cycle;
\pgfsetlinewidth{0.020000\du}
\pgfsetdash{}{0pt}
\pgfsetdash{}{0pt}
\pgfsetmiterjoin
\definecolor{dialinecolor}{rgb}{0.000000, 0.000000, 0.000000}
\pgfsetstrokecolor{dialinecolor}
\draw (27.082500\du,11.203540\du)--(27.082500\du,12.078540\du)--(27.982500\du,12.078540\du)--(27.982500\du,11.203540\du)--cycle;
% setfont left to latex
\definecolor{dialinecolor}{rgb}{0.000000, 0.000000, 0.000000}
\pgfsetstrokecolor{dialinecolor}
\node at (27.532500\du,11.761596\du){$|$};
\definecolor{dialinecolor}{rgb}{1.000000, 1.000000, 1.000000}
\pgfsetfillcolor{dialinecolor}
\fill (21.507500\du,11.178540\du)--(21.507500\du,12.103540\du)--(22.457500\du,12.103540\du)--(22.457500\du,11.178540\du)--cycle;
\pgfsetlinewidth{0.020000\du}
\pgfsetdash{}{0pt}
\pgfsetdash{}{0pt}
\pgfsetmiterjoin
\definecolor{dialinecolor}{rgb}{0.000000, 0.000000, 0.000000}
\pgfsetstrokecolor{dialinecolor}
\draw (21.507500\du,11.178540\du)--(21.507500\du,12.103540\du)--(22.457500\du,12.103540\du)--(22.457500\du,11.178540\du)--cycle;
% setfont left to latex
\definecolor{dialinecolor}{rgb}{0.000000, 0.000000, 0.000000}
\pgfsetstrokecolor{dialinecolor}
\node at (21.982500\du,11.761596\du){$|$};
\definecolor{dialinecolor}{rgb}{1.000000, 1.000000, 1.000000}
\pgfsetfillcolor{dialinecolor}
\fill (20.382500\du,9.717730\du)--(20.382500\du,10.667730\du)--(21.307500\du,10.667730\du)--(21.307500\du,9.717730\du)--cycle;
\pgfsetlinewidth{0.020000\du}
\pgfsetdash{}{0pt}
\pgfsetdash{}{0pt}
\pgfsetmiterjoin
\definecolor{dialinecolor}{rgb}{0.000000, 0.000000, 0.000000}
\pgfsetstrokecolor{dialinecolor}
\draw (20.382500\du,9.717730\du)--(20.382500\du,10.667730\du)--(21.307500\du,10.667730\du)--(21.307500\du,9.717730\du)--cycle;
% setfont left to latex
\definecolor{dialinecolor}{rgb}{0.000000, 0.000000, 0.000000}
\pgfsetstrokecolor{dialinecolor}
\node at (20.845000\du,10.313286\du){$+$};
\pgfsetlinewidth{0.020000\du}
\pgfsetdash{}{0pt}
\pgfsetdash{}{0pt}
\pgfsetbuttcap
{
\definecolor{dialinecolor}{rgb}{0.000000, 0.000000, 0.000000}
\pgfsetfillcolor{dialinecolor}
% was here!!!
\pgfsetarrowsend{latex}
\definecolor{dialinecolor}{rgb}{0.000000, 0.000000, 0.000000}
\pgfsetstrokecolor{dialinecolor}
\draw (24.642726\du,8.428648\du)--(26.953057\du,9.896135\du);
}
\pgfsetlinewidth{0.020000\du}
\pgfsetdash{}{0pt}
\pgfsetdash{}{0pt}
\pgfsetbuttcap
{
\definecolor{dialinecolor}{rgb}{0.000000, 0.000000, 0.000000}
\pgfsetfillcolor{dialinecolor}
% was here!!!
\pgfsetarrowsend{latex}
\definecolor{dialinecolor}{rgb}{0.000000, 0.000000, 0.000000}
\pgfsetstrokecolor{dialinecolor}
\draw (27.457628\du,10.677150\du)--(27.497728\du,11.193393\du);
}
\pgfsetlinewidth{0.020000\du}
\pgfsetdash{}{0pt}
\pgfsetdash{}{0pt}
\pgfsetbuttcap
{
\definecolor{dialinecolor}{rgb}{0.000000, 0.000000, 0.000000}
\pgfsetfillcolor{dialinecolor}
% was here!!!
\pgfsetarrowsend{latex}
\definecolor{dialinecolor}{rgb}{0.000000, 0.000000, 0.000000}
\pgfsetstrokecolor{dialinecolor}
\draw (27.538372\du,12.088401\du)--(27.547978\du,12.820166\du);
}
\definecolor{dialinecolor}{rgb}{1.000000, 1.000000, 1.000000}
\pgfsetfillcolor{dialinecolor}
\fill (24.832500\du,11.191040\du)--(24.832500\du,12.091040\du)--(25.807500\du,12.091040\du)--(25.807500\du,11.191040\du)--cycle;
\pgfsetlinewidth{0.020000\du}
\pgfsetdash{}{0pt}
\pgfsetdash{}{0pt}
\pgfsetmiterjoin
\definecolor{dialinecolor}{rgb}{0.000000, 0.000000, 0.000000}
\pgfsetstrokecolor{dialinecolor}
\draw (24.832500\du,11.191040\du)--(24.832500\du,12.091040\du)--(25.807500\du,12.091040\du)--(25.807500\du,11.191040\du)--cycle;
% setfont left to latex
\definecolor{dialinecolor}{rgb}{0.000000, 0.000000, 0.000000}
\pgfsetstrokecolor{dialinecolor}
\node at (25.320000\du,11.761596\du){$|$};
\pgfsetlinewidth{0.020000\du}
\pgfsetdash{}{0pt}
\pgfsetdash{}{0pt}
\pgfsetbuttcap
{
\definecolor{dialinecolor}{rgb}{0.000000, 0.000000, 0.000000}
\pgfsetfillcolor{dialinecolor}
% was here!!!
\pgfsetarrowsend{latex}
\definecolor{dialinecolor}{rgb}{0.000000, 0.000000, 0.000000}
\pgfsetstrokecolor{dialinecolor}
\draw (26.947808\du,10.518388\du)--(25.817314\du,11.298056\du);
}
\pgfsetlinewidth{0.020000\du}
\pgfsetdash{}{0pt}
\pgfsetdash{}{0pt}
\pgfsetbuttcap
{
\definecolor{dialinecolor}{rgb}{0.000000, 0.000000, 0.000000}
\pgfsetfillcolor{dialinecolor}
% was here!!!
\pgfsetarrowsend{latex}
\definecolor{dialinecolor}{rgb}{0.000000, 0.000000, 0.000000}
\pgfsetstrokecolor{dialinecolor}
\draw (25.333062\du,12.097859\du)--(25.350826\du,12.719115\du);
}
\pgfsetlinewidth{0.020000\du}
\pgfsetdash{}{0pt}
\pgfsetdash{}{0pt}
\pgfsetbuttcap
{
\definecolor{dialinecolor}{rgb}{0.000000, 0.000000, 0.000000}
\pgfsetfillcolor{dialinecolor}
% was here!!!
\pgfsetarrowsend{latex}
\definecolor{dialinecolor}{rgb}{0.000000, 0.000000, 0.000000}
\pgfsetstrokecolor{dialinecolor}
\draw (24.400258\du,8.544736\du)--(24.816635\du,9.249798\du);
}
\pgfsetlinewidth{0.020000\du}
\pgfsetdash{}{0pt}
\pgfsetdash{}{0pt}
\pgfsetbuttcap
{
\definecolor{dialinecolor}{rgb}{0.000000, 0.000000, 0.000000}
\pgfsetfillcolor{dialinecolor}
% was here!!!
\pgfsetarrowsend{latex}
\definecolor{dialinecolor}{rgb}{0.000000, 0.000000, 0.000000}
\pgfsetstrokecolor{dialinecolor}
\draw (23.962560\du,8.546806\du)--(23.447524\du,9.772871\du);
}
\pgfsetlinewidth{0.020000\du}
\pgfsetdash{}{0pt}
\pgfsetdash{}{0pt}
\pgfsetbuttcap
{
\definecolor{dialinecolor}{rgb}{0.000000, 0.000000, 0.000000}
\pgfsetfillcolor{dialinecolor}
% was here!!!
\pgfsetarrowsend{latex}
\definecolor{dialinecolor}{rgb}{0.000000, 0.000000, 0.000000}
\pgfsetstrokecolor{dialinecolor}
\draw (23.275029\du,10.677695\du)--(23.293053\du,11.142950\du);
}
\definecolor{dialinecolor}{rgb}{1.000000, 1.000000, 1.000000}
\pgfsetfillcolor{dialinecolor}
\fill (19.635000\du,11.178540\du)--(19.635000\du,12.103540\du)--(20.585000\du,12.103540\du)--(20.585000\du,11.178540\du)--cycle;
\pgfsetlinewidth{0.020000\du}
\pgfsetdash{}{0pt}
\pgfsetdash{}{0pt}
\pgfsetmiterjoin
\definecolor{dialinecolor}{rgb}{0.000000, 0.000000, 0.000000}
\pgfsetstrokecolor{dialinecolor}
\draw (19.635000\du,11.178540\du)--(19.635000\du,12.103540\du)--(20.585000\du,12.103540\du)--(20.585000\du,11.178540\du)--cycle;
% setfont left to latex
\definecolor{dialinecolor}{rgb}{0.000000, 0.000000, 0.000000}
\pgfsetstrokecolor{dialinecolor}
\node at (20.110000\du,11.761596\du){$|$};
\pgfsetlinewidth{0.020000\du}
\pgfsetdash{}{0pt}
\pgfsetdash{}{0pt}
\pgfsetbuttcap
{
\definecolor{dialinecolor}{rgb}{0.000000, 0.000000, 0.000000}
\pgfsetfillcolor{dialinecolor}
% was here!!!
\pgfsetarrowsend{latex}
\definecolor{dialinecolor}{rgb}{0.000000, 0.000000, 0.000000}
\pgfsetstrokecolor{dialinecolor}
\draw (21.225463\du,10.677150\du)--(21.611202\du,11.168288\du);
}
\pgfsetlinewidth{0.020000\du}
\pgfsetdash{}{0pt}
\pgfsetdash{}{0pt}
\pgfsetbuttcap
{
\definecolor{dialinecolor}{rgb}{0.000000, 0.000000, 0.000000}
\pgfsetfillcolor{dialinecolor}
% was here!!!
\pgfsetarrowsend{latex}
\definecolor{dialinecolor}{rgb}{0.000000, 0.000000, 0.000000}
\pgfsetstrokecolor{dialinecolor}
\draw (20.599163\du,10.677150\du)--(20.349916\du,11.168288\du);
}
\pgfsetlinewidth{0.020000\du}
\pgfsetdash{}{0pt}
\pgfsetdash{}{0pt}
\pgfsetbuttcap
{
\definecolor{dialinecolor}{rgb}{0.000000, 0.000000, 0.000000}
\pgfsetfillcolor{dialinecolor}
% was here!!!
\pgfsetarrowsend{latex}
\definecolor{dialinecolor}{rgb}{0.000000, 0.000000, 0.000000}
\pgfsetstrokecolor{dialinecolor}
\draw (20.097182\du,12.113721\du)--(20.078639\du,12.797506\du);
}
\pgfsetlinewidth{0.020000\du}
\pgfsetdash{}{0pt}
\pgfsetdash{}{0pt}
\pgfsetbuttcap
{
\definecolor{dialinecolor}{rgb}{0.000000, 0.000000, 0.000000}
\pgfsetfillcolor{dialinecolor}
% was here!!!
\pgfsetarrowsend{latex}
\definecolor{dialinecolor}{rgb}{0.000000, 0.000000, 0.000000}
\pgfsetstrokecolor{dialinecolor}
\draw (22.153535\du,12.113735\du)--(22.350263\du,12.657440\du);
}
\pgfsetlinewidth{0.020000\du}
\pgfsetdash{}{0pt}
\pgfsetdash{}{0pt}
\pgfsetbuttcap
{
\definecolor{dialinecolor}{rgb}{0.000000, 0.000000, 0.000000}
\pgfsetfillcolor{dialinecolor}
% was here!!!
\pgfsetarrowsend{latex}
\definecolor{dialinecolor}{rgb}{0.000000, 0.000000, 0.000000}
\pgfsetstrokecolor{dialinecolor}
\draw (23.647905\du,8.425855\du)--(21.317522\du,9.894865\du);
}
% setfont left to latex
\definecolor{dialinecolor}{rgb}{0.000000, 0.000000, 0.000000}
\pgfsetstrokecolor{dialinecolor}
\node[anchor=west] at (21.340000\du,8.166810\du){match 1};
\end{tikzpicture}
 &
\relsize{-2} % Graphic for TeX using PGF
% Title: /home/ameen/Desktop/match2.dia
% Creator: Dia v0.97.1
% CreationDate: Sat Jan 11 00:36:28 2014
% For: ameen
% \usepackage{tikz}
% The following commands are not supported in PSTricks at present
% We define them conditionally, so when they are implemented,
% this pgf file will use them.
\ifx\du\undefined
  \newlength{\du}
\fi
\setlength{\du}{15\unitlength}
\begin{tikzpicture}
\pgftransformxscale{1.000000}
\pgftransformyscale{-1.000000}
\definecolor{dialinecolor}{rgb}{0.000000, 0.000000, 0.000000}
\pgfsetstrokecolor{dialinecolor}
\definecolor{dialinecolor}{rgb}{1.000000, 1.000000, 1.000000}
\pgfsetfillcolor{dialinecolor}
\definecolor{dialinecolor}{rgb}{1.000000, 1.000000, 1.000000}
\pgfsetfillcolor{dialinecolor}
\fill (24.100350\du,11.350000\du)--(24.100350\du,12.250000\du)--(25.075350\du,12.250000\du)--(25.075350\du,11.350000\du)--cycle;
\pgfsetlinewidth{0.020000\du}
\pgfsetdash{}{0pt}
\pgfsetdash{}{0pt}
\pgfsetmiterjoin
\definecolor{dialinecolor}{rgb}{0.000000, 0.000000, 0.000000}
\pgfsetstrokecolor{dialinecolor}
\draw (24.100350\du,11.350000\du)--(24.100350\du,12.250000\du)--(25.075350\du,12.250000\du)--(25.075350\du,11.350000\du)--cycle;
% setfont left to latex
\definecolor{dialinecolor}{rgb}{0.000000, 0.000000, 0.000000}
\pgfsetstrokecolor{dialinecolor}
\node at (24.587850\du,11.920556\du){()};
\definecolor{dialinecolor}{rgb}{1.000000, 1.000000, 1.000000}
\pgfsetfillcolor{dialinecolor}
\pgfpathellipse{\pgfpoint{20.350327\du}{17.174574\du}}{\pgfpoint{1.216577\du}{0\du}}{\pgfpoint{0\du}{0.632074\du}}
\pgfusepath{fill}
\pgfsetlinewidth{0.020000\du}
\pgfsetdash{}{0pt}
\pgfsetdash{}{0pt}
\pgfsetmiterjoin
\definecolor{dialinecolor}{rgb}{0.000000, 0.000000, 0.000000}
\pgfsetstrokecolor{dialinecolor}
\pgfpathellipse{\pgfpoint{20.350327\du}{17.174574\du}}{\pgfpoint{1.216577\du}{0\du}}{\pgfpoint{0\du}{0.632074\du}}
\pgfusepath{stroke}
% setfont left to latex
\definecolor{dialinecolor}{rgb}{0.000000, 0.000000, 0.000000}
\pgfsetstrokecolor{dialinecolor}
\node at (20.350327\du,17.295129\du){p7=\RL{المبنى}};
\definecolor{dialinecolor}{rgb}{1.000000, 1.000000, 1.000000}
\pgfsetfillcolor{dialinecolor}
\pgfpathellipse{\pgfpoint{23.081828\du}{17.180054\du}}{\pgfpoint{1.082178\du}{0\du}}{\pgfpoint{0\du}{0.637554\du}}
\pgfusepath{fill}
\pgfsetlinewidth{0.020000\du}
\pgfsetdash{}{0pt}
\pgfsetdash{}{0pt}
\pgfsetmiterjoin
\definecolor{dialinecolor}{rgb}{0.000000, 0.000000, 0.000000}
\pgfsetstrokecolor{dialinecolor}
\pgfpathellipse{\pgfpoint{23.081828\du}{17.180054\du}}{\pgfpoint{1.082178\du}{0\du}}{\pgfpoint{0\du}{0.637554\du}}
\pgfusepath{stroke}
% setfont left to latex
\definecolor{dialinecolor}{rgb}{0.000000, 0.000000, 0.000000}
\pgfsetstrokecolor{dialinecolor}
\node at (23.081828\du,17.300610\du){r4=\RL{مقربة}};
\definecolor{dialinecolor}{rgb}{1.000000, 1.000000, 1.000000}
\pgfsetfillcolor{dialinecolor}
\fill (21.875350\du,13.465230\du)--(21.875350\du,14.380230\du)--(22.809250\du,14.380230\du)--(22.809250\du,13.465230\du)--cycle;
\pgfsetlinewidth{0.020000\du}
\pgfsetdash{}{0pt}
\pgfsetdash{}{0pt}
\pgfsetmiterjoin
\definecolor{dialinecolor}{rgb}{0.000000, 0.000000, 0.000000}
\pgfsetstrokecolor{dialinecolor}
\draw (21.875350\du,13.465230\du)--(21.875350\du,14.380230\du)--(22.809250\du,14.380230\du)--(22.809250\du,13.465230\du)--cycle;
% setfont left to latex
\definecolor{dialinecolor}{rgb}{0.000000, 0.000000, 0.000000}
\pgfsetstrokecolor{dialinecolor}
\node at (22.342300\du,14.043286\du){\^{}2};
\definecolor{dialinecolor}{rgb}{1.000000, 1.000000, 1.000000}
\pgfsetfillcolor{dialinecolor}
\pgfpathellipse{\pgfpoint{21.712850\du}{15.353540\du}}{\pgfpoint{0.537500\du}{0\du}}{\pgfpoint{0\du}{0.437500\du}}
\pgfusepath{fill}
\pgfsetlinewidth{0.020000\du}
\pgfsetdash{}{0pt}
\pgfsetdash{}{0pt}
\pgfsetmiterjoin
\definecolor{dialinecolor}{rgb}{0.000000, 0.000000, 0.000000}
\pgfsetstrokecolor{dialinecolor}
\pgfpathellipse{\pgfpoint{21.712850\du}{15.353540\du}}{\pgfpoint{0.537500\du}{0\du}}{\pgfpoint{0\du}{0.437500\du}}
\pgfusepath{stroke}
% setfont left to latex
\definecolor{dialinecolor}{rgb}{0.000000, 0.000000, 0.000000}
\pgfsetstrokecolor{dialinecolor}
\node at (21.712850\du,15.474096\du){\RL{هذا}};
\definecolor{dialinecolor}{rgb}{1.000000, 1.000000, 1.000000}
\pgfsetfillcolor{dialinecolor}
\pgfpathellipse{\pgfpoint{22.926600\du}{15.353543\du}}{\pgfpoint{0.573750\du}{0\du}}{\pgfpoint{0\du}{0.469243\du}}
\pgfusepath{fill}
\pgfsetlinewidth{0.020000\du}
\pgfsetdash{}{0pt}
\pgfsetdash{}{0pt}
\pgfsetmiterjoin
\definecolor{dialinecolor}{rgb}{0.000000, 0.000000, 0.000000}
\pgfsetstrokecolor{dialinecolor}
\pgfpathellipse{\pgfpoint{22.926600\du}{15.353543\du}}{\pgfpoint{0.573750\du}{0\du}}{\pgfpoint{0\du}{0.469243\du}}
\pgfusepath{stroke}
% setfont left to latex
\definecolor{dialinecolor}{rgb}{0.000000, 0.000000, 0.000000}
\pgfsetstrokecolor{dialinecolor}
\node at (22.926600\du,15.474098\du){\RL{من}};
\definecolor{dialinecolor}{rgb}{1.000000, 1.000000, 1.000000}
\pgfsetfillcolor{dialinecolor}
\pgfpathellipse{\pgfpoint{27.589711\du}{17.121042\du}}{\pgfpoint{1.061861\du}{0\du}}{\pgfpoint{0\du}{0.578542\du}}
\pgfusepath{fill}
\pgfsetlinewidth{0.020000\du}
\pgfsetdash{}{0pt}
\pgfsetdash{}{0pt}
\pgfsetmiterjoin
\definecolor{dialinecolor}{rgb}{0.000000, 0.000000, 0.000000}
\pgfsetstrokecolor{dialinecolor}
\pgfpathellipse{\pgfpoint{27.589711\du}{17.121042\du}}{\pgfpoint{1.061861\du}{0\du}}{\pgfpoint{0\du}{0.578542\du}}
\pgfusepath{stroke}
% setfont left to latex
\definecolor{dialinecolor}{rgb}{0.000000, 0.000000, 0.000000}
\pgfsetstrokecolor{dialinecolor}
\node at (27.589711\du,17.241597\du){p5=\RL{دبي}};
\definecolor{dialinecolor}{rgb}{1.000000, 1.000000, 1.000000}
\pgfsetfillcolor{dialinecolor}
\pgfpathellipse{\pgfpoint{25.317211\du}{17.121042\du}}{\pgfpoint{1.061861\du}{0\du}}{\pgfpoint{0\du}{0.578542\du}}
\pgfusepath{fill}
\pgfsetlinewidth{0.020000\du}
\pgfsetdash{}{0pt}
\pgfsetdash{}{0pt}
\pgfsetmiterjoin
\definecolor{dialinecolor}{rgb}{0.000000, 0.000000, 0.000000}
\pgfsetstrokecolor{dialinecolor}
\pgfpathellipse{\pgfpoint{25.317211\du}{17.121042\du}}{\pgfpoint{1.061861\du}{0\du}}{\pgfpoint{0\du}{0.578542\du}}
\pgfusepath{stroke}
% setfont left to latex
\definecolor{dialinecolor}{rgb}{0.000000, 0.000000, 0.000000}
\pgfsetstrokecolor{dialinecolor}
\node at (25.317211\du,17.241597\du){p6=\RL{مول}};
\definecolor{dialinecolor}{rgb}{1.000000, 1.000000, 1.000000}
\pgfsetfillcolor{dialinecolor}
\pgfpathellipse{\pgfpoint{24.916600\du}{15.353542\du}}{\pgfpoint{0.658750\du}{0\du}}{\pgfpoint{0\du}{0.578542\du}}
\pgfusepath{fill}
\pgfsetlinewidth{0.020000\du}
\pgfsetdash{}{0pt}
\pgfsetdash{}{0pt}
\pgfsetmiterjoin
\definecolor{dialinecolor}{rgb}{0.000000, 0.000000, 0.000000}
\pgfsetstrokecolor{dialinecolor}
\pgfpathellipse{\pgfpoint{24.916600\du}{15.353542\du}}{\pgfpoint{0.658750\du}{0\du}}{\pgfpoint{0\du}{0.578542\du}}
\pgfusepath{stroke}
% setfont left to latex
\definecolor{dialinecolor}{rgb}{0.000000, 0.000000, 0.000000}
\pgfsetstrokecolor{dialinecolor}
\node at (24.916600\du,15.474097\du){\RL{على}};
\definecolor{dialinecolor}{rgb}{1.000000, 1.000000, 1.000000}
\pgfsetfillcolor{dialinecolor}
\fill (19.827850\du,13.430230\du)--(19.827850\du,14.380230\du)--(20.752850\du,14.380230\du)--(20.752850\du,13.430230\du)--cycle;
\pgfsetlinewidth{0.020000\du}
\pgfsetdash{}{0pt}
\pgfsetdash{}{0pt}
\pgfsetmiterjoin
\definecolor{dialinecolor}{rgb}{0.000000, 0.000000, 0.000000}
\pgfsetstrokecolor{dialinecolor}
\draw (19.827850\du,13.430230\du)--(19.827850\du,14.380230\du)--(20.752850\du,14.380230\du)--(20.752850\du,13.430230\du)--cycle;
% setfont left to latex
\definecolor{dialinecolor}{rgb}{0.000000, 0.000000, 0.000000}
\pgfsetstrokecolor{dialinecolor}
\node at (20.290350\du,14.025786\du){$+$};
\definecolor{dialinecolor}{rgb}{1.000000, 1.000000, 1.000000}
\pgfsetfillcolor{dialinecolor}
\fill (19.852850\du,14.891040\du)--(19.852850\du,15.816040\du)--(20.802850\du,15.816040\du)--(20.802850\du,14.891040\du)--cycle;
\pgfsetlinewidth{0.020000\du}
\pgfsetdash{}{0pt}
\pgfsetdash{}{0pt}
\pgfsetmiterjoin
\definecolor{dialinecolor}{rgb}{0.000000, 0.000000, 0.000000}
\pgfsetstrokecolor{dialinecolor}
\draw (19.852850\du,14.891040\du)--(19.852850\du,15.816040\du)--(20.802850\du,15.816040\du)--(20.802850\du,14.891040\du)--cycle;
% setfont left to latex
\definecolor{dialinecolor}{rgb}{0.000000, 0.000000, 0.000000}
\pgfsetstrokecolor{dialinecolor}
\node at (20.327850\du,15.474096\du){$|$};
\definecolor{dialinecolor}{rgb}{1.000000, 1.000000, 1.000000}
\pgfsetfillcolor{dialinecolor}
\fill (24.475350\du,13.465230\du)--(24.475350\du,14.380230\du)--(25.486750\du,14.380230\du)--(25.486750\du,13.465230\du)--cycle;
\pgfsetlinewidth{0.020000\du}
\pgfsetdash{}{0pt}
\pgfsetdash{}{0pt}
\pgfsetmiterjoin
\definecolor{dialinecolor}{rgb}{0.000000, 0.000000, 0.000000}
\pgfsetstrokecolor{dialinecolor}
\draw (24.475350\du,13.465230\du)--(24.475350\du,14.380230\du)--(25.486750\du,14.380230\du)--(25.486750\du,13.465230\du)--cycle;
% setfont left to latex
\definecolor{dialinecolor}{rgb}{0.000000, 0.000000, 0.000000}
\pgfsetstrokecolor{dialinecolor}
\node at (24.981050\du,14.043286\du){?};
\pgfsetlinewidth{0.020000\du}
\pgfsetdash{}{0pt}
\pgfsetdash{}{0pt}
\pgfsetbuttcap
{
\definecolor{dialinecolor}{rgb}{0.000000, 0.000000, 0.000000}
\pgfsetfillcolor{dialinecolor}
% was here!!!
\pgfsetarrowsend{latex}
\definecolor{dialinecolor}{rgb}{0.000000, 0.000000, 0.000000}
\pgfsetstrokecolor{dialinecolor}
\draw (24.459070\du,12.260050\du)--(23.260613\du,16.541370\du);
}
\pgfsetlinewidth{0.020000\du}
\pgfsetdash{}{0pt}
\pgfsetdash{}{0pt}
\pgfsetbuttcap
{
\definecolor{dialinecolor}{rgb}{0.000000, 0.000000, 0.000000}
\pgfsetfillcolor{dialinecolor}
% was here!!!
\pgfsetarrowsend{latex}
\definecolor{dialinecolor}{rgb}{0.000000, 0.000000, 0.000000}
\pgfsetstrokecolor{dialinecolor}
\draw (24.090007\du,12.043880\du)--(20.763012\du,13.673686\du);
}
\pgfsetlinewidth{0.020000\du}
\pgfsetdash{}{0pt}
\pgfsetdash{}{0pt}
\pgfsetbuttcap
{
\definecolor{dialinecolor}{rgb}{0.000000, 0.000000, 0.000000}
\pgfsetfillcolor{dialinecolor}
% was here!!!
\pgfsetarrowsend{latex}
\definecolor{dialinecolor}{rgb}{0.000000, 0.000000, 0.000000}
\pgfsetstrokecolor{dialinecolor}
\draw (20.302893\du,14.389650\du)--(20.315609\du,14.880788\du);
}
\pgfsetlinewidth{0.020000\du}
\pgfsetdash{}{0pt}
\pgfsetdash{}{0pt}
\pgfsetbuttcap
{
\definecolor{dialinecolor}{rgb}{0.000000, 0.000000, 0.000000}
\pgfsetfillcolor{dialinecolor}
% was here!!!
\pgfsetarrowsend{latex}
\definecolor{dialinecolor}{rgb}{0.000000, 0.000000, 0.000000}
\pgfsetstrokecolor{dialinecolor}
\draw (20.333667\du,15.824804\du)--(20.342397\du,16.532143\du);
}
\pgfsetlinewidth{0.020000\du}
\pgfsetdash{}{0pt}
\pgfsetdash{}{0pt}
\pgfsetbuttcap
{
\definecolor{dialinecolor}{rgb}{0.000000, 0.000000, 0.000000}
\pgfsetfillcolor{dialinecolor}
% was here!!!
\pgfsetarrowsend{latex}
\definecolor{dialinecolor}{rgb}{0.000000, 0.000000, 0.000000}
\pgfsetstrokecolor{dialinecolor}
\draw (24.672711\du,12.258128\du)--(24.894654\du,13.456310\du);
}
\pgfsetlinewidth{0.020000\du}
\pgfsetdash{}{0pt}
\pgfsetdash{}{0pt}
\pgfsetbuttcap
{
\definecolor{dialinecolor}{rgb}{0.000000, 0.000000, 0.000000}
\pgfsetfillcolor{dialinecolor}
% was here!!!
\pgfsetarrowsend{latex}
\definecolor{dialinecolor}{rgb}{0.000000, 0.000000, 0.000000}
\pgfsetstrokecolor{dialinecolor}
\draw (24.959997\du,14.390119\du)--(24.943066\du,14.765987\du);
}
\pgfsetlinewidth{0.020000\du}
\pgfsetdash{}{0pt}
\pgfsetdash{}{0pt}
\pgfsetbuttcap
{
\definecolor{dialinecolor}{rgb}{0.000000, 0.000000, 0.000000}
\pgfsetfillcolor{dialinecolor}
% was here!!!
\pgfsetarrowsend{latex}
\definecolor{dialinecolor}{rgb}{0.000000, 0.000000, 0.000000}
\pgfsetstrokecolor{dialinecolor}
\draw (24.103215\du,12.258128\du)--(22.818712\du,13.472375\du);
}
\pgfsetlinewidth{0.020000\du}
\pgfsetdash{}{0pt}
\pgfsetdash{}{0pt}
\pgfsetbuttcap
{
\definecolor{dialinecolor}{rgb}{0.000000, 0.000000, 0.000000}
\pgfsetfillcolor{dialinecolor}
% was here!!!
\pgfsetarrowsend{latex}
\definecolor{dialinecolor}{rgb}{0.000000, 0.000000, 0.000000}
\pgfsetstrokecolor{dialinecolor}
\draw (22.136684\du,14.390119\du)--(21.898028\du,14.932611\du);
}
\pgfsetlinewidth{0.020000\du}
\pgfsetdash{}{0pt}
\pgfsetdash{}{0pt}
\pgfsetbuttcap
{
\definecolor{dialinecolor}{rgb}{0.000000, 0.000000, 0.000000}
\pgfsetfillcolor{dialinecolor}
% was here!!!
\pgfsetarrowsend{latex}
\definecolor{dialinecolor}{rgb}{0.000000, 0.000000, 0.000000}
\pgfsetstrokecolor{dialinecolor}
\draw (22.533168\du,14.390120\du)--(22.741011\du,14.899078\du);
}
\definecolor{dialinecolor}{rgb}{1.000000, 1.000000, 1.000000}
\pgfsetfillcolor{dialinecolor}
\fill (27.375350\du,14.891040\du)--(27.375350\du,15.816040\du)--(28.325350\du,15.816040\du)--(28.325350\du,14.891040\du)--cycle;
\pgfsetlinewidth{0.020000\du}
\pgfsetdash{}{0pt}
\pgfsetdash{}{0pt}
\pgfsetmiterjoin
\definecolor{dialinecolor}{rgb}{0.000000, 0.000000, 0.000000}
\pgfsetstrokecolor{dialinecolor}
\draw (27.375350\du,14.891040\du)--(27.375350\du,15.816040\du)--(28.325350\du,15.816040\du)--(28.325350\du,14.891040\du)--cycle;
% setfont left to latex
\definecolor{dialinecolor}{rgb}{0.000000, 0.000000, 0.000000}
\pgfsetstrokecolor{dialinecolor}
\node at (27.850350\du,15.474096\du){$|$};
\definecolor{dialinecolor}{rgb}{1.000000, 1.000000, 1.000000}
\pgfsetfillcolor{dialinecolor}
\fill (26.825350\du,13.430230\du)--(26.825350\du,14.380230\du)--(27.750350\du,14.380230\du)--(27.750350\du,13.430230\du)--cycle;
\pgfsetlinewidth{0.020000\du}
\pgfsetdash{}{0pt}
\pgfsetdash{}{0pt}
\pgfsetmiterjoin
\definecolor{dialinecolor}{rgb}{0.000000, 0.000000, 0.000000}
\pgfsetstrokecolor{dialinecolor}
\draw (26.825350\du,13.430230\du)--(26.825350\du,14.380230\du)--(27.750350\du,14.380230\du)--(27.750350\du,13.430230\du)--cycle;
% setfont left to latex
\definecolor{dialinecolor}{rgb}{0.000000, 0.000000, 0.000000}
\pgfsetstrokecolor{dialinecolor}
\node at (27.287850\du,14.025786\du){$+$};
\definecolor{dialinecolor}{rgb}{1.000000, 1.000000, 1.000000}
\pgfsetfillcolor{dialinecolor}
\fill (25.727850\du,14.891040\du)--(25.727850\du,15.816040\du)--(26.677850\du,15.816040\du)--(26.677850\du,14.891040\du)--cycle;
\pgfsetlinewidth{0.020000\du}
\pgfsetdash{}{0pt}
\pgfsetdash{}{0pt}
\pgfsetmiterjoin
\definecolor{dialinecolor}{rgb}{0.000000, 0.000000, 0.000000}
\pgfsetstrokecolor{dialinecolor}
\draw (25.727850\du,14.891040\du)--(25.727850\du,15.816040\du)--(26.677850\du,15.816040\du)--(26.677850\du,14.891040\du)--cycle;
% setfont left to latex
\definecolor{dialinecolor}{rgb}{0.000000, 0.000000, 0.000000}
\pgfsetstrokecolor{dialinecolor}
\node at (26.202850\du,15.474096\du){$|$};
\pgfsetlinewidth{0.020000\du}
\pgfsetdash{}{0pt}
\pgfsetdash{}{0pt}
\pgfsetbuttcap
{
\definecolor{dialinecolor}{rgb}{0.000000, 0.000000, 0.000000}
\pgfsetfillcolor{dialinecolor}
% was here!!!
\pgfsetarrowsend{latex}
\definecolor{dialinecolor}{rgb}{0.000000, 0.000000, 0.000000}
\pgfsetstrokecolor{dialinecolor}
\draw (27.475991\du,14.389650\du)--(27.666741\du,14.880788\du);
}
\pgfsetlinewidth{0.020000\du}
\pgfsetdash{}{0pt}
\pgfsetdash{}{0pt}
\pgfsetbuttcap
{
\definecolor{dialinecolor}{rgb}{0.000000, 0.000000, 0.000000}
\pgfsetfillcolor{dialinecolor}
% was here!!!
\pgfsetarrowsend{latex}
\definecolor{dialinecolor}{rgb}{0.000000, 0.000000, 0.000000}
\pgfsetstrokecolor{dialinecolor}
\draw (26.924947\du,14.389650\du)--(26.557011\du,14.880788\du);
}
\pgfsetlinewidth{0.020000\du}
\pgfsetdash{}{0pt}
\pgfsetdash{}{0pt}
\pgfsetbuttcap
{
\definecolor{dialinecolor}{rgb}{0.000000, 0.000000, 0.000000}
\pgfsetfillcolor{dialinecolor}
% was here!!!
\pgfsetarrowsend{latex}
\definecolor{dialinecolor}{rgb}{0.000000, 0.000000, 0.000000}
\pgfsetstrokecolor{dialinecolor}
\draw (25.084871\du,12.187535\du)--(26.815218\du,13.536712\du);
}
\pgfsetlinewidth{0.020000\du}
\pgfsetdash{}{0pt}
\pgfsetdash{}{0pt}
\pgfsetbuttcap
{
\definecolor{dialinecolor}{rgb}{0.000000, 0.000000, 0.000000}
\pgfsetfillcolor{dialinecolor}
% was here!!!
\pgfsetarrowsend{latex}
\definecolor{dialinecolor}{rgb}{0.000000, 0.000000, 0.000000}
\pgfsetstrokecolor{dialinecolor}
\draw (27.780863\du,15.824759\du)--(27.675742\du,16.537628\du);
}
\pgfsetlinewidth{0.020000\du}
\pgfsetdash{}{0pt}
\pgfsetdash{}{0pt}
\pgfsetbuttcap
{
\definecolor{dialinecolor}{rgb}{0.000000, 0.000000, 0.000000}
\pgfsetfillcolor{dialinecolor}
% was here!!!
\pgfsetarrowsend{latex}
\definecolor{dialinecolor}{rgb}{0.000000, 0.000000, 0.000000}
\pgfsetstrokecolor{dialinecolor}
\draw (25.966737\du,15.824759\du)--(25.601325\du,16.554026\du);
}
% setfont left to latex
\definecolor{dialinecolor}{rgb}{0.000000, 0.000000, 0.000000}
\pgfsetstrokecolor{dialinecolor}
\node[anchor=west] at (21.807850\du,11.829310\du){match 2};
% setfont left to latex
\definecolor{dialinecolor}{rgb}{0.000000, 0.000000, 0.000000}
\pgfsetstrokecolor{dialinecolor}
\node[anchor=west] at (19.550350\du,17.274600\du){};
\end{tikzpicture}
 \\
\end{tabular}
}
  \caption{\label{fig:taskMRE}Matches of regular expression $(P|N)\!+~O?~R~O$\^{}$2~(P|N|U)+$}
\end{figure}
\transtrue
\setcode{standard}

The match trees in Figure~\ref{fig:taskMRE} illustrate two matches of the expression computed by \framework.
The first match tree refers to the text 
\RL{brj _hlyfT bAlqrb mn AltqA.t` Al-'wl}(Khalifa Tower next to the first intersection).
The second match tree refers to the text 
\RL{dby mwl `l_A mqrbT mn h_dA Almbn_A}(Dubai Mall is located near this building).
%The leaf nodes of the trees are entities and the edges and internal nodes are text, 
%morphology-based, 
%and word distance based relational entities. 
The leaf nodes of the match trees are matches to formulae and the internal nodes represent roots to sub-expression matches.
For instance, \RL{brj _hlyfT} in match 1 tree corresponds to the sub-expression $(P|N)+$.

\subsection{User-defined relations and actions}

A relation $R$ is defined by the user as a tuple 
$\langle e_1,e_2,r\rangle$ where 
$e_1,e_2,$ and $r$ are identifiers associated with 
sub-expressions of an expression $f$.
Matches of $R$ are a set of labeled binary edges
where matches of $e_1$ and $e_2$ are the source and destination nodes
and matches of $r$ are the edge labels.
We denote $\ldbrack \langle e_1,e_2,r\rangle \rdbrack$ to be the set of matches of $R$, and refer to them as user defined relational entities.

We are interested in constructing the relational entity graph in Figure~\ref{fig:intromotiv}.
Let $e_1,o_1,r,o_2,$ and $e_2$ be identifiers to the sub-expressions 
$(P|N)+$,$O?$,$R$,$O\wedge 2$, and $(P|N|U)+$, respectively.
We select $(P|N)+$ to denote an entity as it captures a composite place name such as \cci{Khalifa tower}.
The matches to $e_1$, $r$, $o_2$, and $e_2$ in match 1 (Fig.~\ref{fig:taskMRE}) are \RL{brj _hlyfT} (Khalifa Tower), \RL{bAlqrb} (next), \RL{mn} (to), and \RL{AltqA.t` Al-'wl} (first intersection).
Note that there is no match to the optional $O$ formula in match 1.
Similarly, the matches to $e_1$, $o_1$, $r$, $o_2$, and $e_2$ in the second match tree are
\RL{dby mwl} (Dubai Mall), \RL{`l_A} (is located), \RL{mqrbT} (near), \RL{mn h_dA} (this), and \RL{Almbn_A} (building), respectively.

\cci{Relation($e_1$,$e_2$,$r$)} creates the 
edge labeled \cci{next to} between \cci{Khalifa tower} and 
\cci{intersection 1} nodes from match 1, and the 
edge labeled \cci{near} between \cci{Dubai Mall} and \cci{the building} nodes from match 2.
\cci{Relation($r$,$e_1$,$o_1$)} creates the edge labeled \cci{prep} 
between \cci{Dubai Mall} and \cci{near} nodes from match 2. 
\cci{Relation($r$,$e_2$,$o_2$)} creates the edge labeled \cci{from} 
between \cci{intersection 1} and \cci{next to} nodes in 
match 1, and the 
edge labeled \cci{from this} between \cci{near} and \cci{the building} 
nodes in match 2.

After the relation construction phase, we are interested to relate the discovered entities
% and relational entities 
that express the same concept.
\framework provides the extended synonym feature of second order as a default cross-reference relation ($Syn^2$).
%\framework provides the \cci{isA} predicate as a default cross-reference relation.
In Figure~\ref{fig:intromotiv}, triggering this feature creates the edge between the nodes \cci{Khalifa Tower} and \cci{The building}.
The edge is labeled with \cci{isA}.

Moreover, \framework allows advanced users to write C++ code snippets 
to process matches of sub-expressions.
Each sub-expression can be associated with two computational actions: \cci{pre-match} and \cci{on-match}.
Users can use these actions to compute statistical features, 
store intermediate results, 
or apply intelligent entity inference techniques
as we show later in the numerical extraction example of Section~\ref{sec:sec:number}.
\framework provides an API that enriches the actions with detailed access to
all solution features of an expression or a formula match including 
text, position, length, equivalent numerical value when applicable, 
and morphological features.
Once \framework computes all match trees, it traverses each tree to 
execute the user defined \cci{pre-match} actions in pre-order manner
and the \cci{on-match} actions in post-order manner.

\begin{comment}
\subsection{\framework simulator}

The set of tag types ${\cal T}$ contains tuples of the form $\langle l,f,d\rangle$ 
where $l$ is a text label with a descriptive name, 
$f$ is an MRE, and $d$ is a visualization legend 
with font and color information.

For each word $t_i\in T, 0\le i < M, M=|T|$.
\framework computes a Boolean value 
($\{\mathit{true}, \mathit{false}\}$)
for all MBFs. 
Then, it computes the set of MBF tags
$R_i=\{(t_i,tt)| tt=\langle l,f,d\rangle \wedge
f~\mathit{is~an~MBF} \wedge f(t_i)\} \subseteq T \times {\cal T}$
which tags a word $t_i$ with $\mathit{tt}$ 
iff the MBF $f$ associated with
tag type $\mathit{tt}$ is true for $t_i$. 
The MBF evaluation results in a sequence of tag sets 
$\langle R_0, R_1, \ldots, R_{n-1}\rangle$.
If a word $t_i$ has no tag type match, 
its tag set $R_i$ is by default the singleton $O=\{\mathit{NONE}\}$.
%and $t_i$ is referred to as a {\em null} word.

%\subsection{MRE and action simulation}

For each MRE, 
\framework generates its equivalent non-deterministic finite automaton (NFA) in the typical manner~\cite{sipser2006introduction}. 
%Each MRE operation has its equivalent representation in an NFA. 
We support the upto operation ($f$\^{}$x$), which is not directly 
supported in ~\cite{sipser2006introduction}, by 
%it is equivalent to $f?|ff|\dots|\underbrace{f}_{x \text{ times}}$ which has an NFA mapping. 
%we can expand it into a standard regular expression form, for example
expanding it into a regular expression form; for example 
$f$\^{}$3$ is equivalent to $f?|ff|fff$. 
Consider the example of Figure~\ref{fig:intromotiv} and the
corresponding expression $(P|N)\!+~O?~R~O^\wedge 2~(P|N|U)+$. 
Figure~\ref{fig:nfaEx} shows part of the corresponding NFA where
$q_8, q_9, \dots, q_{13}$ represent NFA states,
and edges are transitions based on MBF tags such as 
$P,$ and $N$.
Edges labeled with the empty string $\epsilon$ are non-deterministic.
%The expression uses operations $|$, $?$, $+$, and $\wedge$ to relate places {\tt P}, names of persons {\tt N}, relative positions {\tt R}, numerical terms {\tt U}, and other {\tt O}.

\framework simulates the generated NFA over the sequence of tag sets matching the MBF formulae.
A simulation match $m$ of an expression $f$ is a tree where the root is the 
MRE expression, the internal nodes are the MRE and MBF operations, and the 
leaves are matches of the MAT terms of $f$.
The leave matches form a vector of tags
$\langle r_k,r_{k+1},\dots,r_j\rangle$ 
corresponding to the text sequence 
$\langle t_k,t_{k+1},\dots,t_j\rangle$ where $r_{\ell}\in R_{\ell},0\le k\le \ell \le j < n$.
%
%If the simulation has a match $\langle r_m,r_{m+1},\dots,r_n\rangle$ 
%where $0\le m\le n$, 
%the next simulation starts at $R_{n+1}$. 
%This disallows overlap of matches for the same MRE. 
%
%In case the NFA simulation has no match, 
%the next simulation starts at $R_{m+1}$. 
If we have more than one match %starting at $R_k$ where $0\le k\le n$, 
\framework returns the longest. 
%
Figure~\ref{fig:motiv}(b) shows two match trees of $\mathit{dir}$ extracted 
from the text of Figure~\ref{fig:intromotiv}.
\RL{dby} and \RL{mwl} are leaf nodes referring to name of place tags ($P$). 
The $+$, $|$, and $?$ MRE operations
are internal nodes.

%\framework maintains a function $\phi\subset Q\times\Phi$, 
%where $\mathit{Q}$ is the set of states in the NFA and $\Phi$ is the set of sub-expressions in the MRE.
%So $(q,f)\in \phi$ iff state $q\in Q$ was generated by \framework to correspond to sub-expression $f\in \Phi$. 
%\framework uses $\phi$ to compute a match tree with respect to the MRE regular expression. 
%It also uses $\phi$ and the match sequence to compute the sequence of computational actions of an MRE match. 

\setcode{utf8}
\setarab
\transfalse

\begin{figure}[tb!]
\centering
\resizebox{0.7\columnwidth}{!}{
	\relsize{-1} % Graphic for TeX using PGF
% Title: /home/ameen/Desktop/Thesis Figures/reNFA.dia
% Creator: Dia v0.97.1
% CreationDate: Tue Mar 18 10:06:56 2014
% For: ameen
% \usepackage{tikz}
% The following commands are not supported in PSTricks at present
% We define them conditionally, so when they are implemented,
% this pgf file will use them.
\ifx\du\undefined
  \newlength{\du}
\fi
\setlength{\du}{15\unitlength}
\begin{tikzpicture}
\pgftransformxscale{1.000000}
\pgftransformyscale{-1.000000}
\definecolor{dialinecolor}{rgb}{0.000000, 0.000000, 0.000000}
\pgfsetstrokecolor{dialinecolor}
\definecolor{dialinecolor}{rgb}{1.000000, 1.000000, 1.000000}
\pgfsetfillcolor{dialinecolor}
\definecolor{dialinecolor}{rgb}{1.000000, 1.000000, 1.000000}
\pgfsetfillcolor{dialinecolor}
\pgfpathellipse{\pgfpoint{19.124361\du}{14.988512\du}}{\pgfpoint{0.796261\du}{0\du}}{\pgfpoint{0\du}{0.513512\du}}
\pgfusepath{fill}
\pgfsetlinewidth{0.040000\du}
\pgfsetdash{}{0pt}
\pgfsetdash{}{0pt}
\pgfsetmiterjoin
\definecolor{dialinecolor}{rgb}{0.000000, 0.000000, 0.000000}
\pgfsetstrokecolor{dialinecolor}
\pgfpathellipse{\pgfpoint{19.124361\du}{14.988512\du}}{\pgfpoint{0.796261\du}{0\du}}{\pgfpoint{0\du}{0.513512\du}}
\pgfusepath{stroke}
% setfont left to latex
\definecolor{dialinecolor}{rgb}{0.000000, 0.000000, 0.000000}
\pgfsetstrokecolor{dialinecolor}
\node at (19.124361\du,15.091845\du){$q_{8}$};
\definecolor{dialinecolor}{rgb}{1.000000, 1.000000, 1.000000}
\pgfsetfillcolor{dialinecolor}
\pgfpathellipse{\pgfpoint{28.389361\du}{14.983512\du}}{\pgfpoint{0.796261\du}{0\du}}{\pgfpoint{0\du}{0.513512\du}}
\pgfusepath{fill}
\pgfsetlinewidth{0.040000\du}
\pgfsetdash{}{0pt}
\pgfsetdash{}{0pt}
\pgfsetmiterjoin
\definecolor{dialinecolor}{rgb}{0.000000, 0.000000, 0.000000}
\pgfsetstrokecolor{dialinecolor}
\pgfpathellipse{\pgfpoint{28.389361\du}{14.983512\du}}{\pgfpoint{0.796261\du}{0\du}}{\pgfpoint{0\du}{0.513512\du}}
\pgfusepath{stroke}
% setfont left to latex
\definecolor{dialinecolor}{rgb}{0.000000, 0.000000, 0.000000}
\pgfsetstrokecolor{dialinecolor}
\node at (28.389361\du,15.086845\du){$q_{13}$};
\definecolor{dialinecolor}{rgb}{1.000000, 1.000000, 1.000000}
\pgfsetfillcolor{dialinecolor}
\pgfpathellipse{\pgfpoint{23.716861\du}{13.933512\du}}{\pgfpoint{0.796261\du}{0\du}}{\pgfpoint{0\du}{0.513512\du}}
\pgfusepath{fill}
\pgfsetlinewidth{0.040000\du}
\pgfsetdash{}{0pt}
\pgfsetdash{}{0pt}
\pgfsetmiterjoin
\definecolor{dialinecolor}{rgb}{0.000000, 0.000000, 0.000000}
\pgfsetstrokecolor{dialinecolor}
\pgfpathellipse{\pgfpoint{23.716861\du}{13.933512\du}}{\pgfpoint{0.796261\du}{0\du}}{\pgfpoint{0\du}{0.513512\du}}
\pgfusepath{stroke}
% setfont left to latex
\definecolor{dialinecolor}{rgb}{0.000000, 0.000000, 0.000000}
\pgfsetstrokecolor{dialinecolor}
\node at (23.716861\du,14.036845\du){$q_{10}$};
\definecolor{dialinecolor}{rgb}{1.000000, 1.000000, 1.000000}
\pgfsetfillcolor{dialinecolor}
\pgfpathellipse{\pgfpoint{23.744361\du}{16.033512\du}}{\pgfpoint{0.796261\du}{0\du}}{\pgfpoint{0\du}{0.513512\du}}
\pgfusepath{fill}
\pgfsetlinewidth{0.040000\du}
\pgfsetdash{}{0pt}
\pgfsetdash{}{0pt}
\pgfsetmiterjoin
\definecolor{dialinecolor}{rgb}{0.000000, 0.000000, 0.000000}
\pgfsetstrokecolor{dialinecolor}
\pgfpathellipse{\pgfpoint{23.744361\du}{16.033512\du}}{\pgfpoint{0.796261\du}{0\du}}{\pgfpoint{0\du}{0.513512\du}}
\pgfusepath{stroke}
% setfont left to latex
\definecolor{dialinecolor}{rgb}{0.000000, 0.000000, 0.000000}
\pgfsetstrokecolor{dialinecolor}
\node at (23.744361\du,16.136845\du){$q_{11}$};
\definecolor{dialinecolor}{rgb}{1.000000, 1.000000, 1.000000}
\pgfsetfillcolor{dialinecolor}
\pgfpathellipse{\pgfpoint{21.321861\du}{15.008512\du}}{\pgfpoint{0.796261\du}{0\du}}{\pgfpoint{0\du}{0.513512\du}}
\pgfusepath{fill}
\pgfsetlinewidth{0.040000\du}
\pgfsetdash{}{0pt}
\pgfsetdash{}{0pt}
\pgfsetmiterjoin
\definecolor{dialinecolor}{rgb}{0.000000, 0.000000, 0.000000}
\pgfsetstrokecolor{dialinecolor}
\pgfpathellipse{\pgfpoint{21.321861\du}{15.008512\du}}{\pgfpoint{0.796261\du}{0\du}}{\pgfpoint{0\du}{0.513512\du}}
\pgfusepath{stroke}
% setfont left to latex
\definecolor{dialinecolor}{rgb}{0.000000, 0.000000, 0.000000}
\pgfsetstrokecolor{dialinecolor}
\node at (21.321861\du,15.111845\du){$q_{9}$};
\definecolor{dialinecolor}{rgb}{1.000000, 1.000000, 1.000000}
\pgfsetfillcolor{dialinecolor}
\pgfpathellipse{\pgfpoint{26.174361\du}{15.008512\du}}{\pgfpoint{0.796261\du}{0\du}}{\pgfpoint{0\du}{0.513512\du}}
\pgfusepath{fill}
\pgfsetlinewidth{0.040000\du}
\pgfsetdash{}{0pt}
\pgfsetdash{}{0pt}
\pgfsetmiterjoin
\definecolor{dialinecolor}{rgb}{0.000000, 0.000000, 0.000000}
\pgfsetstrokecolor{dialinecolor}
\pgfpathellipse{\pgfpoint{26.174361\du}{15.008512\du}}{\pgfpoint{0.796261\du}{0\du}}{\pgfpoint{0\du}{0.513512\du}}
\pgfusepath{stroke}
% setfont left to latex
\definecolor{dialinecolor}{rgb}{0.000000, 0.000000, 0.000000}
\pgfsetstrokecolor{dialinecolor}
\node at (26.174361\du,15.111845\du){$q_{12}$};
% setfont left to latex
\definecolor{dialinecolor}{rgb}{0.000000, 0.000000, 0.000000}
\pgfsetstrokecolor{dialinecolor}
\node[anchor=west] at (24.875600\du,14.115000\du){$\epsilon$};
% setfont left to latex
\definecolor{dialinecolor}{rgb}{0.000000, 0.000000, 0.000000}
\pgfsetstrokecolor{dialinecolor}
\node[anchor=west] at (23.731250\du,12.975000\du){$\epsilon$};
% setfont left to latex
\definecolor{dialinecolor}{rgb}{0.000000, 0.000000, 0.000000}
\pgfsetstrokecolor{dialinecolor}
\node[anchor=west] at (20.131250\du,14.750000\du){$\epsilon$};
% setfont left to latex
\definecolor{dialinecolor}{rgb}{0.000000, 0.000000, 0.000000}
\pgfsetstrokecolor{dialinecolor}
\node[anchor=west] at (24.906250\du,15.950000\du){$\epsilon$};
% setfont left to latex
\definecolor{dialinecolor}{rgb}{0.000000, 0.000000, 0.000000}
\pgfsetstrokecolor{dialinecolor}
\node[anchor=west] at (27.163100\du,14.690000\du){$\epsilon$};
% setfont left to latex
\definecolor{dialinecolor}{rgb}{0.000000, 0.000000, 0.000000}
\pgfsetstrokecolor{dialinecolor}
\node[anchor=west] at (23.745600\du,17.200000\du){$\epsilon$};
% setfont left to latex
\definecolor{dialinecolor}{rgb}{0.000000, 0.000000, 0.000000}
\pgfsetstrokecolor{dialinecolor}
\node[anchor=west] at (22\du,14.175000\du){P};
% setfont left to latex
\definecolor{dialinecolor}{rgb}{0.000000, 0.000000, 0.000000}
\pgfsetstrokecolor{dialinecolor}
\node[anchor=west] at (22\du,15.890000\du){N};
% setfont left to latex
\definecolor{dialinecolor}{rgb}{0.000000, 0.000000, 0.000000}
\pgfsetstrokecolor{dialinecolor}
\node[anchor=west] at (16.743800\du,15.025000\du){\dots};
% setfont left to latex
\definecolor{dialinecolor}{rgb}{0.000000, 0.000000, 0.000000}
\pgfsetstrokecolor{dialinecolor}
\node[anchor=west] at (29.846300\du,15.007500\du){\dots};
\pgfsetlinewidth{0.020000\du}
\pgfsetdash{}{0pt}
\pgfsetdash{}{0pt}
\pgfsetbuttcap
{
\definecolor{dialinecolor}{rgb}{0.000000, 0.000000, 0.000000}
\pgfsetfillcolor{dialinecolor}
% was here!!!
\pgfsetarrowsend{latex}
\definecolor{dialinecolor}{rgb}{0.000000, 0.000000, 0.000000}
\pgfsetstrokecolor{dialinecolor}
\draw (19.920623\du,14.988512\du)--(20.505612\du,14.996861\du);
}
\pgfsetlinewidth{0.020000\du}
\pgfsetdash{}{0pt}
\pgfsetdash{}{0pt}
\pgfsetbuttcap
{
\definecolor{dialinecolor}{rgb}{0.000000, 0.000000, 0.000000}
\pgfsetfillcolor{dialinecolor}
% was here!!!
\pgfsetarrowsend{latex}
\definecolor{dialinecolor}{rgb}{0.000000, 0.000000, 0.000000}
\pgfsetstrokecolor{dialinecolor}
\draw (26.970623\du,15.008512\du)--(27.593100\du,14.983512\du);
}
\pgfsetlinewidth{0.020000\du}
\pgfsetdash{}{0pt}
\pgfsetdash{}{0pt}
\pgfsetbuttcap
{
\definecolor{dialinecolor}{rgb}{0.000000, 0.000000, 0.000000}
\pgfsetfillcolor{dialinecolor}
% was here!!!
\pgfsetarrowsend{latex}
\definecolor{dialinecolor}{rgb}{0.000000, 0.000000, 0.000000}
\pgfsetstrokecolor{dialinecolor}
\draw (21.884903\du,14.645404\du)--(22.981212\du,14.130024\du);
}
\pgfsetlinewidth{0.020000\du}
\pgfsetdash{}{0pt}
\pgfsetdash{}{0pt}
\pgfsetbuttcap
{
\definecolor{dialinecolor}{rgb}{0.000000, 0.000000, 0.000000}
\pgfsetfillcolor{dialinecolor}
% was here!!!
\pgfsetarrowsend{latex}
\definecolor{dialinecolor}{rgb}{0.000000, 0.000000, 0.000000}
\pgfsetstrokecolor{dialinecolor}
\draw (21.884903\du,15.371619\du)--(23.008712\du,15.836999\du);
}
\pgfsetlinewidth{0.020000\du}
\pgfsetdash{}{0pt}
\pgfsetdash{}{0pt}
\pgfsetbuttcap
{
\definecolor{dialinecolor}{rgb}{0.000000, 0.000000, 0.000000}
\pgfsetfillcolor{dialinecolor}
% was here!!!
\pgfsetarrowsend{latex}
\definecolor{dialinecolor}{rgb}{0.000000, 0.000000, 0.000000}
\pgfsetstrokecolor{dialinecolor}
\draw (24.452511\du,14.130024\du)--(25.533712\du,14.681652\du);
}
\pgfsetlinewidth{0.020000\du}
\pgfsetdash{}{0pt}
\pgfsetdash{}{0pt}
\pgfsetbuttcap
{
\definecolor{dialinecolor}{rgb}{0.000000, 0.000000, 0.000000}
\pgfsetfillcolor{dialinecolor}
% was here!!!
\pgfsetarrowsend{latex}
\definecolor{dialinecolor}{rgb}{0.000000, 0.000000, 0.000000}
\pgfsetstrokecolor{dialinecolor}
\draw (24.480011\du,15.836999\du)--(25.522020\du,15.327487\du);
}
\pgfsetlinewidth{0.020000\du}
\pgfsetdash{}{0pt}
\pgfsetdash{}{0pt}
\pgfsetbuttcap
{
\definecolor{dialinecolor}{rgb}{0.000000, 0.000000, 0.000000}
\pgfsetfillcolor{dialinecolor}
% was here!!!
\pgfsetarrowsend{latex}
\definecolor{dialinecolor}{rgb}{0.000000, 0.000000, 0.000000}
\pgfsetstrokecolor{dialinecolor}
\pgfpathmoveto{\pgfpoint{26.174378\du}{14.495024\du}}
\pgfpatharc{326}{215}{2.935389\du and 2.935389\du}
\pgfusepath{stroke}
}
\pgfsetlinewidth{0.020000\du}
\pgfsetdash{}{0pt}
\pgfsetdash{}{0pt}
\pgfsetbuttcap
{
\definecolor{dialinecolor}{rgb}{0.000000, 0.000000, 0.000000}
\pgfsetfillcolor{dialinecolor}
% was here!!!
\pgfsetarrowsend{latex}
\definecolor{dialinecolor}{rgb}{0.000000, 0.000000, 0.000000}
\pgfsetstrokecolor{dialinecolor}
\draw (17.643892\du,14.986527\du)--(18.328100\du,14.988512\du);
}
\pgfsetlinewidth{0.020000\du}
\pgfsetdash{}{0pt}
\pgfsetdash{}{0pt}
\pgfsetbuttcap
{
\definecolor{dialinecolor}{rgb}{0.000000, 0.000000, 0.000000}
\pgfsetfillcolor{dialinecolor}
% was here!!!
\pgfsetarrowsend{latex}
\definecolor{dialinecolor}{rgb}{0.000000, 0.000000, 0.000000}
\pgfsetstrokecolor{dialinecolor}
\draw (29.185623\du,14.983512\du)--(29.725000\du,15.000000\du);
}
\pgfsetlinewidth{0.020000\du}
\pgfsetdash{}{0pt}
\pgfsetdash{}{0pt}
\pgfsetbuttcap
{
\definecolor{dialinecolor}{rgb}{0.000000, 0.000000, 0.000000}
\pgfsetfillcolor{dialinecolor}
% was here!!!
\pgfsetarrowsend{latex}
\definecolor{dialinecolor}{rgb}{0.000000, 0.000000, 0.000000}
\pgfsetstrokecolor{dialinecolor}
\pgfpathmoveto{\pgfpoint{19.428607\du}{15.462634\du}}
\pgfpatharc{123}{58}{8.037881\du and 8.037881\du}
\pgfusepath{stroke}
}
\end{tikzpicture}

}
  \caption{Equivalent NFA of direction expression}
  \label{fig:nfaEx}
\end{figure}
\transtrue
\setcode{standard}

%\begin{figure}[tb!]
%{ \relsize{-1}
%\begin{framed}
%\begin{multicols}{2}
%\begin{itemize}
%\item Annotated Expression: \\
%\\
%{ \relsize{-2.5}
%%\begin{itemize}
%%  \item 
%  $\stackrel{e_1}{(P|N)+}~\stackrel{o_1}{\mathit{O}?}~\stackrel{r}{R}~\stackrel{o_2}{\mathit{O^{\wedge}2}}~\stackrel{e_2}{(P|N|U)+}$
%%\end{itemize}
%}
%\end{itemize}
%\begin{itemize}
%\item User defined semantic \\relations:
%\begin{itemize}
%\item $\langle e_1,e_2,r\rangle$
%\item $\langle r,e_1,o_1\rangle$
%\item $\langle r,e_2,o_2\rangle$
%\end{itemize}
%\end{itemize}
%\columnbreak
%\setcode{utf8}
%\transfalse
%\resizebox{0.9\columnwidth}{!}{
%	\relsize{+1.5} % Graphic for TeX using PGF
% Title: /home/ameen/Desktop/ergraphEx.dia
% Creator: Dia v0.97.1
% CreationDate: Sat Jan 11 03:49:31 2014
% For: ameen
% \usepackage{tikz}
% The following commands are not supported in PSTricks at present
% We define them conditionally, so when they are implemented,
% this pgf file will use them.
\ifx\du\undefined
  \newlength{\du}
\fi
\setlength{\du}{30\unitlength}
\begin{tikzpicture}
\pgftransformxscale{1.000000}
\pgftransformyscale{-1.000000}
\definecolor{dialinecolor}{rgb}{0.000000, 0.000000, 0.000000}
\pgfsetstrokecolor{dialinecolor}
\definecolor{dialinecolor}{rgb}{1.000000, 1.000000, 1.000000}
\pgfsetfillcolor{dialinecolor}
\definecolor{dialinecolor}{rgb}{1.000000, 1.000000, 1.000000}
\pgfsetfillcolor{dialinecolor}
\pgfpathellipse{\pgfpoint{4.324130\du}{10.573360\du}}{\pgfpoint{1.223030\du}{0\du}}{\pgfpoint{0\du}{0.734470\du}}
\pgfusepath{fill}
\pgfsetlinewidth{0.050000\du}
\pgfsetdash{}{0pt}
\pgfsetdash{}{0pt}
\pgfsetmiterjoin
\definecolor{dialinecolor}{rgb}{0.000000, 0.000000, 0.000000}
\pgfsetstrokecolor{dialinecolor}
\pgfpathellipse{\pgfpoint{4.324130\du}{10.573360\du}}{\pgfpoint{1.223030\du}{0\du}}{\pgfpoint{0\du}{0.734470\du}}
\pgfusepath{stroke}
% setfont left to latex
\definecolor{dialinecolor}{rgb}{0.000000, 0.000000, 0.000000}
\pgfsetstrokecolor{dialinecolor}
\node at (4.324130\du,10.465026\du){\RL{دبي مول}};
% setfont left to latex
\definecolor{dialinecolor}{rgb}{0.000000, 0.000000, 0.000000}
\pgfsetstrokecolor{dialinecolor}
\node at (4.324130\du,10.888360\du){Dubai Mall};
\definecolor{dialinecolor}{rgb}{1.000000, 1.000000, 1.000000}
\pgfsetfillcolor{dialinecolor}
\pgfpathellipse{\pgfpoint{6.415976\du}{8.893981\du}}{\pgfpoint{1.395976\du}{0\du}}{\pgfpoint{0\du}{0.720041\du}}
\pgfusepath{fill}
\pgfsetlinewidth{0.050000\du}
\pgfsetdash{}{0pt}
\pgfsetdash{}{0pt}
\pgfsetmiterjoin
\definecolor{dialinecolor}{rgb}{0.000000, 0.000000, 0.000000}
\pgfsetstrokecolor{dialinecolor}
\pgfpathellipse{\pgfpoint{6.415976\du}{8.893981\du}}{\pgfpoint{1.395976\du}{0\du}}{\pgfpoint{0\du}{0.720041\du}}
\pgfusepath{stroke}
% setfont left to latex
\definecolor{dialinecolor}{rgb}{0.000000, 0.000000, 0.000000}
\pgfsetstrokecolor{dialinecolor}
\node at (6.415976\du,8.785647\du){\RL{المبنى}};
% setfont left to latex
\definecolor{dialinecolor}{rgb}{0.000000, 0.000000, 0.000000}
\pgfsetstrokecolor{dialinecolor}
\node at (6.415976\du,9.208981\du){the building};
\definecolor{dialinecolor}{rgb}{1.000000, 1.000000, 1.000000}
\pgfsetfillcolor{dialinecolor}
\pgfpathellipse{\pgfpoint{8.776400\du}{10.616657\du}}{\pgfpoint{0.927400\du}{0\du}}{\pgfpoint{0\du}{0.546667\du}}
\pgfusepath{fill}
\pgfsetlinewidth{0.050000\du}
\pgfsetdash{}{0pt}
\pgfsetdash{}{0pt}
\pgfsetmiterjoin
\definecolor{dialinecolor}{rgb}{0.000000, 0.000000, 0.000000}
\pgfsetstrokecolor{dialinecolor}
\pgfpathellipse{\pgfpoint{8.776400\du}{10.616657\du}}{\pgfpoint{0.927400\du}{0\du}}{\pgfpoint{0\du}{0.546667\du}}
\pgfusepath{stroke}
% setfont left to latex
\definecolor{dialinecolor}{rgb}{0.000000, 0.000000, 0.000000}
\pgfsetstrokecolor{dialinecolor}
\node at (8.776400\du,10.508324\du){\RL{مقربة}};
% setfont left to latex
\definecolor{dialinecolor}{rgb}{0.000000, 0.000000, 0.000000}
\pgfsetstrokecolor{dialinecolor}
\node at (8.776400\du,10.931657\du){near};
\definecolor{dialinecolor}{rgb}{1.000000, 1.000000, 1.000000}
\pgfsetfillcolor{dialinecolor}
\pgfpathellipse{\pgfpoint{4.451892\du}{6.494554\du}}{\pgfpoint{1.451892\du}{0\du}}{\pgfpoint{0\du}{0.787194\du}}
\pgfusepath{fill}
\pgfsetlinewidth{0.050000\du}
\pgfsetdash{}{0pt}
\pgfsetdash{}{0pt}
\pgfsetmiterjoin
\definecolor{dialinecolor}{rgb}{0.000000, 0.000000, 0.000000}
\pgfsetstrokecolor{dialinecolor}
\pgfpathellipse{\pgfpoint{4.451892\du}{6.494554\du}}{\pgfpoint{1.451892\du}{0\du}}{\pgfpoint{0\du}{0.787194\du}}
\pgfusepath{stroke}
% setfont left to latex
\definecolor{dialinecolor}{rgb}{0.000000, 0.000000, 0.000000}
\pgfsetstrokecolor{dialinecolor}
\node at (4.451892\du,6.386221\du){\RL{برج خليفة}};
% setfont left to latex
\definecolor{dialinecolor}{rgb}{0.000000, 0.000000, 0.000000}
\pgfsetstrokecolor{dialinecolor}
\node at (4.451892\du,6.809554\du){Khalifa tower};
\definecolor{dialinecolor}{rgb}{1.000000, 1.000000, 1.000000}
\pgfsetfillcolor{dialinecolor}
\pgfpathellipse{\pgfpoint{6.414368\du}{4.903560\du}}{\pgfpoint{1.590768\du}{0\du}}{\pgfpoint{0\du}{0.722790\du}}
\pgfusepath{fill}
\pgfsetlinewidth{0.050000\du}
\pgfsetdash{}{0pt}
\pgfsetdash{}{0pt}
\pgfsetmiterjoin
\definecolor{dialinecolor}{rgb}{0.000000, 0.000000, 0.000000}
\pgfsetstrokecolor{dialinecolor}
\pgfpathellipse{\pgfpoint{6.414368\du}{4.903560\du}}{\pgfpoint{1.590768\du}{0\du}}{\pgfpoint{0\du}{0.722790\du}}
\pgfusepath{stroke}
% setfont left to latex
\definecolor{dialinecolor}{rgb}{0.000000, 0.000000, 0.000000}
\pgfsetstrokecolor{dialinecolor}
\node at (6.414368\du,4.795227\du){\RL{التقاطع الأول}};
% setfont left to latex
\definecolor{dialinecolor}{rgb}{0.000000, 0.000000, 0.000000}
\pgfsetstrokecolor{dialinecolor}
\node at (6.414368\du,5.218560\du){intersection 1};
\definecolor{dialinecolor}{rgb}{1.000000, 1.000000, 1.000000}
\pgfsetfillcolor{dialinecolor}
\pgfpathellipse{\pgfpoint{8.800968\du}{6.567226\du}}{\pgfpoint{1.083468\du}{0\du}}{\pgfpoint{0\du}{0.584486\du}}
\pgfusepath{fill}
\pgfsetlinewidth{0.050000\du}
\pgfsetdash{}{0pt}
\pgfsetdash{}{0pt}
\pgfsetmiterjoin
\definecolor{dialinecolor}{rgb}{0.000000, 0.000000, 0.000000}
\pgfsetstrokecolor{dialinecolor}
\pgfpathellipse{\pgfpoint{8.800968\du}{6.567226\du}}{\pgfpoint{1.083468\du}{0\du}}{\pgfpoint{0\du}{0.584486\du}}
\pgfusepath{stroke}
% setfont left to latex
\definecolor{dialinecolor}{rgb}{0.000000, 0.000000, 0.000000}
\pgfsetstrokecolor{dialinecolor}
\node at (8.800968\du,6.458893\du){\RL{بالقرب}};
% setfont left to latex
\definecolor{dialinecolor}{rgb}{0.000000, 0.000000, 0.000000}
\pgfsetstrokecolor{dialinecolor}
\node at (8.800968\du,6.882226\du){next to};
% setfont left to latex
\definecolor{dialinecolor}{rgb}{0.000000, 0.000000, 0.000000}
\pgfsetstrokecolor{dialinecolor}
\node at (6.673600\du,10.362490\du){\RL{على}};
% setfont left to latex
\definecolor{dialinecolor}{rgb}{0.000000, 0.000000, 0.000000}
\pgfsetstrokecolor{dialinecolor}
\node at (6.673600\du,10.785823\du){prep};
% setfont left to latex
\definecolor{dialinecolor}{rgb}{0.000000, 0.000000, 0.000000}
\pgfsetstrokecolor{dialinecolor}
\node at (9.098600\du,9.312490\du){\RL{من هذا}};
% setfont left to latex
\definecolor{dialinecolor}{rgb}{0.000000, 0.000000, 0.000000}
\pgfsetstrokecolor{dialinecolor}
\node at (9.098600\du,9.735823\du){from this};
\definecolor{dialinecolor}{rgb}{1.000000, 1.000000, 1.000000}
\pgfsetfillcolor{dialinecolor}
\fill (3.913600\du,8.797490\du)--(3.913600\du,9.615823\du)--(4.733600\du,9.615823\du)--(4.733600\du,8.797490\du)--cycle;
% setfont left to latex
\definecolor{dialinecolor}{rgb}{0.000000, 0.000000, 0.000000}
\pgfsetstrokecolor{dialinecolor}
\node at (4.323600\du,9.112490\du){\RL{مقربة}};
% setfont left to latex
\definecolor{dialinecolor}{rgb}{0.000000, 0.000000, 0.000000}
\pgfsetstrokecolor{dialinecolor}
\node at (4.323600\du,9.535823\du){near};
% setfont left to latex
\definecolor{dialinecolor}{rgb}{0.000000, 0.000000, 0.000000}
\pgfsetstrokecolor{dialinecolor}
\node[anchor=west] at (3.823600\du,8.012500\du){\cci{isSyn}};
% setfont left to latex
\definecolor{dialinecolor}{rgb}{0.000000, 0.000000, 0.000000}
\pgfsetstrokecolor{dialinecolor}
\node at (3.848700\du,4.987490\du){\RL{بالقرب}};
% setfont left to latex
\definecolor{dialinecolor}{rgb}{0.000000, 0.000000, 0.000000}
\pgfsetstrokecolor{dialinecolor}
\node at (3.848700\du,5.410823\du){next to};
% setfont left to latex
\definecolor{dialinecolor}{rgb}{0.000000, 0.000000, 0.000000}
\pgfsetstrokecolor{dialinecolor}
\node at (8.892600\du,5.337490\du){\RL{من}};
% setfont left to latex
\definecolor{dialinecolor}{rgb}{0.000000, 0.000000, 0.000000}
\pgfsetstrokecolor{dialinecolor}
\node at (8.892600\du,5.760823\du){from};
% setfont left to latex
\definecolor{dialinecolor}{rgb}{0.000000, 0.000000, 0.000000}
\pgfsetstrokecolor{dialinecolor}
\node[anchor=west] at (7.103800\du,7.837500\du){};
\pgfsetlinewidth{0.050000\du}
\pgfsetdash{}{0pt}
\pgfsetdash{}{0pt}
\pgfsetbuttcap
{
\definecolor{dialinecolor}{rgb}{0.000000, 0.000000, 0.000000}
\pgfsetfillcolor{dialinecolor}
% was here!!!
\definecolor{dialinecolor}{rgb}{0.000000, 0.000000, 0.000000}
\pgfsetstrokecolor{dialinecolor}
\pgfpathmoveto{\pgfpoint{8.386343\du}{6.027332\du}}
\pgfpatharc{1}{-61}{0.960686\du and 0.960686\du}
\pgfusepath{stroke}
}
\pgfsetlinewidth{0.050000\du}
\pgfsetdash{}{0pt}
\pgfsetdash{}{0pt}
\pgfsetbuttcap
{
\definecolor{dialinecolor}{rgb}{0.000000, 0.000000, 0.000000}
\pgfsetfillcolor{dialinecolor}
% was here!!!
\definecolor{dialinecolor}{rgb}{0.000000, 0.000000, 0.000000}
\pgfsetstrokecolor{dialinecolor}
\pgfpathmoveto{\pgfpoint{4.944705\du}{5.180159\du}}
\pgfpatharc{266}{181}{0.530542\du and 0.530542\du}
\pgfusepath{stroke}
}
\pgfsetlinewidth{0.050000\du}
\pgfsetdash{}{0pt}
\pgfsetdash{}{0pt}
\pgfsetbuttcap
{
\definecolor{dialinecolor}{rgb}{0.000000, 0.000000, 0.000000}
\pgfsetfillcolor{dialinecolor}
% was here!!!
\definecolor{dialinecolor}{rgb}{0.000000, 0.000000, 0.000000}
\pgfsetstrokecolor{dialinecolor}
\pgfpathmoveto{\pgfpoint{5.126282\du}{9.169518\du}}
\pgfpatharc{243}{167}{0.649761\du and 0.649761\du}
\pgfusepath{stroke}
}
\pgfsetlinewidth{0.050000\du}
\pgfsetdash{}{0pt}
\pgfsetdash{}{0pt}
\pgfsetbuttcap
{
\definecolor{dialinecolor}{rgb}{0.000000, 0.000000, 0.000000}
\pgfsetfillcolor{dialinecolor}
% was here!!!
\definecolor{dialinecolor}{rgb}{0.000000, 0.000000, 0.000000}
\pgfsetstrokecolor{dialinecolor}
\pgfpathmoveto{\pgfpoint{8.421500\du}{10.111607\du}}
\pgfpatharc{350}{297}{1.328688\du and 1.328688\du}
\pgfusepath{stroke}
}
\pgfsetlinewidth{0.050000\du}
\pgfsetdash{}{0pt}
\pgfsetdash{}{0pt}
\pgfsetbuttcap
{
\definecolor{dialinecolor}{rgb}{0.000000, 0.000000, 0.000000}
\pgfsetfillcolor{dialinecolor}
% was here!!!
\definecolor{dialinecolor}{rgb}{0.000000, 0.000000, 0.000000}
\pgfsetstrokecolor{dialinecolor}
\pgfpathmoveto{\pgfpoint{5.547041\du}{10.573278\du}}
\pgfpatharc{125}{58}{2.099405\du and 2.099405\du}
\pgfusepath{stroke}
}
\pgfsetlinewidth{0.050000\du}
\pgfsetdash{}{0pt}
\pgfsetdash{}{0pt}
\pgfsetbuttcap
{
\definecolor{dialinecolor}{rgb}{0.000000, 0.000000, 0.000000}
\pgfsetfillcolor{dialinecolor}
% was here!!!
\definecolor{dialinecolor}{rgb}{0.000000, 0.000000, 0.000000}
\pgfsetstrokecolor{dialinecolor}
\pgfpathmoveto{\pgfpoint{4.451867\du}{7.281656\du}}
\pgfpatharc{165}{113}{1.674544\du and 1.674544\du}
\pgfusepath{stroke}
}
% setfont left to latex
\definecolor{dialinecolor}{rgb}{0.000000, 0.000000, 0.000000}
\pgfsetstrokecolor{dialinecolor}
\node[anchor=west] at (6.318700\du,7.975000\du){$e_2$};
% setfont left to latex
\definecolor{dialinecolor}{rgb}{0.000000, 0.000000, 0.000000}
\pgfsetstrokecolor{dialinecolor}
\node[anchor=west] at (4.268700\du,8.825000\du){$r_4$};
% setfont left to latex
\definecolor{dialinecolor}{rgb}{0.000000, 0.000000, 0.000000}
\pgfsetstrokecolor{dialinecolor}
\node[anchor=west] at (4.198700\du,11.665000\du){$e_1$};
% setfont left to latex
\definecolor{dialinecolor}{rgb}{0.000000, 0.000000, 0.000000}
\pgfsetstrokecolor{dialinecolor}
\node[anchor=west] at (8.868700\du,11.525000\du){$r_4$};
% setfont left to latex
\definecolor{dialinecolor}{rgb}{0.000000, 0.000000, 0.000000}
\pgfsetstrokecolor{dialinecolor}
\node[anchor=west] at (8.881200\du,8.915000\du){$o_2$};
% setfont left to latex
\definecolor{dialinecolor}{rgb}{0.000000, 0.000000, 0.000000}
\pgfsetstrokecolor{dialinecolor}
\node[anchor=west] at (6.533700\du,11.315000\du){$o_1$};
% setfont left to latex
\definecolor{dialinecolor}{rgb}{0.000000, 0.000000, 0.000000}
\pgfsetstrokecolor{dialinecolor}
\node[anchor=west] at (3.046200\du,7.390000\du){$e_1$};
% setfont left to latex
\definecolor{dialinecolor}{rgb}{0.000000, 0.000000, 0.000000}
\pgfsetstrokecolor{dialinecolor}
\node[anchor=west] at (6.298700\du,3.940000\du){$e_2$};
% setfont left to latex
\definecolor{dialinecolor}{rgb}{0.000000, 0.000000, 0.000000}
\pgfsetstrokecolor{dialinecolor}
\node[anchor=west] at (8.753700\du,5.015000\du){$o_2$};
% setfont left to latex
\definecolor{dialinecolor}{rgb}{0.000000, 0.000000, 0.000000}
\pgfsetstrokecolor{dialinecolor}
\node[anchor=west] at (3.781200\du,4.590000\du){$r_1$};
% setfont left to latex
\definecolor{dialinecolor}{rgb}{0.000000, 0.000000, 0.000000}
\pgfsetstrokecolor{dialinecolor}
\node[anchor=west] at (8.708700\du,7.490000\du){$r_1$};
\end{tikzpicture}

%}
%\transtrue
%\setcode{standard}
%\end{multicols}
%\end{framed}
%}
%\caption{User-defined semantic relation example}
%\label{fig:srEx}
%\end{figure}

\framework computes the relational entities in a user defined 
relation $R=\ldbrack \langle e_1,e_2,r\rangle \rdbrack$
$\subseteq \ldbrack e_1 \rdbrack \times \ldbrack e_2 \rdbrack \times \ldbrack r \rdbrack$ 
to be the elements of $\ldbrack e_1 \rdbrack \times \ldbrack e_2 \rdbrack \times \ldbrack r \rdbrack$
with the smallest nonzero positive distance between the source and the destination
where the distance is the number of words between the matches. 

%\framework supports the following configurations for the subexpressions of $e_i$ and $e_j$:
%\begin{itemize}
%\item $\left\vert{e_i}\right\vert=1\wedge\left\vert{e_j}\right\vert=1$: Relate $e_i$ match to $e_j$ match with edge labelled by $r$.
%\item $i=j\wedge\left\vert{e_i}\right\vert>1$: Relate sequences of $e_i$ matches with edges labelled by $r$.
%\item $i\neq j\wedge\left\vert{e_i}\right\vert>1\wedge\left\vert{e_j}\right\vert=1$: Relate each $e_i$ match to the single $e_j$ match with an edge labelled by $r$. The reverse applies for $\left\vert{e_i}\right\vert=1\wedge\left\vert{e_j}\right\vert>1$.
%\item $i\neq j\wedge\left\vert{e_i}\right\vert>1\wedge\left\vert{e_j}\right\vert>1$: Relate each $e_i$ match to each $e_j$ match with an edge labelled by $r$.
%\end{itemize}
%
In Figure~\ref{fig:motiv}(b), \framework names the subexpressions 
$(P|N)+$, $(P|N|U)+$, $O?$, $O\wedge 2$, and $R$, 
as $e_1, e_2, o_1, o_2,$ and $r$, respectively. 
The user defines the semantic relations
$\langle e_1, e_2, r\rangle$, 
$\langle r, e_1, o_1\rangle$, and 
$\langle r,e_2,o_2\rangle$.

The matches of $e_1,~e_2,~o_1,~o_2,$ and $r$ from the second match tree 
of Figure~\ref{fig:motiv}(b) are \RL{dby mwl}(Dubai Mall), 
\RL{Almbn_A}(the building), \RL{`l_A}(prep), \RL{mn h_dA}(from this), 
and \RL{mqrbT}(near), respectively. 
\framework~constructs the semantic relation matches and builds the lower part of the 
entity-relation graph shown in Figure~\ref{fig:intromotiv}.

For the first match \RL{brj _hlyfT bAlqrb mn AltqA.t` Al-'-wl},
the matches of $e_1,~e_2,~o_2,$ and $r$ are 
\RL{brj _hlyfT}(Khalifa tower), \RL{AltqA.t` Al-'-wl}(intersection 1), 
\RL{mn}(from), and \RL{bAlqrb}(next to), respectively. 
\framework~doesn't construct the relation $\langle r, e_1, o_1\rangle$ 
since $o_1$ has no match. 
Therefore, we get the upper part of the entity-relation graph shown 
in Figure~\ref{fig:intromotiv}.

After computing the relational entities, 
\framework~computes a cross-reference relation between the extracted entities
using a second order synonymy feature ($Syn^2$).
The \cci{isA} edge in the 
graph of Figure~\ref{fig:intromotiv} shows the cross-reference relation 
between \RL{brj _hlyfT}(Khalifa tower) from the first match 
with \RL{Almbn_A}(the building) from the second match.
\end{comment}