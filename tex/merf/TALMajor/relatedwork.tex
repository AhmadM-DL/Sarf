In this section we review the literature on 
entity and relation IE and on automatic and manual annotation
techniques and compare to \framework.

%SystemT\cite{chiticariu2010systemt}, 
%TEXTMARKER\cite{atzmueller2008rule}, 
%Urbain\cite{urbain2012user}, 
%QARAB\cite{hammo2002qarab}

\begin{table}[tb!]
\caption{Comaprison of \framework with SystemT, 
TEXTMARKER, 
Urbain, 
QARAB}
\resizebox{\columnwidth}{!}{
\begin{tabular}{l|c|c|c|c|c}
%{p{2.7cm}|p{.7cm}|p{1.2cm}|p{1.2cm}|p{1.2cm}|p{1.2cm}}
Features & \framework & SystemT & TEXT\linebreak MARKER & Urbain & QARAB\\ \hline
Query type & MRE & AQL & matching \linebreak rules & natural language & natural language \\ \hline
Morphology support& $\checked$ & - & - & OpenNLP & Parser \\ \hline
Relations & $\checked$ & - & - & $\checked$ & - \\ \hline
Actions & $\checked$ & - & - & - & - \\ \hline
Editor & $\checked$ & - & $\checked$ & - & - \\ \hline
Tag visualization & $\checked$ & - & $\checked$ & - & - \\ \hline
Graph visualization & $\checked$ & - & - & - & - \\
\end{tabular}
}
\label{tab:iecomp}
\end{table}

{\bf Information Extraction.~}
The common pattern specification language (CPSL) targets system independent 
IE specifications. 
CPSL consists of 
(1) a declaration part specifying names and labels, 
(2) a rule definition part specifying patterns, regular expressions and associated actions, and 
(3) and macro text substitution part~\cite{appelt1998common}.
\framework extends CPSL with Arabic morphological features, code actions, and user defined relations.

SystemT~\cite{chiticariu2010systemt}
aims to overcome the performance and expressivity limitations of CPSL.
It is based on an algebraic approach to declarative information extraction,
uses the declarative annotation query language (AQL), and uses
an optimizer to generate high performance execution plans for the AQL rules. 
%The AQL query is translated into an algebraic expression and an optimizer selects an execution plan. 
%\framework overcomes the limitations described by SystemT. 
\framework supports multiple tags per word, and supports the MRE conjunction operator
which allows for overcoming the overlapping annotation problem. 
%es the overfor each word and this overcomes the lossy sequencing caused 
%by random annotation dropping. 
%\framework introduces the conjunction operation in the MRE to overcome the expressivity limitations of expressing rules with overlapping annotations. 

TEXTMARKER is a semi-automatic rule-based IE system 
for structured data acquisition\cite{atzmueller2008rule}.
%TEXTMARKER requires user defined rules that consider features of the text 
%However, constructing the rules requires manual work of a domain specialist. 
Both TEXTMARKER and \framework provide the user with GUI editor and result visualizer. 
%\framework also differs in that it provides graph visualization for regular expression and user defisemantic relation matches.

The work in~\cite{urbain2012user} presents a user 
driven relational model and targets entity and relation extraction. 
The user enters a natural language query, and uses the OpenNLP toolkit to 
extract tags and relations from the query. 
Then it extends the extracted entities into a query model,
and uses the query model to extract matches from the document. 
Similar to \framework, the system constructs entities and relations. 

QARAB is an Arabic question answering system that 
takes an Arabic natural language query and provides short answers for it~\cite{hammo2002qarab}. 
QARAB uses traditional information retrieval techniques and an outdated Arabic NLP 
analyzer that computes limited features of Arabic words compared 
to the morphological analysis of \framework. 
QARAB then classifies the query against a set of known question types. 

Table~\ref{tab:iecomp} summarizes the comparison between \framework and
other systems. \framework differs in that it provides 
code actions, user defined relations, and an interactive 
graph visualization of the relational entities. 
It also differs in that it fully supports Arabic morphological analysis
while only QARAB supports Arabic linguistic features using a parser, and 
the work in ~\cite{urbain2012user} uses OpenNLP that currently lacks full
support for Arabic morphological features. 
Similar to TEXTMARKER, \framework has the advantage of providing 
a user friendly interactive interface to edit the entity and relational 
specifications and visualize the results. 

DUALIST is an annotation system for quickly building classifiers for 
text processing tasks using active learning and semi-supervised learning~\cite{settles2011closing}. 
The classifier is interactive as it queries the annotator 
for entity detetction correction and annotation correction. 
\framework doesn't support classification tasks. 
However, \framework provides an interactive GUI where the user can edit MBF and MRE tags. 
This interactive environment contributes to the regular expression 
extraction and semantic relation construction which increases the overall accuracy.

WordNet is a lexical reference system that mimics human lexical memory 
%inspired by psycholinguistic theories 
and that relates words based on their semantic values and their functional 
categories: nouns, verbs, adjectives, adverbs, and function words~\cite{intro}. 
The $Syn^k$ feature in \framework is inspired by WordNet.

{\bf Annotation tools.~}
An overview of annotation tools and their Arabic-English word alignment issues 
concludes with a set of rules and guidelines needed in an Arabic annotation alignment tool
~\cite{kholidy2010towards}. 
The work in \cite{dukes2011supervised} presents a collaborative effort towards morphological 
and syntactic annotation of the Quran. 
\cite{dorr2010interlingual} presents a framework for interlingual 
annotation of parallel text corpora with multi-level representations. 
\cite{kulick2010consistent} presents the integration of the 
Standard Arabic Morphological Analyzer (SAMA) into the workflow of the Arabic Treebank.

MMAX2 is a manual multi-level linguistic annotation tool with an XML 
based data model~\cite{mmax2}. 
It enables the user to create, browse, visualize, and query annotations
and may be able to resolve coreference tags. 
BRAT~\cite{brat} and WordFreak~\cite{wordfreak} are manual 
multi-lingual user friendly web-based annotators that allow the construction 
of entity and relation annotation corpora~\cite{brat}. 
They can be extended through plug-ins to enable automatic annotators 
and customized annotation visualization specifications.
Knowtator~\cite{ogren2006knowtator} is a general purpose incremental text annotation tool 
implemented as a Prot\'eg\'e~\cite{gennari2003evolution} plug-in. 
Prot\'eg\'e is an open-source platform with a suite of tools to construct domain 
models and knowledge-based applications with ontologies. 
However, it doesn't support the Arabic language.

\framework differs from MMAX2, BRAT, WordFreak, and Knowtator in that it is 
an automatic annotator that allows manual corrections 
and sophisticated tag type and relation specifications over
Arabic morphological features.

The work in \cite{smrz2004morphotrees} presents %the extension of TrEd, 
a customizable general purpose tree editor, with the Arabic MorphoTrees annotations. 
The MorphoTrees present the morphological analyses in a hierarchical organization 
based on common features. 

Task specific annotation tools such as~\cite{alrahabi2006semantic}
uses enunciation semantic maps to automatically annotate 
directly reported Arabic and French speech. 
AraTation is another task specific tool for semantic annotation of 
Arabic news using web ontology based semantic maps~\cite{saleh2009aratation}.
We differ in that \framework is general, and not task specific, and it uses 
morphology-based features as atomic terms.
Fassieh is a commercial Arabic text annotation tool that enables the production of large 
Arabic text corpora~\cite{attia2009fassieh}. 
The tool supports Arabic text factorization including morphological analysis, POS tagging, 
full phonetic transcription, and lexical semantics analysis in an automatic mode. 
Fassieh is not directly accessible to the research community and requires commercial licensing. 
\framework is open source and differs in that it allows the user to build tag types 
and extract entities and relations from text.

\begin{comment}
\subsection{Existing annotation and entity extraction tools}

Researchers proposed and evaluated empirical and 
knowledge-based techniques to extract entities and relational entities
from text. 
We briefly review them here and we discuss them and compare to them
in Section~\ref{sec:related}. 
The work in~\cite{ekbal2010named} presents a language independent 
approach for NER extraction using {\em support vector machines}. 
The work in~\cite{abdelrahman2010integrated} integrates 
a semi-supervised bootstrapping pattern recognition technique, 
and a supervised classifier based on {\em conditional random fields}
to solve NER problems. 

Knowledge-based techniques such 
as~\cite{zaghouani2010adapting,traboulsi2009arabic} propose local grammars 
with morphological stemming to perform NER. 
%The work in~
\cite{ZaMaHaCicling2012Entity} presents a method for extracting entities, 
events, and relations amongst them from Arabic text using a hierarchy of manually built 
finite state machines driven by morphological features and graph 
transformation algorithms. 
Such techniques require advanced linguistic and programming expertise. 
QARAB is a question answering system that takes 
Arabic natural language queries and provides short answers~\cite{hammo2002qarab}. 

Researchers also proposed systems for automatic IE based 
on user specifications. 
CPSL is a common pattern specification language for finite-state 
grammar~\cite{appelt1998common}. 
%\framework is an extension to CPSL with action execution and relation construction. 
%It defines three sections for declaration, rule definition, and macros. 
The work in \cite{chiticariu2010systemt} presents SystemT, a system based on 
an algebraic Approach to Declarative information extraction (IE). 
%It uses a declarative rule language and an optimizer. 
TEXTMARKER is a rule-based IE system designed to extract structured data from 
text \cite{atzmueller2008rule}. 
\cite{urbain2012user} presents a user-driven relational model 
requiring a user natural language query to extract entities and relational entities.
\end{comment}