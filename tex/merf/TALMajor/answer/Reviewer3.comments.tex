\section*{Reviewer 3 comments} 
\textit{Comments to the Author}



This work is an improvement step in a continuous previous work 
(Saref,MATAr) which leads to a complete project that can be 
used as a tool for other researchers in the field to help 
improving the Arabic language computations and processings.


It depends on the morphological analyser (Saref) results 
to define tags for flexible corpus annotation that is automatic 
but allow the users to correct and fix the tagging. 
In additions it extract entitiy and relational entityies from 
the input document.


\begin{enumerate}[leftmargin=0mm,label=\bfseries CommentR3.\arabic*]
\item \label{Review.3.1}
As a reader, It wasn't very easy to followup with the writing 
style. I needed to return back to ``Saref'' and ``MATAr'' papers 
to understand the terminology used.
\answer{
  We added a background section with definitions... 
}

First, I think we can ask 
"What would a researcher reading this paper need to know?", the answer would be:

\item \label{Review.3.2}
1- ``What is new in MERF?''. and this was clear for the reader 
    which is mainly
    a)(entity and relational entity extraction).
    b) defining tag of words based on a synonymic relation
\answer{
  ...
}

\item \label{Review.3.3}
2- ``How to use MERF?''
(you can have more explanations in your appendex and could have 
``help'' button in your API. This part was also relatively 
clear for the reader.
\answer{
  ... 
}


\item \label{Review.3.4}
3- The most important part is ``how MERF was built 
(methodology)?''. but I found this part was not very clear.
\answer{
  We added a methodology section...
}

4-and for sure ``how it was evaluated and good results discussion''. hich needs more attention.
\answer{
  We added subsection Results.X to discuss the evaluation 
  strategy. 
  We also added extended the discussion section... 
}


The following are some detailed recomendations:


\item \label{Review.3.5}
1-In the introduction, I think you need to omit ``We discuss the importance of the morphological featureotivs supported in MERF terms in Section 3; in brief, morphological preprocessing iskey to Arabic NLP.'' , in page 2 since it will be repeated in page 5.
\answer{
  Done. 
}


\item \label{Review.3.6}
2- in the introduction you illusterated a target example, 
so why don you need the Motivation section. 
What about if combine both.
\answer{
  Thank you. We combine both as motivation and we include that
  as a subsection of the introduction. 
}

\item \label{Review.3.7}
In addition, at page 9 ``the edges and internal nodes are text, 
morphology-based, and word distance based relational entities'' 
is not clear.
\answer{
  We rephrase the sentence to read ... 
}


\item \label{Review.3.8}
please always define your terminology before using it and 
provide examples. suggest the ``formula" was not so clear. 
I suggest using regular grammar insted since you are using 
regular expression and the match tree can be considered as 
parse tree.
\answer{
  We use formula to define predicates that range over 
  the morphological features. 
  We consider the matches of the formulae as tokens/tags (maybe
  we call them tag types.)
  We then use regular expressions (maybe we should use rules). 
  ... 
  We modify the paper by adopting the suggested terminology: 
  rule for regular expression and parse tree for match tree. 
}

\item \label{Review.3.9}
2- Section ``Existing annotation and entity extraction tools” 
need to be with section 6 or section 6 need to be combined 
with it in as one section but leaving the comparision to 
the analysis section.
\answer{
  We thank the reviewer for the comment. 
  We moved Section 1.1 to Section 6. 
}


\item \label{Review.3.10}
3- Section Background:Morphological analysis, 
the definition explaned in page 9 did not match table 1 at page 7.
\answer{
  We fix the definition to reflect that a prefix (and an suffix)
  could be a concatenation of other prefixes. 
}


\item \label{Review.3.11}
4- Section 5 can be combined with section 2. 
you can introduce the user friendly interface using the 
specified example. 
Section 2 seemed not very imporatnt as stand alone section.
\answer{
  We merged Section 2 with Section 1. 
  We move Section 5 to become Section 2 so that it follows 
  the motivation directly. 
}

5- Section MERF/


\item \label{Review.3.12}
a) figure 3 did not make it clear for the reader to 
understand the process. Also, the figure location is far 
away from its explanation.

\answer{
We split Figure 3 into three Figures and provide explanations
for both. 
a. Arabic Text (Consider MBBF as input coming fro another figure) ----> Tags.
b. Rest of the figure... 
c. RTC + Diff. + Visualizatiin 
}


\item \label{Review.3.13}
b) Syn$^k$ is nicely explaned but I did not recognize it as 
your own idea. if this is your idea to use English translation 
as a pivot to extract Arabic synonyms then it needs to have 
more focus because I think this is new.
\answer{
Up to our knowledge, we introduced the idea. 
We highlight the concept more in the paper. 
}

\item \label{Review.3.14}
what is $2^{gloss}$ and $2^{s}$?
\answer{
  The power sets of... We clarify that in the paper. 
}


\item \label{Review.3.15}
please to make it easier for the reader to followup, 
let the example in figure 4 also be defined as the formula 
variables. w,{u},..so on
\answer{
  We do that now.
}


\item \label{Review.3.16}
c) In MRE section,page 10, $k <=7$ why? 
\answer{
  Computational reasons ... say it in the paper. 
}

\item \label{Review.3.17}
also, A belongs-to F , A is a morphological feature, 
then how CF belongs-to A. A is defined as one item not a set.
\answer{
Thank you. This is a mistake. We corrected that in teh paper to 
read as follows... 
}


\item \label{Review.3.18}
MRT definition is not clear.
\answer{
We clarify the definition of MRE in the paper. 
}


\item \label{Review.3.19}
"MRE" and "user defined relations and actions " sections 
were hard to be understood.
\answer{
We rewrote both sections... 
}

\item \label{Review.3.20}
MREF simulator: please provide examples soon after the 
definitions to make it easier for the reader. 
The reader can be lost so fast within the variable definitions. 
\answer{
We provide running examples with the definitions....
}


\item \label{Review.3.21}
for each word you are computing all the formulas values, 
so the comlexity is O(text length X number of formulas), 
what is the range of number of formulas ?
\answer{
  ... 
  The complexity is (\#solutions for all words) times 
  (\#number of user defined formulae). 
  We provide no limit on that.
  Practically, users define up to 10 formula. 
  We include this discussion in the paper .. 
}


\item \label{Review.3.22}
you are using R as tuple defining relation, 
then bellow you used it as set of feature vectors for the text. 
This is confusing the reader. 
I needed to read the definition in MATAr paper to understand 
what R is.
\answer{
  We cleaned the definitions... 
}


\item \label{Review.3.23}
d) In general, what was the data structure used? 
how would the annotated text be saved?
\answer{
  We used XXX data structures... 
}


\item \label{Review.3.24}
6- At section Results, comparing the development time 
did not strength the work. 
This is my first time to see researchers compare their 
algorithm in term of line of code and developing time. 
I would suggest comparing the algorithm complexity 
(linear , exponential, etc), and comparing how fast a user can 
accomplish similar jobs in the different applications. 
This can be done by assiging similar corpus annotation job 
and information extraction while computeing time needed by 
the user 
``as a survey like comparison for qualitative evaluation''. 
you can search for ``user interface system evaluation''. 
There is huristic evaluation where you give weight for job 
complexity and record time needed to complete it.

\answer{
By development time we meant the time it required the 
users to build an automated annotator similar to 
the MERF based annotator. 
We clarify that in the paper ... 

We conducted a survey as requested and we compare the 
manual annotation time without MERF to the annotation
time with MERF. 
}

\item \label{Review.3.24}
Also, comparison with previous work done by (Zaraket ) 
in different vesion seems like comparing with your previous work 
only and not with others. 
It would be nice to compare others work too because user 
interface could be introduced in different view.
\answer{
 ?????   
}

\item \label{Review.3.24}
What is the data size of your evaluation? 
I couldn't find how presion and recal was counted. 
you mentioned books but did you covered all the text in those 
books ? and how long was it? you can define it in term of 
number of relation extracted , tags added, expesion processed, 
sentences,..etc
\answer{
  We include more information about the data and 
  the evaluation process... 
}


\item \label{Review.3.24}
7- At section Discussion, I strongly urge you to discuss 
WHY the precission and recall have these values.
\answer{
We extend the discussion section to discuss ... 
}

\item \label{Review.3.25}
Clarity:  
It was not very clear how the work was done 
``Methodology part''. 
Having ``Existing annotation and entity extraction tools'' 
under unspecified section is misleading the reader. 


\item \label{Review.3.25}
unnumbered subsections was misleading me as a reader. 
\answer{
  Fixed. CLS file was not compatible with arabtex package. 
}

\item \label{Review.3.25}
table 1 at page 7 while referring to it only at page 9. 
I think it need to be closer to where it was mentioned in the text. 
\answer{
  Done.
}

\item \label{Review.3.25}
``MERF regular expressions support operators such as concatenation, zero or one, zero or more, one or more, up to $M$” what is M? 
it could be for example 0<=M<=|chunk| Section 4 “MERF”please add “Methedology”
\answer{
  $M$ is ... we clarified that in the paper. 
}

\item \label{Review.3.25}
Correctness: 
1- in the abstract, the sentence ``These techniques and 
tools require expertise in linguistics and programming and 
lack support of Arabic morphological analysis which is key to 
process Arabic text''
gives the meaning that ``it require lack support'' so I think it 
needs to be reformulated 
\answer{
  We rephrased the sentence as requested. 
}


\item \label{Review.3.25}
2-``defines tag types and''.// unify the font here and 
many other places in the paper. 
for example look at words(,correct,commercial)
\answer{
  Emphasize ... 
}

\item \label{Review.3.25}
Originality:  
This work will contribute in facilitating Arabic corpus 
annotation, if it was made available for other researchers in 
the field to access and use it, although there is no 
referencing to where it can be accessed. 
this is an improvement of a per-existing work ``MATAr: Morphology-based Tagger for Arabic''
\answer{
  The work is available on github... 
}

\end{enumerate}
