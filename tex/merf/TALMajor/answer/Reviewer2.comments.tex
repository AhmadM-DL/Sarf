\section*{Reviewer 2 comments}
\textit{Comments to the Author}



The submitted paper is dealing with an interesting topic 
of knowledge extraction in context of presenting a tool 
for extracting Entities and Relationships from Arabic texts. 
The tool is mainly based on an intensive interactive process, 
where a user has to define tag types, and associates them 
with regular expressions defined over Boolean formulae, 
which are in turn defined over matches of Arabic morphological 
features extended by additional synonyms. 
A major feature of this system is providing the user with a 
user-friendly GUI, however it does not support classification 
tasks.

\noindent\answer{
  Thank you for finding our work interesting. 
}


Despite the interesting motivation and the importance of 
this topic, the presentation shows some shortcomings in the 
following sense:

\begin{enumerate}[leftmargin=0mm,label=\bfseries CommentR2.\arabic*]


\item \label{Review.2.1} 
The conceptual and implementation aspects of the framework 
were introduced in not clear and differentiated form. 
The authors are advised to separate these issues in formal 
definitions and to present the resulted software tool in 
terms of more clear software techniques.

\answer{
We introduced the following formal definitions...

We clarify the implementation in the following manner. 
Tranformative algorithms: 1. RE to FSM
                          2. Entities and relations to garphs
Generative algorithms: 1. Actions to code
... 
}

\itodo{
Pending...
}


\item \label{Review.2.2} 
Despite the motivation, that framework is proceeding from a 
morphological aspect, this feature came to short in the 
presentation; e.g. sets of morphological solutions.

\answer{
  Section XXX discussed morphological analysis in Arabic. 
  We extend that in ...
  We highlight that by referring to the Section in the 
  Introduction. 
}

\item \label{Review.2.3} 
Finally, I believe this framework should be also expressed 
in terms of the complexity and decidability problems, 
particularly when evaluating the required time of different 
development tasks. 

\answer{
  The problem is decidable: regular grammar rules to FSM. 
  However, the problem might be not tractable: NDFSM to FSM might 
  be exponential. 
  In practice Table~\ref{tab:results} provides runtime results... 
  We modified the paper to include the above.
}

\done{
Done.
}
    
\item \label{Review.2.4} 
Providing the system with GUI is an advantage, 
However the system requires probably to some extend 
good linguistic and formal languages expertise to optimize 
the benefit of the implements tool.

\answer{
We discuss the target audience of the MERF tool in Section XXX. 
We prepared a set of 45 minutes slides based on the regular
expression chapter in Martin Jurafsky's book to... 
}

\done{
You note is accurate.
An advantage of the GUI is that it enables a user with mediocre 
linguistic skills to perform information extraction tasks and 
benefit from the framewrok.
}

\end{enumerate}
