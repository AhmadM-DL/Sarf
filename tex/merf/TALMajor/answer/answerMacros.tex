%%%%%%%%%%%%%%%%%%%%%%%%%%%%%%%%%%%%%%%%%%%%%%
\newcommand{\MATH}[1]{\ensuremath{#1}\xspace}

%%%%%%%%%%%%%%%%%%%%%%%%%%%%%%%%%%%%%%%%%%%%%%%%%%%%%%%%%%%%%%%%%%%%%%%%%%%%%%%%%%%%%%%%%%%%
%%%%%%%%%%%%%%%%%%%%%%%% Lists
\newcommand{\be}{\begin{itemize}}
\newcommand{\ee}{\end{itemize}}
\newcommand{\bdn}{\begin{description}}
\newcommand{\edn}{\end{description}}
\newcommand{\bn}{\begin{enumerate}}
\newcommand{\en}{\end{enumerate}}
\renewcommand{\i}{\item}

\newenvironment{closeitemize}{\begin{list}%
{$\bullet$}%
{\setlength{\itemsep}{-0.2\baselineskip}%
\setlength{\topsep}{0.2\baselineskip}}}%
{\end{list}}




%%%%%%%%%%%%%%%%%%%%%%%%%%%%%% MISC

\newcommand{\remove}[1]{}

%\newcommand{\case}[2]{\vspace{1.5ex} \noindent \textit{Case} #1: \emph{#2}.}
\newcommand{\scase}[2]{\vspace{1.5ex} \noindent \textit{Subcase} #1: #2.}
\newcommand{\sscase}[2]{\vspace{1.0ex} \noindent \textit{Subsubcase} #1: #2.}

%\newcommand{\defn}[1]{\textit{#1}}
\newcommand{\defi}[1]{\textit{#1}:}

\newcommand{\lb}{\linebreak}

%%%%%%%%%%%%%%%%%%%%%%%%%%%% FOOTNOTES AND THEOREMS


\newtheorem{definition}{Definition}

\newtheorem{theorem}{Theorem}
\newtheorem{lemma}{Lemma}
\newtheorem{proposition}{Proposition}
\newtheorem{corollary}{Corollary}
\renewcommand{\thefootnote}{\arabic{footnote}}


\newtheorem{helper}{Helper Function}
\AtBeginEnvironment{helper}{\small}

%%%%%%%%%%%%%%%%%%%%%%%%%%%%%%%%%%% Text
\newcommand{\smpage}{\noindent \parbox{\textwidth}}
\newcommand{\samep}{\parbox{\textwidth}}   % put onto the same page


\newcommand{\mathid}[1]{\ensuremath{\mathit{#1}}\xspace}

\newcommand{\bc}{\begin{center}}
\newcommand{\ec}{\end{center}}
%\newcommand{\ul}{\underline}
\newcommand{\bs}{\bigskip}
%\newcommand{\ms}{\medskip}
%\renewcommand{\ss}{\smallskip}

\newcommand{\bfg}{\begin{figure}}
\newcommand{\efg}{\end{figure}}

\renewcommand{\ss}{\smallskip}
\newcommand{\Eg}{E.g.,\xspace}
\newcommand{\eg}{e.g.,\xspace}
\newcommand{\ie}{i.e.,\xspace}


\newcommand{\intr}{\empi}
\newcommand{\intrdef}{\emph}

\newcommand{\emp}[1]{\textbf{#1}}
\newcommand{\empp}[1]{\emph{#1}}
\newcommand{\empb}[1]{\textbf{#1}}
\newcommand{\empi}[1]{\textit{#1}}
\newcommand{\empbi}[1]{\textbf{\textit{#1}}}

\newcommand{\ind}{\hspace*{3.0em}}

%%%%%From Nancy's book directory:  This spaces the symbols within a
%%%%%word nicely, in math mode.
\newcommand{\ms}[1]{%
        \relax\ifmmode
                \mathord{\mathcode`\-="702D\it #1\mathcode`\-="2200}%
        \else
                $\mathord{\mathcode`\-="702D\it #1\mathcode`\-="2200}$%
        \fi
}


\newcommand{\cmnt}{\`//}


%%%%%%%%%%%%%%%%%%%%%%% GENERAL MATH SYMBOLS %%%%%%%%%%%%%%%%%%%%%%%%%%%%%%

\newcommand{\Adj}{\mathop{\rm Adj}\nolimits}
\newcommand{\abs}[1]{\left| #1\right|}
\newcommand{\card}[1]{\left| #1\right|}
\newcommand{\ar}{\rightarrow}
%\newcommand{\ar}{\longrightarrow}
\newcommand{\al}{\alpha}
\renewcommand{\b}[1]{\overline{#1}}
\newcommand{\cat}{\mathbin{\frown}}
\newcommand{\choice}{\mbox{$[\hspace*{-1.0pt}]$}}
\newcommand{\ceil}[1]{\left\lceil #1 \right\rceil}
\renewcommand{\d}{\, . \,}      % separator in quantified formulae
\newcommand{\df}{\triangleq}
%\newcommand{\df}{\mbox{$\:\stackrel{\rm df}{=\!\!=}\:$}}
\newcommand{\dn}{\mbox{$\hspace{-0.1em}\downarrow\hspace{-0.1em}$}}
\newcommand{\Ex}{\mathop{\rm Ex}}
\newcommand{\es}{``"}   %empty string
\newcommand{\expect}[1]{{\rm E}\left[ #1 \right]}
\newcommand{\expectsq}[1]{{\rm E}^2\left[ #1 \right]}
\newcommand{\floor}[1]{\left\lfloor #1 \right\rfloor}
\renewcommand{\ge}{\geqslant}
\newcommand{\given}{\mid}
\newcommand{\halfind}{\hspace*{1.5em}}
\newcommand{\ifof}{\Longleftrightarrow} % logical equivalence
\newcommand{\img}{\mathrm{Image}}
\newcommand{\ints}{\cap}
\renewcommand{\l}{\ell}
\newcommand{\la}[1]{\mbox{$\, \stackrel{#1}{\rightarrow} \,$}}
\renewcommand{\le}{\leqslant}
\newcommand{\lra}{\mbox{$\longrightarrow$}}
%\newcommand{\mod}{\ \mathrm{mod}\ }
\newcommand{\oneton}{\{1,\,\ldots, n\}}
\newcommand{\n}{\ensuremath{[1:n]}} 
\newcommand{\paren}[1]{\left( #1 \right)}
\newcommand{\pair}[2]{\ensuremath{(#1, #2)}}
\newcommand{\pind}{\hspace*{3.0em}}
\newcommand{\pj}{\!\upharpoonright\!}
\newcommand{\proj}{\pj}
\newcommand{\preimg}{\mathrm{PreImage}}
\newcommand{\pl}{\!\parallel\!}
\newcommand{\s}{\mbox{$\hspace{-1pt}-\hspace{-2pt}$}}
\newcommand{\set}[1]{\ensuremath{\{ #1 \}}\xspace}
%\newcommand{\set}[1]{\{#1\}}
\newcommand{\seq}{\approx}
\newcommand{\spc}{\mbox{\vspace{-0.25in}}}
\newcommand{\stt}{\ | \ }
%\newcommand{\st}[1]{\ensuremath{[ #1 ]}\xspace}
\newcommand{\sub}{\subseteq}
\newcommand{\twodots}{\mathinner{\ldotp\ldotp}}
\newcommand{\tiff}{\textup{\ iff\ }}
\newcommand{\tl}[1]{\mbox{$\tilde{#1}$}}% abbreviated tilde
\newcommand{\tpl}[1]{\ensuremath{\langle #1 \rangle}}
\newcommand{\un}{\cup}
\newcommand{\union}{\bigcup}
\newcommand{\up}{\mbox{$\hspace{-0.1em}\uparrow\hspace{-0.1em}$}}
\newcommand{\Var}{\mathop{\rm Var}\nolimits}
\newcommand{\variance}[1]{{\rm Var}\left[ #1 \right]}
\newcommand{\w}{\omega}

\newcommand{\defcase}{\noindent\ensuremath{\bullet}\xspace}

%%%%%%%%%%%%%%%%%%% Ints, Reals, etc %%%%%%%%%%%%%%%%%%%%%%%%%%%%%

\newcommand{\reals}{{\mathbb R}}
\newcommand{\integers}{{\mathbb Z}}
\newcommand{\naturals}{{\mathbb N}}
\newcommand{\nat}{\naturals}
\newcommand{\nats}{\naturals}
\newcommand{\rationals}{{\mathbb Q}}
\newcommand{\complex}{{\mathbb C}}
\newcommand{\complexes}{{\mathbb C}}



%%%%%%%%%%%%%%%%%%%%%%%%% MATH MACROS AND ENVIRONMENTS %%%%%%%%%%%%%%%%%%%%%%%%%%

%\newcommand{\se}[3]{\ensuremath{#1[#2 .. #3]}}
\newcommand{\se}[3]{\ensuremath{#1[#2 \! : \! #3]}}
%\newcommand{\se}[3]{\ensuremath{#1[#2,\ldots,#3]}}


\newcommand{\ang}[1]{\ifmmode{\left\langle #1 \right\rangle}
   \else{$\left\langle${#1}$\right\rangle$}\fi}
        % the \if allows use outside mathmode,
        % but will swallow following space there!

\newcommand{\struct}[2]{\raisebox{-0.1in}{$\stackrel { \displaystyle
#1} {\scriptstyle #2}\,$}}

\newcommand{\pbx}[2]{\stackrel{\fbox{\begin{Beqnarray*} #1\\.\\.\\.\\#2 \end{Beqnarray*}}}{}}      % proof box

\newcommand{\asrt}[1]{\ensuremath{\{ #1 \}}}

%%%%%%%%%%%%%%%%%%%%%%% Hoare Logic Symbols
\newcommand{\lng}{\langle}
\newcommand{\ra}{\rangle}
\newcommand{\htp}[3]{\ensuremath{\{#1\}\,#2\,\{#3\}}}
\newcommand{\htptc}[3]{\ensuremath{\langle#1\rangle\,#2\,\langle#3\rangle}}
\newcommand{\var}{\ensuremath{\varphi}}

\newcommand{\as}[1]{\ensuremath{\{#1\}}}               % partial correctness asserion
%\newcommand{\ts}[1]{\ensuremath{\langle #1 \rangle}}   % termination correctness asserion

\newcommand{\ch}{\mbox{[\hspace{-0.15ex}]}}

\newcommand{\pa}[1]{\{\ensuremath{#1}\}}
%\newcommand{\ta}[1]{$<$#1$>$}
\newcommand{\ta}[1]{$\langle$\ensuremath{#1}$\rangle$}


\newcommand{\satt}{\equiv}
\newcommand{\sat}{\models}
\newcommand{\satf}{\mbox{\ensuremath{=\hspace*{-5pt}|}}}   % satisfiable - backward turnstile
%\newcommand{\satf}{\mbox{\ensuremath{=\!\!\!\!|}}}   % satisfiable - backward turnstile


\newcommand{\yld}{\vdash}
\newcommand{\yldd}{\equiv}
%\newcommand{\yldd}{\dashv \vdash}

\newcommand{\prob}{\ensuremath{\Pm \sat \pair{\Pre}{\Post}|_b }}



\newcommand{\pc}[1]{{#1}}    %% font for pseudocode
%\newcommand{\pc}[1]{\codetext{#1}}    %% font for pseudocode



%%%%%%%%%%%%%%%%%%%%%%%% JUSTIFY EXAMPLE MACROS
\newcommand{\valpha}{\ensuremath{\mathit{alpha}}}
\newcommand{\numeric}{\ensuremath{\mathit{numeric}}}
\newcommand{\wspace}{\ensuremath{\mathit{wspace}}}
\newcommand{\vnewline}{\ensuremath{\mathit{newline}}}
\newcommand{\blank}{\ensuremath{\mathit{blank}}}
\newcommand{\word}{\ensuremath{\mathit{word}}}
\newcommand{\lline}{\ensuremath{\mathit{line}}}
\newcommand{\para}{\ensuremath{\mathit{para}}}
\newcommand{\numWords}{\ensuremath{\mathit{numWords}}}
\newcommand{\selectWord}{\ensuremath{\mathit{selectWord}}}
\newcommand{\numLines}{\ensuremath{\mathit{numLines}}}


%\RequirePackage{metre}

\newcommand{\nl}{\ensuremath{\backslash n}}
\newcommand{\bl}{\textvisiblespace}
\newcommand{\tx}{tx}



%%%%%%%%%%%%%% MACROS FOR NUMBERED LINES IN CODE TABBING ENVIRONMENT %%%%%%%%%%%%%%%%

\newcounter{lctr}

\newcommand{\li}{\addtocounter{lctr}{1}\arabic{lctr}.}
\newcommand{\lio}[1]{\addtocounter{lctr}{1}\arabic{lctr}.\>\ensuremath{#1}\\}
\newcommand{\lit}[1]{\addtocounter{lctr}{1}\arabic{lctr}.\>\>\ensuremath{#1}\\}
\newcommand{\lih}[1]{\addtocounter{lctr}{1}\arabic{lctr}.\>\>\>\ensuremath{#1}\\}

%%%% 2'nd argument is for a commment
\newcommand{\lioc}[2]{\addtocounter{lctr}{1}\arabic{lctr}.\>\ensuremath{#1}\`{#2}\\}
\newcommand{\litc}[2]{\addtocounter{lctr}{1}\arabic{lctr}.\>\>\ensuremath{#1}\`{#2}\\}
\newcommand{\lihc}[2]{\addtocounter{lctr}{1}\arabic{lctr}.\>\>\>\ensuremath{#1}\`{#2}\\}




%%%%%%%%%%%%%%%%%%%%%%%%%%% PSEUDOCODE SECTION

\newcommand{\gt}{\ensuremath{:=}}   % assignment
%\newcommand{\gt}{\ensuremath{\leftarrow}}   % assignment
\newcommand{\swap}{\ensuremath{\leftrightarrow}}   % swap two vars

\newcommand{\pseudocode}[1]{\ensuremath{\mathbf{#1}}\xspace}
%\newcommand{\pseudocode}[1]{\ensuremath{\mathbf{#1}\ }}
\newcommand{\pseudocodensp}[1]{\ensuremath{\mathbf{#1}}}
%\newcommand{\pseudocode}[1]{\mbox{${\bf #1}$}\xspace}

\newcommand{\IFC}[1]{\pseudocode{if}\ (\ensuremath{#1})}
\newcommand{\WHILEC}[1]{\pseudocode{while}\ (\ensuremath{#1})}
\newcommand{\RETURNE}[1]{\pseudocodensp{return}(\ensuremath{#1})}

\newcommand{\IF}{\pseudocode{if}}
\newcommand{\FI}{\pseudocode{fi}}
\newcommand{\THEN}{\pseudocode{then}}
\newcommand{\ELSE}{\pseudocode{else}}
\newcommand{\ELSF}{\pseudocode{else\ if}}
\newcommand{\ENDIF}{\pseudocodensp{endif}}

\newcommand{\WHILE}{\pseudocode{while}}
\newcommand{\ENDWHILE}{\pseudocode{endwhile}}
\newcommand{\FOR}{\pseudocode{for}}
\newcommand{\FORALL}{\pseudocode{forall}}
\newcommand{\FOREACH}{\pseudocode{foreach}}
\newcommand{\CONTINUE}{\pseudocode{continue}}
\newcommand{\ENDFOR}{\pseudocodensp{endfor}}
\newcommand{\DO}{\pseudocode{do}}
\newcommand{\OD}{\pseudocode{od}}

\newcommand{\BEGIN}{\pseudocode{begin}}
\newcommand{\END}{\pseudocode{end}}
\newcommand{\PROC}{\pseudocode{procedure}}
\newcommand{\CALL}{\pseudocode{call}}
\newcommand{\VAL}{\pseudocode{value}}
\newcommand{\VALRES}{\pseudocode{value\!-\!result}}
\newcommand{\RES}{\pseudocode{result}}
\newcommand{\RETURN}{\pseudocodensp{return}}
\newcommand{\DOWNTO}{\pseudocode{downto}}
\newcommand{\TO}{\pseudocode{to}}

\newcommand{\function}{\pseudocode{function}}
\newcommand{\operation}{\pseudocode{operation}}

\newcommand{\newo}{\pseudocode{new}}
\renewcommand{\int}{\pseudocode{int}}

\newcommand{\skipp}{\pseudocode{skip}}






\newcommand{\csrtl}{\adjustbox{width=3.5em}{\ensuremath{{\cal C/S}}-\ensuremath{\mathit{RTL}}}\xspace}

\newcommand{\flfl}{\ensuremath{\mathit{ff}}\xspace}
\newcommand{\gflfl}{\ensuremath{\mathit{gff}}\xspace}
\newcommand{\gicflfl}{\ensuremath{\mathit{gicff}}\xspace}

\newcommand{\vflfl}{\ensuremath{\mathbf{ff}}\xspace}
\newcommand{\vgflfl}{\ensuremath{\mathbf{gff}}\xspace}
\newcommand{\vgicflfl}{\ensuremath{\mathbf{gicff}}\xspace}

\newcommand{\vI}{\ensuremath{\mathbf{I}}\xspace}
\newcommand{\vO}{\ensuremath{\mathbf{O}}\xspace}
\newcommand{\vRi}{\ensuremath{\mathbf{RI}}\xspace}
\newcommand{\vL}{\ensuremath{\mathbf{L}}\xspace}
\newcommand{\vDelta}{\ensuremath{\pmb{\delta}}\xspace}


%%%%%% macros for helper functions 
\newcommand{\bfit}[1]{\ensuremath{\textsc{\small  #1}}}
\newcommand{\htbullet}{\scalebox{0.7}{$\bullet$}\xspace}
\newcommand{\htfunc}[1]{\cellcolor{gray!10}\textbf{\textsc{\small #1}}}

\newcommand{\alwaysB}{\bfit{alwaysB}\xspace}
\newcommand{\clkedged}{\bfit{ClkEdged}\xspace}
\newcommand{\wireAssign}{\bfit{wireAssign}\xspace}
\newcommand{\asgnmtType}{\bfit{asgnmtType}\xspace}
\newcommand{\belongsTo}{\bfit{belongsTo}\xspace}
\newcommand{\compAStmt}{\bfit{compAStmt}\xspace}
\newcommand{\compCStmt}{\bfit{compCStmt}\xspace}
\newcommand{\createWire}{\bfit{createWire}\xspace}
\newcommand{\elseB}{\bfit{elseB}\xspace}
\newcommand{\exprsn}{\bfit{exprsn}\xspace}
\newcommand{\isPrm}{\bfit{isPrm}\xspace}
\newcommand{\selectRule}{\bfit{selectRule}\xspace}
\newcommand{\interconnectRule}{\bfit{interconnectRule}\xspace}
\newcommand{\ConcernWeave}{\bfit{ConcernWeave}\xspace}
\newcommand{\PrmRTLTransform}{\bfit{PrmRTLTransform}\xspace}
\newcommand{\PLIPCCompute}{\bfit{PLIPCCompute}\xspace}
\newcommand{\FFIPCCompute}{\bfit{FFIPCCompute}\xspace}
\newcommand{\followingPrmLps}{\bfit{followingPrmLps}\xspace}
\newcommand{\gffInst}{\bfit{gffInst}\xspace}
\newcommand{\gicffInst}{\bfit{gicffInst}\xspace}
\newcommand{\arrindex}{\bfit{index}\xspace}
\newcommand{\iterator}{\bfit{iterator}\xspace}
\newcommand{\incrementExp}{\bfit{incrementExp}\xspace}
\newcommand{\initExp}{\bfit{initExp}\xspace}
\newcommand{\loopB}{\bfit{loopB}\xspace}
\newcommand{\nestedPrmLps}{\bfit{nestedPrmLps}\xspace}
\newcommand{\parent}{\bfit{parent}\xspace}
\newcommand{\parentBlock}{\bfit{parentBlock}\xspace}
\newcommand{\precedes}{\bfit{precedes}\xspace}
\newcommand{\process}{\bfit{process}\xspace}
\newcommand{\replaceVar}{\bfit{replaceVar}\xspace}
\newcommand{\stmtType}{\bfit{stmtType}\xspace}
\newcommand{\target}{\bfit{target}\xspace}
\newcommand{\thenB}{\bfit{thenB}\xspace}
\newcommand{\vars}{\bfit{vars}\xspace}

\newcommand{\rtltransform}{\bfit{RTLTransform}\xspace}
\newcommand{\ipccompute}{\bfit{InstPathCondCompute}\xspace}
\newcommand{\ffInfer}{\bfit{ffInfer}\xspace}
\newcommand{\RegWireInst}{\bfit{RegWireInst}\xspace}
\newcommand{\SeqBlkConstruct}{\bfit{SeqBlkConstruct}\xspace}
\newcommand{\CombAssignConstruct}{\bfit{CombAssignConstruct}\xspace} %this (before:\CombRegWireConstruct) creates the initialization assignments 
\newcommand{\CombFLogicConstruct}{\bfit{CombFLogicConstruct}\xspace} %this (incorrectly before:\CombAssignConstruct) alters the  functional logic
%\newcommand{\CombRegWireConstruct}{\bfit{CombRegWireConstruct}\xspace}
%\newcommand{\CombAssignConstruct}{\bfit{CombAssignConstruct}\xspace}
\newcommand{\handleExprsn}{\bfit{handleExprsn}\xspace}
\newcommand{\CombBlkConstruct}{\bfit{CombBlkConstruct}\xspace}
\newcommand{\ProgConstruct}{\bfit{ProgConstruct}\xspace}
% parametrized 
\newcommand{\InstCondCompute}{\bfit{InstCondCompute}\xspace}
\newcommand{\PrmSeqBlkConstruct}{\bfit{PrmSeqBlkConstruct}\xspace}
\newcommand{\PrmCombBlkConstruct}{\bfit{\PrmCombBlkConstruct}\xspace}
\newcommand{\GnrtLpConstruct}{\bfit{GnrtLpConstruct}\xspace}
%\newcommand{\PrmCombAsgnConstruct}{\bfit{PrmCombAsgnConstruct}\xspace}
\newcommand{\LpAssignConstruct}{\bfit{LpAssignConstruct}\xspace}
\newcommand{\PrmCombAssignConstruct}{\bfit{PrmCombAssignConstruct}\xspace}
\newcommand{\InstPathCondCompute}{\bfit{InstPathCondCompute}\xspace}
\newcommand{\PathCondCompute}{\bfit{PathCondCompute}\xspace}
\newcommand{\IPCMutex}{\bfit{IPCMutex}\xspace}

%%%%%%%%%%%%%%%%%%%%%%%%%%%% Boolean Constants




\newcommand{\T}{\mbox{\rm T}}
\newcommand{\F}{\mbox{\rm F}}
\newcommand{\Fa}{\mbox{\rm F}}

\newcommand{\true}{\ensuremath{\mathit{true}}}
\renewcommand{\tt}{\ensuremath{\mathit{tt}}}
\newcommand{\vtt}{\ensuremath{\mathit{vtt}}\xspace}
%\newcommand{\true}{\mbox{\textsc{true}}\xspace}

\newcommand{\false}{\ensuremath{\mathit{false}}}
%\newcommand{\false}{\mbox{\textsc{false}}\xspace}

%\newcommand{\TT}{\top}
%\newcommand{\FF}{\bot}


\newcommand{\andw}{\mbox{ and }}

%\newcommand{\ANDW}{\mbox{\bf and \xspace}}
%\newcommand{\OR}{{\bf or} }
%\newcommand{\NOT}{{\bf not} }




%%%%%%%%%%%%%%%%%%%%%%%%% Boolean Connectives

\renewcommand{\iff}{\equiv}
%\renewcommand{\iff}{\Leftrightarrow}
\newcommand{\imp}{\Rightarrow}
\newcommand{\ev}{\equiv}
%\newcommand{\implies}{\Rightarrow}

%%%%%%%%%%%%%%%%%%%%%%%%%%%%%%%%%%% QUANTIFIERS
\newcommand{\qt}[3]{#1\,#2\,#3}
\newcommand{\qtr}[4]{(#1\,#2 : #3 : #4)}

\newcommand{\Q}{\mbox{${\bf Q}$}\xspace}
\newcommand{\q}{\mbox{${\bf q}$}\xspace}

%% quantifiers
\newcommand{\AND}{\bigwedge}
\newcommand{\INTER}{\bigcap}
\newcommand{\OR}{\bigvee}
\newcommand{\UN}{\bigcup}
\newcommand{\SUM}{\Sigma}
%\newcommand{\INT}{\bigcap}

%% Logical Quantifiers
\newcommand{\fa}{\ensuremath{\forall\,}}
%\newcommand{\fa}{\mbox{${\mathbf{\forall}}$}\xspace}
\newcommand{\ex}{\ensuremath{\exists\,}}
%\newcommand{\ex}{\mbox{${\mathbf{\exists}}$}\xspace}
\newcommand{\LQ}{\mbox{${\bf LQ}$}\xspace}


%% Arithmetic Quantifiers
%\renewcommand{\S}{\Sigma\,}
%\renewcommand{\S}{\mbox{${\mathbf{\Sigma}}$}\xspace}
\renewcommand{\P}{\Pi\,}
\newcommand{\N}{\ensuremath{\mathrm{N}\,}}
\newcommand{\AQ}{\ensuremath{\mathrm{AQ}\,}}
\newcommand{\MAX}{\ensuremath{\mathrm{MAX}\,}}
\newcommand{\MIN}{\ensuremath{\mathrm{MIN}\,}}

%%%%%%%%%%%%%%%%% ENVIRONMENTS
\newcommand{\bd}{\begin{definition}}
\newcommand{\ed}{\end{definition}}

%%%%%%%%%%%%%%%%% EQUATION ENVIRONMENTS
\newcommand{\beqn}{\begin{centeqn}}
\newcommand{\eeqn}{\end{centeqn}}
\newcommand{\beqnnbsp}{\begin{centeqn-nbsp}}
\newcommand{\eeqnnbsp}{\end{centeqn-nbsp}}
\newcommand{\bleqn}[1]{\begin{centlabeqn}{#1}}
\newcommand{\eleqn}{\end{centlabeqn}}
\newcommand{\bleqnnbsp}[1]{\begin{centlabeqn-nbsp}{#1}}
\newcommand{\eleqnnbsp}{\end{centlabeqn-nbsp}}

\newenvironment{centeqn}	% centered equation environment
   {{\ss\\ \hspace*{\fill}}}
   {\hspace*{\fill}\ss\\}

\newsavebox{\EqnLabel}
\newenvironment{centlabeqn}[1]	% centered labeled equation environment
   {\sbox{\EqnLabel}{#1}
    {\medskip\\ \hspace*{\fill}}
   }
   {\hfill{\makebox[0in][r]{\usebox{\EqnLabel}}}\medskip\\}

\newenvironment{centeqn-nbsp} % centered equation - no vertical space at the bottom
   {{\medskip\\ \hspace*{\fill}}}
   {\hspace*{\fill}}

%%%%%%%%%%%%%%%% SpecCheck Specific macros

\newcommand{\cd}[1]{{\small\texttt{#1}}}
\newcommand{\cci}[1]{\mbox{\small \textup{\texttt{#1}}}}
%\newcommand{\cci}[1]{{\small\texttt{#1}}}
%\newcommand{\chci}[1]{`\cci{#1}'}
\newcommand{\chci}[1]{\cci{#1}}

\newcommand{\good}{\cci{good}\xspace}
\newcommand{\bad}{\cci{bad}\xspace}
\newcommand{\dc}{\cci{dontCare}\xspace}
\newcommand{\tgood}{\tiny\texttt{good}\xspace}
\newcommand{\tbad}{\tiny\texttt{bad}\xspace}
\newcommand{\tdc}{\tiny\texttt{dontCare}\xspace}

\newcommand{\ord}{\sqsubseteq}
\newcommand{\ordne}{\sqsubset}

\newcommand{\white}{\ensuremath{\mathit{white}}\xspace}
\newcommand{\black}{\ensuremath{\mathit{black}}\xspace}

\newcommand{\ls}{\cd{ls}}
\newcommand{\af}{f}

\newcommand{\vit}{{\cal T}_I}
\newcommand{\vst}{{\cal T}_S}
\newcommand{\vocrm}{\mathit{voc}_R^M}
\newcommand{\tp}{\mathit{type}}
\newcommand{\tll}{\mathit{tail}}
\newcommand{\hdd}{\mathit{head}}
\newcommand{\Data}{\mathrm{Data}}
\newcommand{\func}{\mathit{func}}
\newcommand{\port}{\mathit{port}}
\newcommand{\export}{\mathit{export}}
\newcommand{\guard}{\mathit{guard}}
\newcommand{\source}{\mathit{src}}
\newcommand{\dest}{\mathit{dest}}
%\DeclareMathOperator{\var}{var}

\newcommand{\ignore}[1]{}
%\newcommand{\ie}{i.e.,}
\newcommand{\secref}[1]{Section~\ref{#1}}
\newcommand{\figref}[1]{Fig.~\ref{#1}}
\newcommand{\defas}{\stackrel{\mathrm{\scriptscriptstyle def}}{=}}

%\newcommand{\true}{\ensuremath{\mathtt{true}}}
%\newcommand{\false}{\ensuremath{\mathtt{false}}}
\newcommand{\goesto}[1][]{\stackrel{#1}{\longrightarrow}} % -->
\newcommand{\valu}[1]{\mathbf{#1}}

\newcommand{\ff}{\ensuremath{\vec{ff}}}
\newcommand{\RI}{\ensuremath{\vec{RI}}}
\newcommand{\I}{\ensuremath{\vec{I}}}
\newcommand{\OO}{\ensuremath{\vec{O}}}
\newcommand{\dlt}{\ensuremath{\vec{\delta}}}
\newcommand{\LL}{\ensuremath{\vec{L}}}
\newcommand{\itodo}[1]{\todo[inline,color=green]{FZ: #1}}
%\newcommand{\itodo}[1]{}

%by Maya
\newcommand\codebold[1]{\textbf{#1}}
\definecolor{darkblue}{rgb}{0.0,0.0,0.6}
\definecolor{darkgreen}{rgb}{0.0,0.6,0.0}
\definecolor{lightgray}{gray}{0.93}
\definecolor{darkblue}{rgb}{0.0,0.0,0.6}
\definecolor{darkgreen}{rgb}{0.0,0.6,0.0}
\definecolor{darkred}{rgb}{0.6,0.0,0.0}
\definecolor{atomictangerine}{rgb}{1.0, 0.6, 0.4}
\definecolor{auburn}{rgb}{0.43, 0.21, 0.1}
\definecolor{arylideyellow}{rgb}{0.91, 0.84, 0.42}
\definecolor{brass}{rgb}{0.71, 0.65, 0.26}
\definecolor{burgundy}{rgb}{0.5, 0.0, 0.13}
\definecolor{bulgarianrose}{rgb}{0.28, 0.02, 0.03}
\definecolor{chocolate}{rgb}{0.48, 0.25, 0.0}
\definecolor{darkraspberry}{rgb}{0.53, 0.15, 0.34}
\definecolor{lightbrown}{rgb}{0.71, 0.4, 0.11}
\definecolor{brown(web)}{rgb}{0.65, 0.16, 0.16}
\definecolor{darkbrown}{rgb}{0.4, 0.26, 0.13}
\definecolor{magenta(process)}{rgb}{1.0, 0.0, 0.56}
\definecolor{background}{HTML}{EEEEEE}
\definecolor{lightviolet}{RGB}{93,71,139}
\definecolor{lighterviolet}{RGB}{159,121,239}
\definecolor{javadocblue}{rgb}{0.25,0.35,0.75} % javadoc

%%%%%%%%%%%%%%%%%%%%%%%%%%%%%%%%%%%%%%%%%%%%%%%%%%%%%%%%%%%%%%%%%%%%%%%%%%%%%%
\lstset{ %
language=C,                % choose the language of the code
basicstyle=\tiny,       % the size of the fonts that are used for the code
numberstyle=\tiny,      % the size of the fonts that are used for the line-numbers
frame=tb,
rulesep=.4pt,
mathescape, % Allows escaping to (La)TeX mode within $$,
stringstyle=\color{redb},
morekeywords={Boolean, dotogether, foreach, endfor, endif, Let},
showstringspaces = false,
basicstyle=\scriptsize\ttfamily,
keywordstyle=\color{darkblue}\bf,
commentstyle=\color{darkgreen},
numbers=left,
stepnumber=1,
%
breaklines=true
}

\lstdefinestyle{c-code}{%
  language=C,                % choose the language of the code
  basicstyle=\tiny,       % the size of the fonts that are used for the code
  numberstyle=\tiny,      % the size of the fonts that are used for the line-numbers
  frame=Ltbr,
  rulesep=.4pt,
  mathescape, % Allows escaping to (La)TeX mode within $$,
  stringstyle=\color{red},% \color{redb}
  %keywords=[1]{elseif, return, else, int, bool, if, Boolean, do-together, foreach, endfor, endif, while, to, Let},
%keywords=[1]{declaration,program,list-of-statements,statement,conditional,loop,sync,assignment,target,bexpr,expr,return-statement,type,variable-decl,function-decl,property-decl,precondition,postcondition,property,quantifier,range,term,id,access,function-call,call-argument,argument-list,call-arg-list},
 % keywords=[2]{@dotogether,@pre,@post,if,while,return,forall,exists,int,bool,num,else,constant},
  showstringspaces = false,
  escapeinside={/*@}{@*/},
  basicstyle=\scriptsize\ttfamily,
   keywords=[1]{foreach,endfor,in,if,elseif,endif,to,else,let,be},
   keywords=[2]{append,next-list,v-next-list,v-expr,pc-next-list,decl-list,init-list,a-index,a-expr,a-next-statement},
  keywordstyle=[3]\color{darkbrown}\bf,
  keywordstyle=[2]\color{darkred}\bf,
  keywordstyle=[1]\color{darkblue}\bf,
  commentstyle=\color{darkgreen},
  alsoletter={-0123456789},
  captionpos=b, %
  breaklines=true,
  backgroundcolor=\color{background},
}

%\lstdefinelanguage{veriloggg}{
%    language=Verilog,
%    alsoletter={&},
%}
\lstdefinestyle{verilog-code}{%
  language=Verilog,                % choose the language of the code
  %basicstyle=\tiny,       % the size of the fonts that are used for the code
  numbers=none, %add numbers=left if want numbers to appear on left of the code lines
  numberstyle=\tiny,      % the size of the fonts that are used for the line-numbers
  frame=none,
  rulesep=.4pt,
  mathescape, % Allows escaping to (La)TeX mode within $$,
  stringstyle=\color{red},% \color{redb}
  %keywords=[1]{elseif, return, else, int, bool, if, Boolean, do-together, foreach, endfor, endif, while, to, Let},
  %keywords=[1]{declaration,program,list-of-statements,statement,conditional,loop,sync,assignment,target,bexpr,expr,return-statement,type,variable-decl,function-decl,property-decl,precondition,postcondition,property,quantifier,range,term,id,access,function-call,call-argument,argument-list,call-arg-list},
 % keywords=[2]{@dotogether,@pre,@post,if,while,return,forall,exists,int,bool,num,else,constant},
  showstringspaces = false,
  escapeinside={/*@}{@*/},
  basicstyle=\scriptsize\ttfamily,
  identifierstyle=\color{blue}\it,
  keywords=[1]{always,reg,wire,posedge,negedge,assign,else,if,begin,end,parameter,module,endmodule,input,output, inputs, outputs,integer, or, generate, for, and, xor,dff, dff2},
  keywords=[2]{0,1,2,3,4,5,6,7,8,9},
  keywords=[3]{@,<,<=,==,>,>=,||,&&,!=},%,\&,!,\#,+,-,<,=,(,),;,[,]},
  keywords=[4]{lssd,gff,gicff},
  keywordstyle=[4]\color{darkgreen}\bf,
  keywordstyle=[3]\color{lightviolet}\bf,
  keywordstyle=[2]\color{red},
  keywordstyle=[1]\color{darkred}\bf,
  commentstyle=\color{darkgreen},
  alsoletter={-0123456789},
  captionpos=b, %
  breaklines=true,
backgroundcolor={},
%  backgroundcolor=\color{white},
 literate=
            {\&}{{\textcolor{lightviolet}{\&}}}{1}
            {[}{{\textcolor{lighterviolet}{[}}}{1}
            {]}{{\textcolor{lighterviolet}{]}}}{1}
            {(}{{\textcolor{lighterviolet}{(}}}{1}
            {)}{{\textcolor{lighterviolet}{)}}}{1}
            {!}{{\textcolor{lightviolet}{!}}}{1}
            {<=}{{\textcolor{lightviolet}{<=}}}{1}
            {=}{{\textcolor{lightviolet}{=}}}{1}
            {;}{{\textcolor{lightviolet}{;}}}{1},
}


\lstdefinestyle{alg}{%
  language=matlab,                % choose the language of the code
  %basicstyle=\tiny, % the size of the fonts that are used for the code
  numberstyle=\normal,% the size of the fonts that are used for the line-numbers
  frame=none, %Ltbr,
  rulesep=1pt,
  mathescape, % Allows escaping to (La)TeX mode within $$,
  stringstyle=\color{red},% \color{redb}
  %keywords=[1]{elseif, return, else, int, bool, if, Boolean, do-together, foreach, endfor, endif, while, to, Let},
%keywords=[1]{declaration,program,list-of-statements,statement,conditional,loop,sync,assignment,target,bexpr,expr,return-statement,type,variable-decl,function-decl,property-decl,precondition,postcondition,property,quantifier,range,term,id,access,function-call,call-argument,argument-list,call-arg-list},
 % keywords=[2]{@dotogether,@pre,@post,if,while,return,forall,exists,int,bool,num,else,constant},
  showstringspaces = false,
  escapeinside={/*@}{@*/},
  %basicstyle=\scriptsize,
Add a comment to this line
  identifierstyle=\color{blue}\it,
%
  keywords=[1]{foreach,endfor,in,elseif,endif,to,let,Then,If,For,Else,such,that,if,then,else},
  keywords=[2]{for,end,begin,generate},
  keywords=[3]{
  InstPathCondCompute, InstPathCondCompute1,  InstPathCondCompute2,  InstPathCondCompute3,  InstPathCondCompute4,  
  iterator,loopB, ThenB, ElseB, nestedPrmLps,followingPrmLps, parentBlock,replaceVar,
  stmtType,initExp,incrementExp,exprsn,clkEdged,IPCcompute,ffInfer,RegWireInst,
  SeqBlkConstruct,RTLTransform,CombRegWireConstruct,Process,
  CombAssignConstruct,CombBlkConstruct,ProgConstruct,target,belongsTo,createWire,
  lookUpWire,gffInst,gicffInst,vars,asgnmtType,handleExprsn,compCStmt,compAStmt,compLStmt,init,increment,InstCondRTLTransform,
  PathCondCompute,PLIPCCompute,FFIPCCompute,IPCMutex,
  InstCondCompute,PrmSeqBlkConstruct,PrmCombBlkConstruct,GnrtLpConstruct,PrmCombAsgnConstruct,LpAssignConstruct,index},
  keywords=[4]{Before,After,},
  keywords=[5]{gff,gicff,},
%
  keywordstyle=[5]\color{darkgreen}\bf,
  keywordstyle=[4]\color{black}\it, 
  keywordstyle=[3]\color{black}\sc, 
  keywordstyle=[2]\color{darkred}\bf,
  keywordstyle=[1]\color{darkblue}\bf,
%
  commentstyle=\color{darkgreen},
  morecomment=[l][\color{darkgreen}\bf]{//},
  morecomment=[s][\color{javadocblue}\bf]{/**}{*/},
  alsoletter={-0123456789'},
  captionpos=b, %
  breaklines=true,
  backgroundcolor=\color{lightgray},
}



\newcommand{\answer}[1]{{\bf Response:~}\\ {\color{blue}#1} $\blacksquare$} 
