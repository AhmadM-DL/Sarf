\vocalize

\def\pp{\ensuremath{{\cal P}}} % prefix
\def\ss{\ensuremath{{\cal S}}} % stem
\def\xx{\ensuremath{{\cal X}}} % suffix
\def\PP{\ensuremath{\mathit{POS}}} % pos
\def\GG{\ensuremath{\mathit{GLOSS}}} % gloss
\def\AC{\ensuremath{\mathit{CAT}}} % category

%~\footnote{In this document, we use the default ArabTeX transliteration style ZDMG.}
%{\bf Morphological analyzer (Sarf).~}
Morphological analysis is key to Arabic 
NLP
due to the exceptional degree of ambiguity in writing, 
the rich morphology, and the complex word derivation 
system~\cite{arabicmorph,shahrour2016camelparser,pasha2014madamira}.
Short vowels, also known as diacritics, are typically omitted in Arabic text
and inferred by readers~\cite{habash2006arabic}. 
For example, the word \RL{bn}%
can be interpreted as \RL{bon} {\tt ``coffee''} with a {\em damma} diacritic on the 
letter \utfrl{بـ} or 
\vocalize \RL{bin} (son of) with a 
{\em kasra} diacritic on the letter \utfrl{بـ}.

Morphological analysis is required even for tokenization of Arabic text. 
The position of an Arabic letter in a word 
(beginning, middle, end, and standalone) changes
its visual form.
Some letters have non-connecting end forms which allows visual
word separation without the need of a white space separator. 
For example, the word \utfrl{ياسمين} can be interpreted as
the ``Jasmine'' flower, 
as well as \notrutfrl{يا} (the calling word) followed by
the word \notrutfrl{سمين} (obese). 
Consider the sentence 
%\utfrl{ذهب الولدإلى المدرسة} 
\notrrl{AlmdrsT}\notrrl{_dhb alwald-il_A}
\arabfalse \RL{_dhb alwald-il_A almdrsT} \arabtrue
(the kid went to school). 
The letters \notrutfrl{د} and \notrutfrl{ى} have 
non-connecting end of word forms and the words 
\notrutfrl{الولد},\notrutfrl{الى}, and\notrutfrl{المدرسة} 
are visually separable, 
yet there is no space character in between.
Newspaper articles with text justification requirements, 
SMS messages, and automatically digitized documents
are examples where such problems occur. 

\framework is integrated with {\em Sarf}, 
an in-house open source Arabic morphological analyzer based on 
finite state transducers~\cite{ZaMaColing2012DemosSarf}. 
Given an Arabic word, Sarf returns 
a set of morphological solutions. 
A word might have more than one solution 
due to multiple possible segmentations and multiple tags associated 
with each word. 
A morphological solution is the internal structure of the word 
composed of several morphemes including 
{\em affixes} ({\em prefixes} and {\em suffixes}), and a
{\em stem}, where each morpheme is associated with tags such as 
POS, gloss, and category tags~\cite{arabicmorph,habash2010introduction}.

Prefixes attach before the stem and a word can have multiple prefixes. 
Suffixes attach after the stem and a word can have multiple suffixes. 
Infixes are inserted inside the stem to form a new stem. 
In this work we consider a set of stems that includes infix morphological changes. 
The part-of-speech tag, referred to as POS, 
assigns a morpho-syntactic tag for a morpheme. 
The gloss is a brief semantic notation of morpheme in English. 
A morpheme might have multiple glosses as it could stand for multiple meanings. 
The category is a custom tag that we assign to multiple morphemes. 
For example, we define the {\tt Name of Person} category to include proper names.
%The category is a user defined tag that the user can assign to several 
%morphemes.
%For example, the user can define a temporal category to include the 
%prefix \utfrl{س} (will) and the time unit \utfrl{ساعة} (hour). 

We denote by 
\ss,
\pp,
\xx,
\PP,
\GG, and 
\AC, the set of 
all stems,
prefixes,
suffixes,
POS,
gloss, 
and user defined category tags, respectively.
Let $T=\langle t_1,t_2,\ldots,t_M\rangle$ be a set of Arabic words denoting the 
text documents.
\framework uses Sarf to compute a set of morphological solutions $M(t)=\{m_1,m_2,\ldots,m_N\}$
for each word $t\in T$. 
%
Each morphological solution $m\in M(t)$ is a tuple of the form 
$\langle p,s,x,P,G,C\rangle \in \pp \times \ss \times \xx \times \PP \times \GG \times \AC$ 
where $p=p_1 \ldots p_{|p|}$, $x=x_1 \ldots x_{|x|}$, 
$P=P_{p_1}\ldots P_{p_{|p|}} P_s P_{x_1}\ldots P_{x_{|x|}}$,
$G=G_{p_1}\ldots G_{p_{|p|}} G_s G_{x_1}\ldots G_{x_{|x|}}$, and
$C=C_{p_1} \ldots C_{p_{|p|}} C_s C_{x_1}\ldots C_{x_{|x|}}$. 
$P_{p_i}, G_{p_i},$ and $C_{p_i},1\le i \le |p|$ are the POS, Gloss and 
category tags of prefix $p_i$. 
$P_{x_j}, G_{x_j},$ and $C_{x_j},1\le j \le |x|$ are the POS, Gloss and 
category tags of suffix $x_i$. 
$P_s, G_s,$ and $C_s$ are the POS, Gloss and 
category tags of stem $s$. 
%
Intuitively, 
$p,x,P,G$ and $C$ are concatenations of prefix, suffix, POS, Gloss and
Category values, respectively.

%\setarab
\begin{table}[tb]
  \centering
  \resizebox{0.85\columnwidth}{!}{
    \begin{tabular}{|r|c|c|c|c|c|}
          & \multicolumn{3}{c|}{\textbf{Prefixes}} & \textbf{Stem} & \textbf{Suffix} \\
    \textbf{Data} & \utfrl{فَ} & \utfrl{سَ} & \utfrl{يَ} & \utfrl{أْكُل} & \utfrl{ها} \\
    \textbf{POS} & CONJ+ & FUT+ & IV3MS+ & VERB\_IMPERFECT & IVSUFF\_DO:3FS \\
    \textbf{Gloss} & and/so & will & he/it & eat/consume & it/them/her \\
    \textbf{index} & \multicolumn{3}{c|}{10} & 13 & 16 \\
    \textbf{length} & \multicolumn{3}{c|}{3} & 3 & 2 \\
    \end{tabular}%
    }
    \vspace{-1em}
  \caption{\label{tab:samplesolution}Sample solution vector for \utfrl{فَسَيَأْكُلها}.}
\end{table}%

Table~\ref{tab:samplesolution} shows the morphological analysis
of the word \notrutfrl{فَسَيَأْكُلها}. 
The word is composed of the prefix morphemes 
\utfrl{فَ}, \utfrl{سَ}, and \utfrl{يَ}, followed by the 
stem \utfrl{أْكُل}, and then followed by the suffix morpheme
\utfrl{ها}. 
Each morpheme is associated with a number of morphological features.
The {\tt CONJ},
{\tt FUT}, 
{\tt IV3MS} 
{\tt VERB\_IMPERFECT}, and 
{\tt IVSUFF\_DO:3FS} POS tags indicate
conjunction, 
future, 
third person masculine singular subject pronoun,
an imperfect verb, and 
a third person feminine singular object pronoun, respectively.
The POS and gloss notations follow the Buckwalter notation~\cite{Buckwalter:02}.
