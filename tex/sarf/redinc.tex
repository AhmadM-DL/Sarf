\subsection{Redundancy} 
\label{s:s:redundant}

Consider the partial lexicon of prefixes in Table~\ref{t:affixes}. 
\todo{R1.9.}The first five rows can be replaced with two atomic affix morphemes ( \RL{f} and \RL{y})  
and one partial affix morpheme (\RL{s}) in $L_{p}$, 
and three rules to generate compound morphemes in $R_{pp}$ (r (\RL{f}, \RL{y}), r (\RL{f}, \RL{s}), r (\RL{s}, \RL{y})). 
Representing prefix \RL{ya} (them/both) required four entries, three of them only differ
in their dependency on the added \RL{ya}. 
Representing prefix \RL{w} required the addition of five entries.
\todo{1.10}
With Sarf, the equivalent addition of \RL{ya} (them/both) requires only two rules in $R_{pp}$ that relate the categories of \RL{f} and \RL{s} to that of \RL{y}. 
The addition of \RL{w} requires only one additional entry (\RL{w}) in $L_p$.
The difference is much larger when we consider the full lexicon as will be shown 
in Section~\ref{sec:results}.

\subsection{Inconsistencies} 
\label{s:s:inc}

The entries in Tables~\ref{t:incBAMA} and ~\ref{t:incSAMA} list examples of the 197 and 208 inconsistencies detected in the affix 
lexicons of BAMA version 1.2 and SAMA version 3.1, respectively.
We found a small number of these inconsistencies manually and we computed the full list via comparing $L_p$ and $L_x$ with 
their counterparts computed using our agglutinative affixes. 
Most of the inconsistencies are direct results of partially redundant entries with erroneous tags. Experiments described later in section \ref{s:s:lexicon_consistency} shows that most of the detected inconsistencies are specifically related to gloss (Table~\ref{t:comp:SAMA}). \todo{3.12}
We note that SAMA corrected several BAMA inconsistencies, but also introduced 
several new ones when modifying existing entries to meet new standards and when introducing new entries. 

%%%%%%%%%%%% BAMA inconsistencies table %%%%%%%%%%%%%%%%
\begin{table}[!tbp]
\begin{minipage}{0.95\textwidth}
\relsize{-1.5} {
  \begin{tabular}{p{2.5cm}ccc}
\hline \hline
   & {\bf Affix } & \textbf{Vocalized} & {\bf Inconsistent tag} \\ 
\hline
\multirow{2}{2.5cm}{ {\bf \ref{inc:plus})}
missing plus in {\bf gloss} tag  of {\bf prefix}}
 & \RL{fl} & \RL{fali} & and/so \fbox{+} for/to + the  \\
% & \RL{fl--l} & \RL{falil} & and/so \fbox{+} for/to + the \\
 & \RL{wbAl} & \RL{wabiAl} & and + with/by \fbox{+} the  \\[6pt]
\multirow{3}{2.5cm}{{\bf \ref{inc:so})} 
  missing alternative {\bf gloss} %for fa
  in {\bf prefix} }
 & \RL{f} & \RL{fa} & and/so  \\ 
 & \RL{fb} & \RL{fabi} & and \fbox{/so} + with/by          \\
 & \RL{fk} & \RL{faka} & and \fbox{/so} + like/such as     \\[6pt]
% & \RL{fl} & \RL{fali} & and \fbox{/so} + for/to           \\
% & \RL{fl} & \RL{fali} & and \fbox{/so} + for/to + the    \\
\multirow{4}{2.5cm}{ {\bf \ref{inc:num_gender_subject}) } 
  gender/number qualifier omitted in {\bf gloss} of subject {\bf suffix} }
 & \RL{n} & \RL{n} & they [fem.pl.] $<$verb$>$                     \\
 & \RL{nhm} & \RL{nahom} & they \fbox{[fem.pl.]} $<$verb$>$ them \\
 & \RL{At} & \RL{At} & [fem.pl.]                                \\
 & \RL{Atk} & \RL{Atka} & \fbox{[fem.pl.]} your                \\[6pt]
% & \multicolumn{4}{c}{48 total inconsistencies} \\ 
\multirow{2}{2.5cm}{
  {\bf \ref{inc:num_gender_object}) } 
  also in {\bf gloss} of object {\bf suffix} }
 & \noVocRL{AnhmA}&  \RL{AnihimA}& them \fbox{(both)}                      \\
 & \noVocRL{tkm}& \utfRL{|Bَتْكُم}& it/they/she $<$verb$>$ you \fbox{(pl.)}  \\[6pt]
% & \multicolumn{4}{c}{39 total inconsistencies} \\ 
\multirow{3}{2.5cm}{
  {\bf \ref{inc:both}) } 
  different ways to express them (both) in {\bf gloss} of {\bf suffix} }
 &\noVocRL{thmA}&  \utfRL{|Bَتْهُما} & it/they/she $<$verb$>$ them (both)    \\
 & \noVocRL{AhmA} &  \utfRL{|BBاهُما} & we $<$verb$>$ 
 \colorbox{shadecolor}{\color{white} (both of)} them \fbox{(both)}  \\
\vocalize
 & \RL{nAhmA} &  \utfRL{nAhُmA} & we $<$verb$>$ 
\colorbox{shadecolor}{\color{white} (both of)} them \fbox{(both)}   \\[6pt]
% & \RL{nhmA} &  \utfRL{nَhُmA} & they $<$verb$>$ 
%        \colorbox{shadecolor}{\color{white} (both of)} them \fbox{(both)}  \\
% & \RL{hmA} &  \utfRL{|BBهُما} & they $<$verb$>$ 
%        \colorbox{shadecolor}{\color{white} (both of)} them \fbox{(both)}  \\
\multirow{2}{2.5cm}{
  {\bf \ref{inc:dot}) } 
  `.' omitted  after pl in {\bf gloss} }
 & \utfRL{م} & \utfRL{|Bُّم} & you [masc.pl.] $<$verb$>$              \\
% & \utfRL{وا} & \utfRL{|Bُوا} & you [masc.pl\fbox{.}] $<$verb$>$     \\ 
 & \utfRL{ونا} & \utfRL{|Bُونا} & you [masc.pl\fbox{.}] $<$verb$>$ us \\[6pt]
% & \multicolumn{4}{c}{8 total  inconsistencies } \\        
\multirow{2}{2.5cm}{
  {\bf \ref{inc:pos}) } 
  {\bf POS} tag is not same as vocalized 
}
 & \noVocRL{th} & \utfRL{thِ}     & +ti/PVSUFF\_SUBJ:2FS  \\
 &              &                & +\colorbox{shadecolor}{\color{white} hu/}\fbox{hi/}PVSUFF\_DO:3MS \\
% & \utfRL{ونا}  & \utfRL{|Bَوْنا} & +\colorbox{shadecolor}{\color{white} uw/}\fbox{aw/}PVSUFF\_SUBJ:3MP  \\ 
% &              &               & +nA/PVSUFF\_DO:1P \\
%% & \multicolumn{4}{c}{38 total inconsistencies} \\         
\hline \hline
\end{tabular}
}
\end{minipage}
\caption{Sample BAMA inconsistencies}
\label{t:incBAMA}
\end{table}



The following describes the BAMA inconsistencies illustrated in Table~\ref{t:incBAMA}: 

\begin{enumerate}[(a)]
    %%%%%%%%%%%%%%%%%%% BAMA prefixes
\item \label{inc:plus}  $L_p$ omits a plus (+) symbol that indicates boundaries in compound prefixes. 
\item \label{inc:so}    $L_p$ omits the (so) alternative gloss that corresponds to \RL{f-} in several compound prefixes.
    %%%%%%%%%%%%%%%%%%% BAMA suffixes
\item \label{inc:num_gender_subject} $L_x$ omits gender and number qualifiers 
  that appear within within square brackets
    from several glosses of subject suffixes.
                % not very precise seems sometimes also uses () for subjects like:
                %اكن    اكُنَّ  PVSuff-Ah       they (both) <verb> you (women)  +A/PVSUFF_SUBJ:3MD+kun~a/PVSUFF_DO:2FP  
\item \label{inc:num_gender_object} $L_x$ omits gender and number qualifiers 
    from several glosses of subject suffixes that appear within parenthesis.
\item \label{inc:both} $L_x$
    expresses the dual quantifier as `them (both)' in the majority of the entries, and 
    as `(both of) them' in several entries.
\item \label{inc:dot} $L_x$ omits the dot (`.') symbol from the gloss abbreviation of plural.
\item \label{inc:pos} $L_x$ contains POS tags that are not consistent with the semantics of the vocalized tags for compound affixes.
\end{enumerate}
\

%%%%%%%%%%%% SAMA inconsistencies table %%%%%%%%%%%%%%%%
\begin{table}[!tbp]
\resizebox{0.95\textwidth}{!}{
%\begin{minipage}{0.8\textwidth}
\relsize{-1.5} {
\begin{tabular}{p{2.5cm}cccc}
\hline \hline
 & {\bf Affix } & \textbf{Vocalized} & {\bf Inconsistent tag} & \\ \hline
\multirow{2}{2.5cm}{
  {\bf \ref{inc:relaxed})} 
  missing standalone alef with no hamza {\bf prefix}  forms
}
% & \RL{A} & \utfRL{أَ} & I \\ 
 & \RL{A} & \utfRL{آ} & I  \\ 
% & \RL{A} & \utfRL{أُ} & I  \\
 & \RL{sA} & \utfRL{سَأُ} & I \\[6pt]
\multirow{2}{2.5cm}{
  {\bf \ref{inc:by}) } 
   additional by in {\bf gloss} 
}
 & \RL{w} & \RL{wa} & with      \\ 
 & \RL{wAl-} & \RL{waAl-} & with\colorbox{shadecolor}{\color{white} /by} + the \\[6pt]
\multirow{1}{2.5cm}{ {\bf \ref{inc:fa}) } 
  additional space in vocalized form 
}
& \RL{f-} & \colorbox{shadecolor}{~}\RL{fa-}\colorbox{shadecolor}{~} \\[6pt]
\multirow{1}{2.5cm}{ {\bf \ref{inc:wafa})} 
   wrong {\bf prefix}
}
& \RL{wf-}& \RL{wafa-} & and + so/and \\[6pt]
\multirow{3}{2.5cm}{ {\bf \ref{inc:def}) } 
  missing definite indicator in {\bf suffix gloss} 
}
 & \utfRL{آت} & \utfRL{آتِ} & [fem.pl.] + [def.acc.]                               \\
 & \utfRL{اتك} & \utfRL{اتِكِ} & [fem.pl.] + [def.acc.] + your [fem.sg.]            \\
 & \utfRL{آتك} & \utfRL{  آتِكِ} & [fem.pl.] + [\fbox{def.}acc.] + your [fem.sg.]  \\[6pt]
% & \multicolumn{4}{c}{33 other inconsistencies} \\ 
\multirow{2}{2.5cm}{ {\bf \ref{inc:num_gender}) }
  omitted gender/num  qualifier in {\bf gloss}
}
 & \utfRL{اك}&  \utfRL{|BBاكِ}& we [verb] + you \fbox{[fem.sg.]}          \\
 & \utfRL{نهم}&  \utfRL{|Bْنَهُم}& they \fbox{[fem.pl.]} [verb] + them      \\[6pt]
% & \multicolumn{4}{c}{72 total inconsistencies} \\ 
\multirow{3}{2.5cm }{  {\bf \ref{inc:both:sama}) } 
   different ways to express them (both) in {\bf gloss} of {\bf suffix} 
}
 & \noVocRL{thmA}&  \utfRL{|Bَتْهُما} & it/they/she [verb] them (both)   \\
 & \noVocRL{AhmA} &  \noTrutfRL{|BBاهُما} \noArutfRL{ّاهُما}& we [verb] 
                \colorbox{shadecolor}{\color{white} (both of)} them \fbox{(both)}   \\
 & \noVocRL{nAhmA} &  \noVocRL{nAhomA} & we [verb] 
                \colorbox{shadecolor}{\color{white} (both of)} them \fbox{(both)}   \\[6pt]
% & \noVocRL{nhmA} &  \noVocRL{nahomA} & they [verb]
%                \colorbox{shadecolor}{\color{white} (both of)} them \fbox{(both)}  \\
% & \noVocRL{hmA} &  \utfRL{|BBهُما} & they [verb[ 
%\colorbox{shadecolor}{\color{white} (both of)} them \fbox{(both)}  \\
\multirow{3}{2.5cm}{ {\bf \ref{inc:num_gender_differences}) } 
  leftover BAMA style tags  in {\bf gloss}
}
 & \utfRL{كم} & \utfRL{كُم} & he/it [verb] + you [masc.pl.]        \\
 & \utfRL{تكم} & \utfRL{تُكُم} & I [verb] + you 
                \colorbox{shadecolor}{\color{white}(pl.)} \fbox{[masc.pl.]}  \\
 & \utfRL{كن} & \utfRL{|Bُّكُنَّ}  & I [verb] + you 
                \colorbox{shadecolor}{\color{white}(women) } \fbox{[fem.pl.]}   \\[6pt]
% & \multicolumn{4}{c}{13 other inconsistencies } \\                
\multirow{3}{2.5cm}{ {\bf \ref{inc:jus}) } 
  indicative {\bf gloss} with jussive {\bf POS}
}
% & ~ & ~ & [jus.] ~ ~ ~ (null)/IVSUFF\_MOOD:J  \\ 
 & \utfRL{نا} & \utfRL{نا} & 
       IVSUFF\_MOOD:J,
       \colorbox{shadecolor}{\color{white}[ind.]} \fbox{[jus.]} + us  \\
 & \utfRL{ني} & \utfRL{نِي} & 
       IVSUFF\_MOOD:J,
       \colorbox{shadecolor}{\color{white}[ind.]} \fbox{[jus.]} + me \\[6pt]
\multirow{3}{2.5cm}{ {\bf \ref{inc:dot:SAMA}) } 
   omitted `.'  in {\bf gloss} 
}
 & & \utfRL{|Bُ} & [def.nom.] & \\ 
 & \utfRL{ه} & \utfRL{|Bُهُ} & [def.nom \fbox{.}]  \\
 & \utfRL{تكن} & \utfRL{|Bَتُكُنَّ} & 
[fem.sg.] + [def.nom \fbox{.}] + your [fem.pl \fbox{.}]  \\[6pt]
% & \multicolumn{4}{c}{36 other inconsistencies } \\
\vocalize
\multirow{2}{2.5cm} { {\bf \ref{inc:pos:SAMA})} 
  shadda inconsistent in {\bf POS} 
}
 & \utfRL{ي} & \utfRL{يَ} & ya/POSS\_PRON\_1S \\
 & \utfRL{ي} & \utfRL{|Bِyya} & 
 % iy/NSUFF\_MASC\_PL\_NOM  +\colorbox{shadecolor}{\color{white}$\sim$a/}\fbox{ya/}POSS\_PRON\_1S  \\
 $\ldots$ +\colorbox{shadecolor}{\color{white}$\sim$a/}\fbox{ya/}POSS\_PRON\_1S  \\
\hline \hline
\end{tabular}
}
}
%\end{minipage}
\caption{Sample SAMA inconsistencies}
\vspace{-2em}
\label{t:incSAMA}
\end{table}


The following describes the SAMA inconsistencies illustrated in Table~\ref{t:incSAMA}:
\todo{R3.11 item (\ref{inc:relaxed)} below} 

\arabtrue

\begin{enumerate}[(a)]
    %%%%%%%%%%%%%%%%%%% SAMA prefixes
\item \label{inc:relaxed}
  $L_p$ misses entries for Alef prefixes with omitted hamza or madda due to relaxed
    writing standards which are common in many documents. 
    This is resolved in SAMA for standalone Alef prefixes via preprocessing tokens and flipping all forms of Alef into one form. 
    We report it here since compound prefix entries with Alef are all listed, 
    e.g. \RL{sA}, 
    \utfRL{سأ}, \utfRL{سآ}, and 
    the standalone prefixes are available but commented out.
Sarf addresses this concern by the separation of prefixes into partial and atomic variants, and by 
introducing agglutinative and fusional rules. The rules construct compound prefixes from partial and atomic prefixes. 
This, therefore, enables the introduction of all prefixes that include Alef, with any of its alternative spellings, 
by only adding the Alef alternatives into the atomic prefix set. 
\item \label{inc:by} $L_p$ contains an additional erroneous alternative gloss for \RL{wa-} in only one compound prefix~\RL{wa-}; while correctly not included elsewhere.
\item \label{inc:fa} $L_p$  contains stray spaces in the vocalized tags of one of the \RL{fa-} alternatives.
\item \label{inc:wafa} $L_p$  contains an entry that supports the concatenation of \RL{wa-} and \RL{fa-} conjunctions. 
    This entry is erroneous and is illegal in Standard Arabic.
    %%%%%%%%%%%%%%%%%%% SAMA suffixes
\item \label{inc:def} $L_x$  omits the definite indicator in the gloss of several suffixes.
\item \label{inc:num_gender} 
    $L_x$  omits gender and number qualifiers that appear within square brackets 
    from the gloss tags of several suffixes.
\item \label{inc:both:sama} 
    $L_x$  
    expresses the dual quantifier as `them (both)' in the majority of the entries, and 
    as `(both of) them' in several other entries.
\item \label{inc:num_gender_differences} 
    $L_x$  
    contains number indicators in the gloss tags still expressed in BAMA style.% square brackets versus parentheses
\item \label{inc:jus} 
    $L_x$  contains entries with an 
    {\em indicative} gloss mood and a {\em jussive} POS mood.
\item \label{inc:dot:SAMA} 
    $L_x$  omits dot (`.') for the abbreviation of plural in several gloss tags.
\item \label{inc:pos:SAMA} 
    $L_x$  
    represents a repeated consonant by a shadda in the POS tag where it should not. 
    SAMA POS tags should spell out the repeated consonants if each belongs to one.\todo{R3.19}  
    In SAMA, the repeated consonant (of the shadda) is spelled out
    whenever the consonants has its separated partial POS tag. 
\end{enumerate}
%\vspace{-1em}

In addition, 53 BAMA and 27 SAMA minor differences exist between $L_p$ and $L_x$ of BAMA and SAMA and
their counterparts computed using our agglutinative affixes. 
For example, the BAMA gloss tags for prefixes that contain
`bi/PREP' report `with/by' in some entries and its reverse `by/with' in others. 
In addition, we detected several entries in $L_p$ of SAMA with no category compatibility 
rules in $R_{ps}$, $R_{sx}$, and $R_{px}$.
