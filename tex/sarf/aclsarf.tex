\documentclass[3p,times,procedia]{elsarticle}
\flushbottom

%% The `ecrc' package must be called to make the CRC functionality available
\usepackage{ecrc}
%\usepackage{amsmath}


%% set the volume if you know. Otherwise `00'
\volume{00}

%% set the starting page if not 1
\firstpage{1}

%% Give the name of the journal
\journalname{Procedia Computer Science}

%% Give the author list to appear in the running head
%% Example \runauth{C.V. Radhakrishnan et al.}
\runauth{Author name}

%% Give the abbreviation of the Journal.
\jid{procs}

%\RequirePackage{etex}	
\RequirePackage{cite}
%\RequirePackage{hhline}
\RequirePackage{adjustbox}
\RequirePackage{graphicx}
\RequirePackage[table]{xcolor}
\RequirePackage{multirow} 
%\RequirePackage[ascii]{inputenc}
\RequirePackage[T1]{fontenc}
\RequirePackage[english]{babel}
\RequirePackage{amssymb,amsfonts,textcomp}
%\RequirePackage{color}
\RequirePackage{multicol}
\RequirePackage{array}
\RequirePackage{rotating}

\RequirePackage{hyperref}
\hypersetup{pdftex, colorlinks=true, linkcolor=blue, 
  citecolor=blue, filecolor=blue, urlcolor=blue, 
  pdftitle=, pdfauthor=, pdfsubject=, pdfkeywords=}

\RequirePackage{relsize}
\RequirePackage{fancyvrb}
\RequirePackage{ifthen}
\RequirePackage{amsmath}
\RequirePackage{amsthm}
\RequirePackage{amssymb}
\RequirePackage{latexsym}
\RequirePackage{xspace}
\RequirePackage{url}
%\RequirePackage[pdftex]{graphicx}
%\RequirePackage{color}
%\RequirePackage{mathpartir}
%\RequirePackage{graphviz}
%\RequirePackage{dot2texi}
%\RequirePackage{tikz}
%\usetikzlibrary{shapes,arrows}
%\RequirePackage{pgfplots}
%\pgfplotsset{compat=1.8}

\RequirePackage[colorinlistoftodos]{todonotes}
\RequirePackage{soul}

%---------------------
\RequirePackage{setspace}
\RequirePackage{listings}
\RequirePackage{float}
\RequirePackage{enumerate}
\RequirePackage{framed}
\RequirePackage{pifont}
%\RequirePackage{wrapfig}
%\RequirePackage{subfig}
%\RequirePackage{epsf}
%\RequirePackage{epsfig}
%\RequirePackage{a4}
%\RequirePackage{fancyhdr}
%\RequirePackage{tocbibind} %bibliography in the table of contents
\RequirePackage{amsbsy}
%\RequirePackage{booktabs}
\RequirePackage{paralist}
%\RequirePackage{algorithm}
%\RequirePackage{algpseudocode}
\RequirePackage{chngpage}
\RequirePackage{tabularx}
\RequirePackage{etoolbox}

%for code bold
\RequirePackage{tgcursor}
\RequirePackage{tcolorbox}


%\documentclass[journal]{IEEEtran}
\documentclass[10pt,journal,cspaper,compsoc]{IEEEtran}

%packages 
\usepackage{etex}
\usepackage{tikz}
\usepackage{xspace}
\usepackage{arabtex}
\usepackage{graphicx}
%\usepackage{caption}
%\usepackage{subcaption}
\usepackage{framed}
\usepackage{array}
\usepackage{graphics}
\usepackage{booktabs}
\usepackage{multirow}
\usepackage{amssymb}
\usepackage{amsmath}
\usepackage{mathabx}
\usepackage{utf8}
\bibliographystyle{plain}
\usepackage{subfigure}
\usepackage{todonotes}
\usepackage{hyperref}
\usetikzlibrary{shapes,arrows}
\usepackage{color,soul}
\usepackage{verbatim}
\usepackage{array}
\usepackage{multicol}

\usepackage{fancyvrb}
\usepackage{relsize}
\usepackage{wasysym}
\usepackage{pbox}

\def\showvrb#1{%
\texttt{\detokenize{#1}}%
}
%%%%%%%%%%%%%%%%%%%%%%
%macros
\def\framework{\textsc{MERF}\xspace}
\newcommand{\itodo}[1]{\todo[inline,color=green]{\tiny #1}}

\newcolumntype{L}[1]{>{\raggedright\let\newline\\\arraybackslash\hspace{0pt}}m{#1}}
\newcolumntype{C}[1]{>{\centering\let\newline\\\arraybackslash\hspace{0pt}}m{#1}}
\newcolumntype{R}[1]{>{\raggedleft\let\newline\\\arraybackslash\hspace{0pt}}m{#1}}

\newcommand{\TinyRL}[1]{{\RL{#1}}}
\newcommand{\cci}[1]{{\small \texttt{#1}}}
%%%%%%%%%%%%%%%%%%%%%%%
%other 
% correct bad hyphenation here
\hyphenation{op-tical net-works semi-conduc-tor}


%%% RL commands
\newcommand{\utfrl}[1]{\setcode{utf8}\RL{#1}\setcode{standard}}
\newcommand{\notrutfrl}[1]{\transfalse\setcode{utf8}\RL{#1}\setcode{standard}\transtrue}
\newcommand{\notrrl}[1]{\transfalse\RL{#1}\transtrue}
\newcommand{\noarrl}[1]{\arabfalse\RL{#1}\arabtrue}


\begin{document}
\begin{frontmatter}

\dochead{3rd International Conference on Arabic Computational Linguistics, ACLing 2017, 5--6 November 2017, Dubai, United Arab Emirates}%
%% Use \dochead if there is an article header, e.g. \dochead{Short communication}
%% \dochead can also be used to include a conference title, if directed by the editors
%% e.g. \dochead{17th International Conference on Dynamical Processes in Excited States of Solids}

\title{Sarf: Fast and Application Customizable Arabic Morphological Analyzer}

%% use optional labels to link authors explicitly to addresses:
%% \author[label1,label2]{<author name>}
%% \address[label1]{<address>}
%% \address[label2]{<address>}


%\author[a]{First Author} 
%\author[b]{Second Author}
%\author[a,b]{Third Author\corref{cor1}}

%\address[a]{First affiliation, Address, City and Postcode, Country}
%\address[b]{Second affiliation, Address, City and Postcode, Country}


%% From NLE %%
%\author[F. Zaraket, A. Jaber, J. Makhlouta, and M. Safieddine]
%       {Fadi A. Zaraket\and Ameen Jaber\and Jad Makhlouta\and Maya H. Safieddine\\
%        Electrical and Computer Engineering,\\
%        American university of Beirut}

\begin{abstract}
Rule-based techniques and tools to extract {\em entities} and {\em relational entities} from documents allow users to specify desired entities using natural language questions, finite state automata, regular expressions, structured query language statements, or proprietary scripts.  These techniques and tools require expertise in linguistics and programming and lack support of Arabic {\em morphological analysis} which is key to process Arabic text.
In this work, we present \framework; a morphology-based entity and relational entity extraction framework for Arabic text.  \framework provides a friendly interface where the user, with basic knowledge of linguistic features and regular expressions, defines {\em tag types} and interactively associates them with regular expressions defined over Boolean formulae. Boolean formulae range over matches of Arabic morphological features, and synonymity features. Users define relations with tuples of subexpression matches and can associate code actions with subexpressions. \framework computes feature matches, regular expression matches, and constructs entities and relational entities from the user-defined relations. We evaluated our work with several case studies and compared with existing application-specific techniques. The results show that \framework requires shorter development time and effort compared to existing techniques and produces reasonably accurate results within a reasonable overhead in run time.
\end{abstract}

\begin{keyword}
Type your keywords here, separated by semicolons ; 

%% keywords here, in the form: keyword \sep keyword

%% PACS codes here, in the form: \PACS code \sep code

%% MSC codes here, in the form: \MSC code \sep code
%% or \MSC[2008] code \sep code (2000 is the default)

\end{keyword}
%\cortext[cor1]{Corresponding author. Tel.: +0-000-000-0000 ; fax: +0-000-000-0000.}
\end{frontmatter}

%\correspondingauthor[*]{Corresponding author. Tel.: +0-000-000-0000 ; fax: +0-000-000-0000.}
%\email{author@institute.xxx}


\section{Introduction}
\label{sec:intro}
{\em Natural language processing} (NLP) applications such as 
{\em machine translation} (MT) and {\em information extraction} (IE)
require {\em morphological analysis} to preprocess Arabic text
due to the rich morphological nature of the Arabic language
~\citep*{Benajiba:07,Habash:06}. 
%
%Given a white space and punctuation delimited Arabic word,
Arabic morphological analyzers
return the internal structure of a given Arabic word composed of 
several {\em morphemes} including {\em affixes} ({\em prefixes} 
and {\em suffixes}), {\em clitics} ({\em proclitics} and {\em enclitics}), 
and {\em stems}~\citep*{Sughaiyer:04}. 
The morphological solution consists also of several {\em morphological features}
(tags) associated with the word and its constituent morphemes 
such as {\em part of speech} (POS), transliteration, {\em gloss}, 
and {\em vocalized morpheme form} (VMF) tags (diacriticized form of the morpheme).
%
%The stem is the core of a word. 
The prefix and suffix attach before and after the stem, respectively. 
Clitics are special affixes that attach to the stem to form a word, 
and differ from regular affixes in that 
they play a syntactic role of another word (often omitted) 
%while being bound to the word in question
~\citep*{habash2010introduction}. 
%
%However, some lexicons such as Buckwalter~\citep*{Buckwalter:04} use the terms prefix and suffix to refer to  proclitic and enclitic, respectively.

This work presents Sarf, a \todo{R2.13} {\em fast and application 
customizable morphological analyzer} for Arabic
that is used in several applications for information extraction 
from Arabic text~\citep*{JaZaMatar,ZaMaFlairs2012HadithBio,ZaMa2012IJCLATime,ZaMaHaCicling2012Entity}. 
Sarf provides NLP application developers with an application programming interface (API) 
to control and refine morphological analysis on the fly. 
The developer implements the interfaces in the application.
Sarf calls the interfaces 
on control points such as prefix, stem, suffix, and full solution matches.
%We distinguish between the NLP application developer 
%who uses Sarf to build an NLP application and the end user who provides the text
%and inspects the solutions.
The Sarf API allows the application to 
(1) control and prioritize the analysis, 
(2) refine the solution features, and
(3) define developer categories and associate them with existing morphemes.

%Users of the developed application pass Arabic text as input and the application uses 
%Sarf to perform customizable morphological analysis on the text. 

Sarf is a significant extension of the work in~\citep*{ZaMaColing2012DemosSarf}. 
It represents Arabic affixes as 
agglutinative affix morphemes with fusional affix concatenation rules. 
Simpler agglutinative affix morphemes can be concatenated to form a more complex 
affix~\citep*{Vajda}. 
%Fusional models are similar to agglutinative with the difference that 
%morpehemes which make up the word fuse together several different features meanings~\citep*{Spencer:91}. 
Fusional affix concatenation rules specify affix pairs and use 
regular expressions \todo{R2.13 removed in} as substitution rules to 
compose the resulting orthographic and semantic tags from the 
tags of the original morphemes~\citep*{Spencer:91}.
The Sarf substitution rules are in sync with 
rules and examples on morpheme concatenative properties from
Arabic morphology textbooks~\citep*{Abd00,Abd001}.
This representation resolves consistency, maintenance, and segmentation 
issues of the current approaches in BAMA and SAMA. 
%Sarf also provides the option to use partial diacritics 
%in disambiguating the morphological solutions of a partially diacritized word.

\todo{R1.1}
In this paper, we make several additional contributions including the following. 
\begin{itemize}
    \item Sarf provides an application customizable morphological analyzer where the developer can control and refine the analysis. 
    \item Sarf structures and computations are engineered to reduce computation cost as follows. 
      \begin{itemize}
        \item The organization of the lexical structures into root index tries during runtime.
        \item The traversal structures that allows multiple solutions to be considered at once.
        \item The computation of the solution features in a tree based structure instead of computing all features and filtering later on
        \item The manual optimization of the traversal of the lexicon tries to reduce the typical excessive non-determinism exhibited in the finite state machine based solutions.
      \end{itemize}
    %\item Sarf is a novel Arabic morphological analyzer with agglutinative affixes and fusional affix concatenation rules based on textbook Arabic morphological rules and on the concatenation rules of existing analyzers. 
    \item Sarf solves the ``run-on'' words problem and optionally uses partial diacritics for solution disambiguation. 
    %\item Sarf solves inconsistencies in existing affix lexicons of BAMA and SAMA. In this paper we list and discuss the inconsistencies.
    %\item Sarf solves the correspondence between the morphological solution and the morphological segmentation of the original text problem.
%    \item We evaluate our approach using the ATB data set and report on the effect of our corrections on the annotations.
    \item Sarf is fully implemented and available online as an open source tool.\footnote{\label{fn:online} \url{http://research-fadi.aub.edu.lb/carla/doku.php?id=sarf} }
\end{itemize}

We evaluated Sarf for %segmentation correspondence, lexicon size, lexicon consistency\todo{R3.19}, 
accuracy and runtime efficiency.
%Our results show that Sarf lexicons are smaller than lexicons of existing analyzers and provide coverage for more morphological solutions. 
Sarf %simplifies the lexicon maintenance task, 
provides better accuracy 
and improves run time efficiency as compared to existing analyzers.  
%
\todo{R2.14}We evaluated Sarf also within applications that use its API 
~\citep{JaZaMatar,ZaMaFlairs2012HadithBio,ZaMa2012IJCLATime,ZaMaHaCicling2012Entity}. 
The carried case studies show that Sarf performs better than existing Arabic morphological analyzers in terms of running time. 
They also demonstrate the efficiency and the utility of the Sarf 
application customizable API. 

The rest of this paper is structured as follows. 
\todo{R2.1A} In Section~\ref{sec:motivation}, we motivate Sarf and discuss how it deals with existing morphological analysis challenges. 
In Section~\ref{sec:related}, we compare Sarf to related work. 
In Section~\ref{sec:overview}, we present the structre of Sarf. 
In Section~\ref{sec:api}, we present the interface provided by Sarf for the 
application developer to control and refine the morphological analysis. 
%In Section~\ref{sec:aggaffix}, we present our method to build agglutinative affix morphemes with fusional affix concatenation rules. 
%%In Section~\ref{sec:motivation}, we discuss existing redundancy and inconsistencies in the lexicons of existing morphological analyzers.
In Section~\ref{sec:diac}, we present our method of using partial diacritics to reduce the morphological ambiguity.
\todo{R1.2}In Section~\ref{sec:results}, we discuss the use of Sarf with several NLP tasks and 
present the results of comparing Sarf to other analyzersin terms of speed and accuracy.%, and consistency.
Finally, we conclude and discuss future work in Section~\ref{sec:conclusion}.

%\vspace{-1cm}

\section{Motivation}
\label{sec:motivation}
\todo{R2.1A}
In this section, we discuss issues that %currently
challenge morphological analysis and explain how Sarf resolves them. 

\subsection{Exhaustive enumeration and performance:}

Current morphological analyzers such as 
%Buckwalter~\citep*{Buckwalter:02},
BAMA~\citep*{Buckwalter:02}, 
SAMA~\citep*{SAMAmain, Kulick:10}, \todo{R3.18} 
Beesley~\citep*{Beesley:01}, 
MADAMIRA~\citep*{pasha2014madamira}, 
and ElixirFM~\citep*{Otakar:07} 
take as input white space delimited tokens,
consider them as words,
and enumerate all possible morphological solutions for each word. 

\begin{table}[tb]
\centering
\begin{minipage}{.9\textwidth}
\caption{\label{t:solexample}Morphological solution for \utfRL{وسيلعبونها}(and they will play it)}
\begin{tabular}{lccc} \hline
  & suffix  \noTrutfRL{ونها} & stem  \noTrutfRL{لعبـ}  & prefix \noTrutfRL{وسيـ} \\ \hline
       POS & \cci{IVSUFF\_SUBJ:MP\_MOOD:I+IVSUFF\_DO:3FS} & \cci{VERB\_IMPERFECT} & \cci{CONJ+FUT+IV3MP}  \\
  Transliteration & \cci{uwnahA} & \cci{loEab} & \cci{wa+sa+ya}  
  \\
           Gloss  & \cci{[MASC.PL.]+it/them/her}.  & \cci{play} & \cci{and they will} \\
  \hline% \hline
\end{tabular}
\end{minipage}
\end{table}


For example, given the word \utfRL{وسيلعبونها}\todo{R2.5}\footnote{The paper uses ArabTex package \citep{arabtex} for Arabic text editing and transliteration. ArabTex package is recommended and used by \citep{IntroNHabash}.} (and they will play it), 
an analyzer may return the solution presented in Table~\ref{t:solexample} with 
a prefix, a stem, a suffix, and their corresponding 
POS, transliteration, and gloss tags. 
%\noTrutfRL{وسيـ} with POS tag \cci{wa/CONJ+sa/FUT+ya/IV3MP} 
%and gloss tag \cci{and they will}, 
%the stem 
%\noTrutfRL{لعبـ} with POS tag \cci{loEab/VERB\_IMPERFECT}
%and gloss tag  \cci{play},
%and 
%the suffix \noTrutfRL{ونها} with POS tag \cci{uwna/IVSUFF\_SUBJ:MP\_MOOD:I+hA/IVSUFF\_DO:3FS}
%and gloss tag \cci{[MASC.PL.]+it/them/her}. 
The prefix 
\noTrutfRL{وسيـ} can be further segmented into 
(1) the proclitic \noTrutfRL{و} with POS tag \cci{CONJ} and gloss tag \cci{and},
(2) the proclitic \noTrutfRL{سـ} with POS tag \cci{FUT} and gloss tag \cci{will},
and (3) the prefix \noTrutfRL{ـيـ} with POS tag \cci{IV3MP} and gloss tag \cci{they(people)}.
Similarly, the suffix 
\noTrutfRL{ونها} 
can be segmented into
(1) the suffix \noTrutfRL{ونـ}, forming a circumfix with \noTrutfRL{ـيـ}, 
with POS tag \cci{IVSUFF\_SUBJ:MP\_MOOD:I}
and gloss tag \cci{[MASC.PL.]}, 
and (2) the enclitic \noTrutfRL{ـها} with POS tag \cci{IVSUFF\_DO:3FS}
and gloss tag \cci{it/them/her}. 
%Some analyzers may refer to \noTrutfRL{و} and \noTrutfRL{سـ} as proclitics, 
%and may refer to \noTrutfRL{ـها} as an enclitic.
\novocalize

The exhaustive enumeration of all solutions may hurt performance and may
not be necessary or appropriate
in some applications~\citep*{Maamouri:10}. 
Thus the need of a customizable morphological analyzer that adapts to 
application specific requirements.
%\vspace{-2.35cm}

\subsection{Diacritics and accuracy}

The accuracy of morphological analysis %, with respect to the context,
suffers from inherent difficulties
%of morphological analysis 
of the Arabic language such as omitted diacritics and position dependent letter
forms. 
%
\fullvocalize
Diacritics, i.e. short vowels, such as 
fatha (\RL{|Ba}), damma (\RL{|Bu}), kasra (\RL{|Bi}), 
tanween (i.e. doubled diacritic including \RL{|BaN}, \RL{|BuN}, \RL{|BiN}), 
\transfalse and sokun (\RL{|B}) 
are almost always omitted in written Arabic text as they can be 
inferred by human readers.
The mark \novocalize shadda (\RL{|BB}) \transtrue denotes the repetition
of the marked character and is also often omitted. 
Partial diacritics can help disambiguate solutions. 
%This is a source of disambiguity. 
Consider the unvocalized word \RL{'kl} with nine morphological solutions.  
Its partially vocalized version \utfRL{أكِل} has only two solutions; 
VMF \utfRL{أَكِل} with gloss \cci{I+trust/put in charge}, and 
VMF \utfRL{أُكِلّ} with gloss \cci{I+make tired/wear out}.

Analyzers such as BAMA and SAMA ignore partial diacritics while 
other analyzers such as \citep*{Beesley:01,Chaaben:10,Attia:00} make use of 
the partial diacritics to reduce ambiguity. 
Sarf provides an option that enables the use of existing diacritics for disambiguation, 
and considers the diacritics at morpheme boundaries to generate only the diacritic 
matching solutions, rather than generating all morphological solutions then 
filtering them. 

%\vspace{-1.2cm}

\subsection{Problem of `run-on' words:}

Arabic letters have up to four different forms
corresponding to their position in a word, i.e, beginning,
middle, end, and separate word forms. 
This allows the phrase \transfalse
\RL{il_A\nospace almdrsT} \transtrue \todo{R2.12} (to the school)
to be visually recognizable
as two separate words \RL{il_A} (to) and \RL{almdrsT} (the school) 
without the need of a delimiter space in between. 
The reason is the first word \RL{il_A} ends with
\RL{_A} a non-connecting letter. 
These words, referred to as `run-on' words~\citep*{Buckwalter:04},
occur regularly, and greatly increase the
difficulty of tokenization. 
\todo{2.1A}None of existing morphological analyzers, up to our knowledge, 
resolve this issue. Sarf on the other hand does by the structures and rules it introduces, which are further discussed in the next issue.

\begin{comment}
\vocalize
\begin{table}[!tb]
\begin{minipage}{\textwidth}
\relsize{-1.5} {
\begin{tabular}{ccccc} 
\hline \hline
\textbf{Prefix} & \textbf{Vocalized} & \textbf{Category} & \textbf{Gloss} & \textbf{POS data} \\ \hline
\RL{f} & \RL{fa} & Pref-Wa & and/so & fa/CONJ+ \\
\RL{y} & \RL{ya} & IVPref-hw-ya & he/it & ya/IV3MS+ \\
\RL{fy} & \RL{faya} & IVPref-hw-ya & and/so + he/it & fa/CONJ+ya/IV3MS+ \\
\RL{sy} & \RL{saya} & IVPref-hw-ya & will + he/it & sa/FUT+ya/IV3MS+ \\
\RL{fsy} & \RL{fasaya} & IVPref-hw-ya & and/so + will + he/it & fa/CONJ+sa/FUT+ya/IV3MS+ \\ \\

\RL{y} & \RL{ya} & IVPref-hmA-ya & they (both) & ya/IV3MD+ \\
\RL{fy} & \RL{faya} & IVPref-hmA-ya & and/so + they (both) & fa/CONJ+ya/IV3MD+ \\
\RL{sy} & \RL{saya} & IVPref-hmA-ya & will + they (both) & sa/FUT+ya/IV3MD+ \\
\RL{fsy} & \RL{fasaya} & IVPref-hmA-ya & and/so + will + they (both) & fa/CONJ+sa/FUT+ya/IV3MD+ \\  \\

\RL{w} & \RL{wa} & Pref-Wa & and & wa/CONJ+ \\
\RL{wy} & \RL{waya} & IVPref-hw-ya & and + he/it & wa/CONJ+ya/IV3MS+ \\
\RL{wsy} & \RL{wasaya} & IVPref-hw-ya & and + will + he/it & wa/CONJ+sa/FUT+ya/IV3MS+ \\
\RL{wy} & \RL{waya} & IVPref-hmA-ya & and + they (both) & wa/CONJ+ya/IV3MD+ \\
\RL{wsy} & \RL{wasaya} & IVPref-hmA-ya & and + will + they (both) & wa/CONJ+sa/FUT+ya/IV3MD+ \\
\hline \hline
\end{tabular}
}
\end{minipage}
\caption{Partial BAMA v1.2 prefix lexicon}
%\vspace{-2cm}
\label{t:affixes}
\end{table}


\subsection {Structure of existing analyzers and related problems:}

Concatenative morphological analyzers~\citep{Buckwalter:02,Kulick:10} %BAMA,SAMA
are based on lexicons of prefixes $L_p$, 
stems $L_s$, and suffixes $L_x$. 
As shown in Table~\ref{t:affixes}, each entry in a lexicon includes the morpheme, 
its vocalized form with diacritics, 
a {\em concatenation compatibility category} (CCC) rule, 
a gloss tag, and a POS tag.
Separate CCC rules specify the compatibility of prefix-stem 
$R_{ps}$, stem-suffix $R_{sx}$, and prefix-suffix (circumfix) $R_{px}$ concatenations. 
The affixes \noTrutfRL{و} and 
\noTrutfRL{يـ} in the before mentioned example are valid standalone prefixes, 
and can be concatenated to the stem \noTrutfRL{لعب} \todo{R2.12} (he plays) to form 
\noTrutfRL{ولعب} (and he plays) and
\noTrutfRL{يلعب} (he is playing), respectively. 
%In addition, the morpheme \noTrutfRL{سـ} can connect to \noTrutfRL{يلعب} to form 
%\noTrutfRL{سيلعب}. 
The $L_p$ and $L_x$ lexicons contain also all final forms of concatenated affixes
as shown in Table~\ref{t:affixes} for sample morphemes. 

This is the source of several problems: 
\begin{itemize}
%\vspace{-2cm}
    \item $L_p$ and $L_x$ contain redundant entries which result \todo{R2.13} in maintenance and consistency issues~\citep*{LRECMaamouriKB08,lrecKulickBM10}.
    \item Augmenting $L_p$ and $L_x$ with additional morphemes, such as \utfRL{أَ} (the question glottal hamza), 
      may result in a quadratic increase in the size of the lexicons~\citep*{Hunspell}.
      The additional morpheme may attach to exiting morphemes. 
      Currently, this results in adding all the resulting morphemes to the BAMA and SAMA lexicons. 
    \item The $L_p$ and $L_x$ lexicons are larger than needed especially that 
      they have to account 
      for several forms of a morpheme with varying diacritics.
    \item The concatenated forms in $L_p$ and $L_x$ contain concatenated POS and 
      other tags. 
%        The segmentation correspondence between the prefix concatenated from several morphemes and the tags associated with it is lost. 
%        In several cases, this leads to missing correspondence between the tokens of the morphological 
%        solution and the segmentation of the original word.
The alignment and correspondence between the original word and its
morphemes with the tags of its morphological solution 
are essential to the success of NLP tasks such as MT 
and IE~\citep*{Regina2011,ELRASemmar08}.
The analysis of the example \utfRL{للقضاء} (for/to the justice/judiciary\todo{R2.12 gloss}) using SAMA,
generates the morphological solution \cci{li/PREP + Al/DET + qaDA'/NOUN}.
The issue of the alignment and correspondence between the original word \noTrutfRL{للقضاء} 
and its morphemes arises as the concatenation of the unvocalized prefix morphemes 
\cci{li} and \cci{Al} is not the same as the corresponding prefix 
segment \noTrutfRL{لل} in the original word. 
\todo{R.3.5}
This is not a problem for SAMA, but rather for the applications that 
use the SAMA output such as treebanking that is interested in \cci{li} be present
as a separate leaf in the syntactic tree. 
Sarf provides a general solution for the segmentation correspondence problem 
since the valid compound affixes preserve the input text segmentation. 
In particular, a partial affix \noTrutfRL{ل} \cci{Al/DET}
connects to the atomic affix \noTrutfRL{ل} \cci{li/PREP} and resolves the problem.

%is segmented into two tokens:
%\cci{li/PREP} and \cci{Al/DET + qaDA'/NOUN}~\citep{LRECMaamouriKB08}. 
%The best approximation of the unvocalized entry 
%of each token is
%\noTrnoVocRL{li} and \noTrnoVocRL{Alqa.dA'}, respectively, 
%with an extra letter \RL{A}. 
%This is not a faithful representation of the original text data 
%and the segmentation does not correspond with that of the
%input text.
\end{itemize}

\todo{R3.1}
Alternatively, 
Sarf extends and refines the lexicons and compatibility category rules of BAMA and SAMA 
such that:
\begin{itemize}
  \item it represents only atomic affix morphemes in the lexicons,
  \item it adds prefix-prefix $R_{pp}$ and suffix-suffix $R_{xx}$ agglutinative and 
    fusional rules,
  \item it generates compound affixes from the atomic ones using $R_{pp}$ and $R_{xx}$, and
  \item it extends the stem lexicon with additional words necessary for the NLP applications it considers.
\end{itemize}

Considering the entries of Table \ref{t:affixes} as example, Sarf represents them using 
only five atomic affix morphemes and five 
prefix-prefix rules.\todo{R2.4} 
\end{comment}

\section{Related work}
\label{sec:related}
\begin{table}[tb]
%\caption{Comaprison of Sarf with SAMA~\citep{Kulick:10}, 
%ElixirFM~\citep{Otakar:07}, 
%MADA+TOKAN~\citep{},
%MADAMIRA~\citep{pasha2014madamira}, 
%Beesley~\citep{}, and
%Xerox~\citep{Beesley:01}
%}
\centering
\begin{minipage}{\textwidth}
\caption{\label{tab:comparison}Comparison of Sarf with SAMA, 
ElixirFM, MADA+TOKAN, MADAMIRA, Beesley, and Fassieh}
\relsize{-1} {
  \begin{tabular}{lccccccc}\toprule
             & Sarf & SAMA & ElixirFM & MADA+TOKAN & MADAMIRA & Beesley & Fassieh\\ \colrule
Application customizable & $\checkmark$ & - & - & - &  - & - & - \\[4pt]
Feature selection & $\checkmark$ & - & - &  - & $\checkmark$ & - & - \\[4pt]
Run-on words & $\checkmark$ & - & - & - & - & - & - \\[4pt]
Partial diacritics & $\checkmark$ & - & - & - & - & $\checkmark$ & - \\[4pt]
Affix segmentation & $\checkmark$ & - & functional & tokenization schemes & statistical & - & - \\[4pt]
Root-Pattern & - & - & $\checkmark$ & - & - & $\checkmark$ & $\checkmark$ \\[4pt]
Automated disambiguation & - & - & - & SVM & SVM & - & maximum  aposteriori  \\ 
\botrule
\end{tabular}
}
\end{minipage}
\end{table}


In this section, we review work related to Arabic morphological analyzers,
segmentation correspondence, 
partial diacritics, and application specific analyzers and compare it to Sarf.

Table~\ref{tab:comparison} summarizes the comparison between Sarf 
and related Arabic morphological analyzers. 
Only ElixirFM~\citep{Otakar:07}, Beesley~\citep*{Beesley:03}, and Fassieh~\citep*{attia2009fassieh} 
provide root-pattern analysis of the stem. 
ElixirFM, MADAMIRA~\citep{pasha2014madamira}, and MADA+TOKAN~\citep*{Habash:09} are 
based on BAMA and SAMA and use functional and statistical techniques to 
address the segmentation problem by reverse engineering the multiple tags of the 
affixes. 
Sarf differs in that the segmentation is an output of the morphological
analysis and not a reverse engineering of the multi-tag affixes. 
Sarf is the only analyzer that addresses the 'run-on words' problem 
and solves it while performing the analysis. 
MADA+TOKAN, MADAMIRA, and Fassieh apply morphological disambiguation 
using support vector machines (SVM), 
and {\em maximum a posteriori} (MAP) estimation, respectively.
Beesley and Sarf consider partial diacritics to eliminate morphological 
solutions that are not in agreement with the partial diacritization. 
Sarf provides an application customizable analyzer that enables 
the developer to control and refine the analysis on the fly 
and filter the solution features. 
MADA+TOKAN and MADAMIRA provide \todo{R2.13} partial control over the output, and not the 
analysis, where MADA+TOKAN allows the user to select \todo{R2.13 } from several segmentation schemes and 
MADAMIRA enables the user to select solution features.

Sarf builds upon the lexicon of Buckwalter\citep{Buckwalter:02}.
SAMA is an updated version of BAMA with increased lexicon coverage 
and additional POS tags~\citep*{maamouri2010ldc}.
Sarf differs from Buckwalter and SAMA in that it defines 
agglutinative and fusional affixes using a shorter list of affixes
and a list of concatenation compatibility rules that allow 
prefix-prefix and suffix-suffix concatenations.
This allows Sarf to better maintain %and propagate 
the morphological tags associated with the affixes. 

Buckwalter\citep{Buckwalter:02} and SAMA~\citep{Maamouri:10} 
produce a set of segmentation solutions for
a word, compute the morphological solutions for each segment, 
compute the product of the solutions, eliminate \todo{R2.13 } the 
incompatible solutions, and then report \todo{R2.13 } the valid solutions. 
Sarf traverses the affix and stem structures with the input word
character by character and keeps a stack of morpheme nodes. 
When a morpheme node in a structure is met, 
it is checked for compatibility with the stack of nodes. 
Consequently, Sarf generates only the solutions with valid 
segmentation, and reports only those with compatible stem and affix 
concatenation. 

SAMA was refined to interact with the Arabic Treebank(ATB)~\citep{Maamouri:04}  %Arabic Treebank at the Linguistic Data Consortium 
project after the addition of a large new corpus. 
The algorithmic changes in SAMA were done manually 
to integrate with the ATB format. 
Sarf API allows for customizable refinements 
and allows Sarf to interact with any 
application on the fly
without the modification of the morphological engine itself
as was reportedly required with SAMA~\citep{Maamouri:10} 

Like ElixirFM~\citep{Otakar:07}, Sarf builds on the lexicon
of the Buckwalter analyzer. 
Sarf also uses deterministic parsing with tries and DAGs
to implement the affix and stem structures. 
We think that the inferential-realizational approach 
of ElixirFM that is highly compatible with the 
Arabic linguistic description~\citep*{Badawi:04}
can benefit from many features unique to the Arabic language.
Sarf leaves implementing that to the developer customization 
through the API since in several cases the NLP application that uses the morphological analyzer
needs only a partial linguistic model of Arabic.

MADA+TOKAN~\citep{Habash:09} is a toolkit for Arabic tokenization, diacritization, 
morphological disambiguation, POS tagging, stemming, and lemmatization. 
Sarf performs all those tasks except for morphological disambiguation 
where MADA uses SVM.
\todo{R3.6}
Sarf keeps a stack of the positions that partition text into morphemes 
as part of the solution construction process. 
Therefore, the text segments corresponding to the solution features is preserved 
with no need for further post-processing. 
%a separate segmentor such as TOKAN is not required. 

%Similar to The MADA+TOKAN toolkit, 
%Also, Sarf thinks of the several valid morphological analyses as 
%a richness that should be exploited at a higher abstract level 
%rather than an inaccuracy that should be corrected.

MADAMIRA is a tool for Arabic morphological analysis and disambiguation 
that is based on the general design of MADA, 
an Arabic morphological analyzer and disambiguator, with additional components 
inspired by AMIRA~\citep{pasha2014madamira}, a language independent SVM based analyzer. 
MADAMIRA improves upon the two systems and returns information selectively 
upon the request of the user. 
Sarf provides means for the developer to control and adapt 
the morphological analysis according to application needs. 
Moreover, the API enables the developer to implement high level 
applications such as NER which is provided by AMIRA.


%Xerox rules can be compiled into specialized finite state
%machine (FSM) based analyzers as described in~\citep{Beesley:03}.
%However, the efficiency of the resulting analyzer depends on the
%way the Xerox rules are written. 
%Writing application specific Xerox grammars and rules, or modifying the existing ones, 
%requires the NLP application developer to have deep knowledge 
%of compilation techniques, context free grammars, 
%and morphological analysis. 
%%Other morphological analyzers such as 
%%Amira~\citep{Diab:07,Benajiba:07},
%%and MAGEAD~\citep{Habash:05}%, and MADA+TOKAN~\citep{Habash:09} 
%%use machine learning and support vector machines (SVM) 
%%to enhance the accuracy of the morphological analysis at the expense 
%%of performance.

Beesley~\citep{Beesley:03} compiles 
Xerox rules into specialized finite state
machine (FSM) based morphological analyzer.
The number of machines 
generated by a compiler for Xerox rules cannot be controlled by the developer of the analyzer, 
and the composition of the FSMs into a single framework is a difficult task 
~\citep{Beesley:01}.
Consequently the efficiency of the resulting analyzer depends on the
way the Xerox rules are written. 
Writing application specific Xerox grammars and rules, or modifying the existing ones, 
requires deep knowledge and insight from the NLP application developer
in compilation techniques, context free grammars, 
and morphological analysis. 
Sarf constructs a framework of efficient structures %including 
%a trie and two DAGs 
that encode the stems and the agglutinative and fusional affixes, respectively.
\todo{R.3.15}
The structures are traversed in a manner similar to the Xerox finite state machines, 
however, they are manually optimized to reduce the number of states and the 
number of non-deterministic transitions. 
Control on non-determinism is left to the compilers with Xerox.
Sarf also provides an application customizable API that allows the developer 
to control the analysis. 
Doing the equivalent with Beesley 
requires the modification of the Xerox rules and 
the recompilation of the analyzer.
Unlike Sarf, Beesley provides a root-pattern analysis of the stem.

Fassieh is a commercial Arabic text annotation tool that 
enables the production of large Arabic text corpora~\citep{attia2009fassieh}.
The tool supports Arabic text factorization including 
morphological analysis, POS tagging, full phonetic transcription, 
and lexical semantics analysis in an automatic mode. 
Unlike Sarf, Fassieh provides morphological 
disambiguation and root-pattern analysis. 
However, Fassieh does not provide segmentation of the affix 
and reports it as a whole unit. 
This tool is not directly accessible to the research community 
and requires commercial licensing. 
Sarf differs in that it is an open-source application 
customizable tool that solves the affix segmentation and 'run-on words' problems. 

The work in~\citep*{Attia:10} addresses the detection of 
Arabic Multi-word Expressions (MWE). 
They define MWEs as 'idiosyncratic interpretations 
that cross word boundaries or spaces'. 
Sarf adopts a similar approach for specific entities such as person names 
and place names. 


Several researchers stress the importance of 
correspondence
between the input string and the tokens of the morphological 
solutions. 
Some work uses POS tags and a syntactic morphological agreement
hypothesis to refine 
syntactic boundaries within words~\citep{Regina2011}.
The work in~\citep*{LIMASemmar05}\citep{ELRASemmar08}
uses an extensive lexicon 
with 3,164,000 stems, stem rewrite rules~\citep*{Darwish:02}, 
syntax analysis, proclitics, and enclitics to address the 
same problem. 
Parallel traversal of the input string and the tokens 
of the morphological solution, while accounting for all possible
SAMA normalizations, partially solves the problem as 
reported in~\citep{LRECMaamouriKB08,lrecKulickBM10}.
Later notes in the documentation of the ATB~\citep*{ATB32LDCNum}
indicate that extensive manual work is still required and that
later versions may drop the input tokens. 
\citep{Regina2011} uses syntactic analysis to resolve the same problem. 
%A significant amount of the literature on Arabic NLP uses the Arabic Tree Bank 
%(ATB)~\citep{Maamouri:04} with tags 
%generated from BAMA and SAMA for learning and 
%evaluation~\citep{ShaalanMF10,Benajiba:07,jumaily2011}.

The survey in~\citep{Sughaiyer:04} compares
several morphological analyzers. 
Analyzers such as~\citep*{Khoja:01}\citep{Darwish:02} 
target specific applications in the analyzer itself 
or use a specific set of POS tags as their reference.
Sarf differs in that it is a general morphological 
analyzer that reports all possible solutions. 
It is application customizable in the sense that the API is used to 
control and prioritize the analysis, refine the solution features, 
and associate morphemes with developer-defined categories.

The work in \citep{Chaaben:10,Attia:00,Beesley:01} 
considers partial diacritics and performs \todo{R2.13 } morphological 
disambiguation by filtering the full morphological solutions 
and excluding inconsistent ones. 
This approach constructs several solutions that will be excluded later. 
Sarf considers diacritic consistency at the morpheme level instead of the final 
solution level. 
It checks for diacritic consistency %using Algorithm~\ref{t:consistent} 
between the input morpheme and the candidate VMF features at every 
accept node during the traversal of Sarf structures.
Sarf analysis proceeds with the consistent VMFs
and terminates the inconsistent ones.

\citep{Beesley:01}, \citep{Chaaben:10}, and \citep{Attia:00} present 
analyzers that consider partial diacritics for morphological disambiguation. 
They filter the output morphological analysis based on 
compatibility with input diacritics if found.
Sarf differs in that it considers the diacritics at morpheme boundaries 
to generate only the diacritic matching solutions, rather than 
generating all morphological solutions then filtering them.


\section{Sarf}
\label{sec:overview}
\begin{figure}[tb!]
\centering
\resizebox{\columnwidth}{!}{
	\relsize{+1} \begin{figure}[tb!]
\centering
\resizebox{\columnwidth}{!}{
	\relsize{+1} \begin{figure}[tb!]
\centering
\resizebox{\columnwidth}{!}{
	\relsize{+1} \input{figures/overview.tex}
}
\caption{\framework flow diagram.}
\label{f:overview}
\end{figure}

Figure~\ref{f:overview} shows the flow diagram of \framework. 
%A box node represents a \framework process and 
%an ellipse node represents input and output data objects.
The Arabic text and the reference tag chunks are the primary inputs to \framework.
Solutions, morphology-based Boolean formulae, tags, 
morphology-based regular expressions, 
tag chunks, relation and action definitions, and data structures expressing entities 
and relations are input and output data to \framework processes. 
The morphological analyzer (Sarf), $Syn^k$ detector, 
GUI for Boolean formulae definition, 
visualization annotator, GUI for regular expression and action definition, 
Boolean formula simulator, regular expression simulator, 
relation extraction and action execution, and difference and statistical analyzer 
are \framework processes.

\def\pp{\ensuremath{{\cal P}}} % prefix
\def\ss{\ensuremath{{\cal S}}} % stem
\def\xx{\ensuremath{{\cal X}}} % suffix
\def\PP{\ensuremath{\mathit{POS}}} % pos
\def\GG{\ensuremath{\mathit{GLOSS}}} % gloss
\def\AC{\ensuremath{\mathit{CAT}}} % category

%~\footnote{In this document, we use the default ArabTeX transliteration style ZDMG.}
{\bf Morphological analyzer (Sarf).~}
\label{s:s:morph}
Morphological analysis is key to Arabic NLP~\cite{arabicmorph}.
Short vowels, also known as diacritics, are typically omitted in Arabic text
and inferred by readers~\cite{habash2006arabic}. 
For example, the word \RL{'sd}%
can be interpreted as \RL{'sad} {\tt ``lion''} with a {\em fatha} diacritic on the 
letter \utfrl{سـ} or 
\vocalize \RL{'sodd} (I block) with a 
{\em damma} diacritic on the letter \utfrl{سـ} and {\em shadda} on the letter \RL{d}.

In addition, the position of an Arabic letter in a word 
(beginning, middle, end, and standalone) changes
its visual form.
Some letters have non-connecting end forms which allows visual
word separation without the need of a white space separator. 
This requires morphological analysis for even the simple tokenization 
task of Arabic text.
%For example, the word \utfrl{ياسمين} can be interpreted as
%the ``Jasmine'' flower, 
%as well as \notrutfrl{يا} (the calling word) followed by
%the word \notrutfrl{سمين} (obese). 
For example, consider the sentence 
%\utfrl{ذهب الولدإلى المدرسة} 
\notrrl{AlmdrsT}\notrrl{_dhb alwald-il_A}
\arabfalse \RL{_dhb alwald-il_A almdrsT} \arabtrue
(the kid went to school). 
The letters \notrutfrl{د} and \notrutfrl{ى} have 
non-connecting end of word forms and the words 
\notrutfrl{الولد}, \notrutfrl{الى}, and  \notrutfrl{المدرسة} 
are visually separable, 
yet there is no space character in between.
Newspaper articles with text justification requirements, 
SMS messages, and automatically digitized documents
are examples where such problems often occur. 

\framework is integrated with {\em Sarf}, 
an in-house open source Arabic morphological analyzer based on 
finite state transducers~\cite{ZaMaColing2012DemosSarf}. 
Given an Arabic word, Sarf returns 
a set of morphological solutions. 
A word might have more than one solution 
due to multiple possible segmentations and multiple tags associated 
with each word. 
A morphological solution is the internal structure of the word 
composed of several morphemes including 
{\em affixes} ({\em prefixes} and {\em suffixes}), and a
{\em stem}, where each morpheme is associated with tags such as 
POS, gloss, and category tags~\cite{arabicmorph,habash2010introduction}. 
%POS, gloss,  {\em lemma}, and category tags~\cite{arabicmorph,habash2010introduction}. 
%A morpheme is the smallest linguistic unit 
%that has a meaning and fulfills a grammatical function. 

Prefixes attach before the stem and a word can have multiple prefixes. 
Suffixes attach after the stem and a word can have multiple suffixes. 
Infixes are inserted inside the stem to form a new stem. 
In this work we consider a set of stems that includes infix morphological changes. 
The part-of-speech tag, referred to as POS, 
assigns a morpho-syntactic tag for a morpheme. 
The gloss is a brief semantic notation of morpheme in English. 
A morpheme might have multiple glosses as it could stand for multiple meanings. 
The category is a user defined tag that the user can assign to several 
morphemes.
For example, the user can define a temporal category to include the 
prefix \utfrl{س} (will) and the time unit \utfrl{ساعة} (hour). 
We denote by 
\ss,
\pp,
\xx,
\PP,
\GG, and 
\AC, the set of 
all stems,
prefixes,
suffixes,
POS,
gloss, 
and user defined category tags, respectively. 

%The lemma is a convention choice that assigns a core meaning of word forms that share similar stems. 
%For example, the words \RL{byt}(house), 
%\RL{llbyt}(for the house), 
%and \RL{bywt}(houses) are mapped to the singular noun form \RL{byt} as a lemma.

\vocalize

%\setarab
\begin{table}[tb!]
  \centering
  \caption{Sample solution vector for \utfrl{فَسَيَأْكُلها}.}
  \resizebox{\columnwidth}{!}{
    \begin{tabular}{|r|c|c|c|c|c|}
          & \multicolumn{3}{c|}{\textbf{Prefixes}} & \textbf{Stem} & \textbf{Suffix} \\
    \textbf{Data} & \utfrl{فَ} & \utfrl{سَ} & \utfrl{يَ} & \utfrl{أْكُل} & \utfrl{ها} \\
    \textbf{POS} & CONJ+ & FUT+ & IV3MS+ & VERB\_IMPERFECT & IVSUFF\_DO:3FS \\
    \textbf{Gloss} & and/so & will & he/it & eat/consume & it/them/her \\
    \textbf{index} & \multicolumn{3}{c|}{10} & 13 & 16 \\
    \textbf{length} & \multicolumn{3}{c|}{3} & 3 & 2 \\
    \end{tabular}%
    }
    \vspace{-3em}
  \label{tab:samplesolution}%
\end{table}%

Table~\ref{tab:samplesolution} shows the morphological analysis
of the \utfrl{فَسَيَأْكُلها}. 
The word is composed of the prefix morphemes 
\utfrl{فَ}, \utfrl{سَ}, and \utfrl{يَ}, followed by the 
stem \utfrl{أْكُل}, and then followed by the suffix morpheme
\utfrl{ها}. 
Each morpheme is associated with a number of morphological features.
The {\tt CONJ},
{\tt FUT}, 
{\tt IV3MS} 
{\tt VERB\_IMPERFECT}, and 
{\tt IVSUFF\_DO:3FS} POS tags indicate
conjunction, 
future, 
third person masculine singular subject pronoun,
an imperfect verb, and 
and a third person feminine singular object pronoun, respectively.
The POS and gloss notations follow the Buckwalter notation~\cite{Buckwalter:02}.

%A morpheme can be a stem or an affix. 
%Each morpheme is associated with other morphological features 
%including {\em POS}, {\em gloss}, {\em lemma}, and {\em category} tags. 

%A stem can be a root word, a {\em templatic}, or {\em non-templatic} stem. 
%Templatic stems are formed from roots using template morphological rules. 
%For example, the stem
%\RL{kAtb} (writer) is a subject noun formed from the root verb \RL{ktb} (write). 
%Non-templatic stems tend to be foreign names such as \RL{واشنطن} (Washington). 
%A root is an atomic word with three, four, and rarely five letters.

%An affix can be a {\em prefix}, a {\em suffix}, or an {\em infix}. 
%The lemma is a convention choice that assigns a core meaning of word forms that share similar stems. 
%For example, the words \RL{byt}(house), 
%\RL{llbyt}(for the house), 
%and \RL{bywt}(houses) are mapped to the singular noun form \RL{byt} as a lemma.

%We introduce this feature in detail in Section~\ref{sec:framework}.

The user interacts with \framework~through a user-friendly interfaces to 
specify morphology-based Boolean formulae and regular expressions 
and associate them with tag types that are in turn associated with
visualization legends such as 
foreground and background colors and fonts. 

{\bf Boolean formula simulator.~}
\framework~uses Sarf to compute morphological solutions for the Arabic text, 
and passes the solutions along with the user-defined tag types to the 
Boolean formula simulator.
The simulator interacts with the $Syn^k$ detector, 
computes tag type matches, and produces a tag set for each word. 
%A word might have multiple tags as its morphological solutions could 
%match multiple Boolean formulae. 
%Each tag contains information about the matching word and the relevant tag type name. 
%Word information include word index relative to the text, character position, and text.
%We formally define the MBF and explain its simulation in Section~\ref{sec:framework}.

{\bf Regular expression simulator.~}
\framework~passes the sequence of tag sets and the user-defined 
tag types with morphology based regular expressions to the regular expression 
simulator. 
The simulator computes tag chunks; i.e. sequences of tags sets, that match the 
regular expressions. 
%Each tag chunk contains information about the matching sequence of words and the tag type with the matching regular expression (MRE). 
%We formally define the MRE and explain its simulation in Section~\ref{sec:framework}.

{\bf Relation extraction and action execution.~}
\framework~enables the user to associate code actions to parts of the regular
expressions. 
The user can use an API to access information such as 
text, position, length, and morphological features of the tag chunks that 
match the sub-expression. 
\framework~also allows the user to declare relations between
matches using the relation editor. 
Each relation is specified by a source, destination, and a label tuple. 
The user selects a feature of a match of sub-expressions to specify the 
tuple entities. 

\framework~executes the user-defined actions corresponding to each 
sub-expression match in a tag chunk. 
It also evaluates the semantic relation declarations against the 
tag chunks to compute the relational matches. 
Finally, \framework~uses the \cci{isA} cross-reference relation 
to create relations across tag type matches. 
%The output of this process expresses entities and relations among them. 
%We formally define the semantic relation and explain its construction in Section~\ref{sec:framework}.

{\bf Visualization annotator.~}
\framework presents the resulting tags to the user incrementally in the form of 
text annotated with style and color legends, match trees, and graphs expressing
the relational entities. 
The annotation can be edited by the user in a user-friendly interface. 
Match trees present the match text associated with the relevant tags 
and regular expression structure. 
The graphs are the result of the user defined relations.
%We present \framework's interface in Section~\ref{sec:gui}.

{\bf Agreement and statistical analysis.~}
\framework provides statistical analysis tools that help compare sets 
of tags to reference tags and compute standard agreement and accuracy measures. 
\framework~ provides criteria for comparison including exact match and intersection. 
This comparison tool can be used to edit the automatically generated annotations. 
%\framework~provides the user with an interactive interface to 
%build the reference tags. 
%We explain the analysis process and its interface in Section~\ref{sec:gui}.

}
\caption{\framework flow diagram.}
\label{f:overview}
\end{figure}

Figure~\ref{f:overview} shows the flow diagram of \framework. 
%A box node represents a \framework process and 
%an ellipse node represents input and output data objects.
The Arabic text and the reference tag chunks are the primary inputs to \framework.
Solutions, morphology-based Boolean formulae, tags, 
morphology-based regular expressions, 
tag chunks, relation and action definitions, and data structures expressing entities 
and relations are input and output data to \framework processes. 
The morphological analyzer (Sarf), $Syn^k$ detector, 
GUI for Boolean formulae definition, 
visualization annotator, GUI for regular expression and action definition, 
Boolean formula simulator, regular expression simulator, 
relation extraction and action execution, and difference and statistical analyzer 
are \framework processes.

\def\pp{\ensuremath{{\cal P}}} % prefix
\def\ss{\ensuremath{{\cal S}}} % stem
\def\xx{\ensuremath{{\cal X}}} % suffix
\def\PP{\ensuremath{\mathit{POS}}} % pos
\def\GG{\ensuremath{\mathit{GLOSS}}} % gloss
\def\AC{\ensuremath{\mathit{CAT}}} % category

%~\footnote{In this document, we use the default ArabTeX transliteration style ZDMG.}
{\bf Morphological analyzer (Sarf).~}
\label{s:s:morph}
Morphological analysis is key to Arabic NLP~\cite{arabicmorph}.
Short vowels, also known as diacritics, are typically omitted in Arabic text
and inferred by readers~\cite{habash2006arabic}. 
For example, the word \RL{'sd}%
can be interpreted as \RL{'sad} {\tt ``lion''} with a {\em fatha} diacritic on the 
letter \utfrl{سـ} or 
\vocalize \RL{'sodd} (I block) with a 
{\em damma} diacritic on the letter \utfrl{سـ} and {\em shadda} on the letter \RL{d}.

In addition, the position of an Arabic letter in a word 
(beginning, middle, end, and standalone) changes
its visual form.
Some letters have non-connecting end forms which allows visual
word separation without the need of a white space separator. 
This requires morphological analysis for even the simple tokenization 
task of Arabic text.
%For example, the word \utfrl{ياسمين} can be interpreted as
%the ``Jasmine'' flower, 
%as well as \notrutfrl{يا} (the calling word) followed by
%the word \notrutfrl{سمين} (obese). 
For example, consider the sentence 
%\utfrl{ذهب الولدإلى المدرسة} 
\notrrl{AlmdrsT}\notrrl{_dhb alwald-il_A}
\arabfalse \RL{_dhb alwald-il_A almdrsT} \arabtrue
(the kid went to school). 
The letters \notrutfrl{د} and \notrutfrl{ى} have 
non-connecting end of word forms and the words 
\notrutfrl{الولد}, \notrutfrl{الى}, and  \notrutfrl{المدرسة} 
are visually separable, 
yet there is no space character in between.
Newspaper articles with text justification requirements, 
SMS messages, and automatically digitized documents
are examples where such problems often occur. 

\framework is integrated with {\em Sarf}, 
an in-house open source Arabic morphological analyzer based on 
finite state transducers~\cite{ZaMaColing2012DemosSarf}. 
Given an Arabic word, Sarf returns 
a set of morphological solutions. 
A word might have more than one solution 
due to multiple possible segmentations and multiple tags associated 
with each word. 
A morphological solution is the internal structure of the word 
composed of several morphemes including 
{\em affixes} ({\em prefixes} and {\em suffixes}), and a
{\em stem}, where each morpheme is associated with tags such as 
POS, gloss, and category tags~\cite{arabicmorph,habash2010introduction}. 
%POS, gloss,  {\em lemma}, and category tags~\cite{arabicmorph,habash2010introduction}. 
%A morpheme is the smallest linguistic unit 
%that has a meaning and fulfills a grammatical function. 

Prefixes attach before the stem and a word can have multiple prefixes. 
Suffixes attach after the stem and a word can have multiple suffixes. 
Infixes are inserted inside the stem to form a new stem. 
In this work we consider a set of stems that includes infix morphological changes. 
The part-of-speech tag, referred to as POS, 
assigns a morpho-syntactic tag for a morpheme. 
The gloss is a brief semantic notation of morpheme in English. 
A morpheme might have multiple glosses as it could stand for multiple meanings. 
The category is a user defined tag that the user can assign to several 
morphemes.
For example, the user can define a temporal category to include the 
prefix \utfrl{س} (will) and the time unit \utfrl{ساعة} (hour). 
We denote by 
\ss,
\pp,
\xx,
\PP,
\GG, and 
\AC, the set of 
all stems,
prefixes,
suffixes,
POS,
gloss, 
and user defined category tags, respectively. 

%The lemma is a convention choice that assigns a core meaning of word forms that share similar stems. 
%For example, the words \RL{byt}(house), 
%\RL{llbyt}(for the house), 
%and \RL{bywt}(houses) are mapped to the singular noun form \RL{byt} as a lemma.

\vocalize

%\setarab
\begin{table}[tb!]
  \centering
  \caption{Sample solution vector for \utfrl{فَسَيَأْكُلها}.}
  \resizebox{\columnwidth}{!}{
    \begin{tabular}{|r|c|c|c|c|c|}
          & \multicolumn{3}{c|}{\textbf{Prefixes}} & \textbf{Stem} & \textbf{Suffix} \\
    \textbf{Data} & \utfrl{فَ} & \utfrl{سَ} & \utfrl{يَ} & \utfrl{أْكُل} & \utfrl{ها} \\
    \textbf{POS} & CONJ+ & FUT+ & IV3MS+ & VERB\_IMPERFECT & IVSUFF\_DO:3FS \\
    \textbf{Gloss} & and/so & will & he/it & eat/consume & it/them/her \\
    \textbf{index} & \multicolumn{3}{c|}{10} & 13 & 16 \\
    \textbf{length} & \multicolumn{3}{c|}{3} & 3 & 2 \\
    \end{tabular}%
    }
    \vspace{-3em}
  \label{tab:samplesolution}%
\end{table}%

Table~\ref{tab:samplesolution} shows the morphological analysis
of the \utfrl{فَسَيَأْكُلها}. 
The word is composed of the prefix morphemes 
\utfrl{فَ}, \utfrl{سَ}, and \utfrl{يَ}, followed by the 
stem \utfrl{أْكُل}, and then followed by the suffix morpheme
\utfrl{ها}. 
Each morpheme is associated with a number of morphological features.
The {\tt CONJ},
{\tt FUT}, 
{\tt IV3MS} 
{\tt VERB\_IMPERFECT}, and 
{\tt IVSUFF\_DO:3FS} POS tags indicate
conjunction, 
future, 
third person masculine singular subject pronoun,
an imperfect verb, and 
and a third person feminine singular object pronoun, respectively.
The POS and gloss notations follow the Buckwalter notation~\cite{Buckwalter:02}.

%A morpheme can be a stem or an affix. 
%Each morpheme is associated with other morphological features 
%including {\em POS}, {\em gloss}, {\em lemma}, and {\em category} tags. 

%A stem can be a root word, a {\em templatic}, or {\em non-templatic} stem. 
%Templatic stems are formed from roots using template morphological rules. 
%For example, the stem
%\RL{kAtb} (writer) is a subject noun formed from the root verb \RL{ktb} (write). 
%Non-templatic stems tend to be foreign names such as \RL{واشنطن} (Washington). 
%A root is an atomic word with three, four, and rarely five letters.

%An affix can be a {\em prefix}, a {\em suffix}, or an {\em infix}. 
%The lemma is a convention choice that assigns a core meaning of word forms that share similar stems. 
%For example, the words \RL{byt}(house), 
%\RL{llbyt}(for the house), 
%and \RL{bywt}(houses) are mapped to the singular noun form \RL{byt} as a lemma.

%We introduce this feature in detail in Section~\ref{sec:framework}.

The user interacts with \framework~through a user-friendly interfaces to 
specify morphology-based Boolean formulae and regular expressions 
and associate them with tag types that are in turn associated with
visualization legends such as 
foreground and background colors and fonts. 

{\bf Boolean formula simulator.~}
\framework~uses Sarf to compute morphological solutions for the Arabic text, 
and passes the solutions along with the user-defined tag types to the 
Boolean formula simulator.
The simulator interacts with the $Syn^k$ detector, 
computes tag type matches, and produces a tag set for each word. 
%A word might have multiple tags as its morphological solutions could 
%match multiple Boolean formulae. 
%Each tag contains information about the matching word and the relevant tag type name. 
%Word information include word index relative to the text, character position, and text.
%We formally define the MBF and explain its simulation in Section~\ref{sec:framework}.

{\bf Regular expression simulator.~}
\framework~passes the sequence of tag sets and the user-defined 
tag types with morphology based regular expressions to the regular expression 
simulator. 
The simulator computes tag chunks; i.e. sequences of tags sets, that match the 
regular expressions. 
%Each tag chunk contains information about the matching sequence of words and the tag type with the matching regular expression (MRE). 
%We formally define the MRE and explain its simulation in Section~\ref{sec:framework}.

{\bf Relation extraction and action execution.~}
\framework~enables the user to associate code actions to parts of the regular
expressions. 
The user can use an API to access information such as 
text, position, length, and morphological features of the tag chunks that 
match the sub-expression. 
\framework~also allows the user to declare relations between
matches using the relation editor. 
Each relation is specified by a source, destination, and a label tuple. 
The user selects a feature of a match of sub-expressions to specify the 
tuple entities. 

\framework~executes the user-defined actions corresponding to each 
sub-expression match in a tag chunk. 
It also evaluates the semantic relation declarations against the 
tag chunks to compute the relational matches. 
Finally, \framework~uses the \cci{isA} cross-reference relation 
to create relations across tag type matches. 
%The output of this process expresses entities and relations among them. 
%We formally define the semantic relation and explain its construction in Section~\ref{sec:framework}.

{\bf Visualization annotator.~}
\framework presents the resulting tags to the user incrementally in the form of 
text annotated with style and color legends, match trees, and graphs expressing
the relational entities. 
The annotation can be edited by the user in a user-friendly interface. 
Match trees present the match text associated with the relevant tags 
and regular expression structure. 
The graphs are the result of the user defined relations.
%We present \framework's interface in Section~\ref{sec:gui}.

{\bf Agreement and statistical analysis.~}
\framework provides statistical analysis tools that help compare sets 
of tags to reference tags and compute standard agreement and accuracy measures. 
\framework~ provides criteria for comparison including exact match and intersection. 
This comparison tool can be used to edit the automatically generated annotations. 
%\framework~provides the user with an interactive interface to 
%build the reference tags. 
%We explain the analysis process and its interface in Section~\ref{sec:gui}.

}
\caption{\framework flow diagram.}
\label{f:overview}
\end{figure}

Figure~\ref{f:overview} shows the flow diagram of \framework. 
%A box node represents a \framework process and 
%an ellipse node represents input and output data objects.
The Arabic text and the reference tag chunks are the primary inputs to \framework.
Solutions, morphology-based Boolean formulae, tags, 
morphology-based regular expressions, 
tag chunks, relation and action definitions, and data structures expressing entities 
and relations are input and output data to \framework processes. 
The morphological analyzer (Sarf), $Syn^k$ detector, 
GUI for Boolean formulae definition, 
visualization annotator, GUI for regular expression and action definition, 
Boolean formula simulator, regular expression simulator, 
relation extraction and action execution, and difference and statistical analyzer 
are \framework processes.

\def\pp{\ensuremath{{\cal P}}} % prefix
\def\ss{\ensuremath{{\cal S}}} % stem
\def\xx{\ensuremath{{\cal X}}} % suffix
\def\PP{\ensuremath{\mathit{POS}}} % pos
\def\GG{\ensuremath{\mathit{GLOSS}}} % gloss
\def\AC{\ensuremath{\mathit{CAT}}} % category

%~\footnote{In this document, we use the default ArabTeX transliteration style ZDMG.}
{\bf Morphological analyzer (Sarf).~}
\label{s:s:morph}
Morphological analysis is key to Arabic NLP~\cite{arabicmorph}.
Short vowels, also known as diacritics, are typically omitted in Arabic text
and inferred by readers~\cite{habash2006arabic}. 
For example, the word \RL{'sd}%
can be interpreted as \RL{'sad} {\tt ``lion''} with a {\em fatha} diacritic on the 
letter \utfrl{سـ} or 
\vocalize \RL{'sodd} (I block) with a 
{\em damma} diacritic on the letter \utfrl{سـ} and {\em shadda} on the letter \RL{d}.

In addition, the position of an Arabic letter in a word 
(beginning, middle, end, and standalone) changes
its visual form.
Some letters have non-connecting end forms which allows visual
word separation without the need of a white space separator. 
This requires morphological analysis for even the simple tokenization 
task of Arabic text.
%For example, the word \utfrl{ياسمين} can be interpreted as
%the ``Jasmine'' flower, 
%as well as \notrutfrl{يا} (the calling word) followed by
%the word \notrutfrl{سمين} (obese). 
For example, consider the sentence 
%\utfrl{ذهب الولدإلى المدرسة} 
\notrrl{AlmdrsT}\notrrl{_dhb alwald-il_A}
\arabfalse \RL{_dhb alwald-il_A almdrsT} \arabtrue
(the kid went to school). 
The letters \notrutfrl{د} and \notrutfrl{ى} have 
non-connecting end of word forms and the words 
\notrutfrl{الولد}, \notrutfrl{الى}, and  \notrutfrl{المدرسة} 
are visually separable, 
yet there is no space character in between.
Newspaper articles with text justification requirements, 
SMS messages, and automatically digitized documents
are examples where such problems often occur. 

\framework is integrated with {\em Sarf}, 
an in-house open source Arabic morphological analyzer based on 
finite state transducers~\cite{ZaMaColing2012DemosSarf}. 
Given an Arabic word, Sarf returns 
a set of morphological solutions. 
A word might have more than one solution 
due to multiple possible segmentations and multiple tags associated 
with each word. 
A morphological solution is the internal structure of the word 
composed of several morphemes including 
{\em affixes} ({\em prefixes} and {\em suffixes}), and a
{\em stem}, where each morpheme is associated with tags such as 
POS, gloss, and category tags~\cite{arabicmorph,habash2010introduction}. 
%POS, gloss,  {\em lemma}, and category tags~\cite{arabicmorph,habash2010introduction}. 
%A morpheme is the smallest linguistic unit 
%that has a meaning and fulfills a grammatical function. 

Prefixes attach before the stem and a word can have multiple prefixes. 
Suffixes attach after the stem and a word can have multiple suffixes. 
Infixes are inserted inside the stem to form a new stem. 
In this work we consider a set of stems that includes infix morphological changes. 
The part-of-speech tag, referred to as POS, 
assigns a morpho-syntactic tag for a morpheme. 
The gloss is a brief semantic notation of morpheme in English. 
A morpheme might have multiple glosses as it could stand for multiple meanings. 
The category is a user defined tag that the user can assign to several 
morphemes.
For example, the user can define a temporal category to include the 
prefix \utfrl{س} (will) and the time unit \utfrl{ساعة} (hour). 
We denote by 
\ss,
\pp,
\xx,
\PP,
\GG, and 
\AC, the set of 
all stems,
prefixes,
suffixes,
POS,
gloss, 
and user defined category tags, respectively. 

%The lemma is a convention choice that assigns a core meaning of word forms that share similar stems. 
%For example, the words \RL{byt}(house), 
%\RL{llbyt}(for the house), 
%and \RL{bywt}(houses) are mapped to the singular noun form \RL{byt} as a lemma.

\vocalize

%\setarab
\begin{table}[tb!]
  \centering
  \caption{Sample solution vector for \utfrl{فَسَيَأْكُلها}.}
  \resizebox{\columnwidth}{!}{
    \begin{tabular}{|r|c|c|c|c|c|}
          & \multicolumn{3}{c|}{\textbf{Prefixes}} & \textbf{Stem} & \textbf{Suffix} \\
    \textbf{Data} & \utfrl{فَ} & \utfrl{سَ} & \utfrl{يَ} & \utfrl{أْكُل} & \utfrl{ها} \\
    \textbf{POS} & CONJ+ & FUT+ & IV3MS+ & VERB\_IMPERFECT & IVSUFF\_DO:3FS \\
    \textbf{Gloss} & and/so & will & he/it & eat/consume & it/them/her \\
    \textbf{index} & \multicolumn{3}{c|}{10} & 13 & 16 \\
    \textbf{length} & \multicolumn{3}{c|}{3} & 3 & 2 \\
    \end{tabular}%
    }
    \vspace{-3em}
  \label{tab:samplesolution}%
\end{table}%

Table~\ref{tab:samplesolution} shows the morphological analysis
of the \utfrl{فَسَيَأْكُلها}. 
The word is composed of the prefix morphemes 
\utfrl{فَ}, \utfrl{سَ}, and \utfrl{يَ}, followed by the 
stem \utfrl{أْكُل}, and then followed by the suffix morpheme
\utfrl{ها}. 
Each morpheme is associated with a number of morphological features.
The {\tt CONJ},
{\tt FUT}, 
{\tt IV3MS} 
{\tt VERB\_IMPERFECT}, and 
{\tt IVSUFF\_DO:3FS} POS tags indicate
conjunction, 
future, 
third person masculine singular subject pronoun,
an imperfect verb, and 
and a third person feminine singular object pronoun, respectively.
The POS and gloss notations follow the Buckwalter notation~\cite{Buckwalter:02}.

%A morpheme can be a stem or an affix. 
%Each morpheme is associated with other morphological features 
%including {\em POS}, {\em gloss}, {\em lemma}, and {\em category} tags. 

%A stem can be a root word, a {\em templatic}, or {\em non-templatic} stem. 
%Templatic stems are formed from roots using template morphological rules. 
%For example, the stem
%\RL{kAtb} (writer) is a subject noun formed from the root verb \RL{ktb} (write). 
%Non-templatic stems tend to be foreign names such as \RL{واشنطن} (Washington). 
%A root is an atomic word with three, four, and rarely five letters.

%An affix can be a {\em prefix}, a {\em suffix}, or an {\em infix}. 
%The lemma is a convention choice that assigns a core meaning of word forms that share similar stems. 
%For example, the words \RL{byt}(house), 
%\RL{llbyt}(for the house), 
%and \RL{bywt}(houses) are mapped to the singular noun form \RL{byt} as a lemma.

%We introduce this feature in detail in Section~\ref{sec:framework}.

The user interacts with \framework~through a user-friendly interfaces to 
specify morphology-based Boolean formulae and regular expressions 
and associate them with tag types that are in turn associated with
visualization legends such as 
foreground and background colors and fonts. 

{\bf Boolean formula simulator.~}
\framework~uses Sarf to compute morphological solutions for the Arabic text, 
and passes the solutions along with the user-defined tag types to the 
Boolean formula simulator.
The simulator interacts with the $Syn^k$ detector, 
computes tag type matches, and produces a tag set for each word. 
%A word might have multiple tags as its morphological solutions could 
%match multiple Boolean formulae. 
%Each tag contains information about the matching word and the relevant tag type name. 
%Word information include word index relative to the text, character position, and text.
%We formally define the MBF and explain its simulation in Section~\ref{sec:framework}.

{\bf Regular expression simulator.~}
\framework~passes the sequence of tag sets and the user-defined 
tag types with morphology based regular expressions to the regular expression 
simulator. 
The simulator computes tag chunks; i.e. sequences of tags sets, that match the 
regular expressions. 
%Each tag chunk contains information about the matching sequence of words and the tag type with the matching regular expression (MRE). 
%We formally define the MRE and explain its simulation in Section~\ref{sec:framework}.

{\bf Relation extraction and action execution.~}
\framework~enables the user to associate code actions to parts of the regular
expressions. 
The user can use an API to access information such as 
text, position, length, and morphological features of the tag chunks that 
match the sub-expression. 
\framework~also allows the user to declare relations between
matches using the relation editor. 
Each relation is specified by a source, destination, and a label tuple. 
The user selects a feature of a match of sub-expressions to specify the 
tuple entities. 

\framework~executes the user-defined actions corresponding to each 
sub-expression match in a tag chunk. 
It also evaluates the semantic relation declarations against the 
tag chunks to compute the relational matches. 
Finally, \framework~uses the \cci{isA} cross-reference relation 
to create relations across tag type matches. 
%The output of this process expresses entities and relations among them. 
%We formally define the semantic relation and explain its construction in Section~\ref{sec:framework}.

{\bf Visualization annotator.~}
\framework presents the resulting tags to the user incrementally in the form of 
text annotated with style and color legends, match trees, and graphs expressing
the relational entities. 
The annotation can be edited by the user in a user-friendly interface. 
Match trees present the match text associated with the relevant tags 
and regular expression structure. 
The graphs are the result of the user defined relations.
%We present \framework's interface in Section~\ref{sec:gui}.

{\bf Agreement and statistical analysis.~}
\framework provides statistical analysis tools that help compare sets 
of tags to reference tags and compute standard agreement and accuracy measures. 
\framework~ provides criteria for comparison including exact match and intersection. 
This comparison tool can be used to edit the automatically generated annotations. 
%\framework~provides the user with an interactive interface to 
%build the reference tags. 
%We explain the analysis process and its interface in Section~\ref{sec:gui}.


\section{Application specific API}
\label{sec:api}
\label{s:api}
Sarf provides the developer with an application programming interface (API) 
  that allows to 
  (1) define developer categories and associate them with specific morphemes, 
  (2) provide rules that prioritize and filter the solution features, and 
  (3) control and refine the morphological analysis on the fly at solution {\em control points}.
%
The control points are (1) agglutinative prefix matches, (2) stem matches, 
(3) agglutinative suffix matches, and 
(4) full solution matches. 

\todo{R1.3}
To use the API, an NLP application developer is required to inherit from 
one or more of the prefix, suffix, stem and general 
stemmer C++ classes and implement the \textit{OnMatch} interface. 
The developer configures the feature priority and selection rules 
by using the \textit{setSolutionSetting} method to push the desired features 
into a feature configuration vector. 
The computation returns the features present in vector in the solutions ordered 
by the same order in the vector. 
If the vector is empty, all features are considered. 

The developer implements the \textit{OnMatch} interface that Sarf calls at the 
control points. 
The developer processes the solution features provided at the control point and returns 
a value that instructs Sarf to 
(1) proceed with the analysis ignoring the current solution, 
(2) accept the solution and continue considering other solutions, 
and (3) accept the solution and stop the analysis.
%
The developer can inspect at each control point 
the agglutinative morphemes and their compatibility category tags, 
the VMF, gloss, POS, lemma, and the developer-defined category tags. 

Consider the task that aims to detect words with possible VERB POS tags. 
The API can be implemented to reject the analysis at the stem control point if the POS tag is not a VERB. 
This prevents the analyzer from computing insignificant 
full morphological solutions at early stages of the analysis.
The developer can also use the feature selection filter API to 
disregard features such as VMF, gloss, and categories. 
This reduces the size of the solution trees and 
their corresponding traversal time. 
%
For example, given the word \RL{'kl}, 
with gloss tags such as `he/it eat' and `fruit',
Sarf typically returns nine possible morphological solutions 
with different VMF, POS tags, and gloss tags. 
With the feature selection filter API, Sarf considers only two solutions with 
two solutions with the {\tt VERB\_PERFECT} and {\tt NOUN} POS tags.

Sarf also provides the developer with the ability to alter 
the structure of the morphological solutions so 
that the traversal of the solutions is application specific. 
The developer can provide a factorization order 
of the solution features using the API priority rules. 
Sarf uses the priority rules to build the structures. 
This allows the developer to dismiss analysis earlier at the control points. 
Consider the task with an aim is to detect adjectives 
with positive sentiment. 
Since the POS feature is more limited than the gloss feature, 
it might give priority to the POS over the gloss. 
Hence, it can filter invalid solutions early in 
the analysis.

Moreover, Sarf enables the application developer to define 
categories and associate them with existing morphemes. 
Consider the task of detecting words that indicate 
family relations such as \RL{ibn} (son), \RL{'b} (father), 
and \RL{'m} (mother). 
The developer can define a category called `family connections' 
and associate it with the stems \RL{ibn}, \RL{'b}, \RL{'m} 
or with their relevant glosses. 
The user defined categories are attached to the tags in the 
morpheme nodes as auxiliary tags that can be looked up in constant time. 

%During the analysis, Sarf calls the API at the control points and 
%the API detect the morphemes that fall under the defined category.


%\section{Agglutinative and fusional morphemes}
%\label{sec:aggaffix}
%%\vocalize
Sarf considers three types of affixes: 
%\vspace{-2em}
\begin{itemize}
  \item {\em Atomic affix morphemes } such as \noTrutfRL{يـ} can be affixes on their own and can directly connect to stems using the $R_{ps}$ and $R_{sx}$ rules. 
  \item {\em Partial affix morphemes} such as \noTrutfRL{سـ} can not be affixes on their own and need to connect to other affixes before they connect to a stem.
  \item {\em Compound affixes} are concatenations of atomic and partial affix morphemes as well as other smaller compound affixes. 
    They can connect to stems according to the $R_{ps}$ and $R_{sx}$ rules. 
\end{itemize}

Sarf forms compound affixes from atomic and partial affix morphemes using newly introduced prefix-prefix $R_{pp}$ and suffix-suffix $R_{xx}$ concatenation rules. 
%
Sarf considers $L_p$ and $L_x$ to be lexicons of atomic and partial affix morphemes associated with their tags. 
Sarf forms agglutinative affixes using prefix-prefix $R_{pp}$ and suffix-suffix $R_{xx}$ concatenation or agglutination rules.
A rule $r \in R_{pp} \cup R_{xx}$ takes the compatibility category tags of affixes $a_1$ and $a_2$ and checks whether they can be concatenated. 
If so, the rule takes $a_1$ and $a_2$ and their tags and generates the
affix $a=r(a_1,a_2)$ with its associated tags according to substitution rules based on regular expressions. 
The rules are fusional in the sense that they modify the orthography and the semantics of the resulting 
affixes by more than simple concatenation.

\begin{table}[!tb]
\begin{minipage}{0.95\textwidth}
\relsize{-1.5} {
  \begin{tabular}{lp{4.4cm}cp{3.3cm}}
\hline \hline
{\bf Category 1} & {\bf Category 2} & {\bf Resulting Category}  & {\bf Substitution rules} \\ \hline
        NPref-Bi & NPref-Al & NPref-BiAl &  \\[3pt]
        NPref-Li & NPref-Al & NPref-LiAl & \em r$//$\noTrRL{Al} $||$\noTrRL{l}$\backslash\backslash$ \\[4pt]
 Pref-Wa & none of \{ Pref-0, NPref-La, PVPref-La\} & \$2 & \\[4pt]
%Pref-sa- & \{IVPref* AND NOT IVPref-li-\} & \{\$2\} \\
IVPref-li- & IVPref-*-y* & IVPref-(@1)-liy(@2)  & { \em d$//$he/$||$him/$\backslash\backslash$, ~ d$//$they$||$them$\backslash\backslash\ldots$ ~ d$//$(+2)$||$ to$\backslash\backslash$} \\ 
\hline \hline
\end{tabular}
}
\end{minipage}
\caption{Example rules from $R_{pp}$ }
%\vspace{-1cm}
\label{t:rules}
\end{table}
%
We illustrate this with the example rules in Table~\ref{t:rules}.
Row 1 presents a simple rule that allows the concatenation of prefixes with category \cci{NPref-Bi} such as \RL{bi-} and \RL{ka-}
to prefixes with category \cci{NPref-Al} such as \noTrRL{Al}, the result is the compound prefix with category \cci{ NPref-BiAl}. 
Since no substitution rule is specified, the tags of the resulting prefix are simple concatenations.

Row 2 presents a rule that takes prefixes with category 
\cci{NPref-Li} such as \RL{li-} and prefixes with category \cci{NPref-Al} such as \noTrRL{Al}.
The substitution rule replaces the \noTrRL{Al-} with \noTrRL{l-} resulting in \noTrRL{lil-}. 
The syntax of the substitution rule for the affix form is \cci{r$//$(substring)$||$(replacement)$\backslash\backslash$}. 

The rule in the Row 3 states that prefixes of category \cci{Pref-Wa} can be concatenated with 
prefixes with categories that are neither of 
\cci{Pref-0}, \cci{NPref-La}, and \cci{PVPref-La} categories. 
%The Boolean operators allowed are \cci{AND}, \cci{OR}, and \cci{NOT}.
The resulting category is denoted with $\$2$ %$
which means the category of the second prefix.

Row 4 illustrates the use of the wild character \chci{*} to capture sub-strings of length zero or more in the second category,
and refers to the captured sub-strings in the resulting category using the \chci{@} operator. 
The \chci{@} operator is always followed by a number that denotes the captured \chci{*} expression.
Row 4 has also an example of substitution rules for the gloss (description) tag that start with the letter $d$.
The \cci{+2} pattern in the last substitution rule means that the \cci{to} partial gloss description should be appended after the gloss 
of the second affix.
Substitution rules for POS tags start with the letter $p$.

\todo{R3.10}
The rule introduced in row 4 replaces 24 prefix category rules in BAMA. 
Given the word \utfRL{ليأكل} (for him/it to eat/consume), 
Sarf returns the solution with the prefixes \noTrutfRL{ل} (for) and \noTrutfRL{ي} (him/it), 
and the stem \noTrutfRL{أكل} (eat/consume). 
The prefixes \noTrutfRL{ل} and \noTrutfRL{ي} have the categories 
\cci{IVPref-li-} and \cci{IVPref-hw-ya}, respectively. 
The category \cci{IVPref-li-} is the same as category 1 of the rule in row 4 and 
the category \cci{IVPref-hw-ya} matches the pattern of category 2 of the same rule. 
The first wild character within the pattern captures the sub-string \cci{hw} 
while the second wild character captures the sub-string \cci{a}. 
Sarf recognizes that those two affixes are compatible using this rule, 
and applies the corresponding expression to get the resulting category 
\cci{IVPref-hw-liya} for the compound affix \noTrutfRL{لي}. 
Then, Sarf matches the result category of the compound affix 
with the category \cci{IV\_no-Pref-A} of the stem \noTrutfRL{أكل} 
and returns a morphological solution.

%\vspace{-3em}
\subsection{Building $R_{pp}$ and $R_{xx}$}
\label{ss:affix-affix}

Our method is in line with native Arabic textbooks on morphology and syntax~\citep*{Mosad09,Abd00,Abd001} 
where only atomic and partial affixes are introduced.
The textbooks also list rules to concatenate the affixes and discuss the syntax, semantic, and phonological forms of the
resulting affixes. 
%This representation is more suited for linguists.
For example, the fourth rule in Table~\ref{t:rules} is derived from a textbook rule that states 
\cci{IVPref-li-} prefixes connect to all imperfect verb prefixes and transform the subject pronoun in the gloss 
to an object pronoun.

The method built the rules in four steps: 
%\vspace{-2em}
\begin{enumerate}
    \item In the first step, we encoded textbook morphological rules into patterns.
    \item In the second step, we inspected the BAMA and SAMA affix lexicons and extracted the atomic and partial affixes from them.
    \item Then, we grouped the rest of the BAMA and SAMA affixes into the rules we collected from the textbooks. 
    \item We refined the rules wherever necessary, and we grouped rules that shared the same patterns. 
\end{enumerate}
%\vspace{-1em}

We validated our work by generating all possible affixes and compared them against the BAMA and SAMA affix lexicons. 
The comparison resulted in discovering the BAMA and SAMA inconsistencies listed in Tables ~\ref{t:incBAMA} and ~\ref{t:incSAMA}.


%%\subsection{Redundancy and inconsistencies}
%%\label{sec:motivation}
%\subsection{Redundancy} 
\label{s:s:redundant}

Consider the partial lexicon of prefixes in Table~\ref{t:affixes}. 
\todo{R1.9.}The first five rows can be replaced with two atomic affix morphemes ( \RL{f} and \RL{y})  
and one partial affix morpheme (\RL{s}) in $L_{p}$, 
and three rules to generate compound morphemes in $R_{pp}$ (r (\RL{f}, \RL{y}), r (\RL{f}, \RL{s}), r (\RL{s}, \RL{y})). 
Representing prefix \RL{ya} (them/both) required four entries, three of them only differ
in their dependency on the added \RL{ya}. 
Representing prefix \RL{w} required the addition of five entries.
\todo{1.10}
With Sarf, the equivalent addition of \RL{ya} (them/both) requires only two rules in $R_{pp}$ that relate the categories of \RL{f} and \RL{s} to that of \RL{y}. 
The addition of \RL{w} requires only one additional entry (\RL{w}) in $L_p$.
The difference is much larger when we consider the full lexicon as will be shown 
in Section~\ref{sec:results}.

\subsection{Inconsistencies} 
\label{s:s:inc}

The entries in Tables~\ref{t:incBAMA} and ~\ref{t:incSAMA} list examples of the 197 and 208 inconsistencies detected in the affix 
lexicons of BAMA version 1.2 and SAMA version 3.1, respectively.
We found a small number of these inconsistencies manually and we computed the full list via comparing $L_p$ and $L_x$ with 
their counterparts computed using our agglutinative affixes. 
Most of the inconsistencies are direct results of partially redundant entries with erroneous tags. Experiments described later in section \ref{s:s:lexicon_consistency} shows that most of the detected inconsistencies are specifically related to gloss (Table~\ref{t:comp:SAMA}). \todo{3.12}
We note that SAMA corrected several BAMA inconsistencies, but also introduced 
several new ones when modifying existing entries to meet new standards and when introducing new entries. 

%%%%%%%%%%%% BAMA inconsistencies table %%%%%%%%%%%%%%%%
\begin{table}[!tbp]
\begin{minipage}{0.95\textwidth}
\relsize{-1.5} {
  \begin{tabular}{p{2.5cm}ccc}
\hline \hline
   & {\bf Affix } & \textbf{Vocalized} & {\bf Inconsistent tag} \\ 
\hline
\multirow{2}{2.5cm}{ {\bf \ref{inc:plus})}
missing plus in {\bf gloss} tag  of {\bf prefix}}
 & \RL{fl} & \RL{fali} & and/so \fbox{+} for/to + the  \\
% & \RL{fl--l} & \RL{falil} & and/so \fbox{+} for/to + the \\
 & \RL{wbAl} & \RL{wabiAl} & and + with/by \fbox{+} the  \\[6pt]
\multirow{3}{2.5cm}{{\bf \ref{inc:so})} 
  missing alternative {\bf gloss} %for fa
  in {\bf prefix} }
 & \RL{f} & \RL{fa} & and/so  \\ 
 & \RL{fb} & \RL{fabi} & and \fbox{/so} + with/by          \\
 & \RL{fk} & \RL{faka} & and \fbox{/so} + like/such as     \\[6pt]
% & \RL{fl} & \RL{fali} & and \fbox{/so} + for/to           \\
% & \RL{fl} & \RL{fali} & and \fbox{/so} + for/to + the    \\
\multirow{4}{2.5cm}{ {\bf \ref{inc:num_gender_subject}) } 
  gender/number qualifier omitted in {\bf gloss} of subject {\bf suffix} }
 & \RL{n} & \RL{n} & they [fem.pl.] $<$verb$>$                     \\
 & \RL{nhm} & \RL{nahom} & they \fbox{[fem.pl.]} $<$verb$>$ them \\
 & \RL{At} & \RL{At} & [fem.pl.]                                \\
 & \RL{Atk} & \RL{Atka} & \fbox{[fem.pl.]} your                \\[6pt]
% & \multicolumn{4}{c}{48 total inconsistencies} \\ 
\multirow{2}{2.5cm}{
  {\bf \ref{inc:num_gender_object}) } 
  also in {\bf gloss} of object {\bf suffix} }
 & \noVocRL{AnhmA}&  \RL{AnihimA}& them \fbox{(both)}                      \\
 & \noVocRL{tkm}& \utfRL{|Bَتْكُم}& it/they/she $<$verb$>$ you \fbox{(pl.)}  \\[6pt]
% & \multicolumn{4}{c}{39 total inconsistencies} \\ 
\multirow{3}{2.5cm}{
  {\bf \ref{inc:both}) } 
  different ways to express them (both) in {\bf gloss} of {\bf suffix} }
 &\noVocRL{thmA}&  \utfRL{|Bَتْهُما} & it/they/she $<$verb$>$ them (both)    \\
 & \noVocRL{AhmA} &  \utfRL{|BBاهُما} & we $<$verb$>$ 
 \colorbox{shadecolor}{\color{white} (both of)} them \fbox{(both)}  \\
\vocalize
 & \RL{nAhmA} &  \utfRL{nAhُmA} & we $<$verb$>$ 
\colorbox{shadecolor}{\color{white} (both of)} them \fbox{(both)}   \\[6pt]
% & \RL{nhmA} &  \utfRL{nَhُmA} & they $<$verb$>$ 
%        \colorbox{shadecolor}{\color{white} (both of)} them \fbox{(both)}  \\
% & \RL{hmA} &  \utfRL{|BBهُما} & they $<$verb$>$ 
%        \colorbox{shadecolor}{\color{white} (both of)} them \fbox{(both)}  \\
\multirow{2}{2.5cm}{
  {\bf \ref{inc:dot}) } 
  `.' omitted  after pl in {\bf gloss} }
 & \utfRL{م} & \utfRL{|Bُّم} & you [masc.pl.] $<$verb$>$              \\
% & \utfRL{وا} & \utfRL{|Bُوا} & you [masc.pl\fbox{.}] $<$verb$>$     \\ 
 & \utfRL{ونا} & \utfRL{|Bُونا} & you [masc.pl\fbox{.}] $<$verb$>$ us \\[6pt]
% & \multicolumn{4}{c}{8 total  inconsistencies } \\        
\multirow{2}{2.5cm}{
  {\bf \ref{inc:pos}) } 
  {\bf POS} tag is not same as vocalized 
}
 & \noVocRL{th} & \utfRL{thِ}     & +ti/PVSUFF\_SUBJ:2FS  \\
 &              &                & +\colorbox{shadecolor}{\color{white} hu/}\fbox{hi/}PVSUFF\_DO:3MS \\
% & \utfRL{ونا}  & \utfRL{|Bَوْنا} & +\colorbox{shadecolor}{\color{white} uw/}\fbox{aw/}PVSUFF\_SUBJ:3MP  \\ 
% &              &               & +nA/PVSUFF\_DO:1P \\
%% & \multicolumn{4}{c}{38 total inconsistencies} \\         
\hline \hline
\end{tabular}
}
\end{minipage}
\caption{Sample BAMA inconsistencies}
\label{t:incBAMA}
\end{table}



The following describes the BAMA inconsistencies illustrated in Table~\ref{t:incBAMA}: 

\begin{enumerate}[(a)]
    %%%%%%%%%%%%%%%%%%% BAMA prefixes
\item \label{inc:plus}  $L_p$ omits a plus (+) symbol that indicates boundaries in compound prefixes. 
\item \label{inc:so}    $L_p$ omits the (so) alternative gloss that corresponds to \RL{f-} in several compound prefixes.
    %%%%%%%%%%%%%%%%%%% BAMA suffixes
\item \label{inc:num_gender_subject} $L_x$ omits gender and number qualifiers 
  that appear within within square brackets
    from several glosses of subject suffixes.
                % not very precise seems sometimes also uses () for subjects like:
                %اكن    اكُنَّ  PVSuff-Ah       they (both) <verb> you (women)  +A/PVSUFF_SUBJ:3MD+kun~a/PVSUFF_DO:2FP  
\item \label{inc:num_gender_object} $L_x$ omits gender and number qualifiers 
    from several glosses of subject suffixes that appear within parenthesis.
\item \label{inc:both} $L_x$
    expresses the dual quantifier as `them (both)' in the majority of the entries, and 
    as `(both of) them' in several entries.
\item \label{inc:dot} $L_x$ omits the dot (`.') symbol from the gloss abbreviation of plural.
\item \label{inc:pos} $L_x$ contains POS tags that are not consistent with the semantics of the vocalized tags for compound affixes.
\end{enumerate}
\

%%%%%%%%%%%% SAMA inconsistencies table %%%%%%%%%%%%%%%%
\begin{table}[!tbp]
\resizebox{0.95\textwidth}{!}{
%\begin{minipage}{0.8\textwidth}
\relsize{-1.5} {
\begin{tabular}{p{2.5cm}cccc}
\hline \hline
 & {\bf Affix } & \textbf{Vocalized} & {\bf Inconsistent tag} & \\ \hline
\multirow{2}{2.5cm}{
  {\bf \ref{inc:relaxed})} 
  missing standalone alef with no hamza {\bf prefix}  forms
}
% & \RL{A} & \utfRL{أَ} & I \\ 
 & \RL{A} & \utfRL{آ} & I  \\ 
% & \RL{A} & \utfRL{أُ} & I  \\
 & \RL{sA} & \utfRL{سَأُ} & I \\[6pt]
\multirow{2}{2.5cm}{
  {\bf \ref{inc:by}) } 
   additional by in {\bf gloss} 
}
 & \RL{w} & \RL{wa} & with      \\ 
 & \RL{wAl-} & \RL{waAl-} & with\colorbox{shadecolor}{\color{white} /by} + the \\[6pt]
\multirow{1}{2.5cm}{ {\bf \ref{inc:fa}) } 
  additional space in vocalized form 
}
& \RL{f-} & \colorbox{shadecolor}{~}\RL{fa-}\colorbox{shadecolor}{~} \\[6pt]
\multirow{1}{2.5cm}{ {\bf \ref{inc:wafa})} 
   wrong {\bf prefix}
}
& \RL{wf-}& \RL{wafa-} & and + so/and \\[6pt]
\multirow{3}{2.5cm}{ {\bf \ref{inc:def}) } 
  missing definite indicator in {\bf suffix gloss} 
}
 & \utfRL{آت} & \utfRL{آتِ} & [fem.pl.] + [def.acc.]                               \\
 & \utfRL{اتك} & \utfRL{اتِكِ} & [fem.pl.] + [def.acc.] + your [fem.sg.]            \\
 & \utfRL{آتك} & \utfRL{  آتِكِ} & [fem.pl.] + [\fbox{def.}acc.] + your [fem.sg.]  \\[6pt]
% & \multicolumn{4}{c}{33 other inconsistencies} \\ 
\multirow{2}{2.5cm}{ {\bf \ref{inc:num_gender}) }
  omitted gender/num  qualifier in {\bf gloss}
}
 & \utfRL{اك}&  \utfRL{|BBاكِ}& we [verb] + you \fbox{[fem.sg.]}          \\
 & \utfRL{نهم}&  \utfRL{|Bْنَهُم}& they \fbox{[fem.pl.]} [verb] + them      \\[6pt]
% & \multicolumn{4}{c}{72 total inconsistencies} \\ 
\multirow{3}{2.5cm }{  {\bf \ref{inc:both:sama}) } 
   different ways to express them (both) in {\bf gloss} of {\bf suffix} 
}
 & \noVocRL{thmA}&  \utfRL{|Bَتْهُما} & it/they/she [verb] them (both)   \\
 & \noVocRL{AhmA} &  \noTrutfRL{|BBاهُما} \noArutfRL{ّاهُما}& we [verb] 
                \colorbox{shadecolor}{\color{white} (both of)} them \fbox{(both)}   \\
 & \noVocRL{nAhmA} &  \noVocRL{nAhomA} & we [verb] 
                \colorbox{shadecolor}{\color{white} (both of)} them \fbox{(both)}   \\[6pt]
% & \noVocRL{nhmA} &  \noVocRL{nahomA} & they [verb]
%                \colorbox{shadecolor}{\color{white} (both of)} them \fbox{(both)}  \\
% & \noVocRL{hmA} &  \utfRL{|BBهُما} & they [verb[ 
%\colorbox{shadecolor}{\color{white} (both of)} them \fbox{(both)}  \\
\multirow{3}{2.5cm}{ {\bf \ref{inc:num_gender_differences}) } 
  leftover BAMA style tags  in {\bf gloss}
}
 & \utfRL{كم} & \utfRL{كُم} & he/it [verb] + you [masc.pl.]        \\
 & \utfRL{تكم} & \utfRL{تُكُم} & I [verb] + you 
                \colorbox{shadecolor}{\color{white}(pl.)} \fbox{[masc.pl.]}  \\
 & \utfRL{كن} & \utfRL{|Bُّكُنَّ}  & I [verb] + you 
                \colorbox{shadecolor}{\color{white}(women) } \fbox{[fem.pl.]}   \\[6pt]
% & \multicolumn{4}{c}{13 other inconsistencies } \\                
\multirow{3}{2.5cm}{ {\bf \ref{inc:jus}) } 
  indicative {\bf gloss} with jussive {\bf POS}
}
% & ~ & ~ & [jus.] ~ ~ ~ (null)/IVSUFF\_MOOD:J  \\ 
 & \utfRL{نا} & \utfRL{نا} & 
       IVSUFF\_MOOD:J,
       \colorbox{shadecolor}{\color{white}[ind.]} \fbox{[jus.]} + us  \\
 & \utfRL{ني} & \utfRL{نِي} & 
       IVSUFF\_MOOD:J,
       \colorbox{shadecolor}{\color{white}[ind.]} \fbox{[jus.]} + me \\[6pt]
\multirow{3}{2.5cm}{ {\bf \ref{inc:dot:SAMA}) } 
   omitted `.'  in {\bf gloss} 
}
 & & \utfRL{|Bُ} & [def.nom.] & \\ 
 & \utfRL{ه} & \utfRL{|Bُهُ} & [def.nom \fbox{.}]  \\
 & \utfRL{تكن} & \utfRL{|Bَتُكُنَّ} & 
[fem.sg.] + [def.nom \fbox{.}] + your [fem.pl \fbox{.}]  \\[6pt]
% & \multicolumn{4}{c}{36 other inconsistencies } \\
\vocalize
\multirow{2}{2.5cm} { {\bf \ref{inc:pos:SAMA})} 
  shadda inconsistent in {\bf POS} 
}
 & \utfRL{ي} & \utfRL{يَ} & ya/POSS\_PRON\_1S \\
 & \utfRL{ي} & \utfRL{|Bِyya} & 
 % iy/NSUFF\_MASC\_PL\_NOM  +\colorbox{shadecolor}{\color{white}$\sim$a/}\fbox{ya/}POSS\_PRON\_1S  \\
 $\ldots$ +\colorbox{shadecolor}{\color{white}$\sim$a/}\fbox{ya/}POSS\_PRON\_1S  \\
\hline \hline
\end{tabular}
}
}
%\end{minipage}
\caption{Sample SAMA inconsistencies}
\vspace{-2em}
\label{t:incSAMA}
\end{table}


The following describes the SAMA inconsistencies illustrated in Table~\ref{t:incSAMA}:
\todo{R3.11 item (\ref{inc:relaxed)} below} 

\arabtrue

\begin{enumerate}[(a)]
    %%%%%%%%%%%%%%%%%%% SAMA prefixes
\item \label{inc:relaxed}
  $L_p$ misses entries for Alef prefixes with omitted hamza or madda due to relaxed
    writing standards which are common in many documents. 
    This is resolved in SAMA for standalone Alef prefixes via preprocessing tokens and flipping all forms of Alef into one form. 
    We report it here since compound prefix entries with Alef are all listed, 
    e.g. \RL{sA}, 
    \utfRL{سأ}, \utfRL{سآ}, and 
    the standalone prefixes are available but commented out.
Sarf addresses this concern by the separation of prefixes into partial and atomic variants, and by 
introducing agglutinative and fusional rules. The rules construct compound prefixes from partial and atomic prefixes. 
This, therefore, enables the introduction of all prefixes that include Alef, with any of its alternative spellings, 
by only adding the Alef alternatives into the atomic prefix set. 
\item \label{inc:by} $L_p$ contains an additional erroneous alternative gloss for \RL{wa-} in only one compound prefix~\RL{wa-}; while correctly not included elsewhere.
\item \label{inc:fa} $L_p$  contains stray spaces in the vocalized tags of one of the \RL{fa-} alternatives.
\item \label{inc:wafa} $L_p$  contains an entry that supports the concatenation of \RL{wa-} and \RL{fa-} conjunctions. 
    This entry is erroneous and is illegal in Standard Arabic.
    %%%%%%%%%%%%%%%%%%% SAMA suffixes
\item \label{inc:def} $L_x$  omits the definite indicator in the gloss of several suffixes.
\item \label{inc:num_gender} 
    $L_x$  omits gender and number qualifiers that appear within square brackets 
    from the gloss tags of several suffixes.
\item \label{inc:both:sama} 
    $L_x$  
    expresses the dual quantifier as `them (both)' in the majority of the entries, and 
    as `(both of) them' in several other entries.
\item \label{inc:num_gender_differences} 
    $L_x$  
    contains number indicators in the gloss tags still expressed in BAMA style.% square brackets versus parentheses
\item \label{inc:jus} 
    $L_x$  contains entries with an 
    {\em indicative} gloss mood and a {\em jussive} POS mood.
\item \label{inc:dot:SAMA} 
    $L_x$  omits dot (`.') for the abbreviation of plural in several gloss tags.
\item \label{inc:pos:SAMA} 
    $L_x$  
    represents a repeated consonant by a shadda in the POS tag where it should not. 
    SAMA POS tags should spell out the repeated consonants if each belongs to one.\todo{R3.19}  
    In SAMA, the repeated consonant (of the shadda) is spelled out
    whenever the consonants has its separated partial POS tag. 
\end{enumerate}
%\vspace{-1em}

In addition, 53 BAMA and 27 SAMA minor differences exist between $L_p$ and $L_x$ of BAMA and SAMA and
their counterparts computed using our agglutinative affixes. 
For example, the BAMA gloss tags for prefixes that contain
`bi/PREP' report `with/by' in some entries and its reverse `by/with' in others. 
In addition, we detected several entries in $L_p$ of SAMA with no category compatibility 
rules in $R_{ps}$, $R_{sx}$, and $R_{px}$.


\section{Diacritics}
\label{sec:diac}
Diacritics are short vowels that
are often omitted in Arabic text and inferred by readers from context. 
Their omission adds to the ambiguity problem of Arabic morphological analysis. 
\fullvocalize
The diacritics \RL{|Ba} ({\em fatha}),  \RL{|Bu} ({\em damma}), and 
\transfalse\RL{|B}\transtrue represent
and appears above the letter. 
`a' vowel,  `o' vowel, and consonant, respectively,  and appear above the letter. 
The diacritic \RL{|Bi} represents a `y' vowel and appears below the letter. 
The diacritics 
\RL{|BaN}, \RL{|BuN}, and \RL{|BiN} represent the `a', `o', and `y' vowels followed 
by a phonetically stressed \transtrue \RL{n} consonant. 

The shadda \RL{|BB} mark is not a 
diacritic but is treated typographically as one, and is also often omitted in Arabic text. 
It denotes  a repeated letter, first as consonant, and second 
as vocalized.
Arabic forbids two letters with a consonant diacritic to follow each other. 
%A shadda can be followed by all diacritics except \transfalse\RL{|B}\transtrue. 


Analyzers such as BAMA and SAMA ignore input partial diacritics because 
they consider them to be (1) rare in common corpora, and (2) 
unreliable because of dialect diversity and human errors~\citep*{Elkateb:06,Attia:06b}. 
\todo{R3.13}
The diacritics can be miss-placed for several reasons including: 
(1) writers use a local Arabic dialect pronunciation of the word, 
(2) font effects such as ligatures compensate for miss-placing diacritics, 
and
(3) writers sometimes use diacritics to decorate their text. 
%
However, the work in \citep{Chaaben:10,Attia:00,Beesley:01} considers 
partial diacritics to decrease morphological ambiguity.
The work in~\cite{Maamouri:06} inspected the ATB v3.2 corpus~\citep*{Maamouri:04}
for diacritics and found that 1.364\% of the words were partially diacriticized and 
those diacritics eventually and {\em effectively} reduced morphological ambiguity.
Most of the diacritics we found were tanween marks which indicate indefiniteness, 
followed in frequency by shadda, and then by dhamma that often indicates 
passive tense when associated with verbs. 
Hence, we decided not to force the use of partial diacritics to disambiguate the 
solutions and to provide it as an option.

%Figure~\ref{t:consistent} presents the Diacritic-aware consistency check Algorithm.
Key to partial diacritic analysis is a diacritic-aware consistency check that replaces
standard string matching checks. 
The \cci{ Diacritic-aware consistency check} algorithm takes as input 
two text chunks $c_1$ and $c_2$. 
\todo{R2.14}It checks if the sequence of letters in $c1$ is the same as that in $c2$. 
It also checks the consistency of the sequence of diacritics in $c1$ and in $c2$. 
%It first checks if the sequence of letters, without diacritics, in $c1$ is consistent with that in $c2$. 
%Then, it checks if the sequence of diacritics of letters in  $c1$ is consistent with that of $c2$.
%
%It checks that the sequence of non-diacritic letters, ignoring the 
%diacritics between them, are equal. 
%It also checks that all sequences of diacritics occurring between 
%non-diacritic letters are consistent. 

Two sequences of diacritics are consistent iff:
%\vspace{-3em}
\begin{enumerate}
  \item Both are equal, or
  \item One of the sequences is empty, or
%  \item don't contain different diacritics
  \item If one has a shadda, then the other has no sukoun, or
  \item If one has a shadda and the other has no shadda, then the rest of the diacritics are compared recursively. 
\end{enumerate}

If both checks were positive, then $c1$ and $c2$ are said to be equal with diacritic-aware consistency. \todo{R2.14}

\begin{table}[tb]
\centering
\begin{minipage}{0.9\textwidth}
\caption{\label{t:correspondence}Arabic string comparison with consideration of partial diacritics}
\begin{tabular}{cccc} %\hline
\toprule
& word & string comparison & diacritic-aware consistency check 
%\vspace{0.2em} 
\\\colrule% \cline{2-4}
(a)\label{ex:valid_correspondence} &
	\begin{minipage}{0.1\textwidth}
	\begin{tabular}{c} 
	\utfRL{أَكّل} \\
	{\bf $\updownarrow$} \\
	\utfRL{أكَل} \\
	\end{tabular}
	\end{minipage}
&
	\begin{minipage}{0.3\textwidth}
	\begin{tabular}{ccccc} 
	\noVocRL{'} & \RL{|Ba} & \noVocRL{k} & \noVocRL{|BB} & \noVocRL{l} \\
	$\updownarrow$ & \textcolor{red}{$\updownarrow$} & \textcolor{red}{$\updownarrow$} & 
			\textcolor{red}{$\updownarrow$} & \textcolor{red}{?} \\
	\noVocRL{'} & \noVocRL{k} & \RL{|Ba} & \noVocRL{l} \\
	\end{tabular}
	\end{minipage}
&
	\begin{minipage}{0.4\textwidth}
	\begin{tabular}{ccccc} 
	\noVocRL{'} & \RL{|Ba} & \noVocRL{k} & \noVocRL{|BB} & \noVocRL{l} \\
%	\RL{'} & \fullvocalize \RL{|Ba} \novocalize & \RL{k} & \RL{|BB} & \RL{l} \\
	$\updownarrow$ & \multicolumn{4}{l}{~~~~~$\nearrow$ ~~~~~ $\nearrow$ ~~~~ $\nearrow$}  \\
%	\RL{'} & \RL{k} & \fullvocalize \RL{|Ba} \novocalize & \RL{l} \\
     \noVocRL{'} &  \noVocRL{k} & \RL{|Ba} & \noVocRL{l} \\
	\end{tabular}
	\end{minipage}

\\ \hline
(b)\label{ex:invalid_correspondence} &
	\begin{minipage}{0.1\textwidth}
	\begin{tabular}{c} 
	\utfRL{أُكِل} \\
	{\bf \color{red} $\updownarrow$}  \\
	\utfRL{أكَل} \\
	\end{tabular}
	\end{minipage}
&
	\begin{minipage}{0.3\textwidth}
	\begin{tabular}{ccccc} 
	\noVocRL{'} & \RL{|Bu} & \noVocRL{k} & \RL{|Bi} & \noVocRL{l} \\
	$\updownarrow$ & \textcolor{red}{$\updownarrow$} & \textcolor{red}{$\updownarrow$} & 
			\textcolor{red}{$\updownarrow$} & \textcolor{red}{?} \\
	\noVocRL{'} & \noVocRL{k} & \RL{|Ba} & \noVocRL{l} \\
	\end{tabular}
	\end{minipage}
&
	\begin{minipage}{0.4\textwidth}
	\begin{tabular}{ccccc} 
	\noVocRL{'} & \RL{|Bu} & \noVocRL{k} & \RL{|Bi} & \noVocRL{l} \\
	$\updownarrow$ & \multicolumn{4}{l}{~~~~~$\nearrow$ ~~~~~ \textcolor{red}{$\nearrow$} ~~~~ $\nearrow$}  \\
	\noVocRL{'} & \noVocRL{k} & \RL{|Ba} & \noVocRL{l} \\
	\end{tabular}
	\end{minipage}
\\ %\hline
\botrule
\end{tabular}
\end{minipage}
\end{table}



Table~\ref{t:correspondence} illustrates the 
\cci{diacritic-aware consistency check} as compared to the standard
string comparison with an example. 
Part (a) shows two diacritic-consistent words 
\vocalize
\utfRL{أَكّل} and \RL{'kal} as a fatha \RL{|Ba} is compatible with an empty diacritic 
and a shadda \RL{|BB} is compatible with a fatha \RL{|Ba}. 
%In fact, a shadda is compatible with all diacritics except 
%for the sukun.
%and consequently its disambiguation value is mainly measured 
%through its presence/absence and not through its compatibility. 
Part (b) illustrates two inconsistent diacritizations 
since \RL{|Bi} is incompatible with \RL{|Ba} next to the letter \noVocRL{k}.
%Inconsistencies can be detected when different short vowels are encountered as in 
%part~\hyperref[ex:valid_correspondence]{(b)} of Table~\ref{t:correspondence}
%in which \fullvocalize

\todo{R2.15, R3.14}
Sarf takes an Arabic word as input with possible partial diacritics. 
Then, it traverses ${\cal P}$, ${\cal S}$, and ${\cal X}$ structures 
to compute the morphological solutions. 
For each matching morpheme in the structures, 
Sarf uses the \cci{diacritic-aware consistency check} algorithm 
to check for the consistency of the fully diacriticized morpheme
with the corresponding partially diacriticized chunk of the input word.
The traversal path corresponding to the morpheme dies if the consistency 
check is negative (inconsistent).


\section{Results}
\label{sec:results}
\major{
In this section we evaluate \framework with four case studies. 
We perform a survey like evaluation where developers manually 
built task specific information extraction tools for the case studies 
and other developers built equivalent \framework tools. 
The aim of the comparison is to showcase that \framework enables 
fast development of linguistic applications with similar accuracy 
and a reasonable affordable overhead in computational time. 
We report development time,
size of developed code versus size of grammar,
running time, and 
precision-recall as metrics
of cost, complexity, overhead, and accuracy,
respectively. 

We survey three case studies from the literature: 
(1) narrator chain, (2) temporal entity, and (3) genealogy entity extraction 
tasks, and we use the reported development time for the task specific 
techniques proposed in
ANGE~\cite{ZaMaFlairs2012HadithBio}, 
ATEEMA~\cite{ZaMa2012IJCLATime},  and
GENTREE~\cite{ZaMaHaCicling2012Entity}, respectively. 
%
We also compare a \framework number normalization task to 
a task specific implementation. 

We evaluated ANGE with
Musnad Ahmad, a hadith book, where we constructed an annotated 
golden reference containing 1,865 words.
We evaluated ATEEMA 
with articles from issues of  the Lebanese 
Al-Akhbar newspaper where we constructed 
an annotated golden reference containing 1,677 words. 
For the genealogical tree extraction we used 
an extract from the Genesis biblical text with 1,227 words.
Finally, we used an annotated article from the Lebanese Assafir
newspaper with 1,399 words to evaluate the NUMNORM case study
\footnote{available at \url{http://www.assafir.com} and 
\url{http://www.al-akhbar.com}.}. 
%
In the online appendix%
~\footnote{available at ~\url{-omitted-for-anonymity-}}%http://webfea.fea.aub.edu.lb/fadi/pdfs/merfappendix.pdf
, we report on eight additional \framework case studies.
Manual annotators inspected the outcome and provided corrections where tools 
made mistakes.
The corrections form the manual gold annotation that we compared against.
}

% Table generated by Excel2LaTeX from sheet 'Sheet1'
\begin{table}[tb!]
  \centering
  \caption{\framework compared to task specific applications.}
  \label{tab:results}%
  \resizebox{\columnwidth}{!}{
    \begin{tabular}{l|l|l|l|ll|l}
     \toprule
      \multirow{2}{*}{Task} & Size & Development& Run & \multicolumn{2}{c|}{Accuracy} & \multirow{2}{*}{Ease of Composition}\\
      & (words) & time & time(s) & Recall & \multicolumn{1}{c|}{Precision} & \\
    \midrule
    %ANGE~\cite{ZaMaFlairs2012HadithBio}
      ANGE & 1,865  
    & 2 months & 1.79 & 0.99 & 0.99 & 3000+ lines of code\\
      \framework &  & 3 hours & 7.24 & 0.99 & 0.93  & 8 MBFs and 4 MREs\\
    \midrule
    %ATEEMA~\cite{ZaMa2012IJCLATime}
      ATEEMA & 1,677 & 1.5 months & 2.53 & 0.88  & 0.89  & 1000+ lines of code  \\
      \framework & & 3 hours & 3.14 & 0.91  & 0.81  & 3 MBFs and 2 MREs\\
    \midrule
    %Genealogy tree~\cite{ZaMaHaCicling2012Entity} 
      Genealogy tree & 1,227 
    & 3 weeks & 0.74 & 0.96 & 0.98 & 3000+ lines of code\\
      \framework &  & 4 hours & 2.28 & 0.84 & 0.93  & 3 MBFs and 3 MREs\\
    \midrule
      NUMNORM &  1,399 & 1 week & 0.32 & 0.91 & 0.93  & 500 lines of code\\
      \framework &  & 1 hour & 1.53 & 0.91 & 0.90 & 3 MBFs/1 MRE/57 lines\\
    \bottomrule
    \end{tabular}%
    }
\vspace{-1em}
\end{table}%

Table~\ref{tab:results} reports the development time,
extraction runtime, recall and precision 
of the output MRE tags, 
the size of the task in lines of code or in number of \framework rules, 
for both the standalone task specific and the \framework implementations.
\major{
The development time measures the time required for 
developing the case study.
For instance, ANGE~\cite{ZaMaFlairs2012HadithBio} required two 
months of development by a research assistant with six and 14 hours of 
course work and teaching duties, respectively.
Recall refers to the fraction of the entities correctly detected against the
total number of entities. 
Precision refers to the fraction of correctly
detected entities against the total number of extracted entities. 
}

\major{
Table~\ref{tab:results} provides runtime results of \framework 
compared to the task specific implementations while running 
MBF and MRE simulations jointly.
This is a rough estimate of the complexity of the \framework simulator. 
%
The complexity of the MBF simulation is the total number of morphological 
solutions for all the words multiplied by the number of user-defined MBFs.
We do not provide a limit on the number of user defined formulae.
In practice, we did not encounter more than ten formulae per case study.
As for the complexity of MRE simulation, converting the rules into 
non-deterministic finite state machines (NDFSM) is done once. 
Simulating an NDFSM over the MBF tags is potentially exponential. 
In practice all our case studies terminated within a predetermined 
time bound of less than 30 minutes. 
\framework required reasonably more runtime than the task specific 
implementations and reported acceptable and 
slightly less precision metrics with around
the same recall.
}

%Table~\ref{tab:mbfer} reports the accuracy of the output MBF tags and the user-defined relation 
%construction in \framework for each task.

%Recall refers to the fraction of the entities correctly detected against 
%the total number of entities available. 
%Precision refers to the fraction of correctly detected entities against the 
%total number of extracted entities. 
%Intuitively, precision denotes whether the system generated false positives.

%\subsection{Comparison}
Table~\ref{tab:results} shows that \framework has a clear advantage over 
task specific techniques in the effort required to develop the application at 
a reasonable cost in terms of accuracy and run time. 
Developers needed three hours, three hours, four hours, and one hour 
to develop the narrator chain, temporal entity, genealogy, and number 
normalization case studies using \framework, respectively. 
However, the developers of ANGE, ATEEMA, GENTREE, and 
NUMNORM needed two months, one and a half months, 
three weeks, and one week, respectively. 
\framework needed eight MBFs and four MREs for narrator chain, 
three MBFs and 2 MREs for temporal entity, three MBFs and three MREs for 
genealogy, and three MBFs, one MRE, and 57 lines of code actions for the number normalization tasks. 
However, ANGE, ATEEMA, GENTREE, and NUMNORM required 
3,000+, 1,000+, 3,000+, and 500 lines of code, respectively.

%a runtime of 7.29, 1.53, 3.14, and 2.28 seconds for 
%the narrator chain, temporal, genealogy, and number normalization tasks, respectively. 
%However, ANGE, ATEEMA, GENTREE, and NUMNORM required 1.79, 2.53, 0.74, and 0.32 seconds, respectively. 
%\framework required a runtime of 7.29, 1.53, 3.14, and 2.28 seconds for 
%the narrator chain, temporal, genealogy, and number normalization tasks, respectively. 
%However, ANGE, ATEEMA, GENTREE, and NUMNORM required 1.79, 2.53, 0.74, and 0.32 seconds, respectively. 
%In narrator chain, \framework scored 0.99\% recall and 0.93\% precision while ANGE scored 0.99\% recall and 0.99\% precision. 
%In temporal entity, \framework scored 0.91\% recall and 0.81\% precision while ATEEMA scored 0.88\% recall and 0.89\% precision. 
%In genealogy, \framework scored 0.84\% recall and 0.93\% precision while GENTREE scored 0.96\% recall and 0.98\% precision. 
%In number normalization, \framework scored 0.91\% recall and 0.90\% precision while NUMNORM scored 0.91\% recall and 0.93\% precision.

\setcode{utf8}
\setarab
\begin{table}[tb!]
  \centering
  \caption{Narrator chain example.}
  \begin{Verbatim}[xleftmargin=1.5cm,fontsize=\relsize{-1},commandchars=\\\{\},codes={\catcode`$=3 \catcode`_=8}]
name:   PN ((MEAN)? PN)*;
nar:    name ((NONE)^3 FAM (NONE)^3 name)*;
pbuh:   BLESS GOD UPONHIM GREET;
nchain: ($s_1=$TOLD $s_2=$nar)+ ((PN|FAM|NONE)^8 pbuh)?
\end{Verbatim}
\resizebox{0.8\columnwidth}{!}{
    \begin{tabular}{|c|c|c|c|c|c|c|c|c|c|}
    \toprule 
    \notrrl{القعقاع} & \notrrl{بن} & \notrrl{عمارة} & \notrrl{عن} & \notrrl{جرير} & \notrrl{حدثنا} & \notrrl{سعيد} & \notrrl{بن} & \notrrl{قتيبة} & \notrrl{حدثنا} \\
    \midrule 
    \noarrl{القعقاع} & \noarrl{بن} & \noarrl{عمارة} & \noarrl{عن} & \noarrl{جرير} & \noarrl{حدثنا} & \noarrl{سعيد} & \noarrl{بن} & \noarrl{قتيبة} & \noarrl{حدثنا} \\
    \midrule
    PN    & FAM   & PN    & TOLD  & PN    & TOLD  & PN    & FAM   & PN    & TOLD \\
    \midrule
    name & & name & & name & & name &  & name & \\
    \midrule
    \multicolumn{3}{|c|}{nar} &       & nar   &       & \multicolumn{3}{c|}{nar} &  \\
    \midrule
    \multicolumn{10}{|c|}{nchain}
    \\
    \bottomrule
    \end{tabular}%
}
  \label{tab:nchain}%
  \vspace{-1.5em}
\end{table}%
\setcode{standard}

\subsection{Narrator chain case study}
A narrator chain is a sequence of narrators referencing each other. 
%A sample narrator chain is shown in Table~\ref{tab:nchain}. 
The chain includes proper nouns, paternal entities, and referencing entities. 
ANGE uses Arabic morphological analysis, finite state machines, and graph transformations 
to extract entities and relations including narrator chains~\cite{ZaMaFlairs2012HadithBio}.

\transfalse
Table~\ref{tab:nchain} presents the MREs for the narrator chain case study. 
MBF \cci{PN} checks the abstract category {\tt Name of Person}. 
MBF \cci{FAM} denotes ``family connector'' and checks the stem gloss ``son''. 
MBF \cci{TOLD} denotes referencing between narrators and checks the disjunction of 
the stems \RL{.hd_t}(spoke to), \RL{`n}(about), \RL{sm`}(heard), \RL{'_hbr}(told), and \RL{'nb-'}(inform). 
MBF \cci{MEAN} checks the stem \RL{`ny}(mean). 
MBFs \cci{BLESS}, \cci{GOD}, \cci{UPONHIM}, and \cci{GREET} check the 
stems \RL{.sll_A}, \RL{Al-ll_ah}, \RL{`ly}, and \RL{sllm}, respectively. 
\transtrue

MRE {\em name} is one or more \cci{PN} tags optionally followed 
with a \cci{MEAN} tag. 
MRE \cci{nar} denotes narrator which is a complex Arabic name
composed as a sequence of Arabic names (\cci{name}) 
connected with family indicators (\cci{FAM}). 
The \cci{NONE} tags in \cci{nar} allow for unexpected words 
that can occur between names. 
%a complex Arabic name that is constructed
%a \cci{name} tag followed by zero or more 
%sequences of \cci{FAM} tag followed by \cci{name} tag 
%with up to three \cci{NONE} tags. 
MRE \cci{pbuh} denotes a praise phrase often associated with 
the end of a hadith (peace be upon him), 
and is the satisfied by the sequence of
\cci{BLESS}, \cci{GOD}, \cci{UPONHIM}, and \cci{GREET} tags. 
MRE \cci{nchain} denotes narrator chain, 
and is a sequence of narrators (\cci{nar})
separated with \cci{TOLD} tags, and optionally followed
by a \cci{pbuh} tag. 
%nal sequence of up to eight \cci{PN}, \cci{FAM}, or 
%\cci{NONE} tags followed by a \cci{pbuh} tag. 

The first row in Table~\ref{tab:nchain} is an example narrator chain,
the second is the transliteration, the third 
shows the MBF tags. Rows 4, 5, and 6 show the 
matches for \cci{name}, \cci{nar}, and \cci{nchain},
respectively.
%
\framework assigns the symbols $s_1$ and $s_2$ for the 
MRE sub-expressions \cci{TOLD} and \cci{nar}, respectively. 
We define the relation $\langle s_2,s_2',s_1\rangle$ 
to relate sequences of narrators with edges labeled by the tags of \cci{TOLD} where 
$s_2'$ denotes the next match of \cci{nar} in the one or more MRE subexpression.
%The narrators in the example shown in Table~\ref{tab:nchain} are \RL{qtybT bn s`yd}, \RL{jryr}, and \RL{`mArT bn Alq`qA`}. 
%\transfalse
%The edges relating the entities are labeled by the word set \{\RL{.hdd_tnA}, \RL{.hdd_tnA}, \RL{`n}\} which contains all the %matches of the MBF \cci{TOLD}.
%\transtrue
%
Table~\ref{tab:mbfer} shows that \framework detected almost all the MBF matches 
with 99\% recall and 85\% precision and 
extracted user-defined relations with 98\% recall and 99\% precision.

%For brevity, we omit the details of \framework temporal entity extraction, 
%genealogy tree, and number normalization case studies, describe them shortly
%below and provide a full description in the online Appendix.

\begin{table}[tb!]
  \centering
  \caption{\framework MBF and user-defined relation accuracy }
  \resizebox{0.65\columnwidth}{!}{
    \begin{tabular}{l|c|c|c|c}
     \toprule
     \multirow{2}{*}{Task} & \multicolumn{2}{c|}{MBF accuracy} & \multicolumn{2}{c}{relation accuracy}\\
     & Recall & Precision & Recall & Precision \\
    \midrule
    Narrator chain & 0.99 & 0.85 & 0.99 & 0.98 \\
    Number normalization & 0.99 & 0.99 & 0.97 & 0.95 \\
    Temporal entity & 0.99 & 0.52 & 0.98 & 0.89 \\
    Genealogy tree & 0.99 & 0.75 & 0.81 & 0.96 \\
    \bottomrule
    \end{tabular}%
    }
  \label{tab:mbfer}%
  \vspace{-1.5em}
\end{table}%

\vspace{-1.5em}
\subsection{Temporal entity extraction}
\vspace{-1em}

Temporal entities are text chunks that express temporal information. 
Some represent absolute time such as \RL{Al_hAms mn 'Ab 2010}. 
Others represent relative time such as \RL{b`d _hmsT 'ayAm}, and quantities 
such as \RL{14 ywmA}. 
{\em ATEEMA} presents a temporal entity detection technique for the Arabic language using 
morphological analysis and finite state transducers~\cite{ZaMa2012IJCLATime}. 
%
Table~\ref{tab:mbfer} shows that \framework detected almost all the MBF matches with 99\% recall, 
however it shows low precision (52\%). 
As for the semantic relation construction, \framework presents a 98\% recall and 89\% precision.

\vspace{-2em}
\subsection{Genealogy tree}
\vspace{-1em}

Biblical genealogical lists trace key biblical figures such as Israelite kings and
prophets with family relations. 
The family relations include wife and parenthood. 
A sample genealogical chunk of text is \RL{w wld hArAn lw.tA} 
meaning ``and Haran became the father of Lot''.
%
GENTREE~\cite{ZaMaHaCicling2012Entity} 
automatically extracts the genealogical family trees using morphology, 
finite state machines, and graph transformations. 
Table~\ref{tab:mbfer} shows that \framework detected 
MBF matches with 99\% recall, and 75\% precision, and
extracted relations with 81\% recall and 96\% precision.

\newcommand*{\fvtextcolor}[2]{\textcolor{#1}{#2}}
\begin{figure}[tb!]
\centering
  \begin{tabular}{p{5.5cm}p{5.8cm}}
\begin{Verbatim}[fontsize=\relsize{-1},frame=single,label=TMB algorithm,commandchars=\\\[\]] 
cout << \fvtextcolor[red][$s1.text];
if(isHundred) {
  if(current != 0) {
    previous += current;
  }
  current = currentH * \fvtextcolor[red][$s1.number];
  currentH = 0;
  isHundred = false;
  isKey = true;
} else if(current == 0) {
  current = \fvtextcolor[red][$s1.number];
  isKey = true;
} else if(!isKey) {
  isKey = true;
  current = current * \fvtextcolor[red][$s1.number];
} else {
  previous += current;
  current = \fvtextcolor[red][$s1.number];}
\end{Verbatim}
&
\begin{Verbatim}[fontsize=\relsize{-1},frame=single,label=DT algorithm,commandchars=\\\[\]] 
if(isHundred) {currentH += \fvtextcolor[red][$s0.number];
} else if(current == 0) {
  current = \fvtextcolor[red][$s0.number];
} else if(isKey) {
  previous += current;
  current = \fvtextcolor[red][$s0.number];
} else {current += \fvtextcolor[red][$s0.number]; }
isKey = false;
\end{Verbatim}
\begin{Verbatim}[fontsize=\relsize{-1},frame=single,label=H algorithm,commandchars=\\\[\]] 
isHundred = true;
if(current == 0)  {
  currentH = \fvtextcolor[red][$s2.number];
} else if(!isKey) {
  currentH = current * \fvtextcolor[red][$s2.number];
  current = 0;
} else {currentH = \fvtextcolor[red][$s2.number];}
isKey = false;
\end{Verbatim}
\\ 
\end{tabular}
  \vspace{-2em}
\caption{Actions for TMB, DT, and H MRE expressions.}
  \vspace{-1.5em}
\label{fig:numnormalgo}
\end{figure}

\vspace{-1em}
\subsection{Number normalization}
\label{subsec:numnorm}

We implemented a number normalization extractor using \framework and 
compared it with {\em NUMNORM}, a 
C++ implementation for number normalization. 
First, we defined the MBFs \cci{DT}, \cci{H}, and \cci{TMB}
to denote (1) digits and tens, (2) hundreds, and (3) 
thousands, millions, and billions, respectively.
The \cci{num} MRE 
\cci{(DT|TMB|H)+} is one or more \cci{DT}, \cci{TMB}, or \cci{H} tags. 
\framework assigns the symbols $s_1$, $s_2$, and $s_3$ 
for the sub-expressions \cci{DT}, \cci{TMB}, and \cci{H}, respectively. 
%The actions associated with the sub-expressions \cci{DT}, \cci{TMB}, and \cci{H} are presented in the appendix.
\major{
Figure~\ref{fig:numnormalgo} shows the actions associated with the \cci{DT}, \cci{TMB}, and \cci{H} subexpressions that cumulatively compute the numeric value of the numeric expression match.
The actions use \framework API to access features of the matches such 
as text (\cci{\$s1.text}) and numeric 
value (\cci{\$s1.number}) of literal numbers such as numbers from one to ten.
}
%
Table~\ref{tab:mbfer} shows high accuracy in MBF tagging and relation extraction 
with 99\% and 97\% recall and 99\% and 95\% precision, respectively. 

\vspace{-1.5em}
\subsection{Discussion}
\label{subsec:discuss}
\label{sec:discuss}
The results show that \framework 
provides a friendly environment to develop entity and relational
entity extraction tasks with acceptable 
accuracy and runtime overheads compared to task specific applications. 
%
\framework requires the user to understand and interact with 
basic linguistic concepts such as readable values of morphological 
features, sequences, repetitions, and bounded repetitions. 
The user interacts with the MBF editor to specify basic concepts
and visualize their matches over highlighted text. 
%
Then, the user interacts with the MRE editor to specify 
sequences of the concepts and visualize the matches
in a graph, in conjunction with the highlighted text.

The two levels of interaction allow the user to separate between concepts 
that relate to word features, and more sophisticated entities 
that relate to sequences and context. 
%
The MBF, MRE, and user defined relations 
can be used to generate large annotated corpora in a fast manner. 
\framework visualization can be used later to refine the annotation.
%edit the corpora and fix the annotations.
%
The case studies showed that 
\framework requires some linguistic expertise to successfully execute the 
tasks.
In contrast, the case specific implementations require more sophisticated 
linguistic and programming expertise to attain similar results. 


We notice that ANGE, ATEEMA, and Genealogy tree report higher precision than \framework. 
This is mainly due to their capacity 
to learn words and relations that may not have a match in the 
morphological analyzer based on co-occurrence relations. 
For example, the sequence $p_1 t_1 p_2$ where $p_1$ and 
$p_2$ are persons and $t_1$ is a tell relationship helps
indicate that $x$ is a tell relationship in $p_1 x p_2$ 
even if the morphological analyzer did not return the required
feature for $x$ to match a tell relationship. 
\framework does not have that capacity yet unless it is
encoded in the C++ actions.

%\subsection{ Threats to validity}
%As with any GUI tool, it takes time for the user to get acquainted with the 
%user friendly interface. 
%Relating the subexpressions to their suggested names by \framework might 
%not be directly intuitive to the user. 
%We address the above two concerns with providing the users with 
%a short tutorial video on how to use the tool.



\section{Conclusion}
\label{sec:conclusion}
In this work, we present a morphology-based entity and relational entity extraction framework for Arabic text.
\framework provides a friendly interface where the user defines tag types 
and associates them with regular expressions defined over Boolean formulae.
The Boolean formulae are in turn defined over matches of Arabic morphological features and
a novel extended synonymy feature ($Syn^k$).
\framework allows the user to associate code actions with each regular sub-expression 
and to define semantic relations between sub-expressions. 
%\framework uses Sarf, an Arabic morphological analyzer, 
%to compute morphological and thereafter
%regular expression matches, and relational entities. 
We evaluate \framework with several case studies and compare with existing application-specific 
techniques.
The results show that \framework requires shorter development time and effort compared 
to existing techniques and produces reasonably accurate results within a reasonable 
overhead in run time. 
In the future, \framework will support user-defined cross-reference predicates, 
and will infer morphological features from relevant example words to express a concept.

%Currently, \framework supports one built in cross-reference predicate based on the $Syn^2$ feature. 
%In the future, \framework will support user-defined cross-reference predicates. 
%Currently the user selects the morphological features to specify the MBF. 
%We will explore techniques that can infer the features from example words that the user judges as relevant to the basic concept in question.

{\footnotesize
%\linespread{1.5}
%\bibliographystyle{apacite}
\bibliographystyle{elsarticle-harv}
\bibliography{refs/aclsarf}
}

%\label{lastpage}

\end{document}