The rich nature of Arabic morphology makes morphological analysis 
key for Arabic natural language processing applications. 
%
Arabic morphological analyzers return several
morphological solutions for a given Arabic word. 
Each solution consists of several morphological features
such as part of speech and gloss description
tags. 
%
Often times, applications need only few of those features.
%
This paper presents Sarf, an application customizable 
morphological analyzer for Arabic. 
Sarf provides an interface that allows application developers to 
(1) control and prioritize the analysis, 
(2) refine solution features, and
(3) define categories and associate them with existing morphemes. 

Sarf uses agglutinative and fusional morphemes for
affix representation,
and extends and refines the morpheme lexicons of SAMA and BAMA.
This reduces redundant morphemes, and subsequently inconsistent 
morpheme tags in the lexicons.
It also solves the segmentation correspondence problem between 
an input word and the several parts of the associated morphological 
solution. 
It uses diacritics to refine solutions, and 
solves the `run-on words' problem. 
%
The implementation of Sarf efficiently encodes
the morpheme lexicons.
Sarf was used in several NLP 
applications for information extraction and provided more accurate
solutions than existing solvers with faster running time.