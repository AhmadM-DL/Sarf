{\em Natural language processing} (NLP) applications such as 
{\em machine translation} (MT) and {\em information extraction} (IE)
require {\em morphological analysis} to preprocess Arabic text
due to the rich morphological nature of the Arabic language
~\citep*{Benajiba:07,Habash:06}. 
%
%Given a white space and punctuation delimited Arabic word,
Arabic morphological analyzers
return the internal structure of a given Arabic word composed of 
several {\em morphemes} including {\em affixes} ({\em prefixes} 
and {\em suffixes}), {\em clitics} ({\em proclitics} and {\em enclitics}), 
and {\em stems}~\citep*{Sughaiyer:04}. 
The morphological solution consists also of several {\em morphological features}
(tags) associated with the word and its constituent morphemes 
such as {\em part of speech} (POS), transliteration, {\em gloss}, 
and {\em vocalized morpheme form} (VMF) tags (diacriticized form of the morpheme).
%
%The stem is the core of a word. 
The prefix and suffix attach before and after the stem, respectively. 
Clitics are special affixes that attach to the stem to form a word, 
and differ from regular affixes in that 
they play a syntactic role of another word (often omitted) 
%while being bound to the word in question
~\citep*{habash2010introduction}. 
%
%However, some lexicons such as Buckwalter~\citep*{Buckwalter:04} use the terms prefix and suffix to refer to  proclitic and enclitic, respectively.

This work presents Sarf, a \todo{R2.13} {\em fast and application 
customizable morphological analyzer} for Arabic
that is used in several applications for information extraction 
from Arabic text~\citep*{JaZaMatar,ZaMaFlairs2012HadithBio,ZaMa2012IJCLATime,ZaMaHaCicling2012Entity}. 
Sarf provides NLP application developers with an application programming interface (API) 
to control and refine morphological analysis on the fly. 
The developer implements the interfaces in the application.
Sarf calls the interfaces 
on control points such as prefix, stem, suffix, and full solution matches.
%We distinguish between the NLP application developer 
%who uses Sarf to build an NLP application and the end user who provides the text
%and inspects the solutions.
The Sarf API allows the application to 
(1) control and prioritize the analysis, 
(2) refine the solution features, and
(3) define developer categories and associate them with existing morphemes.

%Users of the developed application pass Arabic text as input and the application uses 
%Sarf to perform customizable morphological analysis on the text. 

Sarf is a significant extension of the work in~\citep*{ZaMaColing2012DemosSarf}. 
It represents Arabic affixes as 
agglutinative affix morphemes with fusional affix concatenation rules. 
Simpler agglutinative affix morphemes can be concatenated to form a more complex 
affix~\citep*{Vajda}. 
%Fusional models are similar to agglutinative with the difference that 
%morpehemes which make up the word fuse together several different features meanings~\citep*{Spencer:91}. 
Fusional affix concatenation rules specify affix pairs and use 
regular expressions \todo{R2.13 removed in} as substitution rules to 
compose the resulting orthographic and semantic tags from the 
tags of the original morphemes~\citep*{Spencer:91}.
The Sarf substitution rules are in sync with 
rules and examples on morpheme concatenative properties from
Arabic morphology textbooks~\citep*{Abd00,Abd001}.
This representation resolves consistency, maintenance, and segmentation 
issues of the current approaches in BAMA and SAMA. 
%Sarf also provides the option to use partial diacritics 
%in disambiguating the morphological solutions of a partially diacritized word.

\todo{R1.1}
In this paper, we make several additional contributions including the following. 
\begin{itemize}
    \item Sarf provides an application customizable morphological analyzer where the developer can control and refine the analysis. 
    \item Sarf structures and computations are engineered to reduce computation cost as follows. 
      \begin{itemize}
        \item The organization of the lexical structures into root index tries during runtime.
        \item The traversal structures that allows multiple solutions to be considered at once.
        \item The computation of the solution features in a tree based structure instead of computing all features and filtering later on
        \item The manual optimization of the traversal of the lexicon tries to reduce the typical excessive non-determinism exhibited in the finite state machine based solutions.
      \end{itemize}
    %\item Sarf is a novel Arabic morphological analyzer with agglutinative affixes and fusional affix concatenation rules based on textbook Arabic morphological rules and on the concatenation rules of existing analyzers. 
    \item Sarf solves the ``run-on'' words problem and optionally uses partial diacritics for solution disambiguation. 
    %\item Sarf solves inconsistencies in existing affix lexicons of BAMA and SAMA. In this paper we list and discuss the inconsistencies.
    %\item Sarf solves the correspondence between the morphological solution and the morphological segmentation of the original text problem.
%    \item We evaluate our approach using the ATB data set and report on the effect of our corrections on the annotations.
    \item Sarf is fully implemented and available online as an open source tool.\footnote{\label{fn:online} \url{http://research-fadi.aub.edu.lb/carla/doku.php?id=sarf} }
\end{itemize}

We evaluated Sarf for %segmentation correspondence, lexicon size, lexicon consistency\todo{R3.19}, 
accuracy and runtime efficiency.
%Our results show that Sarf lexicons are smaller than lexicons of existing analyzers and provide coverage for more morphological solutions. 
Sarf %simplifies the lexicon maintenance task, 
provides better accuracy 
and improves run time efficiency as compared to existing analyzers.  
%
\todo{R2.14}We evaluated Sarf also within applications that use its API 
~\citep{JaZaMatar,ZaMaFlairs2012HadithBio,ZaMa2012IJCLATime,ZaMaHaCicling2012Entity}. 
The carried case studies show that Sarf performs better than existing Arabic morphological analyzers in terms of running time. 
They also demonstrate the efficiency and the utility of the Sarf 
application customizable API. 

The rest of this paper is structured as follows. 
\todo{R2.1A} In Section~\ref{sec:motivation}, we motivate Sarf and discuss how it deals with existing morphological analysis challenges. 
In Section~\ref{sec:related}, we compare Sarf to related work. 
In Section~\ref{sec:overview}, we present the structre of Sarf. 
In Section~\ref{sec:api}, we present the interface provided by Sarf for the 
application developer to control and refine the morphological analysis. 
%In Section~\ref{sec:aggaffix}, we present our method to build agglutinative affix morphemes with fusional affix concatenation rules. 
%%In Section~\ref{sec:motivation}, we discuss existing redundancy and inconsistencies in the lexicons of existing morphological analyzers.
In Section~\ref{sec:diac}, we present our method of using partial diacritics to reduce the morphological ambiguity.
\todo{R1.2}In Section~\ref{sec:results}, we discuss the use of Sarf with several NLP tasks and 
present the results of comparing Sarf to other analyzersin terms of speed and accuracy.%, and consistency.
Finally, we conclude and discuss future work in Section~\ref{sec:conclusion}.

%\vspace{-1cm}