\todo{R2.1A}
In this section, we discuss issues that %currently
challenge morphological analysis and explain how Sarf resolves them. 

\subsection{Exhaustive enumeration and performance:}

Current morphological analyzers such as 
%Buckwalter~\citep*{Buckwalter:02},
BAMA~\citep*{Buckwalter:02}, 
SAMA~\citep*{SAMAmain, Kulick:10}, \todo{R3.18} 
Beesley~\citep*{Beesley:01}, 
MADAMIRA~\citep*{pasha2014madamira}, 
and ElixirFM~\citep*{Otakar:07} 
take as input white space delimited tokens,
consider them as words,
and enumerate all possible morphological solutions for each word. 

\begin{table}[tb]
\centering
\begin{minipage}{.9\textwidth}
\caption{\label{t:solexample}Morphological solution for \utfRL{وسيلعبونها}(and they will play it)}
\begin{tabular}{lccc} \hline
  & suffix  \noTrutfRL{ونها} & stem  \noTrutfRL{لعبـ}  & prefix \noTrutfRL{وسيـ} \\ \hline
       POS & \cci{IVSUFF\_SUBJ:MP\_MOOD:I+IVSUFF\_DO:3FS} & \cci{VERB\_IMPERFECT} & \cci{CONJ+FUT+IV3MP}  \\
  Transliteration & \cci{uwnahA} & \cci{loEab} & \cci{wa+sa+ya}  
  \\
           Gloss  & \cci{[MASC.PL.]+it/them/her}.  & \cci{play} & \cci{and they will} \\
  \hline% \hline
\end{tabular}
\end{minipage}
\end{table}


For example, given the word \utfRL{وسيلعبونها}\todo{R2.5}\footnote{The paper uses ArabTex package \citep{arabtex} for Arabic text editing and transliteration. ArabTex package is recommended and used by \citep{IntroNHabash}.} (and they will play it), 
an analyzer may return the solution presented in Table~\ref{t:solexample} with 
a prefix, a stem, a suffix, and their corresponding 
POS, transliteration, and gloss tags. 
%\noTrutfRL{وسيـ} with POS tag \cci{wa/CONJ+sa/FUT+ya/IV3MP} 
%and gloss tag \cci{and they will}, 
%the stem 
%\noTrutfRL{لعبـ} with POS tag \cci{loEab/VERB\_IMPERFECT}
%and gloss tag  \cci{play},
%and 
%the suffix \noTrutfRL{ونها} with POS tag \cci{uwna/IVSUFF\_SUBJ:MP\_MOOD:I+hA/IVSUFF\_DO:3FS}
%and gloss tag \cci{[MASC.PL.]+it/them/her}. 
The prefix 
\noTrutfRL{وسيـ} can be further segmented into 
(1) the proclitic \noTrutfRL{و} with POS tag \cci{CONJ} and gloss tag \cci{and},
(2) the proclitic \noTrutfRL{سـ} with POS tag \cci{FUT} and gloss tag \cci{will},
and (3) the prefix \noTrutfRL{ـيـ} with POS tag \cci{IV3MP} and gloss tag \cci{they(people)}.
Similarly, the suffix 
\noTrutfRL{ونها} 
can be segmented into
(1) the suffix \noTrutfRL{ونـ}, forming a circumfix with \noTrutfRL{ـيـ}, 
with POS tag \cci{IVSUFF\_SUBJ:MP\_MOOD:I}
and gloss tag \cci{[MASC.PL.]}, 
and (2) the enclitic \noTrutfRL{ـها} with POS tag \cci{IVSUFF\_DO:3FS}
and gloss tag \cci{it/them/her}. 
%Some analyzers may refer to \noTrutfRL{و} and \noTrutfRL{سـ} as proclitics, 
%and may refer to \noTrutfRL{ـها} as an enclitic.
\novocalize

The exhaustive enumeration of all solutions may hurt performance and may
not be necessary or appropriate
in some applications~\citep*{Maamouri:10}. 
Thus the need of a customizable morphological analyzer that adapts to 
application specific requirements.
%\vspace{-2.35cm}

\subsection{Diacritics and accuracy}

The accuracy of morphological analysis %, with respect to the context,
suffers from inherent difficulties
%of morphological analysis 
of the Arabic language such as omitted diacritics and position dependent letter
forms. 
%
\fullvocalize
Diacritics, i.e. short vowels, such as 
fatha (\RL{|Ba}), damma (\RL{|Bu}), kasra (\RL{|Bi}), 
tanween (i.e. doubled diacritic including \RL{|BaN}, \RL{|BuN}, \RL{|BiN}), 
\transfalse and sokun (\RL{|B}) 
are almost always omitted in written Arabic text as they can be 
inferred by human readers.
The mark \novocalize shadda (\RL{|BB}) \transtrue denotes the repetition
of the marked character and is also often omitted. 
Partial diacritics can help disambiguate solutions. 
%This is a source of disambiguity. 
Consider the unvocalized word \RL{'kl} with nine morphological solutions.  
Its partially vocalized version \utfRL{أكِل} has only two solutions; 
VMF \utfRL{أَكِل} with gloss \cci{I+trust/put in charge}, and 
VMF \utfRL{أُكِلّ} with gloss \cci{I+make tired/wear out}.

Analyzers such as BAMA and SAMA ignore partial diacritics while 
other analyzers such as \citep*{Beesley:01,Chaaben:10,Attia:00} make use of 
the partial diacritics to reduce ambiguity. 
Sarf provides an option that enables the use of existing diacritics for disambiguation, 
and considers the diacritics at morpheme boundaries to generate only the diacritic 
matching solutions, rather than generating all morphological solutions then 
filtering them. 

%\vspace{-1.2cm}

\subsection{Problem of `run-on' words:}

Arabic letters have up to four different forms
corresponding to their position in a word, i.e, beginning,
middle, end, and separate word forms. 
This allows the phrase \transfalse
\RL{il_A\nospace almdrsT} \transtrue \todo{R2.12} (to the school)
to be visually recognizable
as two separate words \RL{il_A} (to) and \RL{almdrsT} (the school) 
without the need of a delimiter space in between. 
The reason is the first word \RL{il_A} ends with
\RL{_A} a non-connecting letter. 
These words, referred to as `run-on' words~\citep*{Buckwalter:04},
occur regularly, and greatly increase the
difficulty of tokenization. 
\todo{2.1A}None of existing morphological analyzers, up to our knowledge, 
resolve this issue. Sarf on the other hand does by the structures and rules it introduces, which are further discussed in the next issue.

\begin{comment}
\vocalize
\begin{table}[!tb]
\begin{minipage}{\textwidth}
\relsize{-1.5} {
\begin{tabular}{ccccc} 
\hline \hline
\textbf{Prefix} & \textbf{Vocalized} & \textbf{Category} & \textbf{Gloss} & \textbf{POS data} \\ \hline
\RL{f} & \RL{fa} & Pref-Wa & and/so & fa/CONJ+ \\
\RL{y} & \RL{ya} & IVPref-hw-ya & he/it & ya/IV3MS+ \\
\RL{fy} & \RL{faya} & IVPref-hw-ya & and/so + he/it & fa/CONJ+ya/IV3MS+ \\
\RL{sy} & \RL{saya} & IVPref-hw-ya & will + he/it & sa/FUT+ya/IV3MS+ \\
\RL{fsy} & \RL{fasaya} & IVPref-hw-ya & and/so + will + he/it & fa/CONJ+sa/FUT+ya/IV3MS+ \\ \\

\RL{y} & \RL{ya} & IVPref-hmA-ya & they (both) & ya/IV3MD+ \\
\RL{fy} & \RL{faya} & IVPref-hmA-ya & and/so + they (both) & fa/CONJ+ya/IV3MD+ \\
\RL{sy} & \RL{saya} & IVPref-hmA-ya & will + they (both) & sa/FUT+ya/IV3MD+ \\
\RL{fsy} & \RL{fasaya} & IVPref-hmA-ya & and/so + will + they (both) & fa/CONJ+sa/FUT+ya/IV3MD+ \\  \\

\RL{w} & \RL{wa} & Pref-Wa & and & wa/CONJ+ \\
\RL{wy} & \RL{waya} & IVPref-hw-ya & and + he/it & wa/CONJ+ya/IV3MS+ \\
\RL{wsy} & \RL{wasaya} & IVPref-hw-ya & and + will + he/it & wa/CONJ+sa/FUT+ya/IV3MS+ \\
\RL{wy} & \RL{waya} & IVPref-hmA-ya & and + they (both) & wa/CONJ+ya/IV3MD+ \\
\RL{wsy} & \RL{wasaya} & IVPref-hmA-ya & and + will + they (both) & wa/CONJ+sa/FUT+ya/IV3MD+ \\
\hline \hline
\end{tabular}
}
\end{minipage}
\caption{Partial BAMA v1.2 prefix lexicon}
%\vspace{-2cm}
\label{t:affixes}
\end{table}


\subsection {Structure of existing analyzers and related problems:}

Concatenative morphological analyzers~\citep{Buckwalter:02,Kulick:10} %BAMA,SAMA
are based on lexicons of prefixes $L_p$, 
stems $L_s$, and suffixes $L_x$. 
As shown in Table~\ref{t:affixes}, each entry in a lexicon includes the morpheme, 
its vocalized form with diacritics, 
a {\em concatenation compatibility category} (CCC) rule, 
a gloss tag, and a POS tag.
Separate CCC rules specify the compatibility of prefix-stem 
$R_{ps}$, stem-suffix $R_{sx}$, and prefix-suffix (circumfix) $R_{px}$ concatenations. 
The affixes \noTrutfRL{و} and 
\noTrutfRL{يـ} in the before mentioned example are valid standalone prefixes, 
and can be concatenated to the stem \noTrutfRL{لعب} \todo{R2.12} (he plays) to form 
\noTrutfRL{ولعب} (and he plays) and
\noTrutfRL{يلعب} (he is playing), respectively. 
%In addition, the morpheme \noTrutfRL{سـ} can connect to \noTrutfRL{يلعب} to form 
%\noTrutfRL{سيلعب}. 
The $L_p$ and $L_x$ lexicons contain also all final forms of concatenated affixes
as shown in Table~\ref{t:affixes} for sample morphemes. 

This is the source of several problems: 
\begin{itemize}
%\vspace{-2cm}
    \item $L_p$ and $L_x$ contain redundant entries which result \todo{R2.13} in maintenance and consistency issues~\citep*{LRECMaamouriKB08,lrecKulickBM10}.
    \item Augmenting $L_p$ and $L_x$ with additional morphemes, such as \utfRL{أَ} (the question glottal hamza), 
      may result in a quadratic increase in the size of the lexicons~\citep*{Hunspell}.
      The additional morpheme may attach to exiting morphemes. 
      Currently, this results in adding all the resulting morphemes to the BAMA and SAMA lexicons. 
    \item The $L_p$ and $L_x$ lexicons are larger than needed especially that 
      they have to account 
      for several forms of a morpheme with varying diacritics.
    \item The concatenated forms in $L_p$ and $L_x$ contain concatenated POS and 
      other tags. 
%        The segmentation correspondence between the prefix concatenated from several morphemes and the tags associated with it is lost. 
%        In several cases, this leads to missing correspondence between the tokens of the morphological 
%        solution and the segmentation of the original word.
The alignment and correspondence between the original word and its
morphemes with the tags of its morphological solution 
are essential to the success of NLP tasks such as MT 
and IE~\citep*{Regina2011,ELRASemmar08}.
The analysis of the example \utfRL{للقضاء} (for/to the justice/judiciary\todo{R2.12 gloss}) using SAMA,
generates the morphological solution \cci{li/PREP + Al/DET + qaDA'/NOUN}.
The issue of the alignment and correspondence between the original word \noTrutfRL{للقضاء} 
and its morphemes arises as the concatenation of the unvocalized prefix morphemes 
\cci{li} and \cci{Al} is not the same as the corresponding prefix 
segment \noTrutfRL{لل} in the original word. 
\todo{R.3.5}
This is not a problem for SAMA, but rather for the applications that 
use the SAMA output such as treebanking that is interested in \cci{li} be present
as a separate leaf in the syntactic tree. 
Sarf provides a general solution for the segmentation correspondence problem 
since the valid compound affixes preserve the input text segmentation. 
In particular, a partial affix \noTrutfRL{ل} \cci{Al/DET}
connects to the atomic affix \noTrutfRL{ل} \cci{li/PREP} and resolves the problem.

%is segmented into two tokens:
%\cci{li/PREP} and \cci{Al/DET + qaDA'/NOUN}~\citep{LRECMaamouriKB08}. 
%The best approximation of the unvocalized entry 
%of each token is
%\noTrnoVocRL{li} and \noTrnoVocRL{Alqa.dA'}, respectively, 
%with an extra letter \RL{A}. 
%This is not a faithful representation of the original text data 
%and the segmentation does not correspond with that of the
%input text.
\end{itemize}

\todo{R3.1}
Alternatively, 
Sarf extends and refines the lexicons and compatibility category rules of BAMA and SAMA 
such that:
\begin{itemize}
  \item it represents only atomic affix morphemes in the lexicons,
  \item it adds prefix-prefix $R_{pp}$ and suffix-suffix $R_{xx}$ agglutinative and 
    fusional rules,
  \item it generates compound affixes from the atomic ones using $R_{pp}$ and $R_{xx}$, and
  \item it extends the stem lexicon with additional words necessary for the NLP applications it considers.
\end{itemize}

Considering the entries of Table \ref{t:affixes} as example, Sarf represents them using 
only five atomic affix morphemes and five 
prefix-prefix rules.\todo{R2.4} 
\end{comment}