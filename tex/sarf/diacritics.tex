Diacritics are short vowels that
are often omitted in Arabic text and inferred by readers from context. 
Their omission adds to the ambiguity problem of Arabic morphological analysis. 
\fullvocalize
The diacritics \RL{|Ba} ({\em fatha}),  \RL{|Bu} ({\em damma}), and 
\transfalse\RL{|B}\transtrue represent
and appears above the letter. 
`a' vowel,  `o' vowel, and consonant, respectively,  and appear above the letter. 
The diacritic \RL{|Bi} represents a `y' vowel and appears below the letter. 
The diacritics 
\RL{|BaN}, \RL{|BuN}, and \RL{|BiN} represent the `a', `o', and `y' vowels followed 
by a phonetically stressed \transtrue \RL{n} consonant. 

The shadda \RL{|BB} mark is not a 
diacritic but is treated typographically as one, and is also often omitted in Arabic text. 
It denotes  a repeated letter, first as consonant, and second 
as vocalized.
Arabic forbids two letters with a consonant diacritic to follow each other. 
%A shadda can be followed by all diacritics except \transfalse\RL{|B}\transtrue. 


Analyzers such as BAMA and SAMA ignore input partial diacritics because 
they consider them to be (1) rare in common corpora, and (2) 
unreliable because of dialect diversity and human errors~\citep*{Elkateb:06,Attia:06b}. 
\todo{R3.13}
The diacritics can be miss-placed for several reasons including: 
(1) writers use a local Arabic dialect pronunciation of the word, 
(2) font effects such as ligatures compensate for miss-placing diacritics, 
and
(3) writers sometimes use diacritics to decorate their text. 
%
However, the work in \citep{Chaaben:10,Attia:00,Beesley:01} considers 
partial diacritics to decrease morphological ambiguity.
The work in~\cite{Maamouri:06} inspected the ATB v3.2 corpus~\citep*{Maamouri:04}
for diacritics and found that 1.364\% of the words were partially diacriticized and 
those diacritics eventually and {\em effectively} reduced morphological ambiguity.
Most of the diacritics we found were tanween marks which indicate indefiniteness, 
followed in frequency by shadda, and then by dhamma that often indicates 
passive tense when associated with verbs. 
Hence, we decided not to force the use of partial diacritics to disambiguate the 
solutions and to provide it as an option.

%Figure~\ref{t:consistent} presents the Diacritic-aware consistency check Algorithm.
Key to partial diacritic analysis is a diacritic-aware consistency check that replaces
standard string matching checks. 
The \cci{ Diacritic-aware consistency check} algorithm takes as input 
two text chunks $c_1$ and $c_2$. 
\todo{R2.14}It checks if the sequence of letters in $c1$ is the same as that in $c2$. 
It also checks the consistency of the sequence of diacritics in $c1$ and in $c2$. 
%It first checks if the sequence of letters, without diacritics, in $c1$ is consistent with that in $c2$. 
%Then, it checks if the sequence of diacritics of letters in  $c1$ is consistent with that of $c2$.
%
%It checks that the sequence of non-diacritic letters, ignoring the 
%diacritics between them, are equal. 
%It also checks that all sequences of diacritics occurring between 
%non-diacritic letters are consistent. 

Two sequences of diacritics are consistent iff:
%\vspace{-3em}
\begin{enumerate}
  \item Both are equal, or
  \item One of the sequences is empty, or
%  \item don't contain different diacritics
  \item If one has a shadda, then the other has no sukoun, or
  \item If one has a shadda and the other has no shadda, then the rest of the diacritics are compared recursively. 
\end{enumerate}

If both checks were positive, then $c1$ and $c2$ are said to be equal with diacritic-aware consistency. \todo{R2.14}

\begin{table}[tb]
\centering
\begin{minipage}{0.9\textwidth}
\caption{\label{t:correspondence}Arabic string comparison with consideration of partial diacritics}
\begin{tabular}{cccc} %\hline
\toprule
& word & string comparison & diacritic-aware consistency check 
%\vspace{0.2em} 
\\\colrule% \cline{2-4}
(a)\label{ex:valid_correspondence} &
	\begin{minipage}{0.1\textwidth}
	\begin{tabular}{c} 
	\utfRL{أَكّل} \\
	{\bf $\updownarrow$} \\
	\utfRL{أكَل} \\
	\end{tabular}
	\end{minipage}
&
	\begin{minipage}{0.3\textwidth}
	\begin{tabular}{ccccc} 
	\noVocRL{'} & \RL{|Ba} & \noVocRL{k} & \noVocRL{|BB} & \noVocRL{l} \\
	$\updownarrow$ & \textcolor{red}{$\updownarrow$} & \textcolor{red}{$\updownarrow$} & 
			\textcolor{red}{$\updownarrow$} & \textcolor{red}{?} \\
	\noVocRL{'} & \noVocRL{k} & \RL{|Ba} & \noVocRL{l} \\
	\end{tabular}
	\end{minipage}
&
	\begin{minipage}{0.4\textwidth}
	\begin{tabular}{ccccc} 
	\noVocRL{'} & \RL{|Ba} & \noVocRL{k} & \noVocRL{|BB} & \noVocRL{l} \\
%	\RL{'} & \fullvocalize \RL{|Ba} \novocalize & \RL{k} & \RL{|BB} & \RL{l} \\
	$\updownarrow$ & \multicolumn{4}{l}{~~~~~$\nearrow$ ~~~~~ $\nearrow$ ~~~~ $\nearrow$}  \\
%	\RL{'} & \RL{k} & \fullvocalize \RL{|Ba} \novocalize & \RL{l} \\
     \noVocRL{'} &  \noVocRL{k} & \RL{|Ba} & \noVocRL{l} \\
	\end{tabular}
	\end{minipage}

\\ \hline
(b)\label{ex:invalid_correspondence} &
	\begin{minipage}{0.1\textwidth}
	\begin{tabular}{c} 
	\utfRL{أُكِل} \\
	{\bf \color{red} $\updownarrow$}  \\
	\utfRL{أكَل} \\
	\end{tabular}
	\end{minipage}
&
	\begin{minipage}{0.3\textwidth}
	\begin{tabular}{ccccc} 
	\noVocRL{'} & \RL{|Bu} & \noVocRL{k} & \RL{|Bi} & \noVocRL{l} \\
	$\updownarrow$ & \textcolor{red}{$\updownarrow$} & \textcolor{red}{$\updownarrow$} & 
			\textcolor{red}{$\updownarrow$} & \textcolor{red}{?} \\
	\noVocRL{'} & \noVocRL{k} & \RL{|Ba} & \noVocRL{l} \\
	\end{tabular}
	\end{minipage}
&
	\begin{minipage}{0.4\textwidth}
	\begin{tabular}{ccccc} 
	\noVocRL{'} & \RL{|Bu} & \noVocRL{k} & \RL{|Bi} & \noVocRL{l} \\
	$\updownarrow$ & \multicolumn{4}{l}{~~~~~$\nearrow$ ~~~~~ \textcolor{red}{$\nearrow$} ~~~~ $\nearrow$}  \\
	\noVocRL{'} & \noVocRL{k} & \RL{|Ba} & \noVocRL{l} \\
	\end{tabular}
	\end{minipage}
\\ %\hline
\botrule
\end{tabular}
\end{minipage}
\end{table}



Table~\ref{t:correspondence} illustrates the 
\cci{diacritic-aware consistency check} as compared to the standard
string comparison with an example. 
Part (a) shows two diacritic-consistent words 
\vocalize
\utfRL{أَكّل} and \RL{'kal} as a fatha \RL{|Ba} is compatible with an empty diacritic 
and a shadda \RL{|BB} is compatible with a fatha \RL{|Ba}. 
%In fact, a shadda is compatible with all diacritics except 
%for the sukun.
%and consequently its disambiguation value is mainly measured 
%through its presence/absence and not through its compatibility. 
Part (b) illustrates two inconsistent diacritizations 
since \RL{|Bi} is incompatible with \RL{|Ba} next to the letter \noVocRL{k}.
%Inconsistencies can be detected when different short vowels are encountered as in 
%part~\hyperref[ex:valid_correspondence]{(b)} of Table~\ref{t:correspondence}
%in which \fullvocalize

\todo{R2.15, R3.14}
Sarf takes an Arabic word as input with possible partial diacritics. 
Then, it traverses ${\cal P}$, ${\cal S}$, and ${\cal X}$ structures 
to compute the morphological solutions. 
For each matching morpheme in the structures, 
Sarf uses the \cci{diacritic-aware consistency check} algorithm 
to check for the consistency of the fully diacriticized morpheme
with the corresponding partially diacriticized chunk of the input word.
The traversal path corresponding to the morpheme dies if the consistency 
check is negative (inconsistent).
