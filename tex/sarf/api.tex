\label{s:api}
Sarf provides the developer with an application programming interface (API) 
  that allows to 
  (1) define developer categories and associate them with specific morphemes, 
  (2) provide rules that prioritize and filter the solution features, and 
  (3) control and refine the morphological analysis on the fly at solution {\em control points}.
%
The control points are (1) agglutinative prefix matches, (2) stem matches, 
(3) agglutinative suffix matches, and 
(4) full solution matches. 

\todo{R1.3}
To use the API, an NLP application developer is required to inherit from 
one or more of the prefix, suffix, stem and general 
stemmer C++ classes and implement the \textit{OnMatch} interface. 
The developer configures the feature priority and selection rules 
by using the \textit{setSolutionSetting} method to push the desired features 
into a feature configuration vector. 
The computation returns the features present in vector in the solutions ordered 
by the same order in the vector. 
If the vector is empty, all features are considered. 

The developer implements the \textit{OnMatch} interface that Sarf calls at the 
control points. 
The developer processes the solution features provided at the control point and returns 
a value that instructs Sarf to 
(1) proceed with the analysis ignoring the current solution, 
(2) accept the solution and continue considering other solutions, 
and (3) accept the solution and stop the analysis.
%
The developer can inspect at each control point 
the agglutinative morphemes and their compatibility category tags, 
the VMF, gloss, POS, lemma, and the developer-defined category tags. 

Consider the task that aims to detect words with possible VERB POS tags. 
The API can be implemented to reject the analysis at the stem control point if the POS tag is not a VERB. 
This prevents the analyzer from computing insignificant 
full morphological solutions at early stages of the analysis.
The developer can also use the feature selection filter API to 
disregard features such as VMF, gloss, and categories. 
This reduces the size of the solution trees and 
their corresponding traversal time. 
%
For example, given the word \RL{'kl}, 
with gloss tags such as `he/it eat' and `fruit',
Sarf typically returns nine possible morphological solutions 
with different VMF, POS tags, and gloss tags. 
With the feature selection filter API, Sarf considers only two solutions with 
two solutions with the {\tt VERB\_PERFECT} and {\tt NOUN} POS tags.

Sarf also provides the developer with the ability to alter 
the structure of the morphological solutions so 
that the traversal of the solutions is application specific. 
The developer can provide a factorization order 
of the solution features using the API priority rules. 
Sarf uses the priority rules to build the structures. 
This allows the developer to dismiss analysis earlier at the control points. 
Consider the task with an aim is to detect adjectives 
with positive sentiment. 
Since the POS feature is more limited than the gloss feature, 
it might give priority to the POS over the gloss. 
Hence, it can filter invalid solutions early in 
the analysis.

Moreover, Sarf enables the application developer to define 
categories and associate them with existing morphemes. 
Consider the task of detecting words that indicate 
family relations such as \RL{ibn} (son), \RL{'b} (father), 
and \RL{'m} (mother). 
The developer can define a category called `family connections' 
and associate it with the stems \RL{ibn}, \RL{'b}, \RL{'m} 
or with their relevant glosses. 
The user defined categories are attached to the tags in the 
morpheme nodes as auxiliary tags that can be looked up in constant time. 

%During the analysis, Sarf calls the API at the control points and 
%the API detect the morphemes that fall under the defined category.
