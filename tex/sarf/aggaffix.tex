%\vocalize
Sarf considers three types of affixes: 
%\vspace{-2em}
\begin{itemize}
  \item {\em Atomic affix morphemes } such as \noTrutfRL{يـ} can be affixes on their own and can directly connect to stems using the $R_{ps}$ and $R_{sx}$ rules. 
  \item {\em Partial affix morphemes} such as \noTrutfRL{سـ} can not be affixes on their own and need to connect to other affixes before they connect to a stem.
  \item {\em Compound affixes} are concatenations of atomic and partial affix morphemes as well as other smaller compound affixes. 
    They can connect to stems according to the $R_{ps}$ and $R_{sx}$ rules. 
\end{itemize}

Sarf forms compound affixes from atomic and partial affix morphemes using newly introduced prefix-prefix $R_{pp}$ and suffix-suffix $R_{xx}$ concatenation rules. 
%
Sarf considers $L_p$ and $L_x$ to be lexicons of atomic and partial affix morphemes associated with their tags. 
Sarf forms agglutinative affixes using prefix-prefix $R_{pp}$ and suffix-suffix $R_{xx}$ concatenation or agglutination rules.
A rule $r \in R_{pp} \cup R_{xx}$ takes the compatibility category tags of affixes $a_1$ and $a_2$ and checks whether they can be concatenated. 
If so, the rule takes $a_1$ and $a_2$ and their tags and generates the
affix $a=r(a_1,a_2)$ with its associated tags according to substitution rules based on regular expressions. 
The rules are fusional in the sense that they modify the orthography and the semantics of the resulting 
affixes by more than simple concatenation.

\begin{table}[!tb]
\begin{minipage}{0.95\textwidth}
\relsize{-1.5} {
  \begin{tabular}{lp{4.4cm}cp{3.3cm}}
\hline \hline
{\bf Category 1} & {\bf Category 2} & {\bf Resulting Category}  & {\bf Substitution rules} \\ \hline
        NPref-Bi & NPref-Al & NPref-BiAl &  \\[3pt]
        NPref-Li & NPref-Al & NPref-LiAl & \em r$//$\noTrRL{Al} $||$\noTrRL{l}$\backslash\backslash$ \\[4pt]
 Pref-Wa & none of \{ Pref-0, NPref-La, PVPref-La\} & \$2 & \\[4pt]
%Pref-sa- & \{IVPref* AND NOT IVPref-li-\} & \{\$2\} \\
IVPref-li- & IVPref-*-y* & IVPref-(@1)-liy(@2)  & { \em d$//$he/$||$him/$\backslash\backslash$, ~ d$//$they$||$them$\backslash\backslash\ldots$ ~ d$//$(+2)$||$ to$\backslash\backslash$} \\ 
\hline \hline
\end{tabular}
}
\end{minipage}
\caption{Example rules from $R_{pp}$ }
%\vspace{-1cm}
\label{t:rules}
\end{table}
%
We illustrate this with the example rules in Table~\ref{t:rules}.
Row 1 presents a simple rule that allows the concatenation of prefixes with category \cci{NPref-Bi} such as \RL{bi-} and \RL{ka-}
to prefixes with category \cci{NPref-Al} such as \noTrRL{Al}, the result is the compound prefix with category \cci{ NPref-BiAl}. 
Since no substitution rule is specified, the tags of the resulting prefix are simple concatenations.

Row 2 presents a rule that takes prefixes with category 
\cci{NPref-Li} such as \RL{li-} and prefixes with category \cci{NPref-Al} such as \noTrRL{Al}.
The substitution rule replaces the \noTrRL{Al-} with \noTrRL{l-} resulting in \noTrRL{lil-}. 
The syntax of the substitution rule for the affix form is \cci{r$//$(substring)$||$(replacement)$\backslash\backslash$}. 

The rule in the Row 3 states that prefixes of category \cci{Pref-Wa} can be concatenated with 
prefixes with categories that are neither of 
\cci{Pref-0}, \cci{NPref-La}, and \cci{PVPref-La} categories. 
%The Boolean operators allowed are \cci{AND}, \cci{OR}, and \cci{NOT}.
The resulting category is denoted with $\$2$ %$
which means the category of the second prefix.

Row 4 illustrates the use of the wild character \chci{*} to capture sub-strings of length zero or more in the second category,
and refers to the captured sub-strings in the resulting category using the \chci{@} operator. 
The \chci{@} operator is always followed by a number that denotes the captured \chci{*} expression.
Row 4 has also an example of substitution rules for the gloss (description) tag that start with the letter $d$.
The \cci{+2} pattern in the last substitution rule means that the \cci{to} partial gloss description should be appended after the gloss 
of the second affix.
Substitution rules for POS tags start with the letter $p$.

\todo{R3.10}
The rule introduced in row 4 replaces 24 prefix category rules in BAMA. 
Given the word \utfRL{ليأكل} (for him/it to eat/consume), 
Sarf returns the solution with the prefixes \noTrutfRL{ل} (for) and \noTrutfRL{ي} (him/it), 
and the stem \noTrutfRL{أكل} (eat/consume). 
The prefixes \noTrutfRL{ل} and \noTrutfRL{ي} have the categories 
\cci{IVPref-li-} and \cci{IVPref-hw-ya}, respectively. 
The category \cci{IVPref-li-} is the same as category 1 of the rule in row 4 and 
the category \cci{IVPref-hw-ya} matches the pattern of category 2 of the same rule. 
The first wild character within the pattern captures the sub-string \cci{hw} 
while the second wild character captures the sub-string \cci{a}. 
Sarf recognizes that those two affixes are compatible using this rule, 
and applies the corresponding expression to get the resulting category 
\cci{IVPref-hw-liya} for the compound affix \noTrutfRL{لي}. 
Then, Sarf matches the result category of the compound affix 
with the category \cci{IV\_no-Pref-A} of the stem \noTrutfRL{أكل} 
and returns a morphological solution.

%\vspace{-3em}
\subsection{Building $R_{pp}$ and $R_{xx}$}
\label{ss:affix-affix}

Our method is in line with native Arabic textbooks on morphology and syntax~\citep*{Mosad09,Abd00,Abd001} 
where only atomic and partial affixes are introduced.
The textbooks also list rules to concatenate the affixes and discuss the syntax, semantic, and phonological forms of the
resulting affixes. 
%This representation is more suited for linguists.
For example, the fourth rule in Table~\ref{t:rules} is derived from a textbook rule that states 
\cci{IVPref-li-} prefixes connect to all imperfect verb prefixes and transform the subject pronoun in the gloss 
to an object pronoun.

The method built the rules in four steps: 
%\vspace{-2em}
\begin{enumerate}
    \item In the first step, we encoded textbook morphological rules into patterns.
    \item In the second step, we inspected the BAMA and SAMA affix lexicons and extracted the atomic and partial affixes from them.
    \item Then, we grouped the rest of the BAMA and SAMA affixes into the rules we collected from the textbooks. 
    \item We refined the rules wherever necessary, and we grouped rules that shared the same patterns. 
\end{enumerate}
%\vspace{-1em}

We validated our work by generating all possible affixes and compared them against the BAMA and SAMA affix lexicons. 
The comparison resulted in discovering the BAMA and SAMA inconsistencies listed in Tables ~\ref{t:incBAMA} and ~\ref{t:incSAMA}.
